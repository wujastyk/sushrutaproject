% !TeX root = main.tex

\section{Sūtrasthāna, adhyāya 14:  On Blood}


\subsection{Literature} 

\subsubsection{Previous scholarship}

Meulenbeld offered an annotated overview of this chapter and a bibliography
of studies on Indian 

%leeches and their application.\footcite[IA, 209; IB,
%324, n.\,131]{meul-hist}


XXX

\subsection{Translation}

\begin{translation}    
\item [1] Here is the first passage KKK

   \begin{enumerate}
  
       \item 
    And now we shall explain \ldots \se{karma}{fate}

\item 

    \cite[33]{adri-1984} -> Author 1999: 33
\item     
    \citep[33]{adri-1984} -> (Author 1999: 33)
\item     
    \citet[33]{adri-1984} -> Author (1999: 33) 
\item 
    As Ḍalhaṇa remarked on \Su{1.14.22}{64}, \ldots  % quick way to refer to vulgate passages and pages
\item \SS
\item \CS
   \end{enumerate}
    

\end{translation}

