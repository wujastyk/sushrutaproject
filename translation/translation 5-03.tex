% !TeX root = incremental_SS_Translation.tex
\section{Kalpasthāna, adhyāya 3}

\subsection{Introduction}

\subsection{Translation}

\begin{translation}
    \item[1]
And now we shall explain the \se{kalpa}{rule} that is the required knowledge about mobile
poisons.\footnote{In contrast to stationary, plant poisons.  No reference is made to 
Dhanvantari \citep[see][]{birc-2021}.}\q{Come back to the issue of "kalpa".  Look up 
passages in the Kośa.}

\item[2] 

\item[3] 

The full explanation about the sixteen \se{adhiṣṭhāna}{carriers} of the mobile
poisons, that have been mentioned by me in brief, will be stated.


\footnote{ “Carrier” for \se{adhiṣṭhāna}{base, foundation} 
 tries to capture the idea that the author will describe the creatures in which poisons inhere.} 
 
 \item[4]
 In that context, they are: 
 \begin{itemize}
     \item sight and breath,
     \item teeth and nails,
     \item \diff{mouth},
     \item  urine and faeces,
     \item \diff{menstrual blood},
     \item semen,
     \item \diff{penis},
     \item saliva,
     \item \diff{lethal points},
     \item \se{mukhasaṃdaṃśā}{nipping with the mouth},
     \item \se{avaśardhita}{fart},\footnote{This interpretation comes from
     Ḍalhaṇa on \Su{5.3.4}{567}, but he reads \dev{viśardhita}.}
     \item \diff{anus},\footnote{Ḍalhaṇa on \Su{5.3.4}{567} noted this reading.}
     \item bones,
     \item bile,
     \item \se{śūka}{bristles},
and 
     \item corpses.
     
     \end{itemize}
    
    \item[5] TBA
    
    \item [6]
    \diff{The enemies of the king pollute the 
    waters, roads and foodstuffs 
    in enemy territory. The experienced physician, 
    who has learned how to purify things, 
    should clean up those polluted things.} 
    
    \item[7]
    
    Polluted water is slimy and smells of tears.\footnote{\dev{asra} normally
means “tears,” but rarely means “blood.” }  It is covered with froth and
covered with streaks. The frogs and fish die, the birds are crazed and, along with
the wetland creatures, they wander about aimlessly.
    
     \item [8]
     
     Men, horses and elephants who swim in it experience vomiting, delusion,
fever, swelling and sharp pains.\footnote{On the polysemy of \se{nāga}{elephant\slash 
snake}, see \cite{seme-1979}.} 
    He should try to purify that polluted water, after curing their ailments.
     
\item [9]

And so, he should burn
    \gls{dhava}  and 
    \gls{aśvakarṇa}, % Dipterocarpa turbanitus Gaertn., GVDB 28
        as well as
    \gls{pāribhadra}, %Erythrina suberosa, GVDB 245
        with
    \gls{pāṭalā} % GVDB 242 Stereospermum suaveolens DC.
        and 
    \gls{sidhraka} % 
        and 
   \gls{muṣkaka}, % GVDB 312, Schrebera swietenioides Roxb and 
   %Elaeodendron glaucum Pers.
        and with
    \gls{rājadruma} % āragvadha, GVDB 37 Cassia fistula Linn
        and 
    \gls{somavalka}.
Then he should sprinkle that ash, cold, on the waters.

\item [10--11]

And in the same way, putting a handful of the ash in a pot, one may also purify water that 
one wants.

If any one of the limbs of cows, horses, elephants, men or women, touch a place on
the ground that enemies have spoiled with poison, or a ford or rock or a flat
surface, then it swells up and burns and its hair and nails fall out on that
place.\footnote{``Swells up" translates an unclear reading that was probably
    \dev{śūyati}, which may be an irregular form of $\surd$\dev{śū, śvā, śvi} 
    \citep[see][175--176]{whit-root}.}

\item [12]

In that case, one should grind up \gls{anantā} together with all the aromatic items, with 
wines, ???.  Then he shoot should sprinkle  the paths to be used. Then the shoot, and 
together with mud, \ldots

\item [13]


\end{translation}