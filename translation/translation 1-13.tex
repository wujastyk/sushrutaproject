% !TeX root = incremental_SS_Translation.tex

\section{Sūtrasthāna, adhyāya 13:  On Leeches}


\subsection{Literature} 

\subsubsection{Previous scholarship}

Meulenbeld offered an annotated overview of this chapter and a bibliography
of studies on Indian leeches and their application.\footcite[IA, 209; IB,
324, n.\,131]{meul-hist}

A Persian version of this chapter of the \SS\ was included in \emph{Sikandar
    Shāh's Mine of Medicine} (\emph{Ma`din al-\underbar{sh}ifā' i
    Sikandar-\underbar{Sh}āhī}) composed in 1512 by Miyān Bhūwah b.
\underline{Kh}awāṣṣ \underline{Kh}ān.\footcites[96--109]{sidd-1959}
{azee-1971} [231--232]{stor-1971} [IB, 324,
n.\,128]{meul-hist}[8--9]{spez-2019}

More recently, Brooks has explored the sense of touch in relation to
leeching and patient-physician
interactions.\footcite{%
    broo-2020,
    broo-2020b,
    %broo-2018,
    broo-2020c}

\subsection{Translation}

\begin{translation}    
\item [1] 
    And now we shall explain \diff{the chapter} about leeches.
    
\item [3] The leech is for the benefit of kings, rich people, delicate people,
children, the elderly, fearful people and women.  It is said to be the most
gentle means for letting blood.

\item [4]

In that context, one should let blood that is corrupted by wind, bile or
phlegm with a horn, a leech, or a \gls{alābu}, respectively.   Or, each kind
can be be made to flow by any of them in their particular way.\footnote{This
    sentence is hard to construe grammatically, although its meaning seems
    clear. In place of \dev{viśeṣastu}, Cakrapāṇidatta and Ḍalhaṇa both read
    \dev{viśeṣatas}, which helps interpretation (\cite[95]{acar-1939},
    \cite[55]{vulgate}). It is notworthy that the critical syllable \dev{stu} is
    smudged or corrected in both \MScite{Kathmandu NAK 1-1079} and in 1-1146, a
    much later Devanāgarī manuscript.\MSsilent{Kathmandu NAK 1-1146}
      
There is an insertion in the text, printed in parentheses in the
vulgate at \Su{1.13.4}{55} as  \dev{viśeṣatastu visrāvyaṃ
śṛṅgajalaukālābubhirgṛhṇīyāt}.  This insertion is not included in the
earlier edition of the vulgate, but is replaced by
\dev{snigdhaśītarūkṣatvāt} \citep[54]{susr-trikamji2}. Ḍalhaṇa noted that,
“this reading is discussed to some extent by some compilers
(\dev{nibandhakāra}), but it is definitely rejected by most of them,
including Jejjhaṭa.” }

\item[5x]  And there are the following about this:

\item [1.13.5]

The horn of cows is praised for being unctuous, \diff{smooth}, and very
sweet.  Therefore, when wind is troubled, that is good for
bloodletting.\footnote{The vulgate replaced “smooth” with “hot.”}

\item [1.13.5a]

Having a length of seven fingers and a large body the shape of a half moon,
should first be placed into a cut.  A strong person should suck with the
mouth.\footnote{This passage is not found in the vulgate, but it is similar
    to the passage cited by Ḍalhaṇa at \Su{1.13.8}{56} and attributed to
    Bhāluki.  Bhāluki was the author of a \emph{Bhālukitantra} that may have
    predated Jejjaṭa and might even have been one of the sources for the \SS\
    \pvolcite{IA}[689--690 \emph{et passim}]{meul-hist}. The editor Ācārya was
    aware of this reading in the Nepalese manuscripts; see his note 4 on
    \Su{1.13.5}{55, note 4}.}

\item[6]

A leech lives in the cold, is sweet and is born in the water. So when
someone is afflicted by bile, they are suitable for
bloodletting.\footnote{Note that the particular qualities (\emph{guṇa}s) of
    the leech in this and the following verses counteract the quality of the
    affliction.  See \cite[113, table 1]{broo-2018}.}

\item[7]

A \gls{alābu} is well known for being pungent, dry and sharp.  So
when someone is afficted by phlegm it is suitable for bloodletting.

\item[8]

In that context, at the scarified location one should let blood using a horn
wrapped in a covering of a thin bladder, or with a \gls{alābu} with a flame
inside it because of the suction.\footnote{There are questions about the
    wrapping or covering of the horn.  Other versions of the text, and the
    commentator, propose that there may be two coverings, or that cloth may be a
    constituent. Comparison with contemporary horn-bloodletting practice by
    traditional Sudanese healers suggests that a covering over the top hole in
    the horn is desirable when sucking, to prevent the patient's blood entering
    the mouth \citep{pbs-2020}.  Our understanding of this verse is that the
    bladder material is used to cover the mouthpiece and then to block it, in
    order to preserve suction in the horn for a few minutes while the blood is
    let. }

\item[9]

Leeches are called “\emph{jala-ayu-ka}” because \se{jala}{water} is their
\se{āyur}{life}.\footnote{This is a folk etymology.} “Home” (\emph{okas})
    means “{dwelling};” their home is water, so they are called
    “\se{jalaukas}{water-dwellers}.”

\item[10]

There are twelve of them: six are venomous and just the same number are 
non-venomous. 

\item[11]
Here is an explanation of the poisonous ones, together with the therapy:
\begin{itemize}
    \item \se{kṛṣṇā}{Black}
    \item \se{karburā}{Mottled}
    
    \item \se{alagarddā}{Sting-gush}\footnote{Treating \dev{gardā} as
    \dev{galdā} and translating as in RV 8.1.20, with \citet[1023, verse 20
    and cf.\ commentary]{jami-2014}. But if \dev{garda} is to be taken from
    $\surd$\dev{gard} then we might have “crying from the sting.”}
    
    \item \se{indrāyudhā}{Rainbow}
    \item \se{sāmudrikā}{Oceanic}
    \item \se{govandanā}{\diff{Cow-praising}}\footnote{The manuscripts all read 
    \dev{govandanā} against the vulgate's \dev{gocandanā}.}
\end{itemize}
Amongst these, 
\begin{itemize}
    \item The one called a Black is the colour of kohl and has a broad head;

    \item The one called Mottled is like the \gls{varmimatsya}, long with a
\se{chinna}{segmented}, humped belly.
    
    \item  The one called Sting-gush is hairy, has large sides and a black mouth.
    
    \item  The one called Rainbow is coloured like a rainbow, with vertical stripes.
    
    \item  The one called Oceanic is slightly blackish-yellow, and is covered with
    variegated flower patterns. 
    
    \item The one called Govandana is like a cow's testicles, having a bifurcated form 
    on the lower side, and a tiny mouth. 
       
\end{itemize}
When someone is bitten by them, the symptoms are: a swelling at the site of
the bite, excessive itching and fainting, fever, a temperature, and
vomiting. In that context the \se{mahāgada}{Great Antidote} should be
applied in drinks and \se{ālepana}{liniments}, etc.\footnote{The “Great Antidote” is
    described in the Kalpasthāna, at \Su{5.5.61--63ab}{578}.  Ḍalhaṇa and the 
    vulgate included errhines in the list of therapies, and Ḍalhaṇa added that “etc.” 
    indicated showers and baths too.} A bite by the
    Rainbow leech is not treatable.  These venomous ones have been explained
    together with their remedies.

\item[12]

Now the ones without poison.\footnote{The translations of the names of these 
leeches are slightly whimsical, but give a sense of the original; \emph{sāvarikā} 
remains etymologically puzzling.} 
\begin{itemize}
    \item \se{kapilā}{Tawny}
    \item \se{piṅgalā }{Ruddy}
    \item \se{śaṅkumukhī }{Dart-mouth}
    \item \se{mūṣikā }{Mouse}
    \item \se{puṇḍarīkamukhī}{Lotus-mouth}
    \item \se{sāvarikā }{Sāvarikā}
\end{itemize}
Amongst these,
\begin{itemize}
    \item The one called Tawny has sides that look as if they are dyed with
realgar and is the colour of glossy mung beans on the back.\footnote{The
    compound \dev{snigdhamudgavarṇṇā} is supported by all the manuscript
    witnesses and is translated here.  Nevertheless, the reading of the
    vulgate, that separates \dev{snigdhā}, f., “slimy” as an adjective for the
    leech, seems more plausible: “it is slimy and the colour of a mung
    bean.”}
    
    \item The one called Ruddy is a bit red, has a round body, is yellowish, and 
    moves fast.
    
    \item The one called Dart-mouth is the colour of liver, drinks fast and has a long 
    mouth.
    
    \item The one called Mouse is the colour and shape of a mouse and has an 
    undesirable smell.
    
    \item The one called Lotus is the colour of mung beans and has a mouth that looks 
    like a lotus.
    
    \item The one called Sāvarikā has the colour of a lotus leaf and is eighteen 
    centimetres long.  But that one is used when the purpose is an animal. 
\end{itemize}
The non-venomous ones have been explained.

\item [13]

Their lands are Yavana, Pāṇḍya, Sahya, Potana and so on.\footnote{This
    passage is discussed by \citet[109--110, 388--389]{kart-2015}.  At the time
    of the composition of the \SS, Yavana would most likely have referred the
    Hellenistic Greek diaspora communities in Bactria and India
    \parencites[136--137]{law-1984}{mair-2013}{mair-2014}. Unproblematically,
    the Pāṇḍya country is the extreme south-eastern tip of the Indian
    subcontinent \citep[E8, p.\,20 \emph{et passim}]{schw-1978}, and Sahya
    refers to the Western Ghats \citep[D5--7, p.\,20 \emph{et
    passim}]{schw-1978}.  The vulgate reading “Pautana” is not a known toponymn.
    Potana was the ancient capital of the Aśmaka Mahājanapada mentioned in Pali
    sources and in inscriptions at Ajāntā and elsewhere, and identified by
    \textcites[142, 179]{law-1984}[18]{gupt-1989} with Pratiṣṭhāna, modern
    Paithan on the Godavari river.  The recurring ancient epithet describing the
    Aśmaka kingdom is that it was on the Godāvarī, and Paithan is flanked to the
    south west and south east by this river.
    
    Some scholars have identified the name with modern Bodhan in Telangana
\parencites[189]{sirc-1971}[E6, p.\,14, 140 \emph{et
passim}]{schw-1978}[102]{sen-1988}, but this implausible identification
is traceable to a speculative suggestion by \citet[89, n.\,5,
143]{rayc-1953} based on a variant form “Podana” found in some early
manuscripts of the \emph{Mahābhārata}: “This name reminds one of Bodhan
in the Nizam's dominions,” “possibly to be identified with Bodhan.”
    
         Ḍalhaṇa on \Su{1.13.13}{57} anachronistically identified “Yavana”
as the land of the Turks (\dev{turuṣka}) and “Pautana” as the
Mathurā region.  He also noted, as did Cakrapāṇidatta, that this
passage was not included by some authorities on the grounds that
the habitats of poisonous and non-poisonous creatures are defined
by other criteria.}  Those in particular have large bodies and are
strong, they drink rapidly, consume a lot, and are without venom.
    
\end{translation}



% % % % % % % % % % % % % % % % % % % % % % % SS 1.28
