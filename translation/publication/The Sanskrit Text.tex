% !TeX root = surgery.tex
\subsection{The Nepalese Version}
Andrey's contribution here

\subsection{Cakrapāṇidatta and Ḍalhaṇa's Versions}
The commentaries of Cakrapāṇidatta and Ḍalhaṇa, called the \emph{Bhānumatī} and \emph{Nibandhasaṅgraha} respectively, are based on similar versions of the \SS, both of which are significantly different to the Nepalese version. Ḍalhaṇa was aware of Cakrapāṇidatta's work and reiterated many of his predecessor's remarks, so both commentaries are reasonably consistent in their interpretation of the root text. 

Trikamajī Ācārya's edition of the \textit{Sūtrasthāna} of the \emph{Bhānumatī} \citep{acar-1939} duplicates the version of the \SS\ in his edition of the \emph{Nibandhasaṅgraha} \citep{vulgate}, except in few cases where Cakrapāṇidatta glosses a different word or compound to that glossed by Ḍalhaṇa. This creates the somewhat misleading impression that both authors commented on an almost identical root text. However, in the sixteenth chapter of the \textit{Sūtrasthāna}, Ḍalhaṇa comments on four verses (1.16.11–14, \cite[78]{vulgate}) that Cakrapāṇidatta cites separately in his commentary \citep[128–129]{acar-1939}, introducing each one as 'some people read' (\emph{ke cit paṭhanti}). This clearly indicates that Cakrapāṇidatta did not consider these four verses to have been composed by Suśruta, yet Ācārya includes them in the root text of the \emph{Bhānumatī}.

Also, Cakrapāṇidatta does not acknowledge by his usual convention of citing the first words or comment on some of the verses in the version of the \SS\ known to Ḍalhaṇa, which suggests that he either did not know of these verses or was not inclined to accept them. Two examples of this occur in the sixteenth chapter of \textit{Sūtrasthāna} (1.16.20–21 and 32, \cite[79 and 81]{vulgate}). It appears that the manuscript on which Ācārya's edition of the \emph{Bhānumatī} was based does not include the root text.\footnote{See the section below on Ācārya's 1939 edition for details of the sources Ācārya used for this edition.} Therefore, the inclusion of them in his \citet{acar-1939} is a questionable hypothesis. 

% Ḍalhaṇa 1.16.11–14
%The version of 1.16.11–14 known to Ḍalhaṇa \citep[78]{vulgate} has four verses (\emph{śloka}) at this point that are not in the Nepalese manuscripts. The additional verses iterate the types of joins required for ear flaps that are missing, elongated, thick, wide, etc. All four verses were probably absent in the version of the \emph{Suśrutasaṃhitā} known to Cakrapāṇidatta. He cites the verses separately in his commentary, the \emph{Bhānumatī} \citep[128–129]{acar-1939}, introducing each one as 'some people read' (\emph{ke cit paṭhanti}). However,  in Trikamajī Ācārya's edition of the \emph{Sūtrasthāna} of the \emph{Bhānumatī}, the root text is largely identical to the one commented on by Ḍalhaṇa (\cite{vulgate}), even in instances like this where Cakrapāṇidatta's commentary indicates that he was reading a different version of the \emph{Suśrutasaṃhitā}

% Ḍalhaṇa 1.16.20–21 
%Cakrapāṇidatta \citep[131]{acar-1939} does not comment on these verses, nor verse 15 of the Nepalese version, and so the version of the \emph{Suśrutasaṃhitā} known to him may not have included them.

% Ḍalhaṇa 1.16.32
 %Cakrapāṇidatta \citep[133]{acar-1939} does not comment on this additional verse, which suggests that either he did not know of it or was not inclined to accept it.
 
% Both commentators were aware of a version of the \SS\ that was similar to the Nepalese version % See blog.
It is also important to note that both commentators were aware of a version of the \SS\ that was similar to, but not identical with, the Nepalese version. 


\subsection{Differences Between the Nepalese and Subsequent Versions with Examples from SS.1.16.}


