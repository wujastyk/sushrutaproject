% !TeX root = incremental_SS_Translation.tex
\section{Kalpasthāna, adhyāya 1}

\subsection{Literature}

A brief survey of this chapter's contents and a detailed assessment of the
existing research on it to 2002 was provided by Meulenbeld.\footcite[IA,
289--290]{meul-hist} Translations of this chapter since 2000 have appeared by 
\textcites[131--139]{wuja-2003}[3, 1--15]{shar-susr}.

More recently, a discussion of the fourth chapter of this section in the light of
the Nepalese manuscripts was published by Harimoto.\footcite[101--104]{hari-2011}
After a close comparative reading of lists of poisonous snakes, Harimoto concluded
that, “the Nepalese version is internally consistent while the [vulgate] editions
are not.”  Harimoto showed how the vulgate editions,\footnote{The two editions
\cite{susr-trikamji3} and \cite{bhat-1889}, that Harimoto noted present identical
texts.} had been adjusted textually to smooth over inconsistencies, and gave
insights into these editorial processes.



\subsection{Manuscript notes}

\begin{itemize}
    \item \MScite{Kathmandu NAK 5-333} has foliation letter numerals, for example on f.\,323a,
    that are similar to \MScite{Cambridge Add.\ 
    1693},\footnote{Scan at 
    \href{https://cudl.lib.cam.ac.uk/view/MS-ADD-01693/1}{cudl.lib.cam.ac.uk/view/MS-ADD-01693/1}.}
     dated to 
    1165\,\CE\, noted in Bendall's 
    chart of Nepalese letter-numerals \cite[Lithograph V, after p.\,225]{bend-budd}
\end{itemize}

\newpage

\subsection{Translation}

\begin{translation}
 \item[1--2]  And now I shall explain the procedures for safeguarding food and
drink, as were declared by the Venerable Dhanvantari.\footnote{MS H adds in the
margin \dev{atha khalu vatsa suśrutaḥ} “Now begins Vatsa Suśruta.”  This phrase
has been copied here by the scribe from the beginning of the \SS\ chapter in the
\emph{sūtrasthāna} on the rules about food and drink (\Su{1.46.3}{214}).  The
scribe presumably felt, not unreasonably, that this section had common subject
matter with the present chapter.  Further, SS 1.46.3 is the only place in the Nepalese 
transmission of the \SS\ that names Dhanvantari and integrates him into the narrative of the 
\SS\ as the teacher of Suśruta. 
  
 The mention of Dhanvantari here is the only other time in the Nepalese
transmission that this authority is cited as the source of Ayurvedic teaching, and the unique 
occurrence of this actual phrase, “as was declared by the Venerable Dhanvantari.”
See the discussion by \citet[28--32]{kleb-2021b}, who concludes that the earliest
recoverable recension of the \SS\ may have had the phrase only at this point and
not elsewhere in the work.}
 
 \item[3] 

 Divodāsa, the king of the earth, was the foremost supporter of religious
discipline and virtue. With unblemished instruction he taught his students, of
whom Suśruta was the leader.\footnote{This is a quite different statement from
the vulgate \citep[559]{susr-trikamji3} that has Dhanvantari as the teacher, and
calls him the \se{kāśipati}{Lord of Kāśī}.  Ḍalhaṇa followed the vulgate but
explicitly noted the reading before us with small differences: \dev{divodāsaḥ
kṣitipatistapodharmaśrutākaraḥ} “Divodāsa, the king of the earth, was a mine of
traditions about discipline and virtue.”}

\subsection{[Threats to the king]}

\item[4--5]  

Evil-hearted enemies who have plucked up their courage, may seek to harm the king,
who knows nothing of it.  He may be assailed with poisons by or by his own people
who have been subverted, wishing to pour the poison of their anger into any
vulnerability they can find.\footnote{Verses about the use of Venemous Virgins as a weapon
do not appear in the Nepalese manuscripts. Cf.\ \cite[81\,f., 132]{wuja-2003}.  This material 
is present in the commentary of Gayadāsa.} 

\item[6] Therefore, a king should always be protected from poison by a physician.

%A king may be cunningly assailed with poisons by evil-hearted enemies who
%have plucked up their courage, or even by his own people turned traitor,
%wishing to pour the poison of their anger into any chink they can find. Or
%sometimes by women using various concoctions, hoping to make him love
%them.\footnote{On how women of ill-character mix their nail-clippings or
%menstrual blood, etc.\ with the king's food, see
%p.\,\pageref{dusyodara}.} Or again, if a Venomous Virgin is used, a man can
%lose his life instantly.\label{visakanya}
%% \footnote{\label{visakanya}On the `Venomous
%% Virgin', see p.\,\pageref{intro:visakanya}.}

\item [7] 

The racehorse-like fickleness of men's minds is well known. And for this reason, a
king should never trust anyone.\footnote{The verb \root śvas is conjugated as a
first class root in the Nepalese manuscripts.}

\item [8--11]

He should employ a doctor in his \se{mahānasa}{kitchen} who is respected by experts, who 
belongs to a good family, is orthodox, sympathetic, not emaciated, and always busy.

\item [12--13]

The kitchen should be constructed at a recommended location and orientation.  It should
have a lot of light,\footnote{We read \dev{mahacchuciḥ} with the Nepalese manuscripts and 
against the vulgate's \dev{mahacchuci}.  We understand \dev{śucis} as a neuter noun 
meaning “light” following \citet[1050a]{apte-prac}.} have clean utensils and be staffed by 
men 
and
women who have been vetted.\footnote{Verses detailing the ideal staff are omitted in the 
Nepalese manuscripts. 
Cf.\ \cites[560]{susr-trikamji3}[132]{wuja-2003}.}


\item[17--18ab]

The chefs, \se{voḍhāra}{bearers}, and makers of boiled rice soups and cakes and whoever
else might be there, must all be under the strict control of the
doctor.\footnote{The word \dev{saupodanaikapūpika} “chefs for the boiled rice soups
and cakes” is grammatically interesting.  The term \dev{sūpodana} (as opposed to
sūpaudana) is attested in the \emph{Bodhāyanīya\-gṛhyasūtra} 2.10.54 
\citep[68]{shas-1920}.  More pertinently, perhaps, \dev{sūpodana} is attested in
the Bower Manuscript, part II, leaf 11r, line 3 \citep[vol.\,1,
p.\,43]{hoer-bowe}.} 
% 2.11.54 supodana in the Bodh. (from Einoo's cards)
% sūpodana kṣīrodana
% Bower MS 328
% Kāty  otoṣthayoḥ samāse vā.

\item[18cd--19ab]

An expert  knows people's \se{iṅgita}{body language} 
through abnormalities
in voice, movement and facial expression. He should be able to identify 
a poisoner by the following signs.\q{Cf.\ Arthaśāstra 1.21.8.}


\item[19cd--23]

Wanting to speak, he gets confused, when asked a question, he never arrives at an
answer, and he talks a lot of confused nonsense, like a fool.  He laughs for no
reason, cracks his knuckles and scratches at the ground. He gets the shakes and
glances nervously from one person to another. His face is drained of colour, he is
\se{dhyāma}{grimy} and he cuts at things with his nails.\footnote{The word
\dev{dhyāma} is glossed by Ḍalhaṇa (in a variant reading) as someone who is the
colour of dirty clothes \Su{5.1}{560}.}  A poisoner goes the wrong way and is
absent-minded.

\item[25--27]

I shall explain the signs to look for in toothbrush twigs, in food and drink as
well as in \se{abhyaṅga}{massage oil} and \se{avalekhana}{combs}; in
\se{utsādana}{dry rubs} and showers, in \se{kaṣāya}{decoctions} and 
\se{anulepana}{massage ointment};
in \se{sraj}{garlands}, clothes, beds, armour and ornaments; in slippers and footstools, and
on the backs of elephants and horses; in \se{snuff}{nasya}, \se{dhūma}{inhaled
    smoke}, \se{añjana}{eye make-up}, etc., and any other things which are commonly 
    poisoned. Then, I shall also explain the remedy.

\item[28]

% My old Susruta.tex translation has \bird and \animal commands for making 
% indexes.  Convert them to the \se{}{} command that we're using in the 
% present document.
\newcommand\animal[4]{\se{#2}{#1}} 
\let\bird=\animal 

%28
Flies or crows or other creatures that eat 
a poisonous \se{bali}{morsel} served 
from the king's portion, die on the spot. 

\item [29] 

Such food makes a fire crackle violently, and gives it an overpowering colour like
a peacock's throat.

\item[30--33]

%Its flames sputter, it has acrid smoke, and before long it goes out. 

After a \animal{chukar partridge}{cakora}{Alectoris chukar}{Collins 45} looks at
food which has poison mingled with it, its eyes are promptly drained of colour; a
\animal{peacock pheasant}{jīvajīvaka}{Polyplectron bicalcaratum}{Dave BSL 
270,
    273, 274, 281} drops dead.  A \animal{koel}{kokila}{Eudynamys 
    scolopacea}{Collins
    66} changes its song and the \animal{common crane}{kroñca}{Grus 
    grus}{Collins 47}
rises up excitedly.\footnote{The verb \dev{arcchati} “rises up” is a rare form
best known from epic Sanskrit \citep[see][212, \S 7.6.1]{ober-2003}.   The
transmitted form \emph{kroñca} is obviously a colloquial version of Sanskrit
\emph{krauñca}.  Commenting on \Su{1.7.10}{31}, Ḍalhaṇa interestingly gives 
the
colloquial versions of several Sanskrit bird names, even singling out 
pronunciation in the specific location of Kānyakubja.  For \emph{krauñca} he says
that people pronounce it \emph{kurañja} and \emph{koṃci}.  The form 
\emph{koñca} is found in Pāli (see \cite[731]{cone-dict}, who notes that 
Ardhamāgadhī has the same form). Elsewhere, Ḍalhaṇa calls the bird 
\emph{krauñcira} (\Su{1.46.105}{223}),  \emph{krauñci} 
(\Su{6.31.154}{684}), and \emph{kaicara} (\Su{6.58.44}{790}).}  It will excite a
\animal{peacock}{mayūra}{Pavo cristatus}{Collins 39}, and the terrified
\animal{parakeet}{śuka}{Psittacula krameri\slash eupatria\slash
    cyanocephala}{Collins 64} and the \bird{hill myna}{sārikā}{Acridotheres tristis
    tristis, L., etc.}{Ali \#1006, \citet[28\,ff.]{Dave}, \citet[119]{Collins}}
screech. The \animal{swan}{haṃsa}{?}{?} trembles very much, and the
\animal{racket-tailed drongo}{bhṛṅgarāja}{Dicrurus paradiseus}{Collins 123}
churrs.\footnote{Ḍalhaṇa seems confused about the \dev{bhṛṅgarāja}.  He calls it 
a
\se{bhramaraka}{bee} (which is another meaning of the term, but inappropriate 
in
this list) and then says that it is like the \se{dhūmyāṭa}{Drongo} and that people
call it “the king of birds.”} The \se{vṛṣabha}{bull}  sheds tears and the monkey
releases excrement.\footnote{The vulgate replaces \se{vṛṣabha}{bull} with
\se{pṛṣata}{Chital deer}.  The reading \dev{pṛṣata} may perhaps be mistaken for
\dev{vṛṣabha} in the Newa script, although the reading of MS KL 699 is hard to read at this 
point.}

\item[34]

Vapour rising from food which has been abused gives rise to a pain in the
heart, it makes the eyes roll, and it gives one a headache.\footnote{ “Abused” translates 
\dev{upakṣipta}.”  The word's semantic field includes “to hurl against,” and especially “to 
insult verbally, insinuate, accuse.”  In the present context, “abuse” covers the fields in English 
as well as Sanskrit.  The commentator Ḍalhaṇa thought that the word referred food “spoiled 
and given” (\dev{vidūṣitasyānnasya bhoktuṃ dattasya}), but he noted that some people 
read “\dev{ukhākṣipta}” or “thrown into a pan.”  Other translators have commonly 
translated “served,” presumably influenced by Ḍalhaṇa's \dev{dattasya}.}

 
    \end{translation}