% !TeX root = incremental_SS_Translation.tex
\section{Kalpasthāna, adhyāya 1}

\subsection{Literature}

A brief survey of this chapter's contents and a detailed assessment of the
existing research on it to 2002 was provided by Meulenbeld.\footcite[IA,
289--290]{meul-hist} Translations of this chapter since 2000 have appeared by 
\textcites[131--139]{wuja-2003}[3, 1--15]{shar-susr}.

More recently, a discussion of the fourth chapter of this section in the light of
the Nepalese manuscripts was published by Harimoto.\footcite[101--104]{hari-2011}
After a close comparative reading of lists of poisonous snakes, Harimoto concluded
that, “the Nepalese version is internally consistent while the [vulgate] editions
are not.”  Harimoto showed how the vulgate editions,\footnote{The two editions
\cite{susr-trikamji3} and \cite{bhat-1889}, that Harimoto noted present identical
texts.} had been adjusted textually to smooth over inconsistencies, and gave
insights into these editorial processes.



\subsection{Manuscript notes}

\begin{itemize}
    \item \MScite{Kathmandu NAK 5-333} has foliation letter numerals, for example on f.\,323a,
    that are similar to \MScite{Cambridge Add.\ 
    1693},\footnote{Scan at 
    \href{https://cudl.lib.cam.ac.uk/view/MS-ADD-01693/1}{cudl.lib.cam.ac.uk/view/MS-ADD-01693/1}.}
     dated to 
    1165\,\CE\, noted in Bendall's 
    chart of Nepalese letter-numerals \cite[Lithograph V, after p.\,225]{bend-budd}
\end{itemize}

\newpage

\subsection{Translation}

\begin{translation}
 \item[1--2]  And now I shall explain the procedures for safeguarding food and
drink, as were declared by the Venerable Dhanvantari.\footnote{MS H adds in the
margin \dev{atha khalu vatsa suśrutaḥ} “Now begins Vatsa Suśruta.”  This phrase
has been copied here by the scribe from the beginning of the \SS\ chapter in the
\emph{sūtrasthāna} on the rules about food and drink (\Su{1.46.3}{214}).  The
scribe presumably felt, not unreasonably, that this section had common subject
matter with the present chapter.  Further, SS 1.46.3 is the only place in the Nepalese 
transmission of the \SS\ that names Dhanvantari and integrates him into the narrative of the 
\SS\ as the teacher of Suśruta. 
  
 The mention of Dhanvantari here is the only other time in the Nepalese
transmission that this authority is cited as the source of Ayurvedic teaching, and the unique 
occurrence of this actual phrase, “as was declared by the Venerable Dhanvantari.”
See the discussion by \citet[28--32]{kleb-2021b}, who concludes that the earliest
recoverable recension of the \SS\ may have had the phrase only at this point and
not elsewhere in the work.}
 
 \item[3]
 

 Divodāsa, the king of the earth, was the foremost supporter of religious
discipline and virtue. With unblemished instruction he taught his students, of
whom Suśruta was the leader.\footnote{This is a quite different statement from
the vulgate \citep[559]{susr-trikamji3} that has Dhanvantari as the teacher, and
calls him the \se{kāśipati}{Lord of Kāśī}.  Ḍalhaṇa followed the vulgate but
explicitly noted the reading before us with small differences: \dev{divodāsaḥ
kṣitipatistapodharmaśrutākaraḥ} “Divodāsa, the king of the earth, was a mine of
traditions about discipline and virtue.”}

\subsection{[Threats to the king]}

\item[4]  

Evil-hearted enemies who have plucked up their courage, may seek to harm the king,
who knows nothing of it.  He may be assailed with poisons by or by his own people
who have been subverted, wishing to pour the poison of their anger into any
vulnerability they can find.\footnote{Verses about the use of Venemous Virgins as a weapon
do not appear in the Nepalese manuscripts. Cf.\ \cite[81\,f., 132]{wuja-2003}}

%A king may be cunningly assailed with poisons by evil-hearted enemies who
%have plucked up their courage, or even by his own people turned traitor,
%wishing to pour the poison of their anger into any chink they can find. Or
%sometimes by women using various concoctions, hoping to make him love
%them.\footnote{On how women of ill-character mix their nail-clippings or
%menstrual blood, etc.\ with the king's food, see
%p.\,\pageref{dusyodara}.} Or again, if a Venomous Virgin is used, a man can
%lose his life instantly.\label{visakanya}
%% \footnote{\label{visakanya}On the `Venomous
%% Virgin', see p.\,\pageref{intro:visakanya}.}



 
    \end{translation}