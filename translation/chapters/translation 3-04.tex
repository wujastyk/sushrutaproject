%!TeX root = ../incremental_SS_Translation.tex

\chapter{Śārīrasthāna 4:  On the Formation of the Embryo}


\section{Introduction}

This chapter opens with sixteen passages that discuss of the seven
layers of skin that form early in the fetus's life.\footnote{On the
    system of the \dev{kalā} see \volcite{1}[183--184]{josi-maha},
    \cite[227--228]{gupt-1983}, \cite[6]{kutu-1962}, \volcite{1}[247--248
    and notes]{meul-hist}.} This system of dermal and interstitial
    \dev{kalā} was not known to the \CS\ as such.  Rather, the \CS\
    mentioned six kinds of \dev{tvac}\sse{tvac}{skin} with different names
    and characteristics. These were classified not according to
    appearence, as in the \SS, but mainly according to the diseases that
    they supported.\footnote{\label{sa4:kalā}\CS\ on \Ca{4.7.4}{337}. This
        contradiction between the \CS\ and the \SS\ was discussed by the
        commentator Cakrapāṇidatta (\emph{idem}).} The concept of interstitial
        skins was used in the \SS\ as an explanatory mechanism for the stages
        of snake envenomation.\footnote{See p.\,\pageref{ka4:kalā-vega}.} The
            \SS's concept of seven skins dominates the narrative of later
            works.\footnote{For example, the fourteenth-century
                \emph{Śārṅgadharasaṃhitā} (1.5.60, \cite[40]{sast-1931}) and the
                sixteenth-century \emph{Bhāvaprakāśa} (1.3.220--222,
                \volcite{1}[49]{brah-1935}), which gives seven skins, like the \SS,
                but names and describes them in the manner of the \CS.   The \AS\
                (Śārīrasthāna 5.18, \cite[296--297]{atha-1980} gives both views.}

\section{Literature} 

Meulenbeld offered an annotated overview of this chapter and a
bibliography of earlier scholarship to 2002 and, in his notes, citations
of the parallel passages in the \CS.\fvolcite{IA}[247--249]{meul-hist}   

\newpage

\section{Translation}

\begin{translation}

    % First draft by Jan Gerris, 2026-01-17
  
\bigskip       

\item[1]

Next we shall discuss the anatomical chapter on the analysis of the foetus.
 
\item[3]

\newenvironment{tightquote}
    {\list{}{\listparindent 1.5em%
          \itemindent    \listparindent
          \rightmargin   \leftmargin
          \parsep        \z@ \@plus\p@}%
      \item\relax}
  {\endlist}

Fire, and \se{soma}{liquid}, air, sattva, rajas, and tamas, the five
senses, the \se{bhūtātman}{elemental self}, and the mind
are the \sepl{prāṇa}{life principle}.\footnote{\dev{prāṇa} is here used to 
refer to all the components of a living being, not merely the five breaths.
On this passage, and its concept of multiple \dev{prāṇa}, see 
\cite[\S3.2.3]{kleb-2021b}.
On the early history of \dev{prāṇa}, see \cite{zysk-1993,zysk-2007}.
On the expression \dev{agnīṣoma},  “fire and liquid,” see 
\cite{wuja-2004,ange-2021}.

The word “\se{manas}{mind}” is present in the Nepalese version, but not
in the vulgate text.  The commentator Gayadāsa (fl.\ ca.\ 1000)
discussed this term, confirming that the word was present in the \SS\
text available to him, and he noted that his predecessor Jejjaṭa did
not think this was proper (\MS{Cambridge Add.2491}, f.\,33r):
\begin{quote}
\dev{jaḍastu bhūtātmaśabdena 
na mano'bhidhatte/ manograhaṇaṃ nādhīyate/  tanna
bhūtātmamanasorvyaktabhedāt/}\par  But Jejjaṭa does not intend “mind” by
the term “elemental self.”  The mention of the word “mind” is not
taught there. That is because there is an obvious difference between
the elemental self and the mind.
\end{quote}
This suggests that the word \emph{manas} was present in the earliest
\SS\ but was dropped as a result of Jejjaṭa's objection (that then
influenced Candraṭa's revision).}




\item[4]

You see, sperm and menstrual blood, maturing, get seven
\sepl{tvac}{skin}, for it is like the \sepl{santānika}{skin} on heated
milk.\footnote{If we take \dev{śukraśoṇitasyābhipacyamānasya} as a
    genitive absolute, we could read this statement as “even though [the
    admixture of] sperm and menstrual blood is forming, seven skins come
    into existence, like the skins on milk.”
    
 The following characterization of these skins is longer in the
vulgate text because it imports the concept of
“\sepl{adhiṣṭhāna}{foundation}.”  This concept of skins as
“foundations” is present in the \CS\ account of six skins, which are
described as being the foundations of various illnesses. See
pp.\,\pageref{ka4:kalā}, \pageref{sa4:kalā}.  It seems likely that
the vulgate here has been supplemented with material from the \CS.}



 The first  of these, called  “\se{avabhāsinī}{Shining},”  makes all
colours shine and makes  visible five kinds of
\se{chāyā}{complexion}. It is the size of one-eighteenth part of a
rice grain.\footnote{\Dalhana{3.4.4}{350} interpreted \dev{vrīhi} “grain of
    rice” as being a barley corn (a standard unit of measurement).}
 

 The second is called “\se{lohitā}{Red},” the size of one-sixteenth.
  
 The third is “\se{śvetā}{White},” the size of one-twelfth. 
 
 The fourth is “\se{tāmra}{Coppery},” one-eighth in size. 
 
 The fifth is named “\se{vedanī}{Feeling},” one-fifth in size. 
 
 The sixth is 
 named “\se{rohiṇī}{Scarlet},” the size of one rice grain. 
 
 The seventh is named  \se{śukradharā}{Semen-supporter},”  the size of two 
 rice grains.
 
 Since it will be said in the chapter about the belly, \begin{quote}
     Using the \se{vrīhimukha}{rice-tip instrument}, one penetrates
the measure of the thickness of a thumb or a \se{aṅgula}{finger's
    breadth}.\footnote{This sentence is cited from \Su{4.14.18}{461}
    in the chapter calle \emph{Udarāṇāṃ cikitsitam} “On the therapy
    of the abdominal ailments.”  The measure
    \dev{aṅguṣṭhodarapramāṇa} “the size of the belly of a thumb” is
    \dev{viṃśatitamabhāgonaṣaḍyavapramāṇam} “one twentieth less than
    six barleycorns” (5$\frac{19}{20}$) according to
    \Dalhana{4.4.4}{355}. Cf.\ \cite[10]{josi-maha}.  The procedure
    described is the puncturing the abdomen and the insertion of a
    tube to release an abnormal buildup of fluid (ascites); it is
    today called paracentesis.
    
    The “rice-tip” instrument is one of the surgical knives described
in \SS\ \Su{1.8}{35--41}, \cite[tr.][83--86]{wuja-2003}; see
\volcite{1}[257--261]{mukh-1913}. See
Fig.\,\ref{fig:vrihimukha-from-ss-1938} and the artists'
reconstructions at \volcite{2}[plate LXXI]{mukh-1913}.  The
purpose of citing this passage here is not clear; perhaps the
author wished to give an example of measurements being used in a
practical situation. Quantitative instructions are not very common
in the \SS.}
 \end{quote}
    
     \begin{figure}
        \centering
        \includegraphics[width=0.7\linewidth]{"media/vrihimukha from SS 1938"}
        \caption{The \emph{Vrīhimukha} instrument, as illustrated in 
        \cite[37]{vulgate}.}
        \label{fig:vrihimukha-from-ss-1938}
    \end{figure}
    
 
 \newpage
 \begin{tt}
    \raggedright
    \bigskip
    \marginpar{Draft tr.\ from here}



\item[5]

Truly, the seven tissue layers surely arise,  marking the boundaries  between the dhātus and internal organs And there are two śloka’s.

\item[6]

Just as sap/resin is seen/appears in (logs of) wood  being cut, similarly  the tissue really  becomes visible in flesh being cut.

\item[7]

The experts  (have always) known that are covered  both by ligaments and (ca) surrounded  by an integument/serosa/amnion and moreover coated by a mucus  from parts of the tissue layers.

\item[8]

Of these the first one  is called māṃsadharā  — holder of flesh. From it arise flesh, veins, ligaments, arteries, and body openings, each in always renewed  patterns.

\item[9]

As lotuses in a pond grow  in all directions in the soil  having a place in mud(dy) water, so too  the veins (spread) through/in the flesh.

\item[10]

The second is called  raktadharā — holder of blood, situated mainly inside the flesh, and its (tasya ?) blood is especially/in special concentration in veins, in the liver and in the spleen.   (no tasyā found in MMW)

\item[11]

Just as  milk oozes from a tree when struck , so does blood gush  quickly from injured flesh.

\item[12]

The third one (is) named medodharā — holder of fat, moreover fat is surely located inside the abdominal cavities,  inside large bones and (ca) the marrow, of all beings.

\item[13]

Marrow (is) predominantly located on the inside/internal in large  bones; and in other (bones), fat is described as blood-stained.

\item[14]

The fourth one is called  śleṣmadharā — holder of phlegm, which is present in the joints  of all living beings.

\item[15]

Just as a wheel rotates  smoothly on a lubricated axle, so the joints function smoothly because of a sliding phlegm (=synovial fluid).

\item[16]

The fifth one (is) purīṣadharā — holder of faeces, with the four kinds  of food, by falling/descending from the stomach into the colon, and divides the bowels accordingly (?).

\item[17]

And the liver is continuous with the stomach as well as with the entrails  ….. (samāsritāḥ). The holder of faeces layer helps separate the stomach and the faeces.


\item[18]

The sixth one is called pittadharā — holder of bile, which digests the four types of ingested food: eaten, drunk, chewed, and licked.

\item[19]

There is a saying : whatever is eaten, chewed, drunk, or licked  and is entering the stomach, gets digested by the power of bile  at the proper time.

\item[20]

The seventh one  is called śukradharā — holder of semen, which is pervading the entire  body of all living beings.

\item[21]

And there is another saying. Just as ghee is present in milk and molasses in sugarcane juice, that way semen (is present) throughout the bodies of men, as the wise physician would know.

\item[22]

From a place, (located) two fingers to the right of the bladder outlet, the urine vessel, a man’s  sperm flows.

\item[23]

\item[24]

The courses of the channels of the menstrual  flow  of those who have obtained embryos  (= who have become pregnant) are blocked  by the embryo.

But because of this no menstruation is visible  in pregnant women. Then, that, whose downward movement  is obstructed, whose upper part (is) accumulating above, is released into the abdomen. And the remainder begins its movement upward into the breasts. Hence,  pregnant women are showing swollen, protruding breasts.

\item[25]

The liver and spleen both originate from blood. The lungs arise from the froth of blood. The uṇḍuka (a gland)/stomach ( ?) arises from blood waste.

\item[26]

There  are other sayings : The refined essence of blood and also  of phlegm, which is considered superior, when that is being digested  by bile, and is also being chased by wind, ? (anudhāvati). 

\item[27]

From that  his  entrails arise in the rectum and the bladder (arises) in the body. There, while being churned and heated, it becomes painful.

\item[28]

In a living being the tongue arises by which tastes is perceived.

\item[29]

\item[30]

\item[31]

The heart, originating from the essence of blood and phlegm.  That which is truly  the base, (namely) the arteries that carry prāṇa is said to be especially the  seat of consciousness. (OR : That upon which the prāṇa-carrying arteries depend is especially said to be the seat of consciousness.) When enveloped by this inertia, all living beings fall into sleep.

\item[32]

\item[33]

However, they declare  sleep to be Viṣṇu-like, sinless. It naturally  affects all beings. When tamas-dominated phlegm occupies the channels of cognition, that is called Tāmasī Nidrā, which leads to unconsciousness — found at times of dissolution and during dark nights in tamas-predominant individuals. Furthermore, the channels that carry consciousness are (like) torrents of excessive  tamas, kapha takes hold. At such moments, the sleep  called tāmasic  arises  — unconscious. It occurs at the time of cosmic dissolution and  during nights and days of excesses of tamas. The type called  rajobhūyiṣṭhā  (is) of those who are predominantly sattvic  due to some external cause, around midnight, when kapha is diminished (and) vāta is predominant, and  also due to mental agitation. It is known as Vaikārikī (psychogenic sleep).

\item[34]

As it is said. The heart (is) declared the seat of consciousness  of embodied beings,  o Suśruta. Furthermore, when overpowered by such  tamas, sleep enters the body.

\item[35]

Tamas is the cause of sleep, sattva  is said to be the cause of being awake.  As expected , one’s own natural disposition natural (sleep) is indeed declared the most prominent  cause.

\item[36]

But the embodied self, (who is) the Lord, sleeps, experiencing again what was felt in former bodies. With a mind joined with rajas, it grasps  experiences that are both auspicious and inauspicious.

\item[37]

But when dysfunction of the senses (occurs), due to tamas, (it) becomes active. Even though the embodied self is not (truly) asleep, it is said: (it is) asleep.

\item[38]

And daytime sleep  is prohibited in all seasons, except in summer. In cases that are prohibited but exceptions may apply,  it still remains prohibited for children , the elderly, the wounded, the weakened, alcohol consumers, and those who are exhausted due to women/sexual activity, travel, riding vehicles, walking, or physical labour. They praise a moment (of quick sleep) of those who are fasting, and in whom fat, vāta, kapha, and rasa have been diminished.
Indeed, even  for those who have remained awake at night, one should sleep during the day for half the time spent awake. Now,  there is a disorder called  ‘daytime sleep’. In that context), for those who sleep during the day, it is considered contrary to natural law, and it causes aggravation of all the doṣas. Furthermore due to that aggravation, there are/one develops  cough, cold, heaviness in the head, body aches, loss of appetite, and weakness of the digestive fire. Indeed even at night, in those who stay awake, those very doṣas arise due to that cause.

\item[39]

And there are more verses. Therefore, one should not stay awake at night and nor  should one avoid sleep during the day. Knowing that these two (daytime sleep and night-time wakefulness) cause doṣa imbalances, the wise person should go to sleep  in moderation.

\item[40]

Indeed a healthy person is cheerful, endowed with great strength and a radiant complexion. A man (who is neither too fat nor too thin, and who is endowed with grace/charm, may live  a hundred  years.

\item[56] !

(\item[41])

Fainting is generally predominantly caused by pitta; dizziness or vertigo arises from a combination of rajas, pitta, and vāta. Drowsiness is caused by tamas, vāta, and kapha; sleep arises from kapha and tamas.

\item[42]

Insomnia arises from vāta, pitta, mental agitation, depletion, and trauma; it subsides through appropriate opposing therapies.

\item[43]

In insomnia oil massage of the head, body rubs, powder massage of the body and gentle stroking massages are beneficial.

\item[44]

With meals consisting of preparations made from ground rice and wheat flour, prepared  with sugarcane products; meals that are sweet, unctuous and enriched with milk, meat  broths, and similar nourishing items.

\item[45]

At night, it would be appropriate to apply a diet with broths of burrowing animals and birds that scatter grain, as well as products with grapes, of white sugar and sugarcane preparations.

\item[46]

One should arrange soft  and  pleasing beds and seats. In case of sleeplessness ; however , the wise person should also employ  other appropriate measures.

\item[47]

In the case of excessive sleep, emesis and other purificatory measures  are beneficial Fasting), bloodletting, and  inducing mental agitation are also (appropriate).

\item[48]

For those afflicted by kapha, fat, or toxins staying awake at night is beneficial. Daytime sleep, on the other hand, is beneficial for those suffering from intense colic, hiccup, indigestion, or diarrhoea.

\item[49]

Lack of awareness  in the sense objects, yawning, heaviness, and fatigue —if these symptoms appear in someone/in such case, one should diagnose  that as drowsiness, which arises from an affliction of sleep.

\item[50]
\item[51]
\item[52]
\item[53]
\item[54]
\item[55]

\item[56]  see 41

\item[57]

Indeed, the growth of the foetus is due to the essence, the qualities, and the self-nature; it also essentially depends on timely and appropriate nourishment.

\item[58]
\item[59]

\item[60]

As it is said: And sight and hair pores  never  grow. These are fixed  for mortals — thus is  the opinion of Dhanvantari.

\item[61]

Even when the body  is declining), these two  always grow  — regarding as their nature and constitution thus: nails and hair— such is the fact.

\item[62]

There are three basic constitutions: wind, bile, and phlegm.

\item[63] 

However, when there is an intense defect in the union of semen and menstrual blood, the constitution  takes origin  because of that; now hear from me the characteristics of these.

\item[64]

(Their characteristics are as follows):
In that case, the vāta constitution is wakeful, dislikes cold, is unfortunate, and because of that ignoble, afflicted with head diseases/headaches and impaired vision; with cracked hands and feet, dry and scanty hair/baldness, sparse beard, quick-tempered, a tooth-gnasher, weak in strength, and short-lived.


\item[65]

He is unsteady, with unstable friendships, ungrateful, thin and rough, with prominent veins, and talkative; quick in movement, noisy, (with the mind) in the air, and while asleep, he moves  in confusion.

\item[66]

Unstable in intellect, restless in mind, having little in the way of teeth, wealth, possessions, and friends; he speaks only scarcely and incoherently— this (is) a man of vāta constitution.

\item[67]

\item[68]

The pitta-natured person, however, is sweating, tolerant of cold, foul-smelling, yellow (in complexion), or dark; having loose limbs, copper-coloured  nails, eyes, palate/uvula?, tongue, lips , soles, palms, and palate; unfortunate, with wrinkles and early grey hair, of loose bowels, with much hunger but hating much heat, quick to anger, quick to be pleased, and has a medium lifespan.


\item[69]

Intelligent, with a skilled mind, a restrained speaker, radiant in assemblies, and possessing irresistible power; when asleep, he would behold the golden filaments of the palāśa blossom  and even the lightning-flash of fire.

\item[70]

He should not bow to others out of fear, but be gentle  to those who bow to him; and if he shows no liking  for friendly courtesies to those who bow, then here in this world he will always  have agitated speech, for his nature in this world will be one produced by pitta. 

\item[71]

A person of phlegmatic constitution is characterized by a complexion resembling dūrvā grass, vedi, sword blade, moist clay, ariṣṭa wood, sugarcane (or reed) stalk —any one of these  shades. Such a person is  fortunate, pleasant to look at, and fond of sweet things. He is grateful, steadfast, patient, not greedy, strong, deliberate in grasping things, firm in enmity(, and long-lived.

\item[73]

He has a smooth, unctuous body; a firm, well-proportioned, and beautiful frame; he is endowed with prosperity; his voice (resembles the deep sound of) a cloud, a drum, or a lion. When asleep, he may see in dreams lotuses, swans, and ruddy geese, and also delightful bodies of water.

\item[74]

\item[75]

A person of kapha constitution has a firm  grasp 0of the sciences, a stable mind friendship, and after long consideration gives much (in gifts). His speech and words are well-finished and deliberate and  he is always showing respect  to his teachers.

\item[76]

\item[77]

When, in a person’s constitution, the characteristics of two doṣas are seen, that should be recognized as a combined constitution. There are three kinds— [those involving] the accumulations (of two doṣas).

\item[78]

No aggravation, alteration, or depletion arises  from other causes; rather, by the very nature of their constitutions), they know the end of life is known.

\item[79]

Just as a worm  born in poison is not harmed by the poison, in the same way, the body is not afflicted by its own constitution because it was born from it.

\item[80]

Here, some say that the material nature of men is threefold, described in terms of wind, fire, and water. If a man’s body is steady and large, he is (of) the earth-nature, patient in disposition. And one of the ether element is pure and long-lived — so declare the great authorities.

\item[81]

Purity, faith, study in the Vedas, reverence for the teacher, hospitality, and sacrificial worship — these are the characteristics of one belonging to the « brahma-body ».

\item[82]

Valour, authority, great fortune, perpetual knowledge of the scriptures, and the maintenance of servants  — these truly  are the bodily marks of one of Indra-like nature).

\item[83]

Patience, fondness for cold, a tawny/brownish complexion, blond, having yellow/tawny hair, and fondness for water — these are likewise the bodily marks of a of Varuṇa-nature.

\item[84]

Neutrality, patience, the acquisition and accumulation of wealth, and great productive power  — these are the bodily marks of one of Kubera-nature.

\item[85]

Fondness for perfumes and garlands  and love for dance and music, and a habit of amusement— these are indeed the bodily marks  of a Gandharva-nature.

\item[86]

One who fulfills undertakings, is firm in effort , steadfast, endowed with memory, pure, free from attachment, aversion, fear, and ignorance — such a person possesses the nature of Yama.

\item[87]

Let men know (that) the man engaged in chanting the vow of celibacy, performing fire rituals and study  endowed with knowledge and wisdom, possessing the nature of a sage.

\item[88]

These seven are the pure bodies  the passionate ones, know for me: the powerful, the fierce, the brave , the cruel, and the non-envious.

\item[89]

But eating alone, being conditioned and of an asuric nature, is of this sort of character: exclusive adherence (to one’s own view), cruelty, envy, and speech of unrighteousness.

\item[90]  cf. infra

\item[91] cf. infra

\item[92]

Excessive self-praise  is also ca api a bodily characteristic of the rakṣasa nature; likewise, disregard for proper conduct, harshness, and fondness for violence.


\item[93]

Lust for women, shamelessness— a demonic collection of qualities, harshness, restlessness  through toil, quickness to anger, and cowardliness.

\item[90] (cf.supra?)

(A man) given to pleasure and food, fickle, and of snake-like nature — they consider such a man  to be ungenerous, lazy, of bad character, and untrustworthy.

\item[94]

They consider a greedy and  ungenerous man to have the nature of a ghost.

\item[91]  (cf. supra?)

One whose desires are awakened, or who avoids enjoyment,
who constantly eats, and who indeed (is) intolerant, has no fixed abode— this is called a bird-like nature. These six are Rājasic traits; but now learn from me about the Tamas-type.

\item[95]

Poor understanding, slowness, excessive sleep, habitual indulgence in sex, incapacity (to take an initiative), and sorrows — these are to be recognized as beastly qualities.

\item[96]

Instability, foolishness, cowardice, fondness for water, and mutual jostling — these are qualities entirely fish-like in nature.

\item[97]

Lacking spirit and strength of limb, devoted solely to food, without cherished aims or desires  — that would be a man with the nature of a tree).

\item[98]

Thus these threefold constitutions, beginning with the rājasic and subsequently as discussed, have been described. Having recognized one’s bodily constitution, one should act accordingly, so it is said. 

\item[99]

Thus ends the Śārīra, the fourth chapter.

\end{tt}
\end{translation}