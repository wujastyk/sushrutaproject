% !TeX root = ../incremental_SS_Translation.tex

\chapter{Introduction to the Uttaratantra}


The \emph{Uttaratantra} of the \SS\ consists of sixty-six
chapters.  This amounts to more than a third of the size of the whole
\SS.  The vulgate version of the work divides the \Utt\
into five major divisions:
\begin{enumerate}
    \item \emph{Śālakyatantra}, on diseases of the eyes, ears, nose and head.  
    This 
    division includes the description of couching for cataract.\footnote{See 
        \cite{leff-2020} for a recent study.}
        
        \item \emph{Kumāratantra}, on avoiding the threat of demonic attacks on 
        children and 
        one  final chapter on disorders of the female genital tract.
        
        \item \emph{Kāyacikitsātantra}, on the treatment of twenty diseases, 
        starting 
        with fever 
        and continuing with diarrhoea through to urinary disorders.
        
        \item \emph{Bhūtatantra}, on possession by supernatural beings, on 
        epilepsy, 
        and on insanity.
        
        \item  \emph{Tantrabhūṣaṇādhyāya}, on the permutations of the 
        savours,\footnote{See \cite{wuja-2000}.} on living well, on logical rules of 
            interpretation for medicine, and on the combinatorics of the four 
            humours.\footnote{Blood is included as a \se{doṣa}{humour} in this 
            chapter.}
            \end{enumerate}
     \begin{table}
    \centering
      \begin{tabular}{>{\em}l>{\em}l}
        \toprule 
        Nepalese 			& vulgate \\
        \midrule
        Śālakyatantra  	& Śālakyatantra \\
        Kumārabhṛtya	&  Kumārabhṛtya\\
        Kāyacikitsā (+ Daśaka)			 &Kāyacikitsātantra \\
                                        & Bhūtatantra  \\
                                & Tantrabhūṣaṇādhyāya \\
        \bottomrule
    \end{tabular}
    
%    \begin{tabular}{>{\em}lccp{.35\textwidth}}
%        \toprule 
%        & Nepalese & vulgate & notes\\
%        \midrule
%        Śālakyatantra  & 1--26 & 1--26 & Nepalese omits 24 here \\
%        Kumāratantra & 27--\textbf{37} &  27--\textbf{38} & \\
%        Kāyacikitsātantra & \textbf{39}--57  & \textbf{39}--59 & Nepalese omits 
%            38 and  adds 24 after 53 \\
%        Daśaka & 58--66 &                   & Nepalese adds 38 after 58 \\
%        Bhūtatantra & ---  & 60--62  \\
%        Tantrabhūṣaṇādhyāya & --- & 63--66  \\
%        \bottomrule
%    \end{tabular}
    \caption{The division of sections of the \Utt\ in the Nepalese 
        version and in the vulgate \citep{vulgate}.}
    \label{utt:sections}
\end{table}
            
In the Nepalese version of the \SS, however, the chapters are
distributed slightly differently across sections (see
Fig.\,\ref{utt:sections}, \ref{utt:conc}).  In the following, the
chapter numbers are those of the vulgate text.
\begin{itemize}
    \item In the Nepalese \emph{Uttaratantra}, 6.13 and 6.14 are combined and 
    both
    called 6.13.  This causes the following chapter numbers in the Nepalese
    version to be reduced by~1.
    
    \item After vulgate 6.23, the Nepalese version skips the 6.24
    and moves straight to 6.25.  Chapter 6.24 appears later, 
    after 6.53.
    
    \item At the start of the \textit{Kāyacikitsā} section of the \Utt, the
    Nepalese version begins with 6.39, on fever, instead of the vulgate's 6.38.  
    
    \item By contrast, the vulgate version of the \emph{Kāyacikitsā} starts with  
    6.38, on ailments of the female reproductive tract.  This appears as the last 
    chapter of the Nepalese of the \emph{Kāyacikitsā}, after 6.59 (after two 
    chapters on urinary problems). 
    
    \item  The \emph{Bhūtavidyā} and \emph{Tantrabhūṣaṇādhyāya} divisions 
    of the vulgate are not differentiated in witness K.  All the chapters to the end 
    of the \Utt\ are called the \emph{Kāyacikitsā}.  
    
    \item But in witness H, 6.60--6.63 are called \emph{Bhūtavidyā}.  Then, 6.64 
    is called \emph{Kāyacikitsā} again, and the last two chapters of ahe \Utt\ are 
    not assigned to any subdivision. 
    
    \item The last ten chapters of the Nepalese text are included in its 
    \emph{Kāyacikitsā}, as mentioned.  However, the last verse of the Nepalese 
    version, just preceding the scribal colophon,  reads as follows:
    \begin{quote}
They have again declared this group of ten in the \emph{Kāyacikitsita}:
urine diseases, urine blockages, the vagina and the
supernatural beings, epilepsy, insanity, the divisions of
the savours, the rules for the preservation of the healthy
person, the \se{tantrayukti}{rules of interpretation}, the
divisions of the humours.\footnote{\dev{mūtradoṣo mūtrāghāto 
yonyamānuṣameva ca/\\
    apasmāronmādakañcaiva rasabhedastathaiva ca/\\
    svastharakṣāvidhāṇañca tantrayuktiśca doṣabhit/\\
    ityebhirdaśakaṃ proktaṃ punaḥ kāyacikitsite//}}
\end{quote}
So the internal evidence of the Nepalese version, transmitted by K, omits 
reference to the vulgates last two subdivisions.  Instead, it declares that the last 
ten chapters form a group, a decade of chapters.
\end{itemize}
    
 
\begin{table}[t]  
    \centering 
\begin{tabular}{lll}
     \toprule
          Nepalese & Vulgate &  \\
            \midrule
&&\\
     \multicolumn{2}{c}{\emph{Śālakyatantra}}&\\
     1--12 & 1--12 &  \\     
     13 & 13--14 &  \\     
     14--22 & 15--23 &  \\     
     & 24 & $\rightarrow$ Nepalese 51 \\     
     23--24 & 25--26 &  \\
 &&\\
 
     \multicolumn{2}{c}{\emph{Kumārabhṛtya}}&\\
%        Nepalese & Vulgate &  \\
%        \midrule
        25--35 & 27--37 &  \\
                    & 38       & $\rightarrow$ Nepalese 58\\

&&\\

     \multicolumn{2}{c}{\emph{Kāyacikitsā}}&\\
%        Nepalese & Vulgate &  \\
%        \midrule
        36--50 & 39--53 &  \\
        51 & 24 &  \\
        52--57 & 54--59 &  \\
        58 & 38 &  \\
        59--65 & 60--66 &  \\
        \bottomrule
\end{tabular} 
\caption{Concordance of Nepalese and vulgate chapter numbers, according to 
witness K.}
\label{utt:conc}
\end{table}
            
\section{Literature}

Meulenbeld offered an annotated overview of the \Utt\ and a
bibliography of earlier scholarship to
2002.\fvolcite{IA}[300--332]{meul-hist}  He treated the individual chapters 
separately and did not provide reflections on the \Utt\ as a unit with five 
subdivisions.

Meulenbeld also discussed the related issue of whether there might
once have been an \Utt\ attached to the \CS, a
$^*$\emph{Carakottaratantra}.\fvolcite{IA}[99--100]{meul-hist}  While
it seems unlikely that there was ever such a text, there is one verse
that deserves attention, namely \CS\ \Ca{8.12.50}{737}.  This is
almost the last verse of the whole work and it states that certain
general topics about how to achieve the right interpretation of
medicine through \sepl{tantrayukti}{hermeneutic rule} will be treated
at greater length in the sequel, or “Uttara," with a view to providing
true knowledge of the truth of the \se{tantra}{system} from the point
of view of merits and flaws.\footnote{\dev{tasmādetāḥ pravakṣyante
    vistareṇottare punaḥ/ tattvajñānārthamasyaiva tantrasya
    guṇadoṣataḥ//}, \CS\ \Ca{8.12.50}{737}.} This passage is printed in
    parentheses in the vulgate edition, indicating that the editor was
    unsure about its validity as part of the text.  This is probably
    because the commentator Cakrapāṇidatta said that this passage was
    considered spurious by the earlier tradition, because the whole \Utt\
    of the \CS\ was spurious.\footnote{\dev{taṃ cānarṣaṃ vṛddhā vadanti,
        agniveśatantre uttaratantrasyaivānārṣatvāt//}, \emph{ibid},
        \Ca{8.12.50}{737}.} Nevertheless, the passage appeared in the
        manuscripts available to the editor, so he printed the passage.  It
        seems at least arguable that this passage in the \CS\ is actually
        referring to the \Utt\ of the \SS, which carries out exactly that
        program in its chapter  about the \sepl{tantrayukti}{hermeneutic rule}
        for interpreting medical statements.  If this is the case, it is
        evidence that Dṛḍhabala (300--500 \CE), the author of this part of the \CS, 
        was aware of the \Utt\ of the \SS.