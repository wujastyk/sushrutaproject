% !TeX root = ../incremental_SS_Translation.tex

\chapter{Introduction to the Uttaratantra}


The \emph{Uttaratantra} of the \SS\ consists of sixty-six
chapters.  This amounts to more than a third of the size of the whole
\SS.  The colophons of our earliest manuscripts show that the \Utt\
has always been subdivided into five major divisions:
\begin{enumerate}
    \item \emph{Śālakyatantra}, on diseases of the eyes, ears, nose and head.  
    This 
    division includes the description of couching for cataract.\footnote{See 
        \cite{leff-2020} for a recent study.}
        
        \item \emph{Kumāratantra}, on avoiding the threat of demonic attacks on 
        children and 
        one  final chapter on disorders of the female genital tract.
        
        \item \emph{Kāyacikitsātantra}, on the treatment of twenty diseases, 
        starting 
        with fever 
        and continuing with diarrhoea through to urinary disorders.
        
        \item \emph{Bhūtatantra}, on possession by supernatural beings, on 
        epilepsy, 
        and on insanity.
        
        \item  \emph{Tantrabhūṣaṇādhyāya}, on the permutations of the 
        savours,\footnote{See \cite{wuja-2000}.} on living well, on logical rules of 
            interpretation for medicine, and on the combinatorics of the four 
            humours.\footnote{Blood is included as a \se{doṣa}{humour} in this 
            chapter.}
            \end{enumerate}
            
            In the Nepalese version of the \SS, the chapters of these divisions are 
            distributed slightly differently than in the vulgate version, see 
            Fig.\,\ref{utt:chapters}.\q{add the swapping of 27 and 51}
            
            \begin{table}
                \centering
                \begin{tabular}{>{\em}lcc}
                \toprule 
                 & Nepalese & vulgate \\
                \midrule
                Śālakyatantra  & 1--26 & 1--26  \\
                
                Kumāratantra & 27--\textbf{37} &  27--\textbf{38}  \\
                
                Kāyacikitsātantra & \textbf{38}--59  & \textbf{39}--59  \\
                
                Bhūtatantra & 60--62 & 60--62  \\
                
                Tantrabhūṣaṇādhyāya & 63--66& 63--66  \\
                \bottomrule
            \end{tabular}
                \caption{The distribution of chapters of the \Utt\ in the Nepalese 
                version and in the vulgate.}
                \label{utt:chapters}
            \end{table}
            
\section{Literature}

Meulenbeld offered an annotated overview of the \Utt\ and a
bibliography of earlier scholarship to
2002.\fvolcite{IA}[300--332]{meul-hist}  He treated the individual chapters 
separately and did not provide reflections on the \Utt\ as a unit with five 
subdivisions.

Meulenbeld also discussed the related issue of whether there might
once have been an \Utt\ attached to the \CS, a
$^*$\emph{Carakottaratantra}.\fvolcite{IA}[99--100]{meul-hist}  While it 
seems 
unlikely that there was ever such a text, there is one verse that deserves 
attention, namely \CS\ \Ca{8.12.50}{737}.  This is almost the last verse of the 
whole work and it states that certain general topics about how to achieve the 
right interpretation of medicine through \sepl{tantrayukti}{hermeneutic rule} 
will be treated at greater length in the sequel, 
or “Uttara" with a view to providing true knowledge of the truth of the 
\se{tantra}{system} from the point of view of merits and 
flaws.\footnote{\dev{tasmādetāḥ pravakṣyante vistareṇottare punaḥ/ 
tattvajñānārthamasyaiva tantrasya guṇadoṣataḥ//}, \CS\ \Ca{8.12.50}{737}.}
This passage is printed in parentheses in the vulgate edition, indicating that the 
editor was unsure about its validity as part of the text.  This is probably because 
the commentator Cakrapāṇidatta said that this passage was considered 
spurious by the earlier tradition, because the whole \Utt\ of the \CS\ was 
spurious.\footnote{\dev{taṃ cānarṣaṃ vṛddhā vadanti, agniveśatantre 
uttaratantrasyaivānārṣatvāt//}, \Ca{8.12.50}{737}.}  Nevertheless, the 
passage appeared in the manuscripts available to the editor, so he printed the 
passage.  It seems at least arguable that this passage in the \CS\ is actually 
referring to the \Utt\ of the \SS, which carries out exactly that program in its 
chapter~65 about the \sepl{tantrayukti}{hermeneutic rule} for interpreting 
medical statements.  If 
this is the case, it is evidence for Dṛḍhabala, the author of this part of the \CS, 
being aware of the \Utt\ of the \SS.