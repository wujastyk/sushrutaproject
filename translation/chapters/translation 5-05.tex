% !TeX root = ../incremental_SS_Translation.tex
\chapter{Kalpasthāna 5: Therapy for those Bitten by Snakes}

\section{Introduction} 

\section{Literature}

A brief survey of this chapter's contents and a detailed assessment of
the existing research on it to 2002 was provided by
Meulenbeld.\footnote{\volcite{IA}[294--295]{meul-hist}. In addition to the
    translations mentioned by \tvolcite{IB}[314--315]{meul-hist}, a translation
    of this chapter was included in \volcite{3}[35--45]{shar-1999}.} 
    
    
    \newpage
    
\section{Translation}

\begin{translation}
    \item [1]
    Now we shall explain the \se{kalpa}{procedure} that is the therapy for 
    someone bitten by a snake.\footnote{On \dev{kalpa}, see note 
    \ref{arunadatta:kalpa}.}
    
    \item[3] For a person bitten on a limb by any snake, one should first
of all make a strong binding, at four fingers measure above the
bite.\footnote{Application of a tourniquet is deprecated by modern
    establishment medicine, which relies on antivenom medications
    \citep[e.g.,][150--151 et passim in the literature]{pill-2013}.
    
    The vulgate introduces the word \dev{ariṣṭā} at this point.  This may be a 
    borrowing from \Ca{Ci.23.251cd}{582}.}
    
    
    \item[4]
    
    Poison does not move around into the body if it is prevented by
bandages (\emph{ariṣṭā})\sse{ariṣṭā}{bandage} or by any other soft
items of \se{plota}{cloth}, \se{carmānta}{leather} or
bark.\footnote{It is hard to translate the word \dev{ariṣṭā}
    otherwise than “bandage,” as referred to by \dev{badhnīyāt} in the
    previous verse, and apparently similar to items of cloth etc., and
    called a \dev{bandha} in the next verse.  But in general Sanskrit
    literature, including medical literature, the word means either “an
    alcoholic tonic” or “an omen of death,” (\Su{1.30.3}{137}), or is a
    plant name.  This raises a question mark over its unique meaning in
    the present context.  The \AH\ (\Ah{Utt.36.42cd}{910}) seems to be a
    gloss on \dev{ariṣṭā}, saying “An expert in mantras may bind using a
    braid made of silk etc. empowered with mantras” (see also
    \Su{5.5.8}{575}).}
    
\item[5] Where a \se{bandha}{bandage} is not suitable, one should raise
the bite up and then cauterize it.\footnote{The vulgate reads
    \dev{utkṛtya} “having excised” rather than translate \dev{uddhṛtya}
    “having raised up.”} Suction, cutting and cauterizing are recommended in
    all cases.

\item[6] Suction will be good after filling the mouth with
\diff{\se{pāṃśu}{earth}}.\footnote{The vulgate recommends cloth, not
    earth (\Su{5.5.6}{574}).}  Alternatively, the snake should be bitten
    \diff{by the person who knows} they have just been bitten.\footnote{The 
    syntax is odd here, and the vulgate has removed the difficulties. 
    \Dalhana{5.5.6}{574} noted that one should hold the snake firmly and give a 
    good bite to its head and tail (\dev{hastābhyāmupasaṃgṛhya pucche vaktre 
    ca sarpaḥ samyag daṣṭavyaḥ}).}

\item[7] 

Now, one should not cauterize anyone bitten by a Maṇḍalin. Because of the
over-abundance of \se{pittaviṣa}{bile-poison}, that bite will cause
\diff{destruction} as a result of cauterization.

\item [8]

Also, an expert in mantras should, together with mantras, tie on a
\se{ariṣṭā}{bandage}.  They say that if that bandage is tied with ropes
and so on it creates a \se{viṣapūtī}{poison-stink}.

\item[9]





    
    %%%%%%%%%%%%
    
    \item[34] \footnote{After this verse, the vulgate text adds twelve
    verses, 35--46, that do not appear in the Nepalese version.}
    
     \item[78] \footnote{After this verse, the vulgate text adds five
        verses, 79--83, that do not appear in the Nepalese version.}
\end{translation}    