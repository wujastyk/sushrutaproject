%d !TeX root = ../incremental_SS_Translation.tex
\chapter{Kalpasthāna 6: Mice and rats}
\label{mūṣikā}


\section{Introduction}

This chapter is numbered 6 in the Nepalese version, but 7 in the vulgate.

\subsection{Literature}

%A brief survey of this chapter's contents and a detailed assessment
%of the existing research on it to 2002 was provided by
%Meulenbeld.\footnote{\volcite{IA}[295]{meul-hist}. In addition to the
%    translations mentioned by \tvolcite{IB}[314--315]{meul-hist}, a
%    translation of this chapter was included in
%    \volcite{3}[61--66]{shar-1999}.} 



\section{Translation}

\begin{translation}
    
    \item[1] 
    
Now I shall explain the \se{kalpa}{procedure} on the topic of
\se{mūṣikā}{mice}.\footnote{The word \dev{mūṣikā} does not distinguish 
between rats and mice; the same is true for MIA and NIA derivatives
\cite[\#10258]{CDIAL}.}
    
    \item[3] 
    
    
    \end{translation}
