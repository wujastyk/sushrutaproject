% !TeX root = incremental_SS_Translation.tex
% Paras

\chapter{Cikitsāsthāna 4:  On the Treatment of Wind 
    Diseases}

\section{Literature} 

Meulenbeld offered an annotated overview of this chapter and a bibliography
of earlier scholarship to 2002.\fvolcite{IA}[265--266]{meul-hist} 

\section{Translation}

\begin{translation}
    
    \item [1] 
    Now we shall describe the treatment of wind diseases.
    
    \item [2]

    \item [3]
    When the wind enters the stomach and one vomits as a result, one should sequentially administer the six-bearing (\dev{ṣaḍdharaṇa}) remedy with cool water for seven nights.\footnote{The vulgate has the reading \dev{chardayitvā} which means "after making (him) vomit". Thus, vomiting is a part of the treatment. Whereas in the H manuscript, vomiting is the symptom of the ailment that needs to be cured.}

    \item [4]
    The remedy constituting of \gls{citraka}, \gls{indrayavā}, \gls{pāṭhā}, \gls{kaṭukā}, \gls{ativiṣā}, and \gls{abhayā} cures serious diseases and is called the six-bearing (\dev{ṣaḍdharaṇa}).

    \item [5]
    When the wind has entered the abdomen (\dev{pakvāśaya}), one should treat it with evacuation of the bowels (\dev{virecana}) using an unctuous substance. One should also treat it with cleansing enemas and excessively salty foods.\footnote{In H, the reading \dev{prāsāḥ} should be read as \dev{prāśāḥ} for it to mean "foods". Otherwise, \dev{prāsāḥ} means "throwing/discharging" or "darts/spears".}\q{This is a change we should make in the edition.} 

    \item [6]
    Once the wind has entered the lower belly, a cleansing enema is recommended. And, on the wind having entered the ears, etc., the wind-slayer sequence should be executed.\footnote{In the H manuscript reading "\dev{śrotādi}\ldots," there appears to be a double sandhi. The base word (\emph{prātipadika}) for "ear" is "\dev{śrotas}". First, the "\dev{s}" disappears making the word "\dev{śrota}". Then the "\dev{a}" at the end combines with the "\dev{ā}" in the word "\dev{ādi}". Thus, the final form becomes "\dev{śrotādi}" as in H. Also see Nidānasthāna Ch. 1 verse ?? for another example of double sandhi.
    Furthermore, the syllable in H after "\dev{cānila}" is not clear. It could be "\dev{hya}" or "\dev{hā}" or perhaps something else. The reading in the vulgate for this syllable is "\dev{hā}". Thus, the complete word becomes "\dev{anilahā}" which means "the slayer of wind". This makes proper sense in this verse. We have considered this reading ("\dev{anilahā}") for our translation.}\q{You need not give all the grammatical details about śrotādi.  Assume you are talking to knowledgeable Sanskrit scholars.}    

    \item [7]
    On the wind having entered the skin, flesh, and blood, one should rub oil on the body (\dev{abhyaṅga}), apply a poultice on the body (\dev{upanāha}), massage the body (\dev{mardana}), smear ointments on the body (\dev{ālepana}), and do blood-letting (\dev{asṛgvimokṣaṇa}). 

    \item[8]
    On the wind having entered the ligaments, joints, and bones, the wise [physician] should employ the application of an unctuous poultice (\dev{snehopanāha}), cauterization (\dev{agnikarma}), binding (\dev{bandhana}), and massage.

    \item [9]
    On the wind being concealed within the bones, it (wind) should be beaten by churning those body parts with hands. A strong physician should then insert a narrow tube within the bone and suck out the wind completely from the bone.\footnote{The H manuscript has the reading \dev{asthīni} which is the accusative plural form of \dev{asthi}. The accusative case does not make sense here.  The Vulgate has the reading \dev{asthani}, the locative singular form of \dev{asthi}. This reading makes proper sense in the verse. Therefore, we have accepted the Vulgate reading \dev{asthani} for translating this verse.} 
    % Ask Dominik Sir about the vipula types of anustup metre. Refer Ch. 106 of the Gita Darpana book of Svami Ramasukhadasa. It has many vipula variants, etc. of the anustup.        
    % My earlier comment in footnote: Also, the metre of this hemistich of the verse is not according to the \emph{anuṣṭup} metre. This gives us more reason to think that the reading \dev{asthīni} is corrupted.   

    \item[10]
    On the wind having entered the semen, one should perform the treatment for the defects of the semen.\footnote{Ḍalhaṇa comments here that this treatment for the defects of the semen is mentioned [earlier] as the \dev{śukraśoṇitaśuddhi}, the purification of the semen and the blood. This is the \emph{Śārīrasthāna} Ch. 2, \dev{śukraśoṇitaviśuddhi}. The second hemistich of this verse is not a part of this sentence but is a part of the sentence in the next verse. That is because the remedies described in this hemistich are appropriate for the disease described in the first hemistich of the next verse.}

    \item[11]
    The intelligent physician should conquer the wind situated within the whole body by immersion, \textit{kuṭī}, \textit{karṣa}, \textit{prastara}, oil massage, enema, and blood-letting.\footnote{In H, the last syllable \dev{ni} of the compound word does not make sense. The Vulgate has the compound word ending with \dev{bhiḥ} which makes proper sense. For making a meaningful translation, we have accepted the Vulgate reading here. Furthermore, Ḍalhaṇa describes the treatments \textit{kuṭī}, \textit{karṣū}, and \textit{prastara} in his commentary. Regarding blood-letting, he says that because the verse has the plural form \dev{sirāmokṣaiḥ}, five blood vessels have to be drained of blood if the wind is not pacified by oil massage, etc.} Or, in case of the wind situated in one part of the body and contained within it, the intelligent physician should cure it with horns. 
    
\end{translation}
