% !TeX root = incremental_SS_Translation.tex
\chapter{Sūtrasthāna 29: prognostic signs relating to the messenger 
and to dreams}

\section{Literature}

Meulenbeld offered an annotated overview of this chapter and a bibliography
of earlier scholarship to 2002.\fvolcite{IA}[219--220]{meul-hist} 

\citeauthor{gosw-2011} studied the commentaries of Ḍalhaṇa and 
Cakrapāṇidatta on this and the following adhyāyas up to 32, focussing on the 
topic of \se{ariṣṭa}{omens}.  He concluded that both 
authors were influenced by the Indriyasthāna of the \CS\ in their commentaries 
on this topic.\footcite{gosw-2011}

\citet{lu-2025} discussed the reception of the materials of this
chapter by Chinese Buddhists, especially in the work of the second-
and third-century translators of Saṅgharakṣa's \emph{Yogācārabhūmi},
An Shigao (ca.\ 148--180 \CE) and Dharmarakṣa (fl.\,284 \CE).  As Lu
said, “The Sanskrit text fixes the baseline wording” of the Chinese
translations.\footcite[2]{lu-2025}  This fixes the reception of the
\SS\ in China to the mid- to late second century.\footnote{The \SS\ passages
    directly known to the Chinese translators include \SS\ 1.28.31--32 and 
    1.29.18--19ab.  Note that in \SS\ 1.29.19ab, the “fourth” day (\dev{caturthī})
    is not present in the Nepalese version, but is present in the Chinese 
    receptions of the text and in the vulgate.}

\section{Translation}
    
\begin{translation}    
    \item [1] 
\end{translation}