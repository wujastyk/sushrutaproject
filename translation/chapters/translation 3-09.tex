%!TeX root = ../incremental_SS_Translation.tex

\chapter{Śārīrasthāna 9:  An Analysis of the Pipes}

\section{Introduction}


\sse{dhamanī}{pipe}%
\sse{sirā}{duct}%
\sse{srotas}{tube}%
\sse{nāḍī}{tube}%
\sse{snāyu}{sinew}%
\sse{kaṇḍarā}{tendon}%

Ancient Indian physicians and surgeons were aware of the network of
vessels and tendons in the body and they debated their appearence and
functions. There are a number of different names for these, such as
\emph{sirā}, \emph{dhamanī}, \emph{srotas},  \emph{nāḍī},
\emph{snāyu}, and \emph{kaṇḍarā}, and these terms pose a challenge to
the translator.  The first four seem to refer clearly to tubes, and
other translators have used “vein” and “artery” for at least two of
them.  But the heart was not thought of as a pump in the Āyurvedic
view of the body, nor did ancient Indian physicians think that the
blood circulated in the post-Harveian sense. Blood moved outwards
radially from the centre of the body, like water from a water-tank
irrigating fields.  There was certainly no concept of a contrast
between venous and arterial circulation, and several of these vessels
are most commonly seen as being rooted in the navel, not the heart.

\label{intro:tubes} There is also the interesting question of what it
is that was actually considered to be flowing in these vessels.  The
\emph{sirā} vessels do seem to carry blood, but the \emph{dhamanī}
vessels conduct wind (Skt.\,{\small$\sqrt{}$}\emph{dham}, “blow”), an
idea strongly reminiscent of the classical Greek doctrine of
Praxagoras of Cos, whose \emph{pneuma}-carrying arteries started in
the heart and spread out into
\emph{neura}.\footnote{\cites[137]{phil-1973}[137--159 et 
passim]{wals-2016}.} But all the vessels
    seem to be implicated in transporting humours, waste products,
    sensations, and perceptions.\footnote{\citet[344--52]{dasg-hist}
        provided a good discussion of this topic.} In āyurveda (as opposed to
        tantra or yoga), the \emph{nāḍī} vessels are primarily discussed as
        the locus of the pulse, though what it is that pulses is not made
        explicit.  In one case Vāgbhaṭa used \emph{nāḍī} to refer to the
        windpipe.  Faced with the word \emph{snāyu}, one is virtually obliged
        to use its English cognate term “sinew.”  But the \emph{snāyu}s seem
        sometimes to refer to what are today called nerves rather than to
        sinews or tendons, and indeed this is the primary meaning of the term
        as used in several contemporary Indian languages. The word
        \emph{kaṇḍarā} more unambiguously refers to the latter.
        
        \looseness1
        I have chosen to use neutral terms like `pipe',
        `tube', `duct', and `sinew' to translate some of the above terms,
        thus retaining the original distinctions. It is for future
        research to develop a more thorough understanding of the
        body-image which underlies the Sanskrit terminology, and perhaps
        with it a more subtle and appropriate set of English translations
        for these terms.
        
        \subsection{Fluids and their conduits}
        
        The types of fluid in the āyurvedic body include
        \se{rakta}{blood}, milk, semen, \se{prāṇa}{breath}, the \se{rasa}{the
            juice of digested food}, and the humours \se{vāta}{wind},
        \se{pitta}{bile}, and \se{kapha}{phlegm}.
        
        %Energy is present throughout the whole body, and is
        %thought to be cold, oily, and solid.  It is of the \skt{cooling
        %principle}{saumya}, and is believed to give the body its power and
        %nourishment.
        
        %from Tarabout.tex:
        
        These fluids are transported from place to place by three
        principle types of conduit: \se{sirā}{ducts},
        \se{dhamanī}{pipes}, and \se{srotas}{tubes}. Given the
        importance of this system of fluid distribution to the āyurvedic
        physiology, surprisingly little work has been done on clarifying
        what these conduits do, and how they are explained in āyurvedic
        theory (exceptions include \citealt[ii.13]{dasg-hist} and
        \citealt[ch.\,2]{kutu-anci}).
        
        %This lack of interest in the fluid
        %mechanics of the anatomy is not new.  Early Chinese and Japanese
        %medical books abound in diagrams of the body carrying lines
        %illustrating channels of \emph{chi}, but no Indian book or
        %manuscript to my knowledge shows maps of the medical ducts, pipes,
        %or tubes.  In fact, medical illustration does not become
        %established as a tradition in India until relatively recently,
        %only getting underway seriously with the production of Sanskrit
        %and vernacular medical printed texts in the second half of the
        %nineteenth century \citep{wuja-body}.  And that initiative too,
        %being inspired and influenced by Western anatomy, ignores the
        %traditional conduits of the āyurvedic body.
        
        \subsubsection{\se{sirā}{Ducts}}
        
        According to the \emph{Suśruta\-saṃhitā}, the function of the 700
        ducts is to carry wind, bile, phlegm and blood around the body,
        starting from their origin in the navel. In a vivid pair of
        metaphors, one agricultural and one botanical, Suśruta's text
        described the ducts as follows (Su.śā.7.3):
        \begin{quote}
            As a garden or a field is irrigated by water-carrying canals, and
            each part receives nourishment, so the ducts provide nutrition to
            the body by means of their contraction and dilation. Their
            branches are just like the veins on a leaf.
        \end{quote}
        A point of special interest is that the ducts are coloured
        according to what they carry: those carrying wind are
        \se{aruṇa}{yellowish brown}, those carrying bile are dark blue,
        those carrying phlegm are white, and those carrying blood are red
        (Su.śā.7.18). It seems likely that these distinctions are based on
        the observation of different-coloured vessels under the surface of
        the skin.  In yet another simile, Suśruta likens the distribution
        of these ducts from the umbilical centre through the body to the
        spokes radiating from the centre of a wheel (Su.śā.7.7).
        
        \subsubsection{\se{dhamanī}{Pipes}}
        
        There are said to be twenty-four pipes in the body (Su.śā.9). Like
        the ducts, they originate in the navel. From there, ten go up, ten
        down, and four sideways.
        
        Those which go up from the navel support the body by carrying
        \se{viśeṣa}{particular items} such as sound, touch, vision,
        taste, smell, \se{praśvāsa}{out-breath},
        \se{ucchvāsa}{in-breath}, yawning, sneezing, laughter, speech,
        crying, etc.  These ten pipes go from the navel to the heart and
        there each one divides into three branches, thus producing thirty
        pipes.  Ten of these are devoted to carrying the humours, wind,
        bile, and phlegm, as well as blood and nutritive fluid (two pipes
        for each substance).  Eight more carry sense impressions: sound,
        form, taste, and smell (again, two pipes each).  Two pipes are
        used for \se{bhāṣā}{speech}, two for making
        \se{ghoṣa}{sound}, two for sleeping, and two more for waking up.
        Two pipes carry tears.  Two pipes connected to the breasts carry
        women's breast-milk; curiously, in men the same two pipes are said
        to carry semen from the breasts.
        
        Those pipes which go down from the navel carry substances such as
        wind, urine, faeces, semen, and menstrual blood.
        %At this point
        %things become a little complicated.  On reaching the receptacle of
        %bile the pipes separate out the nutritive juice which has resulted
        %from digestion and carry it as refreshment to the whole body, also
        %supplying it to the upper and horizontal pipes. They fill the
        %receptacle of nutritive juice. And they separate out the three
        %principle impurities: urine, faeces, and sweat.
        In between the receptacles of raw and digested food, the pipes
        divide into three branches, as before.  The first ten pipes have
        the same functions as the first ten upward pipes. The next two
        carry food to the intestines, and another two carry water.  Two
        carry urine to the bladder.  Two generate and transport semen,
        and two make it ejaculate.  In women, the same four pipes carry
        and discharge menstrual blood.  Two pipes are connected to the
        intestines and function in defecation. The remaining eight pipes
        supply sweat to the horizontal pipes.
        
        The four pipes which run sideways are said to subdivide hundreds
        of thousands of times, holding the body together in a network.
        Their ends are connected to the hair follicles, and through these
        sweat is carried out and nutritive juice is carried in.  This is
        how massage oils, showers, and ointments can move through the skin
        and affect the body internally.  They are also the means by which
        pleasant and unpleasant sensations of touch are experienced.
        
        \subsubsection{\se{srotas}{Tubes}}
        
        According to Suśruta, there are initially twenty-two tubes in the
        body, two for each of eleven substances (Su.śā.9.12--13; cf.\
        Ca.vi.5). Two of the \se{srotas}{tubes} carry
        \se{prāṇa}{breath}, and are joined to the heart and the
        \se{dhamanī}{pipes} which carry nutritive juice. Two more carry
        food, and are joined to the food-carrying pipes and the stomach.
        Two carry water and are joined to the palate and the
        \se{kloman}{lung}. Two carry nutritive juice and are joined to
        the same places as those carrying breath.  Two carry blood, and
        are joined to the liver, the spleen, and the pipes which carry
        blood. Two carry flesh, and are joined to the ligaments, skin, and
        pipes which carry blood. Two carry fat and are joined to the
        \se{kaṭī}{waist} and the kidneys.  Two carry urine and are joined
        to the bladder and penis. Two carry faeces and are joined to the
        receptacle of digested food and the rectum. Two carry semen and
        are joined to the breasts and testicles.  Two carry menstrual
        blood and are joined to the womb and the pipes which carry
        menstrual blood.  (There is no suggestion that these last pairs
        are specific to either gender.) Caraka adds three more categories
        of tube: two carrying bone, two carrying marrow (completing the
        set of seven basic \se{dhatu}{body elements}), and two carrying
        sweat.  He omits menstrual blood.
        %Caraka's description of the various roots of the tubes is
        %different in small details from Suśruta's, and he also asserts
        %that the humours wind, bile, and phlegm are carried in them
        %indiscriminately and all over the body (Ca.vi.5).
        Like the horizontal pipes, the tubes in the body divide and
        subdivide into innumerable tiny branches.
        
        In contrast to the ducts and pipes, the description of these tubes
        is embedded in a discourse of injury, and the symptoms arising
        from damage to them are listed.
        
        The \SS\ recorded the existence of an ancient disagreement amongst
        physicians as to whether the pipes, ducts and tubes are really
        separate types of vessel, and in particular whether there is a
        significant difference between \sepl{dhamanī}{pipe} and
        \sepl{srotas}{tube}. He argues that there is indeed a difference
        between these three types of vessel: they look different, have
        different connections, and different functions.  The authoritative
        tradition of medical science also asserts their difference.  It is
        merely because of their close proximity, similarity, and small
        size that they are conflated.  The \CS\ also testified to
        contemporary debates about the nature of these vessels; it 
        recorded---and rejected---an extreme view that the human body
        consists only of a conglomeration of tubes.
        
        The translation
        “pipe” for \dev{dhamanī} (from the \root\,\dev{dham} “blow”) is
        intended to suggest the primary function of transporting air;
        “vessel” would be an alternative translation. Adhyāyas 
        7 and 8 of the \emph{Śārīrasthāna} described the \dev{sirā} and adhyāya 
        9 
        the \dev{dhamanī}, with the \dev{srotas} being described at the end of 
        chapter 9.
        
        
        \begin{sloka}
            Conduits such as the ducts grow in the flesh just like lotus roots located 
            in 
            muddy water grow in all directions in the ground.
        \end{sloka}
        
        \begin{quote}
            1.14
            
            From the heart it enters the twenty-four \sepl{dhamanī}{pipe}.
            Ten go up, ten go down, and four are horizontal.  Then, through an
            invisible agency, it nourishes the whole body, day in, day out,
            making it grow, holding it up, and making it go.  One can mark its
            passage as it courses through the body by inference based on
            whether diseases are caused by diminution or by superfluity.  This
            \se{rasa}{nutritive juice} courses through all parts of the body,
            through the humours, body tissues, impurities, and organs.
        \end{quote}
        
        
        \section{Literature}
        
        On these different conduits, see
        \cites[404--406]{wuja-2022}[xlvi--xlvii]{wuja-2003}[26--28]{ray-1980} 
        and 
        the descriptions by 
        \tvolcite{2}[344--352]{dasg-medi}. 
        
        \newpage
        
        \section{Translation}
        
        \begin{translation}
            
\item [1]
            
Next, we shall discuss the analysis  of the \sepl{dhamanī}{pipe}.
            
\item[2]
            
There are twenty-four pipes.  And they originate from the navel.  
            
Regarding that, some teachers have said that there is no difference
between \sepl{sirā}{duct}, \sepl{dhamanī}{pipe} and
\sepl{srotas}{tube}, because pipes and tubes are just types of duct.
But here, it is said that that is not correct. Pipes, tubes  and ducts
are different. Why? Because of the difference in their colours, in
their connections to their roots, because of the distinctions in their
function, and because of traditional doctrine.
            
            
            
        \end{translation}
