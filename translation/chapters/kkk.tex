%!TeX root = ../incremental_SS_Translation.tex

\chapter{Śārīrasthāna 9:  An Analysis of the Pipes}

\section{Introduction}

This problem is acute when one is faced with the Sanskrit terms for the
various tubes, pipes, and connecting ligatures in the body.  Obviously the
ancient physicians and surgeons were aware of the network of vessels and
tendons in the body. There are a number of different names for these, such
as
\se{dhamanī}{pipe}
\sskt{duct}{sirā}
\sskt{tube}{srotas}
\se{nāḍī}{tube}
\sskt{sinew}{snāyu}
\sskt{tendon}{kaṇḍarā}
\emph{sirā},
\emph{dhamanī}, \emph{srotas},  \emph{nāḍī},  
\emph{snāyu}, and
\emph{kaṇḍarā}.  The first four of these seem to refer clearly to 
tubes,
and other translators have used `vein' and `artery' for at least two of
them.  But the heart was not a pump in the āyurvedic view of the body, nor
did the blood circulate in the post-Harveian sense.  There was certainly no
concept of a contrast between venous and arterial circulation, and several
of these vessels are most commonly seen as being rooted in the navel, not
the heart.

\label{intro:tubes} 
There is also the interesting
question of what it is that was actually considered to be flowing
in these vessels.  The \emph{sirā} vessels do seem to carry
blood, but the \emph{dhamanī} vessels conduct wind
(Skt.\,{\small$\sqrt{}$}\emph{dham}, `blow'), an idea strongly
reminiscent of the classical Greek doctrine of Praxagoras of Cos,
whose \emph{pneuma}-carrying arteries started in the heart and
spread out into \emph{neura}.\footnote{\citet[137]{phil-gree}.}
    But all the vessels seem to be implicated in transporting
    humours, waste products, sensations, and
    perceptions.\footnote{\citet[344--52]{dasg-hist} provides a good
        discussion of this topic.} In āyurveda (as opposed to tantra or
        yoga), the \emph{nāḍī} vessels are primarily discussed as 
        the
        locus of the pulse, though what it is that pulses is not made
        explicit.  In one case Vāgbhaṭa uses \emph{nāḍī} 
        to refer to the
        windpipe.  Faced with the word \emph{snāyu}, one is virtually
        obliged to use its English cognate term `sinew'.  But the
        \emph{snāyu}s seem sometimes to refer to what are today called
        nerves rather than to sinews or tendons. The word 
        \emph{kaṇḍarā}
        more unambiguously refers to the latter.
        
        \looseness1
        I have chosen to use neutral terms like `pipe',
        `tube', `duct', and `sinew' to translate some of the above terms,
        thus retaining the original distinctions. It is for future
        research to develop a more thorough understanding of the
        body-image which underlies the Sanskrit terminology, and perhaps
        with it a more subtle and appropriate set of English translations
        for these terms.


\section{Literature}

On these different conduits, see
    \cites[404--406]{wuja-2022}[xlvi--xlvii]{wuja-2003}[26--28]{ray-1980} and 
    the descriptions by 
    \tvolcite{2}[344--352]{dasg-medi}. The translation
    “pipe” for \dev{dhamanī} (from the \root\,\dev{dham} “blow”) is
    intended to suggest the primary function of transporting air;
    “vessel” would be an alternative translation. Adhyāyas 
    7 and 8 of the \emph{Śārīrasthāna} describe the \dev{sirā} and adhyāya 9 
    the \dev{dhamanī}, with the \dev{srotas} being described at the end of 
    chapter 9.
  

\begin{sloka}
Conduits such as the ducts grow in the flesh just like lotus roots located in 
muddy water grow in all directions in the ground.
\end{sloka}

\begin{quote}
    1.14

From the heart it enters the twenty-four \sepl{dhamanī}{pipe}.
Ten go up, ten go down, and four are horizontal.  Then, through an
invisible agency, it nourishes the whole body, day in, day out,
making it grow, holding it up, and making it go.  One can mark its
passage as it courses through the body by inference based on
whether diseases are caused by diminution or by superfluity.  This
\se{rasa}{nutritive juice} courses through all parts of the body,
through the humours, body tissues, impurities, and organs.
\end{quote}

\begin{translation}

\item [1]
  
Next, we shall discuss the analysis  of the \sepl{dhamanī}{pipe}.
  
\item[2]

There are twenty-four pipes.  And they originate from the navel.  

On that topic, some teachers say that there is no difference between 
\sepl{sirā}{duct}, \sepl{dhamanī}{pipe} and \sepl{srotas}{tube}.  
  


\end{translation}
