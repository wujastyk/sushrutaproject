% !TeX encoding = UTF-8
\frenchspacing % don't want to fuss about end-of-sentence spacing in glossaries


\renewcommand\bibfont{\small }    

\printbiblist[%
heading=biblistintoc,
title=Editions and Abbreviations,
notkeyword=botanical
]
{shorthand} % the Abbreviations (shorthands)

\indexprologue{\emph{\footnotesize Numbers after the final 
        colon refer to pages in this book.}} 

\printindex[manuscripts]

\printbibliography[title=General Bibliography,
notkeyword=edition,
notkeyword=shorthand,
notkeyword=botanical,
heading=biblistintoc
]

\newpage

\begin{footnotesize}
    
\chapter{Materia Medica}
        
   
% Take the heading of the Abbreviations down from chapter to section:
\defbibheading{biblistintoc}[\bibname]{%
       \addcontentsline{toc}{section}{#1}%
       \section*{#1}%
       \markboth{#1}{#1}}
       

\printbiblist[
    heading=biblistintoc,
    title=Abbreviations,
    keyword=botanical]{shorthand}
    
%\renewcommand{\glossaryname}{Flora and Fauna}
%\newpage
% \renewcommand{\glossaryname}{Materia Medica}
   
% If you want to print all the glossary entries, 
% use the selection=all option in \GlsXtrLoadResources (see 
% xelatex-glossaries.sty.
%
%\printunsrtglossaries % plant names.  iterate over all defined entries 
%% the setup for the glossaries package is in xelatex-indexing etc.
%%

\apptoglossarypreamble[plants]{\emph{\footnotesize 
        Numbers after the final colon refer to 
        pages in this book.  Silent reference regarding plant names has also been 
        made to Wikipedia, which has well-curated botanical 
        information, including up-to-date nomenclature and synonym 
        lists.}\bigskip}


    \printunsrtglossary[type=plants]
    \printunsrtglossary[type=animals]
    \printunsrtglossary[type=minerals]

\clearpage % prevent the page heading “Glossary” printing on previous page
    \chaptermark{Glossary}\sectionmark{Glossary}
    \printindex[lexical]  
%
    
    \clearpage
% Todo list
\end{footnotesize}

\thispagestyle{empty}