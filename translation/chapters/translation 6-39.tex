% !TeX root = incremental_SS_Translation.tex
% Madhu

\chapter{Uttaratantra 39:  On Fevers and their Management [draft]}

\section{Literature} 

Meulenbeld offered an annotated overview of this chapter and a bibliography
of earlier scholarship to 2002.\fvolcite{IA}[313--317]{meul-hist} 

\section{Remarks on the Nepalese version}

This chapter is numbered 6.39 in the vulgate.


\newpage

\section{Translation}

\begin{translation}
    
 %   Wednesday, Oct 19, 2022
    
    \item[1]  And now we shall explain the chapter on the prevention of 
    fever.\footnote{The present chapter discusses the therapeutics of fever. 
    One would expect this to be preceded by a chapter on the the causes of 
    fever, perhaps in the \emph{Nidānasthāna}, but such a chapter does not 
    occur in the \SS.}
%    
%    % Thursday, Oct 20, 2022
%    
%
%    
%    \item[3.1]  
%    
%    And now, O Suśruta, my child.\footnote{The vulgate's
%    attribution of this chapter to Dhanvantari is, as usual, not
%    present in the Nepalese MSS. What we do have is the vocative of
%    the name Suśruta, to whom this chapter is being addressed by an
%    unnamed speaker.  In accordance with other chapter-beginnings in the 
%    Nepalese MSS, the speaker should be Divodāsa \citep{birc-2021}.}
%    
%    \item[3--5ab]  
%    
%In a previous incarnation, god had extracted ambrosia from the midst
%of the waters, and it gave the three and thirty gods immortality. 
%Suśruta and the other students questioned that god, who was seated.
%
%O best of physicians! Complications relating to the wounds of those
%who are wounded have been stated.  So now, declare it to us in brief
%and at length.\footnote{This suggests that this chapter followed a
%    chapter on wounds.  Yet it follows the chapter on diseases of the
%    female reproductive tract (\emph{yonivyāpat}), or in the Nepalese
%    MSS, the chapter on demons (\emph{grahotpatti}).  This suggests that
%    this chapter was once located at another place in the text, perhaps
%    after \Su{1.22}{107--110} “questions about wound discharges
%    (\emph{vraṇāsrāva})”.  However, the canonical list of sixteen
%    complications\sse{upadrava}{complication} is known from the \CS\ (see 
%    \Cs{6.25.29--31}{671}), not the \SS.}
%    
%    
%\item[5cd--6]    
% 
%The wound of one who is afflicted by a complication is hard to treat.
%    
%The complications of a wounded person, whose flesh has lost its
%strength, are considered to be the most difficult thing to treat, 
%because of  the complete loss of the remaining \se{dhātu}{body
%    tissue}.
%    
%\item[7ab] 
%
%Thus, O best of speakers, please describe all the 
%complications.\footnote{The vulgate says more or less the same, but in more 
%verbose language.}
%    
%\item[8]
%    
%After hearing their statement, the best of the physicians spoke: “To
%begin with, I will explain fever, traditionally known as the king of the group of 
%diseases.”
%
%
%\item[9] 
%
%It has arisen from the fire of Rudra's anger, burning all living
%beings. It is well known by various names amongst different peoples.
%    
%% got to here, 26 Nov 2024
%
%    
%    \item[10] Undoubtedly, the fever occurs here at the beginning of birth and
%    death. Thus fever is stated as the king of all diseases.
%    
% %   % Sunday, Oct 23
%%    
%%    \item[11ab] No one except the gods and human beings tolerate it.
%%    
%%    \item[11cd]  
%%    
%%    \item[12]  
%%    
%%    \item[13ab]  
%%    
%%    \item[13cd] There are sweat obstruction, increased temperature, and 
%%excessive
%%    pain all over the parts of body.
%%    
%%    % Monday, Oct 24
%%    
%%    \item[14ab] Simultaneously here, the disease is instructed to be fever.
%%    
%%   \item[14cd]  
%%    
%%    \item[15] Even though the fever has arisen from different causes, it is
%%    prescribed in eight kinds. In those who eat unwholesome food, the humors
%%    are aggravated.
%%    
%%    \item[16ab] They permeate the body entirely and indeed cause to bring 
%%fever.
%%    
%%   \item[16cd]  
%%    
%%   \item[17]  
%%    
%%   \item[18]  
%%    
%%    \item[19cd] Even by the wrongly performed and unctuousness etc., and 
%%by 
%%the
%%    actions of people.
%%    
%%    \item[20] From the various kinds of hindrances, from the emergence of
%%    diseases, excessive exercise, decay, indigestion, evenness from poison,
%%    and the alteration of balance.
%%    
%%    % Tuesday, Oct 25
%%    
%%    \item[21] From the smell of poisonous flowers, aggravation from the 
%%eclipse 
%%of
%%    stars, curses, magical spells, and suspicion of seizers.
%%    
%%    \item[22] Abnormal delivery in women, using unwholesome things during
%%    childbirth, and in the first breastfeeding. The fever is aggravated by
%%    the humor. (19--22)
%%    
%%    \item[23--24]  These humors, agitated and bewildered in many ways 
%%going 
%%in 
%%the
%%    wrong direction, throw the internal digestive fire out that moves
%%    outside. The inner digestive fire prevents the passage of sweat, raising
%%    the temperature of the person's body. The body becomes very hot and 
%%does
%%    not sweat all over.
%%    
%%    % Thursday, Oct 27
%%    
%%    \item[25]  Exhaustion, discontent, paleness, bad taste, the flow of tears 
%%from
%%    eyes, repeated desire and dislike in cold, air, sun-heat, etc.
%%    
%%    \item[26]  Yawning, pain in limbs, heaviness, hair loss, tastelessness, and
%%    darkness, the suffering person becomes unhappy and cold then the fever
%%    will arise.
%%    
%%    % Friday, Oct 28
%%    
%%    \item[27]  Generally, and, especially, excessive yawning due to breeze, 
%%burning
%%    in the eyes, and no appetite for food are observed in cases of wind,
%%    bile, and phlegm, respectively.
%%    
%%    % Saturday, Oct 29
%%    
%%    \item[28]  In the fever generated by aggravation of all humors, all the
%%    symptoms are mixed up. In the case of either of two, the fever is
%%    mingled with and the involvement of two humors in disease causation is
%%    wisely united.
%%    
%%    ( I am not happy with this translation and will have to rework it.)
%%    
%%    \item[29]  Trembling, inconstant paroxysm, dryness of the throat and lips, 
%%loss
%%    of sleep, decline, collapse, and roughness of the body parts.
%%    
%%    % Sunday, Oct 30
%%    
%%    \item[30]  Headache, pain in the body, pain in speech organs, 
%%repugnance, 
%%the
%%    thickness of the stomach, soothing pain, flatulence, and yawning are the
%%    symptoms of fever originating from the wind.
%%    
%%    \item[31]  Intense attack, rough diarrhea, and internal solid heat, little
%%    sleep and vomiting, inflammation in the throat, lips, mouth appear.
%%    
%%    % Monday, Oct 31
%%    
%%    \item[32]  Babbling, pungent mouth, fainting, heat, intoxication, thirst,
%%    yellowish stool, urine and eyes, and confusion - these symptoms are
%%    found in fever generated by bile.
%%    
%%    \item[33A]  Immobility, fixed, strong attack, sloth, debility, sweetness in 
%%the
%%    mouth, whitish urine and excrement, stultifying, and now disgust as
%%    well.\footnote{
%%        This verse is not in the vulgate edition, but the commentator Ḍalhaṇa
%%        acknowledges it.
%%    }
%%    
%%    % Tuesday, Nov 1
%%    
%%    \item[33ab]  Heaviness, cold, becoming moist, the bristling of the hair of 
%%the
%%    body, and excessive sleepiness.
%%    
%%   \item[33cd]  
%%    
%%   \item[34ab]  
%%    
%%    \item[34cd]  Rheum, loss of appetite, cough, and whiteness of the eyes. 
%%These
%%    symptoms are found in fever generated by phlegm.
%%    
%%   \item[35]  
%%    
%%   \item[36]  
%%    
%%   \item[37]  
%%    
%%   \item[38ab]  
%%    
%%    \item[38cd]  All the symptom is generated from all three humors, and, 
%%now,
%%    listen to me about the specific type.
%%    
%%    \item[39ab]  Not too hot or cold, minimal consciousness, confused looks, 
%%loss
%%    of voice.
%%    
%%   \item[39cd]  
%%    
%%    % Wednesday, Nov 2
%%    
%%   \item[40ab]  
%%    
%%    \item[40cd]  The patient lies down, breathing heavily, affected by 
%%babbling 
%%as
%%    complications.
%%    
%%    \item[41]  Others call it abhinyāsa(I am not so sure about 
%%    it.)\footnote{The commentator Ḍalhaṇa discusses it in detail.}, similar to 
%%Hatauja 
%%    (not so sure about it) \footnote{The commentator Ḍalhaṇa discusses it in 
%%detail.}. 
%%    The fever arising from congested humors is curable with
%%    difficulty; others find the fever incurable. (I am not too sure about
%%    the translation, I think I should rework on it.)
%%    
%%   \item[42]  
%%    
%%   \item[43]  
%%    
%%   \item[44]  
%%    
%%   \item[45ab]  
%%    
%%    \item[45cd--46ab]  (Fever caused by congested humors) becoming more 
%%severe, it
%%    attains calmness or kills the patient on the seventh, tenth, or twelfth
%%    day.
%%    
%%    % Thursday, Nov 3
%%    
%%    \item[46cd]  Fever arisen by two humors combined is taught of three 
%%types
%%    having symptoms of aggravation of the two concerned humors.
%%    
%%    \item[47--48ab]  Thirst, fainting, confusion, burning sensation, loss of 
%%sleep,
%%    headache, dryness of the throat and mouth, nausea, the bristling of the
%%    hair of the body, and complete loss of appetite, severe joint pain and
%%    yawning are the features of fever generated by wind and bile.
%%    
%%    % Friday, Nov 4
%%    
%%    \item[48cd--49]  Numbness, soothing pain in joints, sleepiness, heaviness, 
%%in
%%    the same manner, it arises head-seizure, rheum, cough, sweating. And
%%    lassitude, delirium, and babbling are the features of fever generated by
%%    phlegm and wind.
%%    
%%    \item[50]  Smeared bitterness in the mouth, drowsiness, lassitude, 
%%delirium,
%%    cough, thirst, recurrent burning feeling, and frequent cold are the
%%    features of fever generated by phlegm and bile.
%%    
%%    % Saturday, Nov 5
%%    
%%   \item[51]  
%%    
%%   \item[52]  
%%    
%%   \item[53]  
%%    
%%   \item[54]  
%%    
%%   \item[55]  
%%    
%%   \item[56]  
%%    
%%    \item[57]  Experts say that the fever that occurs every third or fourth cycle
%%    (malaria in the modern sense!) is generated due to the wind.  And the
%%    experts say it springs up overheating and overdrinking (pāna? not too
%%    sure) because of the excessive bile.
%%    
%%    \item[58]  Experts say that a slow fever (Monier-Williams gives this) and a
%%    yellow spot in the white of the eye are caused by the abundance of
%%    phlegm. The chronic fevers (viṣamajvarā) that are connected...
%%    (muktānubandhām- not so sure about it, it either could be 
%%mutvānubandhām
%%    or muñcānubandhāṃ, need to check KL 699 ms). They are generally
%%    generated by two humors jointly. (I should rework on it).
%%    
%%    \item[59]  In fever, when phlegm and wind are situated in the skin, they
%%    initially generate coldness, and they pacify; bile causes a burning
%%    feeling at the end.
%%    
%%    % Sunday, Nov 6
%%    
%%    \item[60]  And then, in the beginning, the bile situated in the skin causes a
%%    burning feeling... (the last part of the pāda is unclear to me, I would
%%    like to see the reading of K if it is available!) and the other two,
%%    wind and phlegm, cause cold at the end when it pacifies.
%%    
%%    \item[61]  These two fevers, such as burning feeling and cold, are 
%%mentioned 
%%to
%%    be generated by two humors jointly. Of them, that began with the burning
%%    feeling is grievous and regarded as the most difficult to cure.
%%    
%%    % Monday, Nov 7
%%    
%%   \item[62ab]  
%%    
%%    \item[62cd--63ab]  As mentioned before, the intermittent fever arising 
%%from 
%%the
%%    wind forcibly approaches one's own time (svaṃ kālaṃ, not too sure about
%%    it), in six divisions of day and night. (I should rework on it)
%%    
%%    \item[63cd--64]  And the intermittent fever never releases the body since 
%%the
%%    patient does not get relief from heaviness, change of 
%%    color\footnote{
%%        H reads gauravavaivarṇyaṃ kārśyebhyo (I am not sure about the 
%%grammar)
%%    }, and emaciation. Still, when the sudden attack passes
%%    away, it is experienced that it is gone.
%%    
%%    % Tuesday, Nov 8
%%    
%%    \item[65]  When there is little humor, the weak fever ... (not so sure about
%%    this, also could not get the meaning of 3\textsuperscript{rd} and
%%    4\textsuperscript{th} pādas, ipānalaḥ, ipa+analaḥ, what would be ipa?
%%    The vulgate reads iva, I should rework it)
%%    
%%    \item[66]  The little humor, produced by the use of unwholesome things,
%%    generates irregular fever after being located in one of the humors. (the
%%    2\textsuperscript{nd} pāda is not clear to me. I should rework it.)
%%    
%%    % Wednesday, Nov 9
%%    
%%    \item[67--68ab]  The humor located constantly in chyle and blood 
%%generates 
%%the
%%    chronic fever that occurs the other day, based on flesh, produces fever
%%    that occurs every day, that in fat produces every third day. The same
%%    located in bone and marrow generates the fever that occurs every four
%%    days, a very vehement and deadly hybrid of diseases.
%%    
%%   \item[68cd]  
%%    
%%    \item[69]  The fever is said to be continuous (santata) and continues 
%%without 
%%a
%%    break for seven, ten, or twelve days.
%%    
%%    % Thursday, Nov 10
%%    
%%    \item[70--71ab]  In a continuous (satata), there is a two times 
%%temperature 
%%rise
%%    in a day and night; on other days (anyedyuṣka); however, there is a
%%    one-time temperature rise in a day and night; In thrice (tṛtīyaka), the
%%    fever occurs on the third day, and in the case of the fourth, the fever
%%    occurs on the fourth day.
%%    
%%    \item[71cd]  Some say that it is an irregular fever caused by the 
%%possession 
%%of
%%    demons.
%%    
%%   \item[72]  
%%    
%%   \item[73]  
%%    
%%   \item[74]  
%%    
%%   \item[75ab]  
%%    
%%    \item[75cd--76ab]  The fever which generates from different types of 
%%injuries
%%    should be considered according to the humors aggravated.
%%    
%%    % Friday, Nov 11
%%    
%%    \item[76cd--78ab]  The symptoms of fever generated by poison are- a 
%%black 
%%face,
%%    then diarrhea, and then loss of appetite, thirst, sting, with fainting.
%%    The symptoms of fever generated by the smell of herbs are- fainting,
%%    \ldots{} (the text is unclear to me in H). The symptoms of fever, caused
%%    by pleasure (lust?), are loss of consciousness, lassitude, debility, and
%%    loss of appetite.
%%    
%%   \item[78cd]  
%%    
%%    \item[79]  The symptoms of fever generated from fear and grief are 
%%babbling,
%%    and from anger there, arises trembling. From curses and magical spells
%%    arise delirium and thirst.
%%    
%%    % Saturday, Nov 12
%%    
%%    \item[80--81ab]  The symptoms of fever generated by the possession of 
%%demons are
%%    agitation, laughter, tears, and trembling. Exertion, depletion, and
%%    inflection of injury aggravate wind.  The wind permeates the whole body
%%    and causes a severe fever.
%%    
%%    \item[81cd--82]  Moreover, from the rise of diseases, inflammation, 
%%indigestion,
%%    and another type of fever produced by the same or different causes. I
%%    shall describe the forms of it according to the order.
%%    
%%    % Sunday, Nov 13
%%    
%%    \item[83]  The symptoms of fever located in the chyle? (rasa) are 
%%heaviness,
%%    heart pain, exhaustion, vomiting, and appetite loss.
%%    
%%    \item[84]  The symptoms of fever in blood acquired from people are
%%    blood-spitting, burning feelings, delirium, vomiting, confusion,
%%    babbling, boils, and thirst.
%%    
%%    \item[85]  Cramp in the back of the leg, thirst, discharge of urine and 
%%stool,
%%    internal heat, burning feelings,... (vikṣepa) and fatigue of the body
%%    are the symptoms of fever that occurs in the flesh.
%%    
%%    % Monday, Nov 14
%%    
%%    \item[86]  The symptoms of fever that occurs in fat are excessive 
%%sweating,
%%    thirst, fainting, babbling, vomiting, bad smell, loss of appetite,
%%    fatigue, and intolerance.
%%    
%%    \item[87]  The symptoms of fever that occurs in bones are ... (bhedontra? 
%%not
%%    clear to me), rumbling of the bowels, wheezing, purging, vomiting, and
%%    deflection of the limbs.
%%    
%%    % Tuesday, Nov 15
%%    
%%   \item[88--89ab]  The symptoms of fever that occurs in the marrow are 
%%feeling 
%%of
%%    darkness, hiccups, cough, cold, vomiting, thirst, internal burning
%%    feelings, difficulties in breathing\q{a kind of asthma?}, and cutting
%%    pain in lethal points of the body. In fever occurring in semen, one can
%%    get death after that.\q{Not happy with the last part.} Stiffness of the
%%    male organ (penis) and the release of semen excellently lead to 
%%    death.\q{connecting with the previous pāda?}
%%    
%%    \item[89cd--90ab]  As fire is extinguished after burning fuel, and poison
%%    subsides after beating body tissues. After killing the patient, the
%%    fever ceases as if it had attained its object.
%%    
%%    \item[90cd--91ab]  As the symptoms of fevers are caused by wind, bile, 
%%and
%%    phlegm, the physician should tell about them; the wise should even talk
%%    in the case of chyle, etc.
%%    
%%    % Wednesday, Nov 16
%%    
%%    \item[91cd--92ab]  The symptoms of fevers that are even stated in body 
%%tissue
%%    should be specified as caused by all congested humors. And that
%%    generated by two humors combined should be reported by the symptoms 
%%of
%%    the involvement of two humors.
%%    
%%    \item[92cd--93ab]  And severe fever should be understood to be caused 
%%by
%%    internal burning feelings, thirst, suppression of urine and stools,...
%%    \q{(atyartha? excessive?)}, and by the growing of wheezing and cough.
%%    
%%    \item[93cd--94ab]  The fever patient who has lost splendor and sense, has
%%    weakness and loss of appetite, and is afflicted with severe, sharp, and
%%    vehemence sickness should be given up.
%%    
%%    % Thursday, Nov 17
%%    
%%    \item[94cd--95ab]  The vigor of fever over the period of three, seven, and
%%    twelve days...\q{for...dvādaśādikaḥ)? not clear to me, is it
%%    dvādaśādhikaḥ?} should become intense by the arising of the humors that
%%    are inferior, moderate, and superior\q{(any better medical terms for
%%    them?)}. That fever is easily curable in a successive manner.
%%    
%%   \item[95cd--96ab]  
%%    
%%    \item[96cd]  Thus the kinds of fevers are described and, now the 
%%treatment 
%%will
%%    be explained.
%%    
%%    \item[97]  Now, the wise (the physician) should make the patient drunk 
%%with 
%%the
%%    clarified butter in the prodrome of fever, then overspreading the
%%    clarified butter, and easiness will be achieved by the patient. (not so
%%    happy with the translation)
%%    
%%    \item[98]  The prescribed action is just for fever generated by wind; in the
%%    case of fever generated by the bile, mild purge, and in phlegm-generated
%%    fever and two humors combined, vomiting is prescribed.  (here,
%%    dvandvakaphajeṣu is not clear to me)
%%    
%%    % Friday, Nov 18
%%    
%%   \item[99]  
%%    
%%    \item[100]  One should know the variety of the former forms 
%%(prākrūpirūpa, 
%%any
%%    better word?) like the fire of smoke, and fasting is wholesome by all
%%    means in the case of fever that is fully manifested forms.
%%    
%%    \item[101cd--102ab]  As long as the patient is bound with fasting, he is
%%    distressed with immobile humors. So long he should rely on a light diet
%%    as if he was purged.
%%    
%%    \item[102cd--103ab]  Fasting should not be done in the period of fever 
%%generated
%%    by the wind, depletion, and of psychic origin and even in those
%%    mentioned as unfit for fasting earlier in the chapter on treating
%%    twofold wounds. (Ci.Ch. 1).
%%    
%%    % Saturday, Nov 19
%%    
%%    \footnote{Not in vulgate.}
%%    
%%    \item[104A]  When the wind blows, the patient feels hungry and thirsty, 
%%he 
%%is
%%    accompanied by confusion, and his mouth dries up. When the morbid
%%    swellings have just arisen, and the wounds have become severe, even in
%%    the case of feebleness, the child, the old, and the pregnant should not
%%    work. (I should rework this.)
%%    
%%    \item[103cd--104ab]  Fasting in the sick person with unsteady humor and
%%    digestive fire digests humors, destroys fever and ingestion and becomes
%%    instrumental in producing appetite, taste, and lightness.
%%    
%%    \item[104cd--105ab]  One should know the patient who properly fasted 
%%caused the
%%    discharge of wind, flatulence, and urine, is intolerant to hunger and
%%    thirst, ...\q{(since the word lagha is not clear to me)}, cheerful with
%%    their senses, and debilitated.\q{(Not too happy with it.)}
%%    
%%    % Sunday, Nov 20
%%    
%%    \item[105cd--106ab]  Excessive fasting generates depletion of strength, 
%%thirst,
%%    fainting, lassitude, sleepiness, confusion, fatigue, side effects,
%%    wheezing, etc.
%%    
%%    \item[106cd--108ab]  The hot water ingests, destroys phlegm, and takes 
%%...
%%    (varccā?) and bile in their regular order. It is wholesome for thirst
%%    and those suffering from fever generated from phlegm and wind.. Then,
%%    indeed, it softens humors and tubes. The cold water does the opposite.
%%    After that, by being consumed by cold water, the fever increases. (I
%%    should rework it)
%%    
%%    \item[108cd]  In fevers generated by bile, alcohol, and poison, cold water
%%    boiled with tiktaka (I could not get whether it is bitters or any recipe
%%    like Agathotes Chirayta or a kind of Khadira or other since the second
%%    hemistich is not in H), should be applied.
%%    
%%    % Monday, Nov 21
%%    
%%   \item[109ab]  
%%    
%%    \item[109cd--110ab]  Of those patients suffering from fever, rice gruel or 
%%any
%%    drink mixed with a small quantity of boiled rice prepared according to
%%    its digestive is wholesome in a decent hour for eating since it is
%%    indigestion, digestive, and light.
%%    
%%   \item[110cd]  
%%    
%%    \item[111--112ab]  When the humor is not digested by fasting, water and 
%%gruel,
%%    then the patient should be treated with a decoction that is digestive,
%%    cordial, and febrifuge, which removes bad taste in the mouth, thirst,
%%    and appetite-loss and the decoction.
%%    
%%    \item[112cd--113]  The decoction of five roots (pañcamūlī) should be used 
%%as 
%%a
%%    digestive in fever generated by wind.  In the fever generated by bile,
%%    the decoction should be made of nutgrass, sharp root (kaṭukā), and
%%    tellicherry bark (indrayava) \q{(not sure about it)} with honey, while in
%%    phlegm-generated fever, the decoction of longer pepper and others,
%%    should be used as digestive.
%%    
%%    % Tuesday, Nov 22
%%    
%%    \q{(Not in vulgate)}
%%    
%%    \item[113A]  When the humors are gathered together, then the combined 
%%digestive
%%    is wholesome.\q{(I am looking for a better translation)}
%%    
%%   \item[114]  
%%    
%%    \item[115--116ab]  In fever, the humor should be understood as ripened 
%%when the
%%    fever is moderate, body is light and movement in waste products, then
%%    the drug should be given.
%%    
%%    The characteristic of the ripened one is due to the alteration of the
%%    natural disposition of the humors.
%%    
%%   \item[116cd]  
%%    
%%   \item[117]  
%%    
%%   \item[118]  
%%    
%%   \item[119ab]  
%%    
%%    \item[119cd--120ab]  Some people think that the medicine should be 
%%given 
%%after
%%    seven nights. Some people feel confident that the medication is worth
%%    giving after ten nights.
%%    
%%    \item[120cd--121ab]  Or, in the bile-generated fever that has arisen for a 
%%short
%%    time, medicine is given. And the medicine should be given even to a
%%    feverish person with a very long fever when the humor is ripened.
%%    
%%    % Wednesday, Nov 23
%%    
%%   \item[121cd]  
%%    
%%   \item[122]  
%%    
%%   \item[123ab]  
%%    
%%    \item[123cd--124ab]  When the cleansed tube is ripened and approaching 
%%the
%%    humors, purging should be given even to the feverish
%%    who\textquotesingle s got a short-timed fever.\q{(I'd need to rework on
%%    it).}
%%    
%%    (not in vulgate)
%%    
%%    \item[124A--124D]  Discharge of saliva, heart palpitation, oppression of 
%%the
%%    chest (a kind of asthma), impurity and appetite loss, lassitude,
%%    debility, indigestion, bad taste in the mouth, and heaviness of the
%%    limbs, heartburn, frequent urination, stupor, strong fever. These are
%%    the symptoms of unripened fever. And in that situation, the medicine
%%    should not given.\q{(I'd need to rework on it and think about the
%%    sequencing of the number).}
%%    
%%    % Thursday, Nov 24
%%    
%%    \item[121cd--122ab]  Being the medicine of undigested matter of 
%%    humor\q{(āmadoṣa?
%%    Not too sure)}, it illuminates the fever. It also pacifies irregular
%%    fever.\q{(2nd hemistich is incomplete)}
%%    
%%    \item[124cd--125ab]  When the ripened humor (not sure about the reading
%%    pakvopyati) is taken away, and it remains in the body, it causes
%%    delirium (madātyaya) or irregular fever, or even it does fecal
%%    discharge.\q{(not too sure about the meaning of vyapada)}
%%    
%%   \item[125cd]  
%%    
%%    \item[126]  As the preparatory treatment, vomiting and then the enema 
%%with 
%%a
%%    herbal decoction, purgation, and the head\textquotesingle s purging
%%    (śiraś virecanam, better translation??) should be treated.
%%    
%%    % Friday, Nov 25
%%    
%%    (127ab--128ab) Gradually, when the patient is strong, vomiting should be
%%    prescribed in the case of phlegm-generated fever. When the patient is
%%    suffering pain and...\q{not so sure about sodāvarte}, an enema with a
%%    herbal decoction should be offered in the case of wind-generated fever.
%%    
%%    (127cd--128cd) Purgation is prescribed in bile dominance; the head
%%    cleansing by errhines should be done. An oil enema should be applied for
%%    the patient who is afflicted with seizing pain in the waist and the back
%%    part of the body and has splendid digestive fire.
%%    
%%    \item[129]  The weak and little humor patient should be treated with
%%    mid--breath. Head cleansing by errhines should be done when the head is
%%    undertaken by phlegm.
%%    
%%    % Saturday, Nov 26
%%    
%%   \item[130]  
%%    
%%   \item[131]  
%%    
%%   \item[132]  
%%    
%%   \item[133]  
%%    
%%    \item[134cd--135ab]  When the fever has arisen from overeating, and the 
%%patient
%%    is strong, he should be treated with fasting. Watery gruel should be
%%    given to the patient who\textquotesingle s got a slow digestive fire and
%%    is suffering from thirst.
%%    
%%   \item[135cd]  
%%    
%%   \item[136ab]  
%%    
%%    \item[136cd]  Broth and boiled rice are always wholesome when the fever 
%%is
%%    generated by fatigue, fasting, and caused by the wind.
%%    
%%   \item[137ab]  
%%    
%%    \item[137cd]  Boiled rice with soup of green gram should also be given 
%%when 
%%the
%%    fever has arisen from phlegm.
%%    
%%    \item[138]  That same cold mixed with sugar is wholesome in a 
%%bile-generated
%%    fever. And the soup of green gram and emblic is used in wind and
%%    bile-generated fevers.
%%    
%%    \item[139]  When the wind and phlegm are dominant in the fever, the 
%%soup of
%%    small-root radish and the soup of neem and black pepper (kolaka),\q{not 
%%so
%%    sure about it, MW mentions others like Cordia Myxa and Alangium
%%    hexapetalum} is wholesome when the fever is having bile and phlegm.
%%    
%%    % Sunday, Nov 27
%%    
%%    \item[140]  The patient who has got a burning feeling, ... 
%%(yardditam,\q{not 
%%sure
%%    about it}), also is lean, starving, and thirsty should be given to drink
%%    the satiating food of parched grain mixed with sugar and honey.
%%    
%%    \item[141--142]  The gruel is not wholesome for a patient with phlegm 
%%and 
%%bile
%%    dominance, when it is summertime, and when the blood bile is high, and
%%    who is a regular wine drinker, they should be treated with soups of sour
%%    or not sour fruits and meat soup of wild animals. Wine should be given
%%    to the patient who has wine-friendly health or can digest\q{(sāmāhāya- 
%%any
%%    better word?)} it.
%%    
%%    % Monday, Nov 28
%%    
%%    (143 \& 144cd) In that case, a mixture of black pepper, long pepper, and
%%    dried ginger should be given when the patient suffers from phlegm and
%%    appetite loss. A person (patient) suffering from a chronic fever and
%%    also (unclear after vaddha pra...) is tied with ... humors, suffering
%%    from thirst, or having a burning feeling, gets relief by taking milk.
%%    
%%   \item[144ab]  
%%    
%%    \item[145--146ab]  However, drinking milk kills a human being (patient) 
%%when he
%%    is young. A very light and moderate amount of food is wholesome in the
%%    case of all types of fevers when the shock of fever goes away.
%%    Otherwise, it increases the shock of fever.
%%    
%%    \item[146cd--147ab]  When the patient has a fever, he should eat 
%%wholesome food,
%%    even if he has appetite loss. If the patient does not eat at the time of
%%    food, he becomes depleted or dies.
%%    
%%    % Tuesday, Nov 29
%%    
%%    \item[148cd--149ab]  
%%    
%%    The feverish should not eat heavy food.  Unwholesome food eaten even at
%%a laxative time (abhiṣyandikāle??) is unsuitable for longevity or
%%healthy happiness.\q{Not so happy with this translation}
%%    
%%    \item[149cd--150ab]  A weak patient with continuous, irregular, and
%%    long-standing fever should be treated with myrobalan and wholesome
%%    foods.
%%    
%%    \item[150cd--151ab]  The patient should be given green grams, lentils,
%%    chick-pea, horse grams, and mat beans at meal time to prepare the soup.
%%    
%%    % Wednesday, Nov 30
%%    
%%    \item[151cd--152]  When the fever patient is vegetarian, the following 
%%vegetable
%%    should be given: the snake gourd leaf, the eggplant (vārtāku?),
%%    kukkula?? pāpacelaka (tried to look into Nadakarni, with no luck),
%%    galls, Indian fumitory (parpaṭaka??), prickly-leaved elephant's foot
%%    (gojihvam), small immature radish (bālamūlaka??), leaf of heart-leaved
%%    moonseed.
%%    
%%    \item[153--154ab]  When the fever patient is non-vegetarian, and the 
%%meat is
%%    wholesome to them, he should be given the meat of bustard-quail, grey
%%    partridge, Indian antelope, chital deer, camel, hare, black-tailed
%%    marshy sparrows (kālapuccha, not so sure about it), roe-deer
%%    (kuraṅga??), and then hog-deer (mṛgamātṛka??).
%%    
%%    \item[154cd--155ab]  Some ... (vyavasthitāḥ??) do not recommend Indian 
%%cranes
%%    (sārasa), common cranes (krauñca), peacocks, chickens, and partridges
%%    because of their heaviness and hotness.
%%    
%%    % Thursday, Dec 1
%%    
%%    \item[155cd--156ab]  But, when the winds aggravates of the fever 
%%patients, 
%%then
%%    they too are recommended for the good quantity and time.
%%    
%%    156cd, 157 missing, 157A is not in the vulgate edition, \item[158]  The 
%%patient
%%    suffering from juvenile fever should give up showering, immersing,
%%    oiling, evacuant, and foods that are hard to digest, day dreaming,
%%    sexual intercourse, physical exercise and winter water (not in vulgate)
%%    and now at this time, in summary, he should give up new grains and anger
%%    as well.
%%    
%%    159A (not in vulgate). The patient experiences phthisis(śoṣa? it is
%%    correct, what about dryness?), vomiting, intoxication, fainting,
%%    delusion, and appetite loss as a side effect of consuming them with
%%    taking a shower and others.
%%    
%%    % Friday, Dec 2
%%    
%%    \item[159]  By the unsettled humors and digestive fire, the fever ...
%%    (sandhukṣita??) and becomes profound, sharp, and shock and even
%%    incurable.\q{( Not happy with it)}
%%    
%%   \item[160]  
%%    
%%    \item[161]  Even recovered from the fever, indeed (not sure about the 
%%meaning
%%    of ``ha'' in this context), when regained, the fever getting from the
%%    unwholesome things in a weak patient cauterizes the body as fire burns a
%%    dry wood into ashes.
%%    
%%    \item[162]  There is no action to avoid when the body is released from 
%%fever
%%    unless the patient has a natural disposition regarding humor and life.
%%    
%%    % Saturday, Dec 3
%%    
%%    \item[163]  Fainting in fever patients is caused even by a minute action in
%%    life. He should be fed while sitting on the bed and therefore made to
%%    urinate and stool.
%%    
%%    \item[164]  When there is appetite loss, body exhaustion, change in 
%%complexion,
%%    impurity in the parts of the body, the patient whose fever is
%%    alleviated, the body should be cleansed ...\q{(the second hemistich is
%%    incomplete)}.
%%    
%%    % Sunday, Dec 4
%%    
%%    \item[165]  The intelligent physician should prevent the patient affected 
%%by
%%    fever from going to sleep too soon. He may have a fever again when it is
%%    cured.
%%    
%%    should prevent the patient affected (?) by fever from going to sleep too
%%    soon
%%    
%%    \item[166]  One should treat all fevers with the opposite of to cause. One
%%    should treat the root disease when the fever has arisen from hard work,
%%    depletion, and infliction of injury.
%%    
%%   \item[167]  
%%    
%%    \item[168]  Therefore, listen to me for the pacifying decoctions that the
%%    learned physician should give in all kinds of fevers.
%%    
%%    % Monday, Dec 5
%%    
%%    \item[169--170ab]  The decoction prepared from long pepper, Indian
%%    sarsaparillas, grapes, dill, and black cardamom and mixed with jaggery
%%    beats the fever generated by wind.  Then, a drink of boiled milk or the
%%    cold decoction of heart-leaved moonseed should be offered.\q{can śṛta
%%    mean here boiled milk? Not happy with the last part}
%%    
%%    \item[170cd--171ab]  Ritual grass, country mallow, devil's weed... \q{the 
%%rest of
%%    the text is unclear to me}. The patient should drink a mixture of sugar
%%    and clarified butter that destroys the fever.
%%    
%%   \item[171cd]  
%%    
%%   \item[172]  
%%    
%%   \item[173]  
%%    
%%   \item[174]  
%%    
%%   \item[175ab]  
%%    
%%    \item[175cd]  The patient should use saunas ointment (śvedālepaḥ??) and
%%    clarified butter massage to the body even in unfavorable situations 
%%\q{(not
%%    so sure about it). {[} ghṛtābhyaṅgonavasthāsu should it be like
%%    ghṛtābhyaṅgo `navasthāsu?, svedā lepaḥ ghṛtābhyaṅgonavasthāsu ca
%%    yojayet{]} (Not so happy with the translation)}
%%    
%    
\end{translation}