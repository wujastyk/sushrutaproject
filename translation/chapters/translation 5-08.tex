% !TeX root = ../incremental_SS_Translation.tex
\chapter{Kalpasthāna 8: Poisonous insects}

\section{Introduction} 

This is the last chapter of the \emph{Kalpasthāna}.  Since the
chapter-colophons of the Nepalese manuscripts commonly end with the
statement, “here ends the \SS\ together with the Uttaratantra,” we can
presume that an older version of the \SS\ ended with the present
chapter.  Added to this, the beginning of the next section of the
work, the Uttaratantra, reads “It being declared in the preceding 120
chapters, from here on, in the latter section, I shall explain the
meanings in detail, fully.\footnote{Note that this is not the reading
    of the vulgate, which says that the Uttaratantra will explain
    everything that was \emph{not} completely explained before.}  Now, I
    shall explain the treatise called “the latter” where diseases
    in their diversity are fully revealed."
    
    

\subsection{Literature} 
A brief survey of this chapter's contents and a detailed assessment of
the existing research on it to 2002 was provided by
Meulenbeld.\footcite[IA, 296--299]{meul-hist} 


\section{Translation}



\begin{translation}

\item[1]
 And now I shall explain the \se{kalpa}{procedure} about 
 insects\sse{kīṭa}{insect}.
 
 


\subsection{Taxonomy of insects}

%\item[3--17ab] 

\item [3] 

Insects originate from the semen, feces, urine, the rot of corpses,
and eggs \diff{of snakes}.  Their
\sse{prakṛti}{character}characters are traditionally divided
into \diff{three}: wind, fire, and water.

\item[4]

Yet others hold the opinion that they are connected with the
\sse{prakṛti}{character}characters of all of the humours. And those
insects are also very fierce and all of them are divided into four
groups:

\item[5]
\begin{enumerate}
    \item \se{uṇḍunābha}{Uṇḍu-navels},
    \item \se{tuṇḍikerī}{snouted},
    \item \se{śriṅgī}{horned},
    \item \se{śatakulimbhaka}{hundred-kulimbhaka}.    
\end{enumerate}

\subsection{Symptoms}

\item[17cd--24] xx

\subsection{Taxonomy according to symptoms and prognosis}

\item[25--27] xx

\item [28]  \gls{godheraka}    \label{godheraka}
    
\item [29] \footnote{See n.\,\ref{galagodika}, 
p.\,\pageref{galagodika}.}


\item[30--41] xx

\subsection{Therapies}

\item[42--56abcd] xx
 
\subsection{Taxonomy of scorpions}
 
 \item [56ef--66] xx
 
 \subsection{Therapies for scorpion-sting}
 
 \item[67--74] xx
 
 \subsection{Symptoms of spider poisoning}
 
 \item[75--89] xx
 
 \subsection{Origin story for spiders}
 
 \item[90--93] xx
 
 \subsection{Taxonomy of spiders}
 
 \item[94--100ab] xx
 
 \subsection{Specific symptoms and treatment for spider poisoning}
 
 \item[100cd--120] XX
 
 \subsection{Untreatable spider poisons}
 
 \item [121--127] xx
 
 \subsection{Curable and incurable}
 
 \item[128--129] xx
 
 \subsection{Therapies for spider poisoning}
 
 \item [130--134] xx
 
\subsection{General therapies for poisoning}

\item [135--139] xx

\subsection{End of the Suśrutasaṃhitā}

\item[140--143] xx
 
\end{translation}