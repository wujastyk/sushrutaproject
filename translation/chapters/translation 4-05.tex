% !TeX root = incremental_SS_Translation.tex
% Paras

\chapter{Cikitsāsthāna 5:  On the Treatment of Serious Wind 
Diseases}

\section{Literature} 

Meulenbeld offered an annotated overview of this chapter and a bibliography
of earlier scholarship to 2002.\fvolcite{IA}[266]{meul-hist} 

\section{Translation}

\begin{translation}
    
    \item [1]
    Now we shall describe the treatment of serious wind diseases.
    
    \item [2]

    \item [3]
    One group says that the blood afflicted by wind (wind-blood) (\textit{vāta-rakta}) is of two types: spreading out over a surface (\dev{uttāna}) and deep (\dev{avagāḍha}).\footnote{Ḍalhaṇa comments \citep[424]{vulgate} that \dev{uttāna} refers to being situated in the skin and flesh, and \dev{avagāḍha} refers to being situated internally.} However, this is not correct.\footnote{In H, the word \dev{tan} should be \dev{tat}.} Why? Just as leprosy, after spreading over a surface it (afflicted blood) becomes deeply situated. Therefore, its being of two different types is refuted.  

    \item[4]
    When the wind is aggravated by fighting a strong person, etc.\footnote{These factors that aggravate the wind are mentioned in \textit{Nidānasthāna}, Ch. 12, text 6.}, one's corrupted blood caused by eating heavy or hot food before the last meal is digested blocks the path of the aggravated wind. It then combines with the wind and simultaneously creates pain due to the wind-blood. This [condition] is called wind-blood (\textit{vāta-śoṇita}). At first, it is situated in the hands and feet.\footnote{In H, the word \dev{tan} should be \dev{tat}.} Later, it spreads throughout the body. Its early forms are pricking pain, burning, itching, ulcer, trembling\footnote{In H, there should not have been the \dev{s} after \dev{stambha}.}, roughness of the skin, pulsation in the blood vessels, tendons, and tubular vessels\footnote{In addition to blood vessels, it would also include the nerves.}, weakness of the thighs, as well as the sudden appearance of dark brown, tawny, or red spots on the soles of the feet, fingers, ankles, and wrists. The disease becomes fully manifest in the person who does not undertake the means to revert the disease or applies a wrong treatment. Its symptoms have been mentioned. Among them, weakness occurs for the one who does not counter the disease.

    \item[5]
    Generally, wind-blood occurs in those who are very delicate, those who eat the wrong foods and enjoy improperly, those who are fat, and even in those who indulge in pleasure.  

    \item[6]
    In that regard, one should treat the patient who is not degenerating due to wasting of life air, thirst, fever, unconsciousness, dyspnea, trembling, and loss of appetite, is not oppressed by the contraction [of limbs], is strong, composed, and has the means.

    \item[7]
    In the treatment, at the beginning itself one should do blood-letting of the wind-affected body part little by little and more than once. That (slow blood-letting) is because of the danger of further aggravation of wind. One should avoid doing blood-letting of the part hardened or weakened by excessive wind.\footnote{In H, the reading \dev{amlāna} does not make sense given the context. Therefore, we have accepted the vulgate reading \dev{mlāna} for the translation.} Thereafter, one should make the patient do the remedies of vomiting, etc. If the wind that is mixed [with blood] or separated is very aggravated then one should make him consume aged ghee or goat-milk. Or, [one can give him] half a measure of oil added with an \textit{akṣa} of \gls{madhuka} and cooked with \gls{pṛśniparṇī}\footnote{Ḍalhaṇa glosses \citep[425]{vulgate} \emph{śṛgālavinnā} as \emph{pṛśniparṇī}.}, or the oil that is sweetened by sugar and honey and cooked with \gls{śuṇṭhī} and \gls{kaśeru}. Or, one should boil milk with an eight times volume of the decoction of the following herbs: \gls{śyāmā}, \gls{rāsnā}, \gls{suṣavī}, \gls{pṛśniparṇī}\footnote{According to Ḍalhaṇa, \emph{śṛgālavinnā} is \emph{pṛśniparṇī}.}, \gls{pīlu}, \gls{śatāvarī}, \gls{śvadaṃṣṭrā}, and \gls{dvipañcamūla}. This milk should then be used to cook oil with the admixture of pastes of \gls{meṣaśṛṅgī}, \gls{śvadaṃṣṭrā}, \gls{madhu}, \gls{nāgabalā}, \gls{bhadradāru}, \gls{vacā}, and \gls{surabhi}. This (resultant) should be utilised in drinks, etc. Or, one should use the oil that is cooked with a decoction of \gls{śatāvarī}, \gls{apāmārga}\footnote{Ḍalhaṇa glosses \citep[425]{vulgate} \textit{mayūraka} as \textit{apāmārga}.}, \gls{dhavaka}, \gls{madhuka}, \gls{kṣīravidārī}, \gls{balā}, \gls{atibalā}, and \gls{tṛṇapañcamūlī}\footnote{Ḍalhaṇa comments \citep[425]{vulgate} that \gls{kuśa}, \gls{kāśa}, \gls{nala}, \gls{darbha}, \gls{kāṇḍa}, and \gls{ikṣuka} are called \textit{tṛna} (grass).}, with the admixture of \gls{kākolī}, etc. Or, one should use the \gls{balā}-oil that is cooked as \emph{śatapāka}.\footnote{\emph{Śatapāka} seems to be an oil that is prepared with a hundred parts of some things similar to \textit{sahasrapāka} that is prepared with one thousand parts of some herbs. Refer \textit{Cikitsāsthāna} Ch. 4 text 29 for the preparation of \textit{sahasrapāka}.} Or, [the affected body part] should be moistened with milk that is boiled with the roots of wind-alleviating herbs, or it should be moistened with sour things.\footnote{Ḍalhaṇa comments \citep[425]{vulgate} that the sour things (\textit{amla)} are \gls{surā}, \gls{sauvīraka}, \gls{tuṣa}-water, etc. \textit{Surā} is some kind of liquor, \textit{sauvīraka} is perhaps the fruit of the jujube tree, and \textit{tuṣa} is perhaps Terminalia Bellerica (\dev{vibhītaka}).} In that regard, five remedies prepared with milk are described. For preparing a poultice, milk should be cooked in ghee, oil, fat, marrow, and \emph{dugdha}\footnote{In the \SS, the word for milk is \textit{kṣīra} or \textit{payas} but not \textit{dugdha}. Therefore, the word \textit{dugdha} here can mean the sap of plants or something that is extracted.} separately with each of these powdered grains or pulses---barley, wheat, sesame, \gls{mudga}, or \gls{māṣa}---that is mixed with unctuous pastes of \gls{kākolī}, \gls{kṣīrakākolī}, \gls{jīvaka}, \gls{ṛṣabhaka}, \gls{balā}, \gls{atibalā}, \gls{pṛśniparṇī}\footnote{\emph{śṛgālavinnā}}, \gls{meṣaśṛṅgī}, \gls{piyāla}, sugar, \gls{kaśeru}\footnote{For \emph{kaśerukā}}, \gls{surabhi}, and \gls{vacā}. Or, the essence of unctuous fruits\footnote{Ḍalhaṇa comments \citep[425]{vulgate} that the unctuous fruits mentioned here are sesame, castor, \gls{atasī}, \gls{vibhītaka}, etc.} can be used as a poultice. Or, a \textit{veśavāra}\footnote{In H, the reading \dev{vaiśavāro} does not make sense. It should have been \dev{veśavāro}, as shown in the vulgate, which is the reading we have accepted here.\\ \textit{Veśavāra} is boneless meat minced, steamed, and added with spices, ghee, etc. Refer to 'Ayurveda Medical Dictionary' by Ranganayakulu Potturu.} prepared from the flesh of a fat \textit{cilicima} fish\footnote{H has the compound word \dev{nalapīnamatsya}. \dev{nalamīna} is a particular fish known as \textit{cilicima} (\dev{cilicimaḥ}). See \textit{Amarakośa}. Also, if the name is \dev{nalamatsya} then the word \dev{pīna} (fat) within the name is not according to proper Sanskrit. But, it can be allowed because the word \dev{matsya} (fish), instead of being a part of the name, can be considered to mean fish in general and thus the word \dev{pīna} becomes its modifier. Thus, \dev{nalapīnamatsya} can mean "a fat fish that is a \dev{nala} (\textit{cilicima})".\\ Ḍalhaṇa says in his comment \citep[425]{vulgate} that \dev{nalamīna} is a type of \dev{rohita} (\textit{rohita}). Monier Williams says that \textit{rohita} is a kind of fish: Cyprinus Rohitaka. Regarding the \textit{rohita} fish, there is a \textit{subhāṣita}: \dev{agādhajalasañcārī na garvaṃ yāti rohitaḥ | aṅguṣṭhodakamātreṇa śapharī pharpharāyate ||} This indicates that \textit{rohita} is a deep water fish.}\q{The webpage https://hindi.shabd.in/vairagya-shatakam-bhag-acharya-arjun-tiwari/post/117629 says that this verse belongs to the \textit{Nītiratna}. I could not find this text.} can be used instead. Or, [one can use] the poultice containing \gls{bilva}-rind\footnote{The word \dev{pesikā} in H should be read \dev{peśikā}.}, \gls{tagara}, \gls{devadāru}, \gls{saralā}, \gls{rāsnā}, \gls{hareṇu}, \gls{kuṣṭha}, \gls{śatapuṣpa}, liquor, yogurt, and whey. Or, [one can use] the ointment prepared by mixing \gls{mātuluṅga}, \emph{amla}\footnote{Perhaps it could mean vinegar or sour curds. Refer to Monier Williams Sanskrit Dictionary.}, salt, and ghee with honey and \gls{śigru}-root. Or else, [one can use] the unctuous sesame paste. 

    \item[8]
    When the [condition of wind-blood] has a predominance of bile, the patient should be made to drink a decoction of grapes, \gls{revataka}-fruit, \gls{payasyā}, \gls{madhuka}, \gls{candana}, and \gls{kāśmarī}. This decoction is sweetened with honey and sugar before consumption. Or, the decoction of \gls{śatāvarī}, \gls{paṭola}, \gls{patra}, \textit{triphalā}, \gls{kaḍurohiṇī}, and \gls{guḍūcī} should be given. [The patient should be administered] ghee that is prepared with sweet, bitter, and astringent [remedies].\footnote{Ḍalhaṇa comments \citep[425]{vulgate} that the sweet remedies are \gls{kākolī}, etc., bitter remedies are \gls{paṭola}, etc., and astringent remedies are \textit{triphalā}, etc.} 
    
    [The patient] should be sprinkled with a decoction of \gls{bisa}, \gls{mṛṇāla}, \gls{bhadraśriya}, and \gls{padmaka} mixed with goat-milk\footnote{The compound word ending with \dev{kaṣāyeṇa} is taken to be a \textit{bahuvrīhi} for \dev{ajākṣīreṇa} (goat-milk).}, or with rice water that is mixed with milk, sugarcane juice, honey, and sugar, or with whey and sour rice gruel mixed with a decoction of grapes and sugarcane. Or else, [the patient] should be sprinkled with ghee that is prepared with \textit{jīvanīya}\footnote{\textit{Jīvanīya} seems to be a group of medicinal herbs. There is an Ayurvedic preparation called \textit{jīvanīya-ghṛta}. Refer to the \textit{Āyurvedīya Śabdakośa} vol. 1.} or sprinkled with ghee that is purified for one hundred times.

    The poultice [to be applied] should be made of rice flour or of the paste of sour rice gruel mixed with \gls{nala}, \gls{vañjula}, \gls{tālīśa}\footnote{\dev{tālīsa} should be read \dev{tālīśa}}, \gls{śṛṅgāṭaka}, \gls{kālodyā}, \gls{gaurī}, \gls{śaivāla}, \gls{padma}, etc. The poultice should be mixed with ghee.

    \item[9]
     The [condition of wind-blood] with a predominance of blood should be treated in the same way. Also, blood-letting should be done repeatedly.

     \item[10]
    However, when the [condition of wind-blood] has a predominance of phlegm, the patient should be made to consume a decoction of \gls{āmalaka} and \gls{haridrā} that is sweetened with honey, or a decoction of \textit{triphalā}, or a paste of \gls{madhuka}, \gls{śṛṅgavera}, \gls{harītakī}, and \gls{tiktarohiṇī}. He should be made to drink \gls{harītakī} with water mixed with a little urine. He should be sprinkled with oil, urine, salty water, and liquor that are acidic\footnote{Reading the word \dev{sukta} in H as \dev{śukta}}. Or, he should be sprinkled with a decoction of \gls{āragvadha}, etc. 

    The patient should be massaged with ghee cooked with sour cream, urine, liquor, \gls{śuktā}\footnote{Monier Williams states Rumex Vesicarius for \textit{śuktā}}, \gls{madhuka}, \gls{śārivā}\footnote{DCS has this entry: Cryptolepsis buchananii Roem. et Schult. (Surapāla (1988), 453) Decalepis hamiltonii Wight et Arn. (Surapāla (1988), 453)}, and \gls{padmaka}.
    
     The poultice should be made of either the paste of white mustard, or the paste of sesame and \gls{aśvagandhā}, or the paste of \gls{dāru}\footnote{According to V.S. Apte, \dev{dāru} can mean \dev{devadāru}}, \gls{śelu}, and \gls{kapittha}, or the paste of honey, \gls{śigru}, and \gls{punarnavā},\footnote{H has a short \dev{a} instead of the long \dev{ā}.} or the paste of dry ginger, long pepper, black pepper,\footnote{\dev{vyoṣatiktā} refers to the group of these three pungent spices.} \gls{pṛthakparṇī}, and \gls{bṛhatī}.\footnote{In H, the Sanskrit syntax does not match up with what the author is trying to say. The name of the fifth paste should also have been in the nominative case, as the other four pastes.}\q{The provisional edition should be modified accordingly.} These five poultices are prepared with salty water. Thus, they have been described.

     \item[11] 
    
\end{translation}
