% !TeX root = incremental_SS_Translation.tex

\chapter{Śārīrasthāna 3:  On Conception and the Development of the 
Embryo}

%First draft, by Jan Gerris, 2023-12-19. 

\section{Literature} 

Meulenbeld offered an annotated overview of this chapter and a
bibliography of earlier scholarship to
2002.\fvolcite{IA}[247--247]{meul-hist}  Important subsequent 
studies of the chapter include those of \citeauthor{das-2003} and of
\citeauthor{krit-2009}.\footcites[ch.\,8, \emph{et 
passim}]{das-2003}{krit-2009,krit-2013}.

\newpage
\section{Translation}


\begin{translation}


\item[1] 

We shall now explain the anatomy that is the descent of the embryo. 

%We are now about to begin to explain how the embryo is conceived, 
%nestles and develops* once it arrives in the body.

\item[3]

Semen is \se{saumya}{of the nature of Soma} and menstrual blood is
\se{āgneya}{of the nature of Agni}.\footnote{On the Saumya--Agni
    classification, see \cites{wuja-2004}{ange-2021}[521--527]{das-2003}. 
    The fiery nature of menstrual blood is already stated in
    \Su{1.14.7}{59}, “\ldots but menstrual blood is of the nature of
    Agni, because the embryo is of the nature of fire and water.”} 
Furthermore, in this context there also exists a proximity of the
other elements (\emph{bhūta})\sse{bhūta}{element}, by way of a minute
special property\sse{aṇu}{minute}\sse{viśeṣa}{special property},
because they help one another and they enter into one
another.\footnote{\Dalhana{3.3.3}{350} glossed \dev{aṇunā viśeṣeṇa}
    “by way of a minute special property” as \dev{sūkṣmaprakāreṇa} “in an
    attenuated manner.”
        
        \Dalhana{3.3.3}{350} drew attention
        to \Su{3.1.21ab}{343} where the idea of this interpenetration
        (\dev{anupraveśa}) is mentioned.}


\item [4]

In this case,
when there is a union of \diff{a husband and wife},
the heat from the body stimulates the wind.

In that case, 
because of the \se{sannipāta}{combination} of fire and wind,
the semen that is ejaculated finds its way to the vagina. 

It is commingled with \se{ārtava}{menstrual blood}, then because
of the joining together of Agni and Soma, what is being mingled together
arrives in the receptacle of the fetus. 

He is referred to by names that express synonyms such as, the knower
of the field, the sentient, the toucher, the smeller, the seer, the
hearer, the taster, the human, the goer, the witness, the creator, the
speaker, \diff{the one who is, “who is the one that is life at the
    start?”}\footnote{The last phrase is awkward.  It translates \dev{yaḥ
    ko'sāvādya āyuriti}, which could be paraphrased, “the one who is the
    answer to the question `who is the one who is life at the outset?'” or
    “\ldots `who is that first one who is life?'.” The text differs from
    he vulgate's \dev{yaḥ ko 'sāv iti}, that omits \dev{ādya āyur}
    (\Su{3.3.4}{350}). Most other early editions print \dev{yo'sāviti}
    (e.g., \cites[v.\,1, 320]{gupt-1835}[313]{bhat-1889}[v.\,3,
    30]{bhat-1908}[v.\,2, 635]{sarm-1895}. \citet[v.\,2, 65]{ghan-1936}
    read \dev{yaḥ ko'sāvity}). No other translators translate this
    phrase, nor does Ḍalhaṇa gloss it.} %
    %    vaktā yo'sāv ity evam - Madhusūdana Gupta 1835
    %    vaktā yo'sāv ity evam - Jīvānandavidyāsāgara
    %    vaktā yo'sāv ity evam - Candrākānta Bhattacharyya p.757
    %    vaktā yaḥ ko'sāv ity evam  - Ghanekar 1936
    %    vaktā yo sāv ity evam - Muralidhara Sarma
    %



%Sperm from the male absorbs heat whereas eggs from the female 
%release heat. With respect to this aspect, the way the different basic 
%elements of 
%matter behave depends on how the elements specifically react with one 
%another 
%and how they form bonds with one another.  


Driven by fate, and impelled by wind, the imperishable, unchanging,
inconceivable \se{bhūtātman}{elemental self} enters into the
\se{garbhāśaya}{uterus} together with sattva, rajas and tamas, gods
and demons, and other entities.\footnote{In the vulgate,
    \dev{bhūtātman} “elemental self” is not the subject of the sentence,
    which then reads less clearly overall.}

\item [5]

In that context, a predominance of sperm leads to a male, a
predominance of menstrual blood leads to a female, and equality of the
two leads to a person who is \se{napuṃsaka}{neither male nor female}.

\item[6ab]

In that context, there is a twelve-night period of the \se{ṛtu}{season}.

\item[3.3.6.1]

$\dag$Approaching a woman in season for intercourse during that first
day, a man becomes \se{anāyuṣya}{devoid of long life}.\footnote{This
    passage appears in the Nepalese version at this point, and is absent
    from the vulgate version.  MS H is the sole witness to the Nepalese
    version at this point and it is damaged, making the interpretation of this
    passage difficult.  In this sentence, a nominative would read
    better than the accusative \dev{anāyuṣyam}.} %
    To the extent that the fetus is deposited at that time, because of
    being expelled it is lost.\footnote{In this and the following sentences, parts of 
    witness H are damaged and impossible to read.} $\dag$
    
    And on the third day, similarly, the body is incomplete and has little duration of 
    life.  For that reason, one should avoid the third night. 
    And seed and menses do not develop the proper quality as expected.     
   $\dag$Just as an object thrown into a river against the flow does not come 
   back.$\dag$  Sperm should be seen the same way.  Therefore the restricted third 
   night should be avoided.  In this context, after seeing the twelve nights of the 
   season, she has no menses. 

\item [6cd]

Some call such women, “having invisible menses."

\item [3.3.9]

And on this:
\begin{quote}
    When the day is over, the lotus inevitably closes.  In the same
way, when the season is over, the woman's uterus
closes.\footnote{The \root \emph{kuc} “close, contract” appears in
    this sense in the \emph{Dhātupāṭha} (1.199 \dev{saṃkocane}) but it
    is not common in literature.  The more common word in this sense
    would be from \root \emph{kuñc} “contract,” although \emph{kuc} is
    the primary IE form \pvolcite{1}[361]{EWA}.}
\end{quote}

% got to here

\bigskip

%\begin{tt}
%    \raggedright
%
%
%\item[6]A boy is conceived when on the first day of the period of twelve days of 
%the cycle* the desire for sexual intercourse is not endlessly postponed. It should 
%not be disregarded that a woman who is definitely pregnant may  suffer a 
%miscarriage; a second pregnancy can miscary as well and even in a third 
%gestation, the body can be incomplete either in form or in number of limbs, and 
%both the strength and the life expectancy can be limited. This is the reason why 
%one should avoid three-nightly intervals. There are also patients who do not 
%exhibit produce menstrual periods or have no sperm production and who do not 
%return to normality.* For that reason, if sperm production has to be observed, a 
%man should avoid the habit of three-nightly advances. In such cases, even after 
%having observed there periods of twelve nights, yet there is no ovulation 
%proper.** Some state that these are amenorrhoeic.*** 
%
%\item[7]Here are some more verses.
%
%\item[8] Lacking
%
%\item[9]As surely as by rule of nature the night-lotus folds its leaves, so truly a 
%woman’s yoni by law of nature is also closing*.
%
%\item[9A]The face of a woman becomes swollen, lively and because of 
%transudation moist like that of an elephant, she longs for intimate contact with a 
%man, talks sweetly, her belly drooping and her head let down/uncared for,…
%
%\item[9B]… her arms, breasts, hips, loins, thighs, her abdomen around the navel, 
%her bottom and buttocks, all are trembling. And she experiences intense 
%happiness and satisfaction, you can tell her a woman after her courses. 
%
%\item[10]The Vāyu then guides the mentrual discharge that comes after being 
%heaped up for a month through the two channels towards the opening of the yoni.
%
%\item[11]Menstruation becomes a regular feat from twelve years onwards and 
%owing to the natural decay of functions it ends from about fifty years onwards.
%
%\item[12]So, if a man desires children, he should have intercourse with his wife 
%during the fertile period of the cycle* and for that particular purpose he should 
%visit her on even days in order to beget a boy and on uneven days for a girl.**
%
%\item[13]In this context, fatigue, lassitude, thirst, a feeling of exhaustion in the 
%thighs, flatulence, an arrest of the menses – and of sperm from the yoni* - with a 
%sensation of shaking heat all suggest that a pregnancy has been obtained very 
%recently.
%
%\item[14]Here are some more verses. It is claimed that a typical early sign of 
%pregnancy is the nipples turning darker*, the appearance, on the midline of the 
%abdomen, of a coloured stripe, (resembling hair)** and sudden vomiting.
%
%\item[15] Lacking
%
%\item[16]From the very beginning of pregnancy the woman should avoid sexual 
%intercourse, exertion, excessive exercise, sleeping by day and waking at night, 
%being terrified, sitting for too long in one position, being all alone, Sneha-krama 
%and other treatments as well as blood-letting at an inappropriate time.
%
%
%\item[17] Lacking
%
%\item[18]So then, in the first month a kalala arises. In the second month a ghana 
%develops that has arisen thanks to blood, ritual oblations and by wind and has 
%become mature with the five essential elements. If there is a lump-like structure, 
%it will be a male. If the structure is oblong or peśī, a girl; if there is a bud-shaped 
%structure  or arbbuda, an individual with undifferentiated external sexual 
%features.* In the third month five protrusions (of hands, feet and head) result 
%from the process of development.  All limbs and all minor body parts become 
%distinguishable (though still) very minute. In the fourth month all limbs and 
%minor 
%body parts become manifest. In the fifth month all limbs and minor body parts 
%become even more individualized. Owing to the formation of an individualized 
%fetal heart, consciousness becomes a distinct separate constituent which is why 
%during the fourth month, that foetus, from the appearance of that organ onwards, 
%forms desires from (all five) objects of sense. Henceforth the lady becomes the 
%double-hearted (or pregnant) one and she makes her desires known.  The 
%two-hearted/pregnant one, (if) disrespected, causes a child to be born who is  
%……………………………….. (kukukūniṃṣaṇṛm),  dwarfish, with eye defects, blind, 
%……………………….. (vānārīsutam). That on which account she desires 
%(something) is also that by means of which she can be gratified. Having obtained 
%(to be) pregnant, she causes a son to be born who is really strong and has a long 
%life expectancy. 
%
%\item[19]And here are some more verses. Indeed that pregnant woman desires 
%…………… (bhoktum) the objects of the senses during the course (of her 
%pregnancy); for fear of injury to the foetus a physician, after having fetched these 
%things, should give any desired object.
%
%\item[20]She should give birth to a son endowed with virtues; if the pregnant 
%woman does not obtain (what she desires), he (the foetus?) (or she, the woman?) 
%also becomes equally insecure him-/her-self.
%
%\item[21]With respect to all those desires of the senses in which the pregnant 
%mother was slighted, she will give birth to a son who is defective in each of all 
%those same corresponding senses.
%
%\item[22]A king in an interview with whom a woman during her pregnancy wins 
%and she gives birth to a son who is wealthy and is highly fortunate.*
%
%\item[23]A pregnant woman, dressed in fine cloth, wearing silk and other things, 
%gives birth to a charming son decorated (alaṇkā) …………. reṣiṇaṃ
%
%
%\item[24]If (she goes) to a hermitage, she brings forth someone who is 
%self-restrained and a stone-pillar of religion, resembling a godhead and begotten 
%in the utmost happiness. Upon seeing someone in a high position designed by 
%birth, she gives birth to a stone-pillar of violence.
%
%\item[25]If she feels like eating the flesh of an Iguana (she produces) a son who 
%is drowsy and who has the nature of a killer; by means of beef meat a son who is 
%wild and who is powerful because he is savage in everything.
%
%\item[26] When from the pregnant woman (there is a wish for meat of) buffalo a 
%son is produced who has fearful red-eyes and who looks shaggy.
%
%\item[27]Lacking
%
%\item[28]Hence, she who during her pregnancy considers what people eat, 
%wishes for her offspring the same via the food habits of the body.
%
%\item[29] And that which has yet to happen again when the child is growing up, 
%should be such that through divine intervention the pregnant woman should 
%produce it during her pregnancy. 
%
%\item[30]In the fifth (month of pregnancy) the mindbecomes more and more 
%awakened; in the sixth intelligence (becomes awakened); in the seventh all the 
%limbs and smaller body parts (are in place); if in the eighth (month) the ojas is 
%not stable in that case the child does not live* - he is provided with a share (of it) 
%by the demons– so then strong excellent meat should be provided to him; if he is 
%not yet caused to be born in the ninth, tenth, eleventh or twelfth  (month), then 
%there is something wrong.
%
%\item[31]Furthermore, the umbilical cord is securely fastened to both 
%juice-carrying vessels of the mother and carries the power (energy?) of the 
%essential juice coming from the food of the mother and what causes (the baby) to 
%live is the distribution of the life juice,* over all the body parts of the not yet 
%(existing) newborn, from the beginning of conception (?) (niḥṣekān), and over 
%(all) the transportation channels, running in all directions because of that 
%intimate connection of the vessels.
%
%\item[32]Mainly, the developments of the foetus are: śaunakasays says that the 
%head develops first because it is at the basis of this (development). 
%Kṛtavīryasaysit states is the heart (which is at the base) of both intellect and 
%mind. Pārāsa’ s son maintains instead that (it is) ……………….. (deraha?-) of the 
%body. Mārkkandeya presumes that hands and feet are first because they are at 
%the basis of movement in the body of the foetus. Subhūti Gautama claims all the 
%limbs and their smaller subparts develop because of their development because 
%the development of all the moving limbs is irretrievably connected, all turned 
%into one and the same direction (of the thorax) together. At the time of early 
%pregnancy, due to their extreme minuteness, they cannot be perceived, like 
%sprouts of bamboo or seeds of mango. Thus, in the manner mango fruits 
%becomes ripe, or as the shine of the hair of the head, or the way marrow appears 
%in bones, step by step these things are seen more accurately, e.g. as an increase 
%of black colour, and they become gradually apparent as the body (takes shape). 
%Due to their feature of being so subtle, the minuteness of the hair of the head 
%(and other examples) makes the black become apparent in this way; just so the 
%growth of bamboo is also explained. Similarly in the beginning of a pregnancy, 
%precisely  because of the minuteness in all limbs and smaller body parts which 
%are present, these are not well perceived (but) because of their increasing 
%degree of blackness they become apparent.
%
%\item[33] It is claimed (that this) is not the consequence of any previous or any 
%(bad or) excellent fate but solely because of the minuteness  they* are not being 
%observed. In that context we shall explain features in the body that are paternal, 
%maternal, connected with rasa, related to the soul, linked to the quietude of mind 
%and relative to the essence of being.** Keeping this in mind, the hair of the head, 
%tears, teeth, nails, the hair of beard and moustache, things made of hard 
%substance (cartilage?)*** are brought about as paternal (elements). Muscle, 
%blood, fat, marrow, the heart, the umbilicus (= the placenta? )****, the liver, the 
%spleen, the intestines, the anus are brought about as the soft maternal 
%(elements). The increase in size of the body, the growth of the child and (its) 
%outward appearance, the gain and loss of its erect attitude are caused by the 
%rasa. The senses, consciousness, duration of life and the intensity of pleasure 
%and pain are related to the spiritual element. We shall discuss later the 
%satva-related things. Valour, healthfulness, strength, complexion and prudence 
%depend on the existential disposition.**
%
%\item[34] In this context a woman in whose right breast  milk appears first,* 
%(whose) right flankis the larger one  and leg shall be lifted first on the right sid, 
%and  who is occupied for the largest part during pregnancy with things that are 
%identified) by male names, and in her sleep receives lotuses, blue lotus 
%blossoms, Kumud-flowers, āmrāmrātaka**-flowers and so on, or precisely with 
%male names, and who has an enhanced facial complexion, it is proclaimed to be 
%likely (bhavetām?) (that) it will lead to the birth of that son. In case of the 
%opposite of this (it will lead to) a girl. She whose both sides are bent down and 
%(whose) aformentioned  belly is bulging  forward, the typical feature from this 
%knowledge is a sexless individual. She whose abdomen is sunk in the middle will 
%produce  …………….. (prābhūtamu?) twins . 
%
%\item[35]And here is (more). Women who sit down to the gods and Brahmins, 
%have the advantage of a ceremonially pure offspring. They produce children with 
%great qualities. In the opposite case however, they have no qualities. 
%
%\item[36] The development of the limbs and the smaller anatomical parts  
%progresses  precisely all according to its own nature. The development of these 
%limbs and the smaller anatomical parts is dependent upon the qualities and 
%conditions which could not be known of the foetus by religion and could not be 
%caused by religion.*
%%iti śārīratṛtīyo 'dhyāyaḥ || 
%
%This is the third chapter of the śarīra.
%
%\end{tt}
    
\end{translation}
