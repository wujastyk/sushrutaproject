%!TeX root = incremental_SS_Translation.tex

\chapter{Śārīrasthāna 1:  A Consideration of All Beings}

\section{Introduction}


The short account of Sāṅkhya philosophy offered in this chapter of the
\SS\ is several times characterized by its authors as being ``special to
physicians".\footnote{3.1.11 \dev{vaidyake tu} ``but in
    medicine\ldots''; 3.1.13 \dev{cikitsite} ``in medicine''; 3.1.16
    \dev{āyurvedaśāstreṣu} ``in the treatises about medicine\ldots '';
    3.1.16 \dev{sa eṣa karmapuruṣaścikitsādhikṛtaḥ} ``it is this agentic
    person that medicine is concerned with.''}  And it does indeed stand
    slightly apart from the major classical forms of Sāṅkhya philosophy in
    some regards.

For example, the description of the evolution from Ahaṅkāra given in  \SS\
3.1.4 (p.\,\pageref{3.1.4}) corresponds more to the
\textit{Māṭharavṛtti} and to the Purāṇas than to other
commentaries on the \emph{Sāṅkhyakārikā}.  As \citeauthor{solo-1974}
pointed out, the description of Ahaṅkāra in Māṭhara's commentary on the
\emph{Sāṅkhyakārikā} (ca.\,1000) is unique in the following regard:
\begin{quote}
    All [early Sāṅkhya commentaries] mention the paryāyas of ahaṃkāra,
viz.\ bhūtādi, vaikṛta and taijasa; but all except M[āṭharavṛtti]
simply state that the 16 are produced from ahaṃkāra and enumerate
them. M[āṭharavṛtti] alone explains here that the five tanmātras
are produced from bhūtādi which is tāmasa, the 11 organs are
produced from vaikṛta which is sāttvika, while both are produced
from taijasa which is rājasa.\footcite[52, 180]{solo-1974}
\end{quote}
 This historically distinct scheme, as also presented in the Purāṇas,
was shown in a clear diagram by \citet[27]{biar-1981}; see
Figure~\ref{fig:biardeau1981-p27}.
            \begin{figure}
                \small
    \centering
    %\includegraphics[height=\textheight]{chapters/media/biardeau1981-p27}
    \begin{tikzpicture}[
        node distance=0.4cm,
        every node/.style={font=\itshape},
        box/.style={align=center}
        ]
        
        % Top nodes
        \node (avyakta) {avyakta = pradhāna};
        \node[right=3cm of avyakta] (purusa) {+ \; \textit{puruṣa}};
        \node[left=3.5cm of avyakta] (kkk1) {}; %DW
        
        % Main vertical line
        \node[below=of avyakta] (mahan) {mahān};
        \node[below=of mahan] (gunas) {(sāttvika \; rājasa \; tāmasa)};
        \node[below=of gunas] (ahamkara) {ahaṅkāra};
        
        % Threefold division with less spacing
        \node[below left=.25cm and 1cm of ahamkara] (vaikarika) 
        {vaikārika};
        \node[below=.25cm of ahamkara] (taijasa) {taijasa};
        \node[below right=.25cm and 1cm of ahamkara] (bhutadi) 
        {bhūtādi};
        
        % Wavy brace under the three nodes
        \draw[decorate, decoration={snake, amplitude=1mm, segment 
            length=6mm}]
        ([yshift=-3mm]vaikarika.south west) -- ([yshift=-3mm]bhutadi.south 
        east);
        
        % From vaikārika
        \node[below=of vaikarika] (dummy1) {};
        \node[below=1.2cm of vaikarika] (buddhi) {5 buddhīndriya};
        \node[below=of buddhi] (karmendriya) {+ 5 karmendriya};
        \node[below=of karmendriya] (manas) {manas = 
            11\textsuperscript{e} 
            indriya};
        % DW
        \node[left=.75cm of manas] (kkk2) {};
        
        % From taijasa
        
        \node[below right=1.2cm and -.8cm of taijasa] (sabda){śabdatanmātra};
        \node[above=of sabda] (dummy2) {}; %DW
        \node[below=of sabda] (akasa) {ākāśa};
        \node[below=of akasa] (sparsa) {sparśatanmātra};
        \node[below=of sparsa] (vayu) {vāyu};
        \node[below=of vayu] (rupa) {rūpatanmātra};
        \node[below=of rupa] (jyotis) {jyotis};
        \node[below=of jyotis] (rasa) {rasatanmātra};
        \node[below=of rasa] (apas) {āpaḥ};
        \node[below=of apas] (gandha) {gandhatanmātra};
        \node[below=of gandha] (sanghata) {saṅghāta};
        
        % Connections
        \draw[->] (avyakta) -- (mahan);
        \draw[->] (mahan) -- (gunas);
        \draw[->] (gunas) -- (ahamkara);
        
        % DW lines on left
        \draw[-] (avyakta) -- (kkk1);
        \draw[-](kkk1) -- (kkk2);
        \draw[->](kkk2) -- (manas);       
        
        % Branches
        %                  \draw[->] (ahamkara) -- (vaikarika);
        %                  \draw[->] (ahamkara) -- (taijasa);
        %                  \draw[->] (ahamkara) -- (bhutadi);
        
        % Left branch
        \draw[->] (dummy1) -- (buddhi);
        \draw[->] (buddhi) -- (karmendriya);
        \draw[->] (karmendriya) -- (manas);
        
        % Diagonal slanted connection back into middle branch
        %\draw[->] (manas.east) .. controls +(3,0) and +(-3,0) .. 
        %(sparsa.west);
        
        % Middle branch
        \draw[->] (dummy2) -- (sabda);
        \draw[->] (sabda) -- (akasa);
        \draw[->] (akasa) -- (sparsa);
        \draw[->] (sparsa) -- (vayu);
        \draw[->] (vayu) -- (rupa);
        \draw[->] (rupa) -- (jyotis);
        \draw[->] (jyotis) -- (rasa);
        \draw[->] (rasa) -- (apas);
        \draw[->] (apas) -- (gandha);
        \draw[->] (gandha) -- (sanghata);
        
    \end{tikzpicture}
    
    \caption{Levels of original creation as presented in the following 
        Purāṇas: \emph{Vāyupurāṇa}, \emph{Brahmāṇḍapurāṇa}, 
        \emph{Viṣṇupurāṇa}, \emph{Mārkaṇḍeyapurāṇa}, 
        and \emph{Kūrmapurāṇa} \citep[after][27]{biar-1981}. 
        See footnote \ref{puraniccosmology}.}
    \label{fig:biardeau1981-p27}
\end{figure}

Another example of the \SS's eclectic account of Sāṅkhya is the list
of homologies given in 3.1.7--8.\label{3.1.7-8}  The evolutes of Prakṛti are
described as having three aspects or instantiations:  
in the physical world (\emph{adhibhūta}), in the individual person
(\emph{adhyātma}), and in the divine realm (\emph{adhideva}).  While
this terminology is reminiscent of very old language from the Upaniṣads,
this specific scheme is not widely known in Sāṅkhya literature, or
anywhere.\footnote{These homologies in the \SS\ were noted by
\citet[55]{comb-2011}. The adjacent topic of the three kinds of
suffering was discussed by  \textcites{stei-2007}{vuka-2023}.  For the
Upaniṣads, one thinks of \emph{Bṛhadāraṇyaka} 1.5.21 (\emph{adhyātma}),
3.7.15 (\emph{adhibhūta, adhyātma}), 2.3.3 (\emph{adhidaivata}),
\emph{et passim}.  In the Pāli Canon, \emph{ajjhattika} (cognate Skt.\
\emph{adhyātmika}) means ``inner, internal, in the physical body,'' as
is clear from, e.g., \emph{Mahāhatthipadopamasutta} \P6 where the
``internal'' earth element is characterized as including head, hair,
body hair, nails, teeth, skin, flesh, sinews, bones, marrow, kidneys,
heart, liver, etc. Ed.\ \pvolcite{1}[185]{tren-1888},
%tr.\,\cite[133--134]{chal-}
tr.\,\cite[279]{nana-1995}.)} 
            \begin{table}[t]
        \small
        \centering
        \caption{Homologies according to Mahābhārata 14.42.27--40.}
        \label{MbhHomologies}
        \medskip
        \begin{tblr}{width=0.8\textwidth,colspec = {XXXX}}
            \toprule
            \emph{bhūta} & \emph{adhyātma} & \emph{adhibhūta} & 
            \emph{adhidaiva} \\
            \midrule
            ākāśa 	  & śrotra & śabda & Diś \\
            marut 	  & tvag & spraśtavya & Vidyut \\
            jyotis 		& cakṣur & rūpa & Sūrya \\
            āp 			& jihvā& rasa	& Soma\\
            pṛthivī    & ghrāṇa & gandha & Vāyu \\
            & pādau & gantavya 		& Viṣṇu \\
            & pāyuḥ & visarga 		& Mitra \\
            &upastha &śukra 		& Prajāpati \\
            & hastau & karman 		& Śakra \\
            & vāk & vaktavya 		& Vahni \\
            & manas & mantavya 	& Candramas \\
            & buddhir& vijñeya 		& Brahmā \\
            \bottomrule
        \end{tblr}
    \end{table}
    %
The scheme is first found in the \emph{Mahābhārata}: see
Table~\ref{MbhHomologies}.\footnote{Referred to in \emph{Mahābhārata}
    6.30.1 \pvolcite{7}[1666--1668]{sukt-1933} and then described in
    detail twice, in 12.300.17--301.14 (\emph{ibid}.\ 15, 1666--1668) and
    in 14.42.27–40 (\emph{ibid}.\ 18, 152--153).  These and the following
    parallels were pointed out by Philipp Maas.} %
    The scheme also appears in the fourteenth- or fifteenth-century
    \emph{Tattvasamāsasūtra} sūtras 7--9 and in its commentaries
    \emph{Kramadīpikā} and the \emph{Tattvayāthārthyadīpanaṭīkā} of
    Bhāvagaṇeśa, both of  which set out homologous triples, equating
    ontologies on the personal, physical and divine
    levels.\footnote{\cite[81--82]{dviv-1996} and
        \cite[15--16]{bhat-1965} respectively.  On the dates of the works,
        see \cites[152--153]{huli-1978}[ch.\,24 et passim]{lars-1987}. See
        also the discussion in \cite[134]{rosu-1978}, cited in
        \volcite{IB}[370, n.\,4]{meul-hist}.}

\label{nityau} Another parallel between the present chapter and the
\emph{Mahābhārata} occurs in 3.1.9.  The text is explaining the
similar and different properties of Puruṣa and Prakṛti. This passage
in the \SS\ is very close in both meaning and wording to
\emph{Mahābhārata}
12.210.6--8:\footnote{\pvolcite{15}[1159]{sukt-1933}.  The parallel was
    pointed out by Christ\`ele Barois.}
    \begin{quote}
        In the same way, both Prakṛti and Puruṣa should be known. 
        But the discerning person should pay particular attention to that 
        special, even greater entity that is different from both Prakṛti and 
        Puruṣa.  They both have no beginning and no end and they both have no 
        characteristics.  They are both eternal, extremely fine, and they are both 
        greater than Mahat.  This is what they have in common. Thus 
        there is another special feature. 
        
        \medskip
        \dev{tadevametau vijñeyāvavyaktapuruṣāvubhau /\\    
        avyaktapuruṣābhyāṃ tu yatsyādanyanmahattaram // 6\\    
        taṃ viśeṣamavekṣeta viśeṣeṇa vicakṣaṇaḥ / \\    
        anādyantāvubhāvetāvaliṅgau cāpyubhāvapi // 7\\    
        ubhau nityau sūkṣmatarau mahadbhyaśca mahattarau / \\    
        sāmānyametadubhayorevaṃ hyanyadviśeṣaṇam // 8}
    \end{quote}


\section{Literature} 

Meulenbeld offered an annotated overview of this chapter and a
bibliography of earlier scholarship to 2002 and, in his notes,
citations of the parallel passages in the
\CS.\fvolcite{IA}[243]{meul-hist}  \citet{lars-1987} provided a major
overview of Sāṅkhya literature. Recent overviews of the classical
Sāṅkhya theory include those of
\textcites[\S2.4]{chat-2021}[ch.\,22]{adam-2022}{ruzs-2025}.
\citet[54--56]{comb-2011} studied the Sāṅkhya concepts specifically in
the \SS.


\cleardoublepage

\section{Translation}

\begin{translation}
    
    \item [1] 
    
    So, now we shall explain the anatomy chapter that is a reflection about all 
    beings.\footnote{The Nepalese version has nouns in apposition 
    (“-\dev{cintā} $\leftrightarrow$ \dev{śārīram}”).  The vulgate makes this a 
    single 
    karmadhāraya compound that is slightly easier to parse.} 
    
    
\subsection{The production of all things}    
\item[3]

%The cause of all beings, called “the unmanifest,” is without a cause,
%is characterised by sattva, rajas and tamas,  has eight forms and
%is the reason for the appearance  of this whole world.

That which is called “the unmanifest” is the causeless cause of all living beings,  
having the characteristics of sattva, rajas and tamas, having eight forms, and 
being the reason for the appearance of this whole world.\footnote{We do
    not translate the polysemic terms \emph{sattva}, \emph{rajas} and
\emph{tamas}, about which a
    large scholarly literature may be consulted.}
    
It is the single basis of the many
\se{kṣetrajña}{witnesses},\footnote{“Witnesses” refers to the
    disembodied, inner selves that witness the world, the most essential
    kernels of personal consciousness. Glossed in \CS\ \Ca{4.1.61}{293} as
    “the unmanifest self, eternal, sovereign, and unchanging”.  Cf.\
    translation and context in \cite[239]{wuja-2023a}, and \citet[132 et
    passim]{rosu-1978}.\label{ksetrajna}} just as the ocean is
    to the beings who live in water.\footnote{The Nepalese witnesses
        differ from the vulgate here, reading \dev{udakaujas} “creatures whose
        power is water.”  This is linguistically and semantically implausible.
        Ḍalhaṇa remarked that there were different interpretations of this
        simile in the vulgate version, \dev{audakānām} “creatures having
        watery character." Some thought it meant ``like rivers, lakes and
        other forms of water are supported by the ocean", while other thought
        it referred to living beings like fish and plants that are supported
        by the ocean." The emendation to \dev{udakaukas} suggested by Philipp
        Maas is compelling semantically and palaeographically.}
    
\item[4]
\label{3.1.4}

From that unmanifest, the Mahat arises, having exactly the same
properties.\footnote{In classical Sāṅkhya theory, \dev{mahat} is a
    synonym for \dev{buddhi}, ``intellect.''  In the present passage, this
    identity is not explicit; rather, it is a cosmological entity.  In the
    cosmology of the \emph{Pātañjalayogaśāstra}, it is pure being,
    \dev{sattāmātra} 2.19 \citep[85]{agas-1904}, it is also sometimes
    designated as the great \dev{ātman} ``great self'' in the sense of a
    universal being.} %
    From that Mahat, which has those same properties, arises the
    Ahaṅkāra, having exactly the same characteristics.\footnote{The
        Ahaṅkāra, etymologically ``the utterance `I',''  is the assertion
        of personal and creative identity. See the classic study by
        \citet{buit-1957} that discusses the several problems raised by the
        term.}  It has three aspects: \se{vaikārika}{mutable},
        \se{taijasa}{fiery} and
        \se{bhūtādi}{elemental}.\footnote{\label{puraniccosmology}These
            technical terms occur in \emph{Sāṅkhyakārikā} 22 as synonyms for
            Ahaṅkāra.  In \emph{Sāṅkhykārikā} 25, they are described as
            emanations coming from Ahaṅkāra \parencites[46--47,
            50]{sast-1948}[187--188, 195--196]{wezl-1998}. They also occur in
            the Purāṇic cosmogonies; \citet[27]{biar-1981} offered a useful
            diagrammatic representation of these showing these relationships,
            reproduced in Figure~\ref{fig:biardeau1981-p27}.  See the
            discussion of these difficult terms by \citet[23--25]{buit-1957}.}

            
            From that mutable Ahaṅkāra the 
            eleven \se{indriya}{faculties} arise, with the very same
            characteristics. It is as follows: ear, skin, eye, tongue, nose,
            speech, hand, genitals, anus, feet and mind.  Amongst these, the first 
            five are the faculties of \se{buddhi}{cognition}; the next five are the 
            faculties of \se{karma}{action}.  The mind has properties of both.
            
            From the Ahaṅkāra as \se{bhūtādi}{starting point for the
    elements}, arise the five \se{tanmātra}{bare entities},
with exactly the same characteristics.\footnote{Earlier,
    the Ahaṅkāra was said to have three aspects, so we would
    here expect a description of the \se{taijasa}{fiery}
    aspect.  But the Nepalese version goes straight to the
    \se{bhūtādi}{elemental} aspect.  The vulgate text inserts
    the fiery aspect alongside the elemental as if it were
    similar in all respects (\dev{taijasasahāya}).}  It is as
    follows: bare sound, bare touch, bare form, bare taste,
    bare smell.\footnote{Or, ``the essence of sound,'' etc.}
    
    From these \se{bhūta}{elements} come \se{ākāśa}{ether}, air, fire,
water and earth; from these come sound, touch, form, taste and
smell, with the same distinctions.\footnote{On “ether,” see footnote 
\ref{ether}.}  In this way these twenty-four
\sepl{tattva}{principle} have been explained.
            
\item[5]    

In this context, entities such as sound are the objects of the
\se{indriya}{faculties} of cognition.  Amongst the faculties of
action, they are: speaking, holding, enjoyment, excretion and walking
respectively.

\item[6]

The eight \sepl{prakṛti}{productive principle} are 
the \se{avyakta}{unmanifest},
\se{mahān}{The Great}, 
the \se{ahaṅkāra}{I-principle}, 
and the five \se{tanmātra}{fine elements}.
The rest are the sixteen \se{vikāra}{modifications}.
    
\item[7]   

And for each of these, the sense object is the 
\se{adhibhūta}{physical entity}.\footnote{There is a question about what
    ``of them'' refers to.  The list that follows has thirteen terms;
    fifteen if one takes hands and feet as duals; seventeen if one takes
    eyes and ears as pairs.  This does not quite correspond to any of the
    previous listings.   The following list only lists the divine and personal
    ontologies; the physical ones are not explicitly listed.
    
    The Nepalese version before emendation had a different meaning at
this point (ignoring grammatical difficulties): "Each and every
one of them has a sovereign with respect to their domain." See the edition's
critical apparatus for details.}  But they themselves are the
\se{adhyātma}{personal aspect}.
The \se{adhideva}{divine aspect} is thus:
Brahmā is the divine aspect of the \se{buddhi}{intellect}, 
Īśvara is of the \se{ahaṃkāra}{sense of the self}, 
the moon is of the mind, 
the directions are of the ear, 
wind is of the skin, 
the sun is of the eyes, 
the waters is of the tongue, 
the earth is of the nose, 
fire is of the voice, 
Indra is of the hands, 
Viṣṇu is of the feet, 
Mitra is of the anus, 
and Prajāpati is of the genitals.\footnote{Expressed as a table 
    in Table~\ref{adhideva}.  On this and the next passage, see discussion 
    above, \pageref{3.1.7-8}.} 

\begin{table}[t]
\centering
\caption{Ontologies on the personal, physical and
        divine levels.}
\bigskip
    
   \DeclareTblrTemplate {contfoot-text}{default}{(\emph{continued\ldots})}  
   \DeclareTblrTemplate{firsthead,lasthead}{default}{}   
\begin{tblr}[
    headsep=0pt,
    presep=0pt,
    ]{colspec={XQ},
       width = 0.7\linewidth,
       rowhead = 1}
\toprule
\emph{Divine} & \emph{Personal}\\
\midrule
 Brahmā & of \se{buddhi}{intellect},\\
Īśvara & of \se{ahaṃkāra}{sense of self},\\
the moon & of mind,\\
the directions & of the ear,\\
wind & of the skin,\\
the sun & of the eyes,\\
the waters & of the tongue,\\
the earth & of the nose,\\
fire & of the voice,\\
Indra & of the hands,\\
Viṣṇu & of the feet,\\
Mitra & of the anus,\\
and Prajāpati & of the genitals.\\
\bottomrule
\end{tblr}
   \label{adhideva}
\end{table}

\item[8]

This whole group lacks consciousness.\footnote{I.e., the group of
    twenty-four \sepl{tattva}{principle}.}  And the twenty-fifth, the
    Person (\emph{puruṣa})\sse{puruṣa}{person}, is the one that causes
    consciousness.  And he is connected to the  \se{kārya}{effects} that
    are the \sepl{karaṇa}{instrument}.\footnote{This sentence is hard to
        understand, but it is underscoring the unique role of the Person. The
        vulgate text at this point has, ``is united by  \se{kāraṇa}{cause} and
        result,'' a quite different and easier reading.  In philosophical
        prose one would hesitate to interpret \se{karaṇa}{instrument} as
        \se{kāraṇa}{cause} As a dvandva, \dev{karaṇakārya}  breaks Pāṇini
        2.2.34 because \dev{karaṇa} has more vowels than \dev{kārya}. Reading
        the compound \dev{karaṇakārya} not as a dvandva, but as a karmadhārya
        seems preferable.} %
        Even though the \se{pradhāna}{productive principle} is
        unconscious, they point out that it is active for the purpose
        of the Person's \se{kaivalya}{freedom}.\footnote{The
            expression ``they point out" suggests reference to outside
            experts.  Since the following milk simile is identical to
            \emph{Sāṅkhyakārikā} 57
            \parencites[184--186]{main-1972}[263]{wezl-1998}, it seems
            certain that the reference is to this text or a lost
            predecessor.  The \emph{Sāṅkhyakārikā} was translated into
            Chinese in the mid-sixth century and may have been composed
            one or two centuries before that time \citep[138]{huli-1978}.}
            %
            On this point, they give the examples of causes like the
            one about milk, etc.\footnote{I.e., the calf  in the
                proximity of the cow causes the milk, in the same way that
                Prakṛti in the proximity of Puruṣa causes evolution.
                \Dalhana{3.1.8}{340} explained that the milk, even though
                it is \se{ajña}{unconscious}, \se{pravartate}{comes forth}
                (in the cow) for the purpose of nourishing a calf. Ḍalhaṇa
                also gave the example of a man's semen that is
                \se{ajña}{incognizant} but is \se{pravartate}{comes forth}
                in the presence of erotic women, at a private party, for
                the purpose of the man's enjoyment.
                \emph{Pātañjalayogaśāstra} 4.17 addressed the same issue
                with the simile of magnetism: the \se{citta}{mind} was
                likened to a piece of iron that is attracted by the magnet
                of sense objects \citep[193, et passim]{agas-1904}.}


% everything except the mahābhūtas and tanmātras - CB

\subsection{Prakṛti and Puruṣa}

\item[9]

From this point onwards we shall describe how Prakṛti and Puruṣa have
similar and different essential properties
(\emph{dharma})\sse{dharma}{essential property}.

Both are without beginning and both are without end; both both are
permanent;\footnote{This is an emended reading of
    the Nepalese witnesses, which both read \dev{anityau} “impermanent.”
    It is inconsistent and contextually incorrect
    to assert that Puruṣa is impermanent.  The vulgate reads
    ``permanent."} both are unsurpassed, both are without 
    \sepl{liṅga}{characteristic} and both are omnipresent.\footnote{See 
        discussion, p.\pageref{nityau}.}
 
But Prakṛiti is single, unconscous, has three guṇas, is essentially a
seed, is essentially creative and has the essential 
property of being in the middle.

The Puruṣas, meanwhile, are multiple and have consciousness.  They do
not have the guṇas, they are not essentially seeds, they are not
essentially creative, and do not have the essential property of being in the 
middle.

\item[10]

Thus, on the assumption that an effect corresponds to its cause, all
these \sepl{viśeṣa}{particular}, consisting of sattva, rajas and tamas,
come into being.

Some people argue that the Person actually consists of these
particulars\footnote{\Dalhana{3.1.10}{340} identifies these as
    the \sepl{tattva}{principle}, beginning with \dev{mahat}.} because he is
    manifested by them and made out of them.\footnote{This opinion of “some 
    people” represents
        a kind of materialist who thinks that the Person is also composed of
        prakṛti's components rather than being distinct and unitary.  This is an 
        outsider view as far as early Sāṅkhya is concerned.
        
        On \dev{añjana} in the compount \dev{tadañjana} “manifested by
them,” \citet{kuip-1953} noted the inadequacy of dictionary
entries for derivatives of roots \dev{añc}\slash\dev{añj}, and
described the root of the present word under no.\,3 “show,
manifest, mark, adorn, honour.” See ibid., \S12, pp.\,76--82.}

\item[11]

In the Vedic tradition, however,\footnote{The variant reading
    \dev{vaidika}, in witness N, probably would refer to the Vedic
    tradition, as it does generally in Sanskrit literature
    \citep[1022]{moni-sans}. Witness H and the vulgate read \dev{vaidyake}
    “in the medical tradition,” which may be a banalization.  The
    subsequent statement listing different views about Prakṛti is not
    known elsewhere in medical literature, but is very close to Vedic
    sources such as the \emph{Śvetāśvataropaniṣad} 1.1--2
    \citep[414--415]{oliv-1998} and \emph{Gauḍapādakārikā} 1.8, 9
    \citep[3--4, 62]{karm-1953}.  \citet[10, n.\,19]{oliv-2017} identified the
    earliest occurence of the term \dev{vaidyaka} as being in Patañjali's 
    \emph{Mahābhāṣya}.}
    
    % Cf.\
    % \emph{Mahābhārata},
    %    13.107.1: \dev{śatāyuruktaḥ puruṣaḥ śatavīryaśca vaidike}
    %    \pvolcite{17.2}[559]{sukt-1933}, where \dev{vaidike} almost
    % certainly
    %    refers to medicine.}

\begin{quote}
    \sse{pṛthudarśin}{people with a wide perspective}%
people with a wide perspective consider \se{svabhāva}{essential
    being}, destiny, time, \se{pariṇāma}{transformation}, the Lord,
and \se{yadṛcchā}{chance} to be
Prakṛti.\footnote{\Dalhana{3.1.11}{341--342} discussed whether
    these six causal entities were to be considered together or separately.
    Ḍalhaṇa seems to have accepted Jejjaṭa's view that these are
    multiple philosophical views, but that physicians consider the
    ultimate cause to be Prakṛti.  He also recorded Gayadāsa's view
    that some thinkers believe that these causes cumulatively
    constitute Prakṛti. Cf.\ the similar discussion in 
    \emph{Śvetāśvataropaniṣad}, \emph{ibid}. \volcite{IB}[370, 
    n.\,5]{meul-hist} provided a bibliography on these topics.}
    \end{quote}
    % Jejjata and Dalhana : multiple views, we think Prakriti.
    % Gayi: these views constitute Prakriti.
    
\item[12]

\begin{quote}
So one should note that the \diff{\sepl{bhūta}{element} that are
    produced}  specifically have their qualities.\footnote{Note that the
    phrase \dev{tato jātātāni bhūtāni} “So, those that are produced,”
    differs from the vulgate text, but was known and accepted by
    Gayadāsa.} From those, the entire \sepl{bhūtagrāma}{group of living
        being} is generated, having their
    \sepl{lakṣaṇa}{characteristic}.\footnote{This passage contains
        potential ambiguities about the polysemic word \dev{bhūta}, whose
        meanings include “elemental substance” (such as earth, air, water,
        etc.)  and “being” (as in creature, animal).  The author of the \SS\
        used the word in both these senses, even in a single passage. The
        keyword \dev{bhūtagrāma} “collection of \dev{bhūta}s” is a case in
        point. It also occurs at \SS\ \Su{1.1.22}{5} where it is described in
        the main \SS\ text as signifying the aggregate of the four types of
        living being, namely those born of sweat, the womb, eggs, and sprouts
        (broadly corresponding to insects, mammals, birds and reptiles, and
        plants). In his comment on the present passage, Ḍalhaṇa agreed with
        this view, glossing \dev{bhūtagrāma} as \dev{sthāvarajaṅgamātmaka}
        “consisting of mobile and stationary beings" (\Dalhana{3.1.12}{341}). 
        Ḍalhaṇa's language here is close to the \emph{Gopathabrāhmaṇa} 1.29
        \parencites[ed.][21--22]{gaas-1919}[tr.][30--31]{paty-1969}.  He also
        defended the connection of “element” with “beings” by noting that
        \dev{tallakṣaṇa} “having their characteristics” meant that physical
        elements like earth have certain characteristics, such as solidity,
        heaviness and roughness, and that the group of living beings have
        these same characteristics of the physical elements, because they
        originate from those elements (\dev{pañcamahābhūtārabdhasya
        bhūtagrāmasya}\ldots, \cite[341]{vulgate}).}

\end{quote}

\item[13--14]

Its \se{upayoga}{applicability} is stated always with regard to
medicine.\footnote{\Dalhana{3.1.14}{341--342} explained that “its”
    refers to discussion of the group of living beings, which starts with
    the five \sepl{mahābhūta}{great element} 
    (\dev{pañcamahābhūtārabdha}).
    See also the previous footnote.} Therefore in therapeutics, no consideration
    is given beyond the elements. Because it has been stated,
    \begin{quote}
        [by saying `Puruṣa'] he has stated that it originates from a
collection of substances beginning with the
\sepl{bhūta}{element}.\footnote{I.e., the “person” is the
    physical subject of medical science, deriving from the
    \sepl{bhūta}{element}, etc.  The internal reference here is to
    \SS\ \Su{1.38}{9}.  In that passage, the \SS\ defined the
    \se{puruṣa}{human being} as a material creature made out of
    the five elements and the physical bodily parts and tissues.}
        \end{quote}
And in \se{āyurveda}{the science of medicine}, it is the elemental
senses that are described, as well as the objects of the
senses.\footnote{The \emph{Yuktidīpikā} discussed the Nyāya view that
    the senses evolve from the elements, but asserted that the followers
    of Sāṅkhya reject this view and propose that the senses evolve
    directly from \se{ahaṅkāra}{identity}
    \parencites[ed.][203]{wezl-1998}[tr.][67--68]{harz-2006}.}

\item[15]

There is a verse on this:
\begin{quote}
    A human being grasps each object of sense by means of their own
corresponding sense organs.  It is an established fact that one
thing cannot be grasped by a different one, because it is
constrained by the equivalence of their origins.\footnote{E.g.,
    the eye can see visual images because the eye and the visual
    images themselves both originate in the element of fire.  But the
    eye cannot see scents or sounds.}
    \end{quote}
    
\item[16]

In the teachings of Ayurveda, witnesses
(\emph{kṣetrajña})\sse{kṣetrajña}{witness} are not considered to be
[both] omnipresent as well as permanent.\footnote{In 3.1.9 above, the
    authors have stated the Sāṅkhya view that the Person is beginningless,
    endless, permanent and omnipresent.  Now, the authors state that
    Āyurvedic physicians have a different view, namely that the Person is
    permanent but not omnipresent.
    
On “witnesses” see footnote \ref{ksetrajna}.} 
From the established opinion of Ayurveda, they bring forward
logical reasons to explain the Person as witnesses that are
permanent but not omnipresent.  Witnesses that are permanent but not omnipresent
transmigrate into the wombs of animals, and into
humans and gods, according to the \se{nimitta}{determining factors} of virtue and
vice.\footnote{The manuscript readings of the Nepalese witnesses
    are difficult here, and emendations have been made to preserve the
    logic of the passage. On \dev{ca} after a dvandva, see 
    \volcite{II.1}[\S70]{wack-1896}.}
    
They may be grasped through inference, they are mobile, they are
extremely fine, they have consciousness, they are eternal, they are
manifested in the conjoining of semen and menstrual blood.  The Person
has been defined as  “an aggregation of the five great elements and
the embodied soul”.\footnote{I.e., in \SS\ \Su{1.1.22}{5} and
    mentioned again in 3.1.14 above.} Therefore, this is indeed the
    “\se{karmapuruṣa}{patient}” who is the subject of
    medicine.\footnote{“Patient” translates \dev{karmapuruṣa}, the “Person
        subject to \se{karma}{action}.”  For this sense of the English word,
        cf.\ the OED's entry: “A person who or thing which undergoes some
        action, or to which something is done; a (passive) recipient. Chiefly
        in contrast with agent” \citep[Patient 4a]{OED}.
        \Dalhana{3.1.16}{342} glossed \dev{karmapuruṣa} both as “the one who
        experiences the results of karma" (\dev{karmaphalabhāk}) and also as
        “the one who receives the results of medical care”
        (\dev{cikitsitakarmaphalam}).  See the discussion by \citet[67, 132, 141, 142, 
        146, 147, 169, 177]{rosu-1978}. The term “\dev{karmapuruṣa}” also
        occurs at \Su{3.8.8}{380}, where it clearly means “patient,” and again
        at \Su{6.65.22}{817}, where \SS\ 1.1.22 is cited as an example of the
        interpretative rule called “\se{prasaṅga}{recontextualization}” (see
        p.\,\pageref{65:prasanga} below).  Note, that while this citation of
        1.1.22 in 6.65.22 is present in the vulgate text of the \SS, it is not
        present in the Nepalese version.}

\subsection{Attributes of the Person}
        
\item[17]

Its \sepl{guṇa}{attribute} are happiness and distress, desire and aversion, and
effort, breathing in and out, closing and opening the eyes,\footnote{A masculine 
dvandva.} 
cognizing, 
thinking, 
intending, 
reflecting, 
remembering, 
knowing, 
deciding, and
perceiving sense objects. %
\footnote{This passage adds the vocabularly of Vaiśeṣika and Nyāya to
    that of Sāṅkhya, showing that the specifically Ayurvedic view of the
    world being presented is more syncretic than individual philosophical
    schools. Cf.\ parallel passages in, for example, \CS\ (ed.\ \cite[294]{vulgate},
    tr.\ \cite[240]{wuja-2023a}), \emph{Nyāyasūtra} 1.1.10 (ed.\
    \cite[27]{josi-1922}, tr.\ \cite[34]{jha-1939}), \emph{Vaiśeṣikasūtra}
    3.2.4 (ed.\ \cite[28]{jamb-1961}, tr.\ \cite[117]{sinh-1928}).}
        
\item[18]

Furthermore, 
not being malicious,
enjoying sharing,
tolerance,
truthfulness,
\se{dharma}{righteousness},
\se{āstikya}{being a believer},\footnote{\Dalhana{3.1.18}{343} defines this
    as someone who believes in such things as 
    dharma, liberation and the world beyond.\label{astikya}}
knowing,
cognizing,
intelligence,\footnote{\Dalhana{3.1.18}{343} glossed \dev{medhā} as 
“having the power to concentrate on books” (\dev{granthāvadhāraṇaśaktiḥ}). 
Cf.\ his similar statement at \Dalhana{1.2.3}{10}, where the qualities of a good 
medical student were described.}
willpower,\footnote{MS N reads \sanengdev{dhṛti}{willpower}, while H reads
    \sanengdev{smṛti}{recollection} and the vulgate reads both.   In Sanskrit 
    literature 
    generally, these two virtues often appear together (and with 
    \sanengdev{medhā}{intelligence}).  The witnesses do not give a compelling 
    reason 
    for choosing any of these options.}
and not being overly attached
are associated with sattva\sse{sattva}{purity}. 

Excessive suffering, \se{apradāna}{being ungenerous},\footnote{The
    vulgate text prefers “excessive wandering about” here, as well as
    “having no willpower.”} bad behaviour, a lack of compassion,
    dishonesty, egoism, hypocrisy, pride, \se{harṣa}{lust}, desire and
    aversion are associated with
    rajas\sse{rajas}{turbulence}.\footnote{\dev{kāmakrodhaḥ} “desire and
        aversion” is a dvandva in the masculine.}

Depression,
being a non-believer,\footnote{See footnote \ref{astikya}.}
unvirtuous behaviour,
\se{buddhinirodha}{unreasonableness},
ignorance,
lack of intelligence,
laziness, and
sleepiness
are associated with tamas.\sse{tāmasa}{stultifying}

\item[19]

Furthermore, sound, the auditory faculty, the aggregate of all
\se{chidra}{intervals} and separateness are associated with the 
\se{āntarīkṣa}{ethereal}.\footnote{The term “ether,” translating
    \dev{antarīkṣa}, refers to the classical element of extension that
    separates all entities and prevents the existence of a
    vacuum.\label{ether}
    
    These characterisations of the elements are
    not typical in Sanskrit philosophical literature.}

The faculty of touch, the aggregate of all \sepl{ceṣṭā}{gesture}, all
\sepl{spanda}{spontaneous movement} of the body and lightness are
associated with air.\footnote{\Dalhana{3.1.19}{343} described “all
    gestures” as “bending and straightening, although others say all
    actions of body, voice and mind”
    (\dev{namanonnamanādisarvakriyāsamūhaḥ, kāyavāṅgmanaḥkriyāsamūha
    ityanye}).}
    
     

Form, 
the faculty of vision,
colour,
warmth,
radiance,
ripening,
intolerance,
and sharpness
are associated with fire.

Taste,
the faculty of taste,
the aggregate of all liquids,
heaviness, 
coldness,
oiliness and 
flow,
are associated with water.

Smell, the olfactory organ, the aggregate of all solid
bodies,\sse{mūrti}{solid body} and heaviness are associated with
earth.\footnote{There is a partial overlap of the elemental
    \se{guṇa}{qualities} listed in this section with those mentioned in
    some anonymous stanzas that appear in the \emph{Yuktidīpikā} on
    \emph{Sāṅkhyakārikā} 38c \citep[225]{wezl-1998}, also referred to by
    Vācaspatimiśra in his \emph{Tattvavaiśāradī} on \PYS\ 3.44
    \citep[??]{pras-1912}.}\q{PYS 3.44?}

\item[20]

In that context, 
ether is mostly sattva,
wind is mostly rajas,
fire is mostly sattva and rajas,
water is mostly sattva and tamas,
earth is mostly tamas. 




\item [21]

And on this there is:

\begin{sloka}
 One should note that all these pervade each other.
 
The manifest characteristic of all these is regarded as being in
each of the separate substances.\footnote{The Sanskrit text of the
    Nepalese version at this point is supported by both witnesses, but is
    harder to construe than the vulgate.  The vulgate's \dev{dravye tu}
    makes a clearer sentence, but \emph{lectio dificilior potior}.  It
is possible that this verse is cited from
    another con  The general sense is that while
    these substances pervade each other, they nevertheless preserve their
    individual characteristics.}
    
\end{sloka}    

\item[22]

\begin{sloka}
The eight productive entities have been declared and the sixteen modifications, 
and, concisely, the witness, according to their own system and another 
system.\footnote{\Dalhana{3.1.22}{344} commented that the “own system” 
was surgery (\dev{śalyatantra}) and the other system was ear, nose and throat 
surgery and Sāṅkhya (\dev{śālākyatantre sāṅkhye ca}).  The sentence could 
also be translated as “depending on itself and depending on another."}
\end{sloka}

\bigskip

This is the first chapter on anatomy.







    
    
\end{translation}
