\documentclass[14pt]{extarticle}
\usepackage{polyglossia,fontspec,xunicode}
\usepackage[normalem]{ulem}
\usepackage[noend,noeledsec,noledgroup]{reledmac}
\usepackage[margin=1in]{geometry}

\arrangementX[A]{paragraph}
\arrangementX[B]{paragraph}
\renewcommand*{\thefootnoteB}{\Roman{footnoteB}}
\arrangementX[C]{paragraph}
\renewcommand*{\thefootnoteC}{\roman{footnoteC}}


\Xarrangement[A]{paragraph}
\Xnotenumfont[A]{\bfseries}
\Xlemmafont[A]{\bfseries}

\setdefaultlanguage{sanskrit}
\setotherlanguage{english}
\newfontfamily\devanagarifont{Brill}
\newfontfamily{\devafont}{Pedantic Devanagari}

\usepackage[Devanagari,DevanagariExtended]{ucharclasses}

\makeatletter
\setTransitionsFor{Devanagari}%
 {\let\curfamily\f@family\let\curshape\f@shape\let\curseries\f@series\devafont}
 {\fontfamily{\curfamily}\fontshape{\curshape}\fontseries{\curseries}\selectfont}
\makeatother

\makeatletter
\setTransitionsFor{DevanagariExtended}%
 {\let\curfamily\f@family\let\curshape\f@shape\let\curseries\f@series\devafont}
 {\fontfamily{\curfamily}\fontshape{\curshape}\fontseries{\curseries}\selectfont}
\makeatother

\begin{document}
    \raggedright
% Manual hyphenation points for Sanskrit words and compounds.
% By Dominik Wujastyk.
% Copyright Dominik Wujastyk 2021.
% Released under a BY-SA Creative Commons license 
% (Attribution-ShareAlike 4.0 International http://creativecommons.org/licenses/by-sa/4.0/).
% This file is still actively growing, slowly but steadily (March 2021) .
%
% These special hyphenations have to be loaded after
% \begin{document}. See
% http://www.tug.org/pipermail/xetex/2008-July/010362.html
% Or use 
% \AtBeginDocument{% Manual hyphenation points for Sanskrit words and compounds.
% By Dominik Wujastyk.
% Copyright Dominik Wujastyk 2021.
% Released under a BY-SA Creative Commons license 
% (Attribution-ShareAlike 4.0 International http://creativecommons.org/licenses/by-sa/4.0/).
% This file is still actively growing, slowly but steadily (March 2021) .
%
% These special hyphenations have to be loaded after
% \begin{document}. See
% http://www.tug.org/pipermail/xetex/2008-July/010362.html
% Or use 
% \AtBeginDocument{% Manual hyphenation points for Sanskrit words and compounds.
% By Dominik Wujastyk.
% Copyright Dominik Wujastyk 2021.
% Released under a BY-SA Creative Commons license 
% (Attribution-ShareAlike 4.0 International http://creativecommons.org/licenses/by-sa/4.0/).
% This file is still actively growing, slowly but steadily (March 2021) .
%
% These special hyphenations have to be loaded after
% \begin{document}. See
% http://www.tug.org/pipermail/xetex/2008-July/010362.html
% Or use 
% \AtBeginDocument{\input{sanskrit-hyphenations}}% should work, but doesn't
% special hyphenations for Sanskrit words tagged in
% Polyglossia.
% *English,\textenglish{},text,and
% *Sanskrit,\textsanskrit{},text.
%
% English (see below for \textsanskrit)
%
\hyphenation{%
    dhanva-ntariṇopa-diṣ-ṭaḥ
    suśruta-nāma-dheyena
    tac-chiṣyeṇa
    kāśyapa-saṃ-hitā
    cikitsā-sthāna
    su-śruta-san-dīpana-bhāṣya
    dṛṣṭi-maṇḍala
    uc-chiṅga-na
    sarva-siddhānta-tattva-cūḍā-maṇi
    tulya-sau-vīrāñja-na
    indra-gopa
    śrī-mad-abhi-nava-guptā-cārya-vi-ra-cita-vi-vṛti-same-tam
    viśva-nātha
śrī-mad-devī-bhāga-vata-mahā-purāṇa
    siddhā-n-ta-sun-dara
    brāhma-sphuṭa-siddh-ānta
    bhū-ta-saṅ-khyā
    bhū-ta-saṃ-khyā
    kathi-ta-pada
    devī-bhā-ga-vata-purāṇa
    devī-bhā-ga-vata-mahā-purāṇa
    Siddhānta-saṃ-hitā-sāra-sam-uc-caya
    sau-ra-pau-rāṇi-ka-mata-sam-artha-na
    Pṛthū-da-ka-svā-min
    Brah-ma-gupta
    Brāh-ma-sphu-ṭa-siddhānta
    siddhānta-sun-dara
    vāsa-nā-bhāṣya
    catur-veda
    bhū-maṇḍala
    jñāna-rāja
    graha-gaṇi-ta-cintā-maṇi
    Śiṣya-dhī-vṛd-dhi-da-tan-tra
    brah-māṇḍa-pu-rā-ṇa
    kūr-ma-pu-rā-ṇa
    jam-bū-dvī-pa
    bhā-ga-vata-pu-rā-ṇa
    kupya-ka
    nandi-suttam
    nandi-sutta
    su-bodhiā-bāī
    asaṅ-khyāta
    saṅ-khyāta
    saṅ-khyā-pra-māṇa
    saṃ-khā-pamāṇa
    nemi-chandra
    anu-yoga-dvāra
    tattvārtha-vārtika
    aka-laṅka
    tri-loka-sāra
    gaṇi-ma-pra-māṇa
    gaṇi-ma-ppa-māṇa
    eka-pra-bhṛti
gaṇaṇā-saṃ-khā
gaṇaṇā-saṅ-khyā
dvi-pra-bhṛti
duppa-bhi-ti-saṃ-khā
vedanābhi-ghāta
Viṣṇu-dharmottara-pu-rāṇa
abhaya-deva-sūri-vi-racita-vṛtti-vi-bhūṣi-tam
abhi-dhar-ma
abhi-dhar-ma-ko-śa
abhi-dhar-ma-ko-śa-bhā-ṣya
abhi-dharma-kośa-bhāṣya
abhi-dharma-kośa-bhāṣyam
abhi-nava
abhyaṃ-karopāhva-vāsu-deva-śāstri-vi-ra-ci-ta-yā
ācārya-śrī-jina-vijayālekhitāgra-vacanālaṃ-kṛtaś-ca
ācāry-opā-hvena
ādhāra
adhi-kāra
adhi-kāras
ādi-nātha
agni-besha
agni-veśa
ahir-budhnya
ahir-budhnya-saṃ-hitā
aita-reya-brāhma-ṇa
akusī-dasya
amara-bharati
Amar-augha-pra-bo-dha
amṛ-ta-siddhi
ānanda-kanda
ānan-da-rā-ya
ānand-āśra-ma-mudraṇā-la-ya
ānand-āśra-ma-saṃ-skṛta-granth-āva-liḥ
anna-pāna-mūlā
anu-ban-dhya-lakṣaṇa-sam-anv-itās
anu-bhav-ād
anu-bhū-ta-viṣayā-sam-pra-moṣa
anu-bhū-ta-viṣayā-sam-pra-moṣaḥ
aparo-kṣā-nu-bhū-ti
app-proxi-mate-ly
ardha-rātrika-karaṇa
ārdha-rātrika-karaṇa
ariya-pary-esana-sutta
arun-dhatī
ārya-bhaṭa
ārya-bhaṭā-cārya-vi-racitam
ārya-bhaṭīya
ārya-bhaṭīyaṃ
ārya-lalita-vistara-nāma-mahā-yāna-sūtra
ārya-mañju-śrī-mūla-kalpa
ārya-mañju-śrī-mūla-kalpaḥ
asaṃ-pra-moṣa
aṣṭāṅga-hṛdaya-saṃ-hitā
aṣṭāṅga-saṃ-graha
asura-bhavana
aśva-ghoṣa
ātaṅka-darpaṇa-vyā-khyā-yā
atha-vā
ava-sāda-na
āyār-aṅga-suttaṃ
ayur-ved
ayur-veda
āyur-veda
āyur-veda-dīpikā
āyur-veda-dīpikā-vyā-khyayā
āyur-ve-da-ra-sā-yana
āyur-veda-sū-tra
ayur-vedic
āyur-vedic
ayur-yog
bādhirya
bahir-deśa-ka
bala-bhadra
bala-kot
bala-krishnan
bāla-kṛṣṇa
bau-dhā-yana-dhar-ma-sūtra
bel-valkar
bhadra-kālī-man-tra-vi-dhi-pra-karaṇa
bhadrā-sana
bhadrā-sanam
bha-ga-vat-pāda
bhaiṣajya-ratnāvalī
bhan-d-ar-kar
bhartṛhari-viracitaḥ
bhaṭṭā-cārya
bhaṭṭot-pala-vi-vṛti-sahitā
Bhiṣag-varāḍha-malla-vi-racita-dīpikā-Kāśī-rāma-vaidya-vi-raci-ta-gūḍhā-rtha-dīpikā-bhyāṃ
bhiṣag-varāḍha-malla-vi-racita-dīpikā-Kāśī-rāma-vaidya-vi-racita-gūḍhārtha-dīpikā-bhyāṃ
bhoja-deva-vi-raci-ta-rāja-mārtaṇḍā-bhi-dha-vṛtti-sam-e-tāni
bhu--va-na-dī-pa-ka
bīja-pallava
bi-kaner
bodhi-sat-tva-bhūmi
brahma-gupta
brahmā-nanda
brahmāṇḍa-mahā-purā-ṇa
brahmāṇḍa-mahā-purā-ṇam
brahma-randhra
brahma-siddh-ānta
brāhma-sphuṭa-siddh-ānta
brāhma-sphu-ṭa-siddhānta
brahma-vi-hāra
brahma-vi-hāras
brahma-yā-mala-tan-tra
Bra-ja-bhāṣā
bṛhad-āraṇya-ka
bṛhad-yā-trā
bṛhad-yogi-yājña-valkya-smṛti
bṛhad-yogī-yājña-valkya-smṛti
bṛhaj-jāta-kam
bṛhat-khe-carī-pra-kāśa
buddhi-tattva-pra-karaṇa
cak-ra-dat-ta
cakra-datta
cakra-pāṇi-datta
cā-luk-ya
caraka-prati-saṃ-s-kṛta
caraka-prati-saṃ-s-kṛte
caraka-saṃ-hitā
casam-ul-lasi-tāmaharṣiṇāsu-śrutenavi-raci-tāsu-śruta-saṃ-hitā
cau-kham-ba
cau-luk-yas
chandi-garh
chara-ka
cha-rīre
chatt-opa-dh-ya-ya
chau-kham-bha
chi-ki-tsi-ta
cid-ghanā-nanda-nātha
ci-ka-ner
com-men-taries
com-men-tary
com-pre-hen-sive-ly
daiva-jñālaṃ-kṛti
daiva-jñālaṅ-kṛti
dāmo-dara-sūnu-Śārṅga-dharācārya-vi-racitā
Dāmodara-sūnu-Śārṅga-dharācārya-vi-racitā
darśanā-ṅkur-ābhi-dhayā
das-gupta
deha-madhya
deha-saṃ-bhava-hetavaḥ
deva-datta
deva-nagari
deva-nāgarī
devā-sura-siddha-gaṇaiḥ
dha-ra-ni-dhar
dharma-megha
dharma-meghaḥ
dhru-vam
dhru-va-sya
dhru-va-yonir
dhyā-na-grahopa-deśā-dhyā-yaś
dṛḍha-śūla-yukta-rakta
dvy-ulbaṇaikolba-ṇ-aiḥ
four-fold
gan-dh-ā-ra
gārgīya-jyoti-ṣa
gārgya-ke-rala-nīla-kaṇṭha-so-ma-sutva-vi-racita-bhāṣyo-pe-tam
garuḍa-mahā-purāṇa
gaurī-kāñcali-kā-tan-tra
gau-tama
gauta-mādi-tra-yo-da-śa-smṛty-ātma-kaḥ
gheraṇḍa-saṃ-hitā
gorakṣa-śata-ka
go-tama
granth-ā-laya
grantha-mālā
gran-tha-śreṇiḥ
grāsa-pramāṇa
guru-maṇḍala-grantha-mālā
gyatso
hari-śāstrī
haṭhābhyāsa-paddhati
haṭha-ratnā-valī
Haṭha-saṅ-keta-candri-kā
haṭha-tattva-kau-mudī
haṭha-yoga
hāyana-rat-na
haya-ta-gran-tha
hema-pra-bha-sūri
hetu-lakṣaṇa-saṃ-sargād
hīna-madhyādhi-kaiś
hindī-vyā-khyā-vi-marśope-taḥ
hoern-le
ijya-rkṣa
ikka-vālaga
indra-dhvaja
indrāṇī-kalpa
indria
Īśāna-śiva-guru-deva-pad-dhati
jābāla-darśanopa-ni-ṣad
jadav-ji
jagan-nā-tha
jala-basti
jal-pa-kal-pa-tāru
jam-bū-dvī-pa-pra-jña-pti
jam-bū-dvī-pa-pra-jña-pti-sūtra
jana-pad-a-sya
jāta-ka-kar-ma-pad-dhati
jaya-siṃha
jinā-agama-grantha-mālā
jin-en-dra-bud-dhi
jīvan-muk-ti-vi-veka
jñā-na-nir-mala
jñā-na-nir-malaṃ
joga-pra-dīpya-kā
jya-rkṣe
Jyo-tiḥ-śās-tra
jyo-ti-ṣa-rāya
jyoti-ṣa-rāya
jyotiṣa-siddhānta-saṃ-graha
jyotiṣa-siddhānta-saṅ-graha
kāka-caṇḍīśvara-kal-pa-tan-tra
kakṣa-puṭa
kali-kāla-sarva-jña
kali-kāla-sarva-jña-śrī-hema-candrācārya-vi-raci-ta
kali-kāla-sarva-jña-śrī-hema-candrācārya-vi-raci-taḥ
kali-yuga
kal-pa
kal-pa-sthāna
kalyāṇa-kāraka
Kāmeśva-ra-siṃha-dara-bhaṅgā-saṃ-skṛta-viśva-vidyā-layaḥ
kapāla-bhāti
karaṇa-tilaka
kar-ma
kar-man
kāṭhaka-saṃ-hitā
kavia-rasu
kavi-raj
keśa-va-śāstrī
ke-vala--rāma
keva-la-rāma
khaṇḍa-khādyaka-tappā
khe-carī-vidyā
knowl-edge
kol-ka-ta
kriyā-krama-karī
kṛṣṇa-pakṣa
kṛtti-kā
kṛtti-kās
kubji-kā-mata-tantra
kula-pañji-kā
kul-karni
ku-māra-saṃ-bhava
kuṭi-pra-veśa
kuṭi-pra-veśika
lakṣ-mī-veṅ-kaṭ-e-ś-va-ra
lit-era-ture
lit-era-tures
locana-roga
mādha-va
mādhava-kara
mādhava-ni-dāna
mādhava-ni-dā-nam
madh-ūni
madhya
mādhyan-dina
madhye
mahā-bhāra-ta
mahā-deva
mahā-kavi-bhartṛ-hari-praṇīta-tvena
maha-mahopa-dhyaya
mahā-maho-pā-dhyā-ya-śrī-vi-jñā-na-bhikṣu-vi-raci-taṃ
mahā-mati-śrī-mādhava-kara-pra-ṇī-taṃ
mahā-mudrā
mahā-muni-śrī-mad-vyāsa-pra-ṇī-ta
mahā-muni-śrī-mad-vyāsa-pra-ṇī-taṃ
maharṣiṇā
maha-rṣi-pra-ṇīta-dharma-śāstra-saṃ-grahaḥ
Maha-rṣi-varya-śrī-yogi-yā-jña-valkya-śiṣya-vi-racitā
mahā-sacca-ka-sutta
mahā-sati-paṭṭhā-na-sutta
mahā-vra-ta
mahā-yāna-sūtrālaṅ-kāra
maitrāya-ṇī-saṃ-hitā
maktab-khānas
māla-jit
māli-nī-vijayot-tara-tan-tra
manaḥ-sam-ā-dhi
mānasol-lāsa
mānava-dharma-śāstra
mandāgni-doṣa
mannar-guḍi
mano-har-lal
mano-ratha-nandin
man-u-script
man-u-scripts
mataṅga-pārame-śvara
mater-ials
matsya-purāṇam
medh-ā-ti-thi
medhā-tithi
mithilā-stha
mithilā-stham
mithilā-sthaṃ
mṛgendra-tantra-vṛtti
mud-rā-yantr-ā-laye
muktā-pīḍa
mūla-pāṭha
muṇḍī-kalpa
mun-sh-ram
Nāda-bindū-pa-ni-ṣat
nāga-bodhi
nāga-buddhi
nakṣa-tra
nara-siṃha
nārā-yaṇa-dāsa
nārā-yaṇa-dāsa
nārā-yaṇa-kaṇṭha
nārā-yaṇa-paṇḍi-ta-kṛtā
nar-ra-tive
nata-rajan
nava-pañca-mayor
nava-re
naya-na-sukho-pā--dhyāya
ni-ban-dha-saṃ-grahā-khya-vyākhya-yā
niban-dha-san-graha
ni-dā-na
nidā-na-sthā-na-sya
ni-dāna-sthānasyaśrī-gaya-dāsācārya-vi-racitayānyāya-candri-kā-khya-pañjikā-vyā-khyayā
nir-anta-ra-pa-da-vyā-khyā
nir-guṇḍī-kalpa
nir-ṇaya-sā-gara
Nir-ṇaya-sāgara
nir-ṇa-ya-sā-gara-mudrā-yantrā-laye
nir-ṇa-ya-sā-ga-ra-yantr-āla-ya
nir-ṇaya-sā-gara-yantr-ā-laye
niśvāsa-kārikā
nīti-śṛṅgāra-vai-rāgyādi-nāmnāsamākhyā-tānāṃ
nityā-nanda
nya-grodha
nya-grodho
nyā-ya-candri-kā-khya-pañji-kā-vyā-khya-yā
nyāya-śās-tra
okaḥ-sātmya
okaḥ-sātmyam
okaḥ-sātmyaṃ
oka-sātmya
oka-sātmyam
oka-sātmyaṃ
oris-sa
oṣṭha-saṃ-puṭa
ousha-da-sala
padma-pra-bha-sūri
Padma-prā-bhṛ-ta-ka
padma-sva-sti-kārdha-candrādike
paitā-maha-siddhā-nta
pañca-karma
pañca-karman
pāñca-rātrā-gama
pañca-siddh-āntikā
paṅkti-śūla
Paraśu-rāma
paraśu-rāma
pari-likh-ya
pāśu-pata-sū-tra-bhāṣya
pātañ-jala-yoga-śās-tra
pātañ-jala-yoga-śās-tra-vi-varaṇa
pat-añ-jali
pat-na
pāva-suya
phiraṅgi-can-dra-cchedyo-pa-yogi-ka
pim-pal-gaon
pipal-gaon
pitta-śleṣ-man
pit-ta-śleṣ-ma-śoṇi-ta
pitta-śoṇi-ta
prā-cīna-rasa-granthaḥ
prā-cya
prā-cya-hindu-gran-tha-śreṇiḥ
prācya-vidyā-saṃ-śodhana-mandira
pra-dhān-in
pra-ka-shan
pra-kaṭa-mūṣā
pra-kṛ-ti-bhū-tāḥ
pra-mā-ṇa-vārt-tika
pra-ṇītā
pra-saṅ-khyāne
pra-śas-ta-pāda-bhāṣya
pra-śna-pra-dīpa
pra-śnārṇa-va-plava
praśnārṇava-plava
pra-śna-vai-ṣṇava
pra-śna-vaiṣṇava
prati-padyate
pra-yatna-śaithilyānan-ta-sam-āpatti-bhyām
prei-sen-danz
punar-vashu
puṇya-pattana
pūrṇi-mā-nta
raghu-nātha
rāja-kīya
rāja-kīya-mudraṇa-yantrā-laya
rāja-śe-khara
rajjv-ābhyas-ya
raj-put
rāj-put
rakta-mokṣa-na
rāma-candra-śāstrī
rāma-kṛṣṇa
rāma-kṛṣṇa-śāstri-ṇā
rama-su-bra-manian
rāmā-yaṇa
rasa-ratnā-kara
rasa-ratnākarāntar-ga-taś
rasa-ratna-sam-uc-caya
rasa-ratna-sam-uc-ca-yaḥ
rasa-vīry-auṣa-dha-pra-bhāvena
rasā-yana
rasendra-maṅgala
rasendra-maṅgalam
rāṣṭra-kūṭa
rāṣṭra-kūṭas
sādhana
śākalya-saṃ-hitā
śāla-grāma-kṛta
śāla-grāma-kṛta
sāmañña-pha-la-sutta
sāmañña-phala-sutta
sama-ran-gana-su-tra-dhara
samā-raṅga-ṇa-sū-tra-dhāra
sama-ra-siṃ-ha
sama-ra-siṃ-haḥ
sāmba-śiva-śāstri
same-taḥ
saṃ-hitā
śāṃ-ka-ra-bhāṣ-ya-sam-etā
sam-rāṭ
saṃ-rāṭ
Sam-rāṭ-siddhānta
Sam-rāṭ-siddhānta-kau-stu-bha
sam-rāṭ-siddhānta-kau-stu-bha
saṃ-sargam
saṃ-sargaṃ
saṃ-s-kṛta
saṃ-s-kṛta-pārasī-ka-pra-da-pra-kāśa
saṃ-śo-dhana
saṃ-śodhitā
saṃ-sthāna
sam-ullasitā
sam-ul-lasi-tam
saṃ-valitā
saṃ-valitā
śāndilyopa-ni-ṣad
śaṅ-kara
śaṅ-kara-bha-ga-vat-pāda
śaṅ-karā-cārya
san-kara-charya
Śaṅ-kara-nārā-yaṇa
sāṅ-kṛt-yā-yana
san-s-krit
śāra-dā-tila-ka-tan-tra
śa-raṅ-ga-deva
śār-dūla-karṇā-va-dāna
śār-dūla-karṇā-va-dāna
śā-rī-ra-sthāna
śārṅga-dhara-saṃ-hitā
Śārṅga-dhara-saṃ-hitā
sar-va-dar-śana-saṅ-gra-ha
sarva-kapha-ja
sarv-arthāvi-veka-khyā-ter
sar-va-śa-rīra-carās
sarva-siddhānta-rāja
Sarva-siddhā-nt-rāja
sarva-vyā-dhi-viṣāpa-ha
sarva-yoga-sam-uc-caya
sar-va-yogeśvareśva-ram
śāstrā-rambha-sam-artha-na
śatakatrayādi-subhāṣitasaṃgrahaḥ
sati-paṭṭhā-na-sutta
ṣaṭ-karma
ṣaṭ-karman
sat-karma-saṅ-graha
sat-karma-saṅ-grahaḥ
ṣaṭ-pañcā-śi-kā
saun-da-ra-nanda
sa-v-āī
schef-tel-o-witz
scholars
sharī-ra
sheth
sid-dha-man-tra
siddha-nanda-na-miśra
siddha-nanda-na-miśraḥ
siddha-nitya-nātha-pra-ṇītaḥ
Siddhānta-saṃ-hitā-sāra-sam-uc-caya
Siddhā-nta-sār-va-bhauma
siddhānta-sindhu
siddhānta-śiro-maṇ
Siddhānta-śiro-maṇi
Siddhā-nta-tat-tva-vi-veka
sid-dha-yoga
siddha-yoga
sid-dhi
sid-dhi-sthā-na
sid-dhi-sthāna
śikhi-sthāna
śiraḥ-karṇā-kṣi-vedana
śiro-bhūṣaṇam
Śivā-nanda-saras-vatī
śiva-saṃ-hitā
śiva-yo-ga-dī-pi-kā
ska-nda-pu-rā-ṇa
śleṣ-man
śleṣ-ma-śoni-ta
sodā-haraṇa-saṃ-s-kṛta-vyā-khyayā
śodha-ka-pusta-kaa
śoṇi-ta
spaṣ-ṭa-krānty-ādhi-kāra
śrī-cakra-pāṇi-datta
śrī-cakra-pāṇi-datta-viracitayā
śrī-ḍalhaṇācārya-vi-raci-tayāni-bandha-saṃ-grahākhya-vyā-khyayā
śrī-dayā-nanda
śrī-hari-kṛṣṇa-ni-bandha-bhava-nam
śrī-hema-candrā-cārya-vi-raci-taḥ
śrī-kaṇtha-dattā-bhyāṃ
śrī-kṛṣṇa-dāsa
śrī-kṛṣṇa-dāsa-śreṣṭhinā
śrīmac-chaṅ-kara-bhaga-vat-pāda-vi-raci-tā
śrī-mad-amara-siṃha-vi-racitam
śrī-mad-bha-ga-vad-gī-tā
śrī-mad-bhaṭṭot-pala-kṛta-saṃ-s-kṛta-ṭīkā-sahitam
śrī-mad-dvai-pā-yana-muni-pra-ṇītaṃ
śrī-mad-vāg-bhaṭa-vi-raci-tam
śrī-maṃ-trī-vi-jaya-siṃha-suta-maṃ-trī-teja-siṃhena
śrī-mat-kalyāṇa-varma-vi-racitā
śrī-mat-sāyaṇa-mādhavācārya-pra-ṇītaḥsarva-darśana-saṃ-grahaḥ
śrī-nitya-nātha-siddha-vi-raci-taḥ
śrī-rāja-śe-khara
śrī-śaṃ-karā-cārya-vi-raci-tam
śrī-vā-cas-pati-vaidya-vi-racita-yā
śrī-vatsa
śrī-veda-vyāsa-pra-ṇīta-mahā-bhā-ratāntar-ga-tā
śrī-veṅkaṭeś-vara
śrī-vi-jaya-rakṣi-ta
sruta-rakta
sruta-raktasya
stambha-karam
sthānāṅga-sūtra
sthira-sukha
sthira-sukham
stra-sthā-na
subhāṣitānāṃ
su-brah-man-ya
su-bra-man-ya
śukla-pakṣa
śukrā-srava
suk-than-kar
su-pariṣkṛta-saṃgrahaḥ
sura-bhi-pra-kash-an
sūrya-dāsa
sūrya-siddhānta
su-shru-ta
su-śru-ta
su-shru-ta-saṃ-hitā
su-śru-ta-saṃ-hitā
su-śru-tena
sutra
sūtra
sūtra-neti
sūtra-ni-dāna-śā-rīra-ci-ki-tsā-kal-pa-sthānot-tara-tan-trātma-kaḥ
sūtra-sthāna
su-varṇa-pra-bhāsot-tama-sū-tra
Su-var-ṇa-pra-bhās-ot-tama-sū-tra
su-varṇa-pra-bhāsotta-ma-sūtra
su-vistṛta-pari-cayātmikyāṅla-prastāvanā-vividha-pāṭhān-tara-pari-śiṣṭādi-sam-anvitaḥ
sva-bhāva-vyādhi-ni-vāraṇa-vi-śiṣṭ-auṣa-dha-cintakās
svā-bhāvika
svā-bhāvikās
sva-cchanda-tantra
śvetāśva-taropa-ni-ṣad
taila-sarpir-ma-dhūni
tait-tirīya-brāhma-ṇa
tājaka-muktā-valeḥ
tājika-kau-stu-bha
tājika-nīla-kaṇṭhī
tājika-yoga-sudhā-ni-dhi
tapo-dhana
tapo-dhanā
tārā-bhakti-su-dhārṇava
tārtīya-yoga-su-sudhā-ni-dhi
tegi-ccha
te-jaḥ-siṃ-ha
ṭhāṇ-āṅga-sutta
ṭīkā-bhyāṃ
ṭīkā-bhyāṃ
tiru-mantiram
tiru-ttoṇṭar-purāṇam
tiru-va-nanta-puram
trai-lok-ya
trai-lokya-pra-kāśa
tri-bhāga
tri-kam-ji
tri-pita-ka
tri-piṭa-ka
tri-vik-ra-mātma-jena
ud-ā-haraṇa
un-mārga-gama-na
upa-ca-ya-bala-varṇa-pra-sādādī-ni
upa-laghana
upa-ni-ṣads
upa-patt-ti
ut-sneha-na
utta-rā-dhya-ya-na
utta-rā-dhya-ya-na-sūtra
uttara-khaṇḍa-khādyaka
uttara-sthāna
uttara-tantra
vācas-pati-miśra-vi-racita-ṭīkā-saṃ-valita
vācas-pati-miśra-vi-racita-ṭīkā-saṃ-valita-vyā-sa-bhā-ṣya-sam-e-tāni
vag-bhata-rasa-ratna-sam-uc-caya
vāg-bhaṭa-rasa-ratna-sam-uc-caya
vaidya-vara-śrī-ḍalhaṇā-cārya-vi-racitayā
vai-śā-kha
vai-śeṣ-ika-sūtra
vāja-sa-neyi-saṃ-hitā
vājī-kara-ṇam
vākya-śeṣa
vākya-śeṣaḥ
vaṅga-sena
vaṅga-sena-saṃ-hitā
varā-ha-mihi-ra
vārāhī-kalpa
vā-rāṇa-seya
va-ra-na-si
var-mam
var-man
var-ṇa-saṃ-khyā
var-ṇa-saṅ-khyā
vā-si-ṣṭha
vasiṣṭha-saṃ-hitā
vā-siṣṭha-saṃ-hitā
Va-sistha-Sam-hita-Yoga-Kanda-With-Comm-ent-ary-Kai-valya-Dham
vastra-dhauti
vasu-bandhu
vāta-pit-ta
vāta-pit-ta-kapha
vāta-pit-ta-kapha-śoṇi-ta
vāta-pitta-kapha-śoṇita-san-nipāta-vai-ṣamya-ni-mittāḥ
vāta-pit-ta-śoṇi-ta
vāta-śleṣ-man
vāta-śleṣ-ma-śoṇi-ta
vāta-śoṇi-ta
vātā-tapika
vātsyā-ya-na
vāya-vīya-saṃ-hitā
vedāṅga-rāya
veezhi-nathan
venkat-raman
vid-vad-vara-śrī-gaṇeśa-daiva-jña-vi-racita
vidya-bhu-sana
vi-jaya-siṃ-ha
vi-jñāna-bhikṣu
Vijñāneśvara-vi-racita-mitākṣarā-vyā-khyā-sam-alaṅ-kṛtā
vi-mā-na
vi-mā-na-sthāna
vimāna-sthā-na
vi-racitā
vi-racita-yāmadhu-kośākhya-vyā-khya-yā
vi-recana
vishveshvar-anand
vi-śiṣṭ-āṃśena
vi-suddhi-magga
vi-vi-dha-tṛṇa-kāṣṭha-pāṣāṇa-pāṃ-su-loha-loṣṭāsthi-bāla-nakha-pūyā-srāva-duṣṭa-vraṇāntar-garbha-śalyo-ddharaṇārthaṃ
vṛd-dha-vṛd-dha-tara-vṛd-dha-tamaiḥ
vṛddha-vṛddha-tara-vṛddha-tamaiḥ
vṛnda-mādhava
vyāḍī-ya-pa-ri-bhā-ṣā-vṛtti
vyā-khya-yā
vy-akta-liṅgādi-dharma-yuk-te
vyā-sa-bhā-ṣya-sam-e-tāni
vyati-krāmati
Xiuyao
yādava-bhaṭṭa
yāda-va-śarma-ṇā
yādava-sūri
yājña-valkya-smṛti
yājña-valkya-smṛtiḥ
yantrā-dhyāya
Yantra-rāja-vicāra-viṃśā-dhyāyī
yavanā-cā-rya
yoga-bhā-ṣya-vyā-khyā-rūpaṃ
yoga-cintā-maṇi
yoga-cintā-maṇiḥ
yoga-ratnā-kara
yoga-sāra-mañjarī
yoga-sāra-sam-uc-caya
yoga-sāra-saṅ-graha
yoga-śikh-opa-ni-ṣat
yoga-tārā-valī
yoga-yājña-val-kya
yoga-yājña-valkya-gītāsūpa-ni-ṣatsu
yogi-yājña-valkya-smṛti
yoshi-mizu
yukta-bhava-deva
}
%%%%%%%%%%%%%%%%%%%%
%Sanskrit:
%%%%%%%%%%%%%%%%%%%%
\textsanskrit{\hyphenation{%
    dhanva-ntariṇopa-diṣ-ṭaḥ
suśruta-nāma-dheyena
tac-chiṣyeṇa
    su-śruta-san-dīpana-bhāṣya
    cikitsā-sthāna
tulya-sau-vīrāñjana
indra-gopa
dṛṣṭi-maṇḍala
uc-chiṅga-na
vi-vi-dha-tṛṇa-kāṣṭha-pāṣāṇa-pāṃ-su-loha-loṣṭāsthi-bāla-nakha-pūyā-srāva-duṣṭa-vraṇāntar-garbha-śalyo-ddharaṇārthaṃ
śrī-ḍalhaṇācārya-vi-raci-tayāni-bandha-saṃ-grahākhya-vyā-khyayā
ni-dāna-sthānasyaśrī-gaya-dāsācārya-vi-racitayānyāya-candri-kā-khya-pañjikā-vyā-khyayā
casam-ul-lasi-tāmaharṣiṇāsu-śrutenavi-raci-tāsu-śruta-saṃ-hitā
bhartṛhari-viracitaḥ
śatakatrayādi-subhāṣitasaṃgrahaḥ
mahā-kavi-bhartṛ-hari-praṇīta-tvena
nīti-śṛṅgāra-vai-rāgyādi-nāmnāsamākhyā-tānāṃ
subhāṣitānāṃ
su-pariṣkṛta-saṃgrahaḥ
su-vistṛta-pari-cayātmikyāṅla-prastāvanā-vividha-pāṭhān-tara-pari-śiṣṭādi-sam-anvitaḥ
ācārya-śrī-jina-vijayālekhitāgra-vacanālaṃ-kṛtaś-ca
abhaya-deva-sūri-vi-racita-vṛtti-vi-bhūṣi-tam
abhi-dhar-ma
abhi-dhar-ma-ko-śa
abhi-dhar-ma-ko-śa-bhā-ṣya
abhi-dharma-kośa-bhāṣyam
abhyaṃ-karopāhva-vāsu-deva-śāstri-vi-racita-yā
agni-veśa
āhā-ra-vi-hā-ra-pra-kṛ-tiṃ
ahir-budhnya
ahir-budhnya-saṃ-hitā
akusī-dasya
alter-na-tively
amara-bharati
amara-bhāratī
āmla
amlīkā
ānan-da-rā-ya
anna-mardanādi-bhiś
anu-bhav-ād
anu-bhū-ta-viṣayā-sam-pra-moṣa
anu-bhū-ta-viṣayā-sam-pra-moṣaḥ
anu-māna
anu-miti-mānasa-vāda
ariya-pary-esana-sutta
ārogya-śālā-karaṇā-sam-arthas
ārogya-śālām
ārogyāyopa-kal-pya
arś-āṃ-si
ar-tha
ar-thaḥ
ārya-bhaṭa
ārya-lalita-vistara-nāma-mahā-yāna-sūtra
ārya-mañju-śrī-mūla-kalpa
ārya-mañju-śrī-mūla-kalpaḥ
asaṃ-pra-moṣa
āsana
āsanam
āsanaṃ
asid-dhe
aṣṭāṅga-hṛdaya
aṣṭāṅga-hṛdaya-saṃ-hitā
aṣṭ-āṅga-saṅ-graha
aṣṭ-āṅgā-yur-veda
aśva-gan-dha-kalpa
aśva-ghoṣa
ātaṅka-darpaṇa
ātaṅka-darpaṇa-vyā-khyā-yā
atha-vā
ātu-r-ā-hā-ra-vi-hā-ra-pra-kṛ-tiṃ
aty-al-pam
auṣa-dha-pāvanādi-śālāś
ava-sāda-na
avic-chin-na-sam-pra-dāya-tvād
āyur-veda
āyur-veda-sāra
āyur-vedod-dhāra-ka-vaid-ya-pañc-ānana-vaid-ya-rat-na-rāja-vaid-ya-paṇḍi-ta-rā-ma-pra-sāda-vaid-yo-pādhyā-ya-vi-ra-ci-tā
bahir-deśa-ka
bala-bhadra
bāla-kṛṣṇa
bau-dhā-yana-dhar-ma-sūtra
bhadrā-sana
bhadrā-sanam
bha-ga-vad-gī-tā
bha-ga-vat-pāda
bhaṭṭot-pala-vi-vṛti-sahitā
bhṛtyāva-satha-saṃ-yuktām
bhū-miṃ
bhu--va-na-dī-pa-ka
bīja-pallava
bodhi-sat-tva-bhūmi
brāhmaṇa-pra-mukha-nānā-sat-tva-vyā-dhi-śānty-ar-tham
brāhmaṇa-pra-mukha-nānā-sat-tve-bhyo
brahmāṇḍa-mahā-purā-ṇa
brahmāṇḍa-mahā-purā-ṇam
brāhma-sphu-ṭa-siddhānta
brahma-vi-hāra
brahma-vi-hāras
bṛhad-āraṇya-ka
bṛhad-yā-trā
bṛhad-yogi-yājña-valkya-smṛti
bṛhad-yogī-yājña-valkya-smṛti
bṛhaj-jāta-kam
cak-ra-dat-ta
cak-ra-pā-ṇi-datta
cā-luk-ya
caraka-prati-saṃ-s-kṛta
caraka-prati-saṃ-s-kṛte
cara-ka-saṃ-hitā
ca-tur-thī-vi-bhak-ti
cau-kham-ba
cau-luk-yas
chau-kham-bha
chun-nam
cikit-sā-saṅ-gra-ha
daiva-jñālaṃ-kṛti
daiva-jñālaṅ-kṛti
darśa-nāṅkur-ābhi-dhayāvyā-khya-yā
deva-nagari
deva-nāgarī
dhar-ma-megha
dhar-ma-meghaḥ
dhyā-na-grahopa-deśā-dhyā-yaś
dṛṣṭ-ān-ta
dṛṣṭ-ār-tha
dvāra-tvam
evaṃ-gṛ-hī-tam
evaṃ-vi-dh-a-sya
gala-gaṇḍa
gala-gaṇḍādi-kar-tṛ-tvaṃ
gan-dh-ā-ra
gar-bha-śa-rī-ram
gaurī-kāñcali-kā-tan-tra
gauta-mādi-tra-yo-da-śa-smṛty-ātma-kaḥ
gheraṇḍa-saṃ-hitā
gran-tha-śreṇi
gran-tha-śreṇiḥ
guru-maṇḍala-grantha-mālā
hari-śāstrī
hari-śās-trī
haṭha-yoga
hāyana-rat-na
hema-pra-bha-sūri
hetv-ābhā-sa
hīna-mithy-āti-yoga
hīna-mithy-āti-yogena
hindī-vyā-khyā-vi-marśope-taḥ
hoern-le
idam
ijya-rkṣe
ikka-vālaga
ity-arthaḥ
jābāla-darśanopa-ni-ṣad
jal-pa-kal-pa-tāru
jam-bū-dvī-pa
jam-bū-dvī-pa-pra-jña-pti
jam-bū-dvī-pa-pra-jña-pti-sūtra
jāta-ka-kar-ma-pad-dhati
jinā-agama-grantha-mālā
jī-vā-nan-da-nam
jñā-na-nir-mala
jñā-na-nir-malaṃ
jya-rkṣe
kāka-caṇḍīśvara-kal-pa-tan-tra
kā-la-gar-bhā-śa-ya-pra-kṛ-tim
kā-la-gar-bhā-śa-ya-pra-kṛ-tiṃ
kali-kāla-sarva-jña
kali-kāla-sarva-jña-śrī-hema-candrācārya-vi-raci-ta
kali-kāla-sarva-jña-śrī-hema-candrācārya-vi-raci-taḥ
kali-yuga
kal-pa-sthāna
kar-ma
kar-man
kārt-snyena
katham
kāvya-mālā
keśa-va-śāstrī
kol-ka-ta
kṛṣṇa-pakṣa
kṛtti-kā
kṛtti-kās
kula-pañji-kā
ku-māra-saṃ-bhava
lab-dhāni
mada-na-phalam
mādha-va
Mādhava-karaaita-reya-brāhma-ṇa
Mādhava-ni-dāna
mādhava-ni-dā-nam
madhu-kośa
madhu-kośākhya-vyā-khya-yā
madhya
madhye
ma-hā-bhū-ta-vi-kā-ra-pra-kṛ-tiṃ
mahā-deva
mahā-mati-śrī-mādhava-kara-pra-ṇī-taṃ
mahā-muni-śrī-mad-vyāsa-pra-ṇī-ta
mahā-muni-śrī-mad-vyāsa-pra-ṇī-taṃ
maha-rṣi-pra-ṇīta-dharma-śāstra-saṃ-grahaḥ
mahā-sacca-ka-sutta
mahau-ṣadhi-pari-cchadāṃ
mahā-vra-ta
mahā-yāna-sūtrālaṅ-kāra
mano-ratha-nandin
matsya-purāṇam
me-dhā-ti-thi
medhā-tithi
mithilā-stha
mithilā-stham
mithilā-sthaṃ
mud-rā-yantr-ā-laye
muktā-pīḍa
mūla-pāṭha
nakṣa-tra
nandi-purāṇoktārogya-śālā-dāna-phala-prāpti-kāmo
nara-siṃha
nara-siṃha-bhāṣya
nārā-ya-ṇa-dāsa
nārā-yaṇa-kaṇṭha
nārā-yaṇa-paṇḍi-ta-kṛtā
nava-pañca-mayor
nidā-na-sthā-na-sya
ni-ghaṇ-ṭu
nir-anta-ra-pa-da-vyā-khyā
nir-ṇaya-sā-gara
nir-ṇaya-sā-gara-yantr-ā-laye
nirūha-vasti
niś-cala-kara
ni-yukta-vaidyāṃ
nya-grodha
nya-grodho
nyāya-śās-tra
nyāya-sū-tra-śaṃ-kar
okaḥ-sātmya
okaḥ-sātmyam
okaḥ-sātmyaṃ
oka-sātmya
oka-sātmyam
oka-sātmyaṃ
oṣṭha-saṃ-puṭa
ousha-da-sala
padma-pra-bha-sūri
padma-sva-sti-kārdha-candrādike
paitā-maha-siddhā-nta
pañca-karma
pañca-karma-bhava-rogāḥ
pañca-karmādhi-kāra
pañca-karma-vi-cāra
pāñca-rātrā-gama
pañca-siddh-āntikā
pari-bhāṣā
pari-likh-ya
pātañ-jala-yoga-śās-tra
pātañ-jala-yoga-śās-tra-vi-varaṇa
pat-añ-jali
pāṭī-gaṇita
pāva-suya
pim-pal-gaon
pipal-gaon
pit-ta-kṛt
pit-ta-śleṣma-ghna
pit-ta-śleṣma-medo-meha-hik-kā-śvā-sa-kā-sāti-sā-ra-cchardi-tṛṣṇā-kṛmi-vi-ṣa-pra-śa-ma-naṃ
prā-cya
prā-cya-hindu-gran-tha-śreṇiḥ
prācya-vidyā-saṃ-śodhana-mandira
pra-dhān-āṅ-gaṃ
pra-dhān-in
pra-ka-shan
pra-kṛ-ti
pra-kṛ-tiṃ
pra-mā-ṇa-vārt-tika
pra-saṅ-khyāne
pra-śas-ta-pāda-bhāṣya
pra-śna-pra-dīpa
pra-śnārṇa-va-plava
praśnārṇava-plava
pra-śna-vaiṣṇava
pra-śna-vai-ṣṇava
prati-padyate
pra-ty-akṣa
pra-yat-na-śai-thilyā-nan-ta-sam-ā-pat-ti-bhyām
pra-yat-na-śai-thilyā-nān-tya-sam-ā-pat-ti-bhyāṃ
pra-yatna-śai-thilya-sya
puṇya-pattana
pūrṇi-mā-nta
rāja-kīya
rajjv-ābhyas-ya
rāma-kṛṣṇa
rasa-ratnā-kara
rasa-vai-śeṣika-sūtra
rogi-svasthī-karaṇānu-ṣṭhāna-mātraṃ
rūkṣa-vasti
sād-guṇya
śākalya-saṃ-hitā
sam-ā-mnāya
sāmañña-pha-la-sutta
sama-ran-gana-su-tra-dhara
samā-raṅga-ṇa-sū-tra-dhāra
sama-ra-siṃ-ha
sama-ra-siṃ-haḥ
saṃ-hitā
sāṃ-sid-dhi-ka
saṃ-śo-dhana
sam-ul-lasi-tam
śāndilyopa-ni-ṣad
śaṅ-kara
śaṅ-kara-bha-ga-vat-pāda
Śaṅ-kara-nārā-yaṇa
saṅ-khyā
sāṅ-kṛt-yā-yana
san-s-krit
sap-tame
śāra-dā-tila-ka-tan-tra
śa-raṅ-ga-deva
śār-dūla-karṇā-va-dāna
śā-rī-ra
śā-rī-ra-sthāna
śārṅga-dhara
śārṅga-dhara-saṃ-hitā
sar-va
sarva-darśana-saṃ-grahaḥ
sar-va-dar-śāna-saṅ-gra-ha
sar-va-dar-śāna-saṅ-gra-haḥ
sarv-arthāvi-veka-khyā-ter
sar-va-tan-tra-sid-dhān-ta
sar-va-tan-tra-sid-dhān-taḥ
sarva-yoga-sam-uc-caya
sar-va-yogeśvareśva-ram
śāstrā-rambha-sam-artha-na
śāstrāram-bha-sam-arthana
ṣaṭ-pañcā-śi-kā
sat-tva
saunda-ra-na-nda
sid-dha
sid-dha-man-tra
sid-dha-man-trā-hvayo
sid-dha-man-tra-pra-kāśa
sid-dha-man-tra-pra-kāśaḥ
sid-dha-man-tra-pra-kāśaś
sid-dh-ān-ta
siddhānta-śiro-maṇ
sid-dha-yoga
sid-dhi-sthāna
śi-va-śar-ma-ṇā
ska-nda-pu-rā-ṇa
sneha-basty-upa-deśāt
sodā-haraṇa-saṃ-s-kṛta-vyā-khyayā
śodha-ka-pusta-kaṃ
śo-dha-na-ci-kitsā
so-ma-val-ka
śrī-mad-devī-bhāga-vata-mahā-purāṇa
srag-dharā-tārā-sto-tra
śrī-hari-kṛṣṇa-ni-bandha-bhava-nam
śrī-hema-candrā-cārya-vi-raci-taḥ
śrī-kaṇtha-datta
śrī-kaṇtha-dattā-bhyāṃ
śrī-kṛṣṇa-dāsa
śrī-mad-amara-siṃha-vi-racitam
śrī-mad-aruṇa-dat-ta-vi-ra-ci-tayā
śrī-mad-bhaṭṭot-pala-kṛta-saṃ-s-kṛta-ṭīkā-sahitam
śrī-mad-dvai-pā-yana-muni-pra-ṇītaṃ
śrī-mad-vāg-bha-ṭa-vi-ra-ci-tam
śrī-maṃ-trī-vi-jaya-siṃha-suta-maṃ-trī-teja-siṃhena
śrī-mat-kalyāṇa-varma-vi-racitā
śrīmat-sāyaṇa-mādhavācārya-pra-ṇītaḥ
śrī-vā-cas-pati-vaidya-vi-racita-yā
śrī-vatsa
śrī-vi-jaya-rakṣi-ta
sthānāṅga-sūtra
sthira-sukha
sthira-sukham
strī-niṣevaṇa
śukla-pakṣa
su-śru-ta-saṃ-hitā
sū-tra
sūtrārthānān-upa-patti-sūca-nāt
sūtra-sthāna
su-varṇa-pra-bhāsot-tama-sū-tra
svalpauṣadha-dāna-mā-tram
śvetāśva-taropa-ni-ṣad
tad-upa-karaṇa-tāmra-kaṭāha-kalasādi-pātra-pari-cchada-nānā-vidha-vyādhi-śānty-ucitauṣadha-gaṇa-yathokta-lakṣaṇa-vaidya-nānā-vidha-pari-cāraka-yutāṃ
tājaka-muktā-valeḥ
tājika-kau-stu-bha
tājika-nīla-kaṇṭhī
tājika-yoga-sudhā-ni-dhi
tāmra-paṭṭādi-li-khi-tāṃ
tan-nir-vāhāya
tapo-dhana
tapo-dhanā
tārā-bhakti-su-dhārṇava
tārtīya-yoga-su-sudhā-ni-dhi
tegi-ccha
te-jaḥ-siṃ-ha
trai-lok-ya
trai-lokya-pra-kāśa
tri-piṭa-ka
tri-var-gaḥ
un-mār-ga-gama-na
upa-de-śa
upa-patt-ti
ut-sneha-na
utta-rā-dhyā-ya-na
uttara-sthāna
uttara-tantra
vāchas-pati
vād-ā-valī
vai-śā-kha
vai-ta-raṇa-vasti
vai-ta-raṇok-ta-guṇa-gaṇa-yu-k-taṃ
vājī-kara-ṇam
vāk-patis
vākya-śeṣa
vākya-śeṣaḥ
varā-ha-mihi-ra
va-ra-na-si
vā-rā-ṇa-sī
var-mam
var-man
varṇa-sam-ā-mnāya
va-siṣṭha-saṃ-hitā
vā-siṣṭha-saṃ-hitā
vasu-bandhu
vasu-bandhu
vāta-ghna-pit-talāl-pa-ka-pha
vātsyā-ya-na
vidya-bhu-sana
vidyā-bhū-ṣaṇa
vi-jaya-siṃ-ha
vi-jñāna-bhikṣu
vi-kal-pa
vi-kamp-i-tum
vi-mā-na-sthāna
vi-racita-yā
vishveshvar-anand
vi-śiṣṭ-āṃśena
viṣṇu-dharmot-tara-purāṇa
viśrāma-gṛha-sahitā
vi-suddhi-magga
vopa-de-vīya-sid-dha-man-tra-pra-kāśe
vyādhi-pratī-kārār-tham
vyāḍī-ya-pa-ri-bhā-ṣā-vṛtti
vyati-krāmati
vy-ava-haranti
yādava-bhaṭṭa
yāda-va-śarma-ṇā
yādava-sūri
yājña-valkya-smṛti
yavanā-cā-rya
yoga-ratnā-kara
yoga-sāra-sam-uc-caya
yoga-sāra-sam-uc-cayaḥ
yoga-sūtra-vi-vara-ṇa
yoga-yājña-valkya
yoga-yājña-valkya-gītāsūpa-ni-ṣatsu
yoga-yājña-valkyaḥ
yogi-yājña-valkya-smṛti
yuk-tiḥ
yuk-tis
}}
\normalfontlatin
\endinput
}% should work, but doesn't
% special hyphenations for Sanskrit words tagged in
% Polyglossia.
% *English,\textenglish{},text,and
% *Sanskrit,\textsanskrit{},text.
%
% English (see below for \textsanskrit)
%
\hyphenation{%
    dhanva-ntariṇopa-diṣ-ṭaḥ
    suśruta-nāma-dheyena
    tac-chiṣyeṇa
    kāśyapa-saṃ-hitā
    cikitsā-sthāna
    su-śruta-san-dīpana-bhāṣya
    dṛṣṭi-maṇḍala
    uc-chiṅga-na
    sarva-siddhānta-tattva-cūḍā-maṇi
    tulya-sau-vīrāñja-na
    indra-gopa
    śrī-mad-abhi-nava-guptā-cārya-vi-ra-cita-vi-vṛti-same-tam
    viśva-nātha
śrī-mad-devī-bhāga-vata-mahā-purāṇa
    siddhā-n-ta-sun-dara
    brāhma-sphuṭa-siddh-ānta
    bhū-ta-saṅ-khyā
    bhū-ta-saṃ-khyā
    kathi-ta-pada
    devī-bhā-ga-vata-purāṇa
    devī-bhā-ga-vata-mahā-purāṇa
    Siddhānta-saṃ-hitā-sāra-sam-uc-caya
    sau-ra-pau-rāṇi-ka-mata-sam-artha-na
    Pṛthū-da-ka-svā-min
    Brah-ma-gupta
    Brāh-ma-sphu-ṭa-siddhānta
    siddhānta-sun-dara
    vāsa-nā-bhāṣya
    catur-veda
    bhū-maṇḍala
    jñāna-rāja
    graha-gaṇi-ta-cintā-maṇi
    Śiṣya-dhī-vṛd-dhi-da-tan-tra
    brah-māṇḍa-pu-rā-ṇa
    kūr-ma-pu-rā-ṇa
    jam-bū-dvī-pa
    bhā-ga-vata-pu-rā-ṇa
    kupya-ka
    nandi-suttam
    nandi-sutta
    su-bodhiā-bāī
    asaṅ-khyāta
    saṅ-khyāta
    saṅ-khyā-pra-māṇa
    saṃ-khā-pamāṇa
    nemi-chandra
    anu-yoga-dvāra
    tattvārtha-vārtika
    aka-laṅka
    tri-loka-sāra
    gaṇi-ma-pra-māṇa
    gaṇi-ma-ppa-māṇa
    eka-pra-bhṛti
gaṇaṇā-saṃ-khā
gaṇaṇā-saṅ-khyā
dvi-pra-bhṛti
duppa-bhi-ti-saṃ-khā
vedanābhi-ghāta
Viṣṇu-dharmottara-pu-rāṇa
abhaya-deva-sūri-vi-racita-vṛtti-vi-bhūṣi-tam
abhi-dhar-ma
abhi-dhar-ma-ko-śa
abhi-dhar-ma-ko-śa-bhā-ṣya
abhi-dharma-kośa-bhāṣya
abhi-dharma-kośa-bhāṣyam
abhi-nava
abhyaṃ-karopāhva-vāsu-deva-śāstri-vi-ra-ci-ta-yā
ācārya-śrī-jina-vijayālekhitāgra-vacanālaṃ-kṛtaś-ca
ācāry-opā-hvena
ādhāra
adhi-kāra
adhi-kāras
ādi-nātha
agni-besha
agni-veśa
ahir-budhnya
ahir-budhnya-saṃ-hitā
aita-reya-brāhma-ṇa
akusī-dasya
amara-bharati
Amar-augha-pra-bo-dha
amṛ-ta-siddhi
ānanda-kanda
ānan-da-rā-ya
ānand-āśra-ma-mudraṇā-la-ya
ānand-āśra-ma-saṃ-skṛta-granth-āva-liḥ
anna-pāna-mūlā
anu-ban-dhya-lakṣaṇa-sam-anv-itās
anu-bhav-ād
anu-bhū-ta-viṣayā-sam-pra-moṣa
anu-bhū-ta-viṣayā-sam-pra-moṣaḥ
aparo-kṣā-nu-bhū-ti
app-proxi-mate-ly
ardha-rātrika-karaṇa
ārdha-rātrika-karaṇa
ariya-pary-esana-sutta
arun-dhatī
ārya-bhaṭa
ārya-bhaṭā-cārya-vi-racitam
ārya-bhaṭīya
ārya-bhaṭīyaṃ
ārya-lalita-vistara-nāma-mahā-yāna-sūtra
ārya-mañju-śrī-mūla-kalpa
ārya-mañju-śrī-mūla-kalpaḥ
asaṃ-pra-moṣa
aṣṭāṅga-hṛdaya-saṃ-hitā
aṣṭāṅga-saṃ-graha
asura-bhavana
aśva-ghoṣa
ātaṅka-darpaṇa-vyā-khyā-yā
atha-vā
ava-sāda-na
āyār-aṅga-suttaṃ
ayur-ved
ayur-veda
āyur-veda
āyur-veda-dīpikā
āyur-veda-dīpikā-vyā-khyayā
āyur-ve-da-ra-sā-yana
āyur-veda-sū-tra
ayur-vedic
āyur-vedic
ayur-yog
bādhirya
bahir-deśa-ka
bala-bhadra
bala-kot
bala-krishnan
bāla-kṛṣṇa
bau-dhā-yana-dhar-ma-sūtra
bel-valkar
bhadra-kālī-man-tra-vi-dhi-pra-karaṇa
bhadrā-sana
bhadrā-sanam
bha-ga-vat-pāda
bhaiṣajya-ratnāvalī
bhan-d-ar-kar
bhartṛhari-viracitaḥ
bhaṭṭā-cārya
bhaṭṭot-pala-vi-vṛti-sahitā
Bhiṣag-varāḍha-malla-vi-racita-dīpikā-Kāśī-rāma-vaidya-vi-raci-ta-gūḍhā-rtha-dīpikā-bhyāṃ
bhiṣag-varāḍha-malla-vi-racita-dīpikā-Kāśī-rāma-vaidya-vi-racita-gūḍhārtha-dīpikā-bhyāṃ
bhoja-deva-vi-raci-ta-rāja-mārtaṇḍā-bhi-dha-vṛtti-sam-e-tāni
bhu--va-na-dī-pa-ka
bīja-pallava
bi-kaner
bodhi-sat-tva-bhūmi
brahma-gupta
brahmā-nanda
brahmāṇḍa-mahā-purā-ṇa
brahmāṇḍa-mahā-purā-ṇam
brahma-randhra
brahma-siddh-ānta
brāhma-sphuṭa-siddh-ānta
brāhma-sphu-ṭa-siddhānta
brahma-vi-hāra
brahma-vi-hāras
brahma-yā-mala-tan-tra
Bra-ja-bhāṣā
bṛhad-āraṇya-ka
bṛhad-yā-trā
bṛhad-yogi-yājña-valkya-smṛti
bṛhad-yogī-yājña-valkya-smṛti
bṛhaj-jāta-kam
bṛhat-khe-carī-pra-kāśa
buddhi-tattva-pra-karaṇa
cak-ra-dat-ta
cakra-datta
cakra-pāṇi-datta
cā-luk-ya
caraka-prati-saṃ-s-kṛta
caraka-prati-saṃ-s-kṛte
caraka-saṃ-hitā
casam-ul-lasi-tāmaharṣiṇāsu-śrutenavi-raci-tāsu-śruta-saṃ-hitā
cau-kham-ba
cau-luk-yas
chandi-garh
chara-ka
cha-rīre
chatt-opa-dh-ya-ya
chau-kham-bha
chi-ki-tsi-ta
cid-ghanā-nanda-nātha
ci-ka-ner
com-men-taries
com-men-tary
com-pre-hen-sive-ly
daiva-jñālaṃ-kṛti
daiva-jñālaṅ-kṛti
dāmo-dara-sūnu-Śārṅga-dharācārya-vi-racitā
Dāmodara-sūnu-Śārṅga-dharācārya-vi-racitā
darśanā-ṅkur-ābhi-dhayā
das-gupta
deha-madhya
deha-saṃ-bhava-hetavaḥ
deva-datta
deva-nagari
deva-nāgarī
devā-sura-siddha-gaṇaiḥ
dha-ra-ni-dhar
dharma-megha
dharma-meghaḥ
dhru-vam
dhru-va-sya
dhru-va-yonir
dhyā-na-grahopa-deśā-dhyā-yaś
dṛḍha-śūla-yukta-rakta
dvy-ulbaṇaikolba-ṇ-aiḥ
four-fold
gan-dh-ā-ra
gārgīya-jyoti-ṣa
gārgya-ke-rala-nīla-kaṇṭha-so-ma-sutva-vi-racita-bhāṣyo-pe-tam
garuḍa-mahā-purāṇa
gaurī-kāñcali-kā-tan-tra
gau-tama
gauta-mādi-tra-yo-da-śa-smṛty-ātma-kaḥ
gheraṇḍa-saṃ-hitā
gorakṣa-śata-ka
go-tama
granth-ā-laya
grantha-mālā
gran-tha-śreṇiḥ
grāsa-pramāṇa
guru-maṇḍala-grantha-mālā
gyatso
hari-śāstrī
haṭhābhyāsa-paddhati
haṭha-ratnā-valī
Haṭha-saṅ-keta-candri-kā
haṭha-tattva-kau-mudī
haṭha-yoga
hāyana-rat-na
haya-ta-gran-tha
hema-pra-bha-sūri
hetu-lakṣaṇa-saṃ-sargād
hīna-madhyādhi-kaiś
hindī-vyā-khyā-vi-marśope-taḥ
hoern-le
ijya-rkṣa
ikka-vālaga
indra-dhvaja
indrāṇī-kalpa
indria
Īśāna-śiva-guru-deva-pad-dhati
jābāla-darśanopa-ni-ṣad
jadav-ji
jagan-nā-tha
jala-basti
jal-pa-kal-pa-tāru
jam-bū-dvī-pa-pra-jña-pti
jam-bū-dvī-pa-pra-jña-pti-sūtra
jana-pad-a-sya
jāta-ka-kar-ma-pad-dhati
jaya-siṃha
jinā-agama-grantha-mālā
jin-en-dra-bud-dhi
jīvan-muk-ti-vi-veka
jñā-na-nir-mala
jñā-na-nir-malaṃ
joga-pra-dīpya-kā
jya-rkṣe
Jyo-tiḥ-śās-tra
jyo-ti-ṣa-rāya
jyoti-ṣa-rāya
jyotiṣa-siddhānta-saṃ-graha
jyotiṣa-siddhānta-saṅ-graha
kāka-caṇḍīśvara-kal-pa-tan-tra
kakṣa-puṭa
kali-kāla-sarva-jña
kali-kāla-sarva-jña-śrī-hema-candrācārya-vi-raci-ta
kali-kāla-sarva-jña-śrī-hema-candrācārya-vi-raci-taḥ
kali-yuga
kal-pa
kal-pa-sthāna
kalyāṇa-kāraka
Kāmeśva-ra-siṃha-dara-bhaṅgā-saṃ-skṛta-viśva-vidyā-layaḥ
kapāla-bhāti
karaṇa-tilaka
kar-ma
kar-man
kāṭhaka-saṃ-hitā
kavia-rasu
kavi-raj
keśa-va-śāstrī
ke-vala--rāma
keva-la-rāma
khaṇḍa-khādyaka-tappā
khe-carī-vidyā
knowl-edge
kol-ka-ta
kriyā-krama-karī
kṛṣṇa-pakṣa
kṛtti-kā
kṛtti-kās
kubji-kā-mata-tantra
kula-pañji-kā
kul-karni
ku-māra-saṃ-bhava
kuṭi-pra-veśa
kuṭi-pra-veśika
lakṣ-mī-veṅ-kaṭ-e-ś-va-ra
lit-era-ture
lit-era-tures
locana-roga
mādha-va
mādhava-kara
mādhava-ni-dāna
mādhava-ni-dā-nam
madh-ūni
madhya
mādhyan-dina
madhye
mahā-bhāra-ta
mahā-deva
mahā-kavi-bhartṛ-hari-praṇīta-tvena
maha-mahopa-dhyaya
mahā-maho-pā-dhyā-ya-śrī-vi-jñā-na-bhikṣu-vi-raci-taṃ
mahā-mati-śrī-mādhava-kara-pra-ṇī-taṃ
mahā-mudrā
mahā-muni-śrī-mad-vyāsa-pra-ṇī-ta
mahā-muni-śrī-mad-vyāsa-pra-ṇī-taṃ
maharṣiṇā
maha-rṣi-pra-ṇīta-dharma-śāstra-saṃ-grahaḥ
Maha-rṣi-varya-śrī-yogi-yā-jña-valkya-śiṣya-vi-racitā
mahā-sacca-ka-sutta
mahā-sati-paṭṭhā-na-sutta
mahā-vra-ta
mahā-yāna-sūtrālaṅ-kāra
maitrāya-ṇī-saṃ-hitā
maktab-khānas
māla-jit
māli-nī-vijayot-tara-tan-tra
manaḥ-sam-ā-dhi
mānasol-lāsa
mānava-dharma-śāstra
mandāgni-doṣa
mannar-guḍi
mano-har-lal
mano-ratha-nandin
man-u-script
man-u-scripts
mataṅga-pārame-śvara
mater-ials
matsya-purāṇam
medh-ā-ti-thi
medhā-tithi
mithilā-stha
mithilā-stham
mithilā-sthaṃ
mṛgendra-tantra-vṛtti
mud-rā-yantr-ā-laye
muktā-pīḍa
mūla-pāṭha
muṇḍī-kalpa
mun-sh-ram
Nāda-bindū-pa-ni-ṣat
nāga-bodhi
nāga-buddhi
nakṣa-tra
nara-siṃha
nārā-yaṇa-dāsa
nārā-yaṇa-dāsa
nārā-yaṇa-kaṇṭha
nārā-yaṇa-paṇḍi-ta-kṛtā
nar-ra-tive
nata-rajan
nava-pañca-mayor
nava-re
naya-na-sukho-pā--dhyāya
ni-ban-dha-saṃ-grahā-khya-vyākhya-yā
niban-dha-san-graha
ni-dā-na
nidā-na-sthā-na-sya
ni-dāna-sthānasyaśrī-gaya-dāsācārya-vi-racitayānyāya-candri-kā-khya-pañjikā-vyā-khyayā
nir-anta-ra-pa-da-vyā-khyā
nir-guṇḍī-kalpa
nir-ṇaya-sā-gara
Nir-ṇaya-sāgara
nir-ṇa-ya-sā-gara-mudrā-yantrā-laye
nir-ṇa-ya-sā-ga-ra-yantr-āla-ya
nir-ṇaya-sā-gara-yantr-ā-laye
niśvāsa-kārikā
nīti-śṛṅgāra-vai-rāgyādi-nāmnāsamākhyā-tānāṃ
nityā-nanda
nya-grodha
nya-grodho
nyā-ya-candri-kā-khya-pañji-kā-vyā-khya-yā
nyāya-śās-tra
okaḥ-sātmya
okaḥ-sātmyam
okaḥ-sātmyaṃ
oka-sātmya
oka-sātmyam
oka-sātmyaṃ
oris-sa
oṣṭha-saṃ-puṭa
ousha-da-sala
padma-pra-bha-sūri
Padma-prā-bhṛ-ta-ka
padma-sva-sti-kārdha-candrādike
paitā-maha-siddhā-nta
pañca-karma
pañca-karman
pāñca-rātrā-gama
pañca-siddh-āntikā
paṅkti-śūla
Paraśu-rāma
paraśu-rāma
pari-likh-ya
pāśu-pata-sū-tra-bhāṣya
pātañ-jala-yoga-śās-tra
pātañ-jala-yoga-śās-tra-vi-varaṇa
pat-añ-jali
pat-na
pāva-suya
phiraṅgi-can-dra-cchedyo-pa-yogi-ka
pim-pal-gaon
pipal-gaon
pitta-śleṣ-man
pit-ta-śleṣ-ma-śoṇi-ta
pitta-śoṇi-ta
prā-cīna-rasa-granthaḥ
prā-cya
prā-cya-hindu-gran-tha-śreṇiḥ
prācya-vidyā-saṃ-śodhana-mandira
pra-dhān-in
pra-ka-shan
pra-kaṭa-mūṣā
pra-kṛ-ti-bhū-tāḥ
pra-mā-ṇa-vārt-tika
pra-ṇītā
pra-saṅ-khyāne
pra-śas-ta-pāda-bhāṣya
pra-śna-pra-dīpa
pra-śnārṇa-va-plava
praśnārṇava-plava
pra-śna-vai-ṣṇava
pra-śna-vaiṣṇava
prati-padyate
pra-yatna-śaithilyānan-ta-sam-āpatti-bhyām
prei-sen-danz
punar-vashu
puṇya-pattana
pūrṇi-mā-nta
raghu-nātha
rāja-kīya
rāja-kīya-mudraṇa-yantrā-laya
rāja-śe-khara
rajjv-ābhyas-ya
raj-put
rāj-put
rakta-mokṣa-na
rāma-candra-śāstrī
rāma-kṛṣṇa
rāma-kṛṣṇa-śāstri-ṇā
rama-su-bra-manian
rāmā-yaṇa
rasa-ratnā-kara
rasa-ratnākarāntar-ga-taś
rasa-ratna-sam-uc-caya
rasa-ratna-sam-uc-ca-yaḥ
rasa-vīry-auṣa-dha-pra-bhāvena
rasā-yana
rasendra-maṅgala
rasendra-maṅgalam
rāṣṭra-kūṭa
rāṣṭra-kūṭas
sādhana
śākalya-saṃ-hitā
śāla-grāma-kṛta
śāla-grāma-kṛta
sāmañña-pha-la-sutta
sāmañña-phala-sutta
sama-ran-gana-su-tra-dhara
samā-raṅga-ṇa-sū-tra-dhāra
sama-ra-siṃ-ha
sama-ra-siṃ-haḥ
sāmba-śiva-śāstri
same-taḥ
saṃ-hitā
śāṃ-ka-ra-bhāṣ-ya-sam-etā
sam-rāṭ
saṃ-rāṭ
Sam-rāṭ-siddhānta
Sam-rāṭ-siddhānta-kau-stu-bha
sam-rāṭ-siddhānta-kau-stu-bha
saṃ-sargam
saṃ-sargaṃ
saṃ-s-kṛta
saṃ-s-kṛta-pārasī-ka-pra-da-pra-kāśa
saṃ-śo-dhana
saṃ-śodhitā
saṃ-sthāna
sam-ullasitā
sam-ul-lasi-tam
saṃ-valitā
saṃ-valitā
śāndilyopa-ni-ṣad
śaṅ-kara
śaṅ-kara-bha-ga-vat-pāda
śaṅ-karā-cārya
san-kara-charya
Śaṅ-kara-nārā-yaṇa
sāṅ-kṛt-yā-yana
san-s-krit
śāra-dā-tila-ka-tan-tra
śa-raṅ-ga-deva
śār-dūla-karṇā-va-dāna
śār-dūla-karṇā-va-dāna
śā-rī-ra-sthāna
śārṅga-dhara-saṃ-hitā
Śārṅga-dhara-saṃ-hitā
sar-va-dar-śana-saṅ-gra-ha
sarva-kapha-ja
sarv-arthāvi-veka-khyā-ter
sar-va-śa-rīra-carās
sarva-siddhānta-rāja
Sarva-siddhā-nt-rāja
sarva-vyā-dhi-viṣāpa-ha
sarva-yoga-sam-uc-caya
sar-va-yogeśvareśva-ram
śāstrā-rambha-sam-artha-na
śatakatrayādi-subhāṣitasaṃgrahaḥ
sati-paṭṭhā-na-sutta
ṣaṭ-karma
ṣaṭ-karman
sat-karma-saṅ-graha
sat-karma-saṅ-grahaḥ
ṣaṭ-pañcā-śi-kā
saun-da-ra-nanda
sa-v-āī
schef-tel-o-witz
scholars
sharī-ra
sheth
sid-dha-man-tra
siddha-nanda-na-miśra
siddha-nanda-na-miśraḥ
siddha-nitya-nātha-pra-ṇītaḥ
Siddhānta-saṃ-hitā-sāra-sam-uc-caya
Siddhā-nta-sār-va-bhauma
siddhānta-sindhu
siddhānta-śiro-maṇ
Siddhānta-śiro-maṇi
Siddhā-nta-tat-tva-vi-veka
sid-dha-yoga
siddha-yoga
sid-dhi
sid-dhi-sthā-na
sid-dhi-sthāna
śikhi-sthāna
śiraḥ-karṇā-kṣi-vedana
śiro-bhūṣaṇam
Śivā-nanda-saras-vatī
śiva-saṃ-hitā
śiva-yo-ga-dī-pi-kā
ska-nda-pu-rā-ṇa
śleṣ-man
śleṣ-ma-śoni-ta
sodā-haraṇa-saṃ-s-kṛta-vyā-khyayā
śodha-ka-pusta-kaa
śoṇi-ta
spaṣ-ṭa-krānty-ādhi-kāra
śrī-cakra-pāṇi-datta
śrī-cakra-pāṇi-datta-viracitayā
śrī-ḍalhaṇācārya-vi-raci-tayāni-bandha-saṃ-grahākhya-vyā-khyayā
śrī-dayā-nanda
śrī-hari-kṛṣṇa-ni-bandha-bhava-nam
śrī-hema-candrā-cārya-vi-raci-taḥ
śrī-kaṇtha-dattā-bhyāṃ
śrī-kṛṣṇa-dāsa
śrī-kṛṣṇa-dāsa-śreṣṭhinā
śrīmac-chaṅ-kara-bhaga-vat-pāda-vi-raci-tā
śrī-mad-amara-siṃha-vi-racitam
śrī-mad-bha-ga-vad-gī-tā
śrī-mad-bhaṭṭot-pala-kṛta-saṃ-s-kṛta-ṭīkā-sahitam
śrī-mad-dvai-pā-yana-muni-pra-ṇītaṃ
śrī-mad-vāg-bhaṭa-vi-raci-tam
śrī-maṃ-trī-vi-jaya-siṃha-suta-maṃ-trī-teja-siṃhena
śrī-mat-kalyāṇa-varma-vi-racitā
śrī-mat-sāyaṇa-mādhavācārya-pra-ṇītaḥsarva-darśana-saṃ-grahaḥ
śrī-nitya-nātha-siddha-vi-raci-taḥ
śrī-rāja-śe-khara
śrī-śaṃ-karā-cārya-vi-raci-tam
śrī-vā-cas-pati-vaidya-vi-racita-yā
śrī-vatsa
śrī-veda-vyāsa-pra-ṇīta-mahā-bhā-ratāntar-ga-tā
śrī-veṅkaṭeś-vara
śrī-vi-jaya-rakṣi-ta
sruta-rakta
sruta-raktasya
stambha-karam
sthānāṅga-sūtra
sthira-sukha
sthira-sukham
stra-sthā-na
subhāṣitānāṃ
su-brah-man-ya
su-bra-man-ya
śukla-pakṣa
śukrā-srava
suk-than-kar
su-pariṣkṛta-saṃgrahaḥ
sura-bhi-pra-kash-an
sūrya-dāsa
sūrya-siddhānta
su-shru-ta
su-śru-ta
su-shru-ta-saṃ-hitā
su-śru-ta-saṃ-hitā
su-śru-tena
sutra
sūtra
sūtra-neti
sūtra-ni-dāna-śā-rīra-ci-ki-tsā-kal-pa-sthānot-tara-tan-trātma-kaḥ
sūtra-sthāna
su-varṇa-pra-bhāsot-tama-sū-tra
Su-var-ṇa-pra-bhās-ot-tama-sū-tra
su-varṇa-pra-bhāsotta-ma-sūtra
su-vistṛta-pari-cayātmikyāṅla-prastāvanā-vividha-pāṭhān-tara-pari-śiṣṭādi-sam-anvitaḥ
sva-bhāva-vyādhi-ni-vāraṇa-vi-śiṣṭ-auṣa-dha-cintakās
svā-bhāvika
svā-bhāvikās
sva-cchanda-tantra
śvetāśva-taropa-ni-ṣad
taila-sarpir-ma-dhūni
tait-tirīya-brāhma-ṇa
tājaka-muktā-valeḥ
tājika-kau-stu-bha
tājika-nīla-kaṇṭhī
tājika-yoga-sudhā-ni-dhi
tapo-dhana
tapo-dhanā
tārā-bhakti-su-dhārṇava
tārtīya-yoga-su-sudhā-ni-dhi
tegi-ccha
te-jaḥ-siṃ-ha
ṭhāṇ-āṅga-sutta
ṭīkā-bhyāṃ
ṭīkā-bhyāṃ
tiru-mantiram
tiru-ttoṇṭar-purāṇam
tiru-va-nanta-puram
trai-lok-ya
trai-lokya-pra-kāśa
tri-bhāga
tri-kam-ji
tri-pita-ka
tri-piṭa-ka
tri-vik-ra-mātma-jena
ud-ā-haraṇa
un-mārga-gama-na
upa-ca-ya-bala-varṇa-pra-sādādī-ni
upa-laghana
upa-ni-ṣads
upa-patt-ti
ut-sneha-na
utta-rā-dhya-ya-na
utta-rā-dhya-ya-na-sūtra
uttara-khaṇḍa-khādyaka
uttara-sthāna
uttara-tantra
vācas-pati-miśra-vi-racita-ṭīkā-saṃ-valita
vācas-pati-miśra-vi-racita-ṭīkā-saṃ-valita-vyā-sa-bhā-ṣya-sam-e-tāni
vag-bhata-rasa-ratna-sam-uc-caya
vāg-bhaṭa-rasa-ratna-sam-uc-caya
vaidya-vara-śrī-ḍalhaṇā-cārya-vi-racitayā
vai-śā-kha
vai-śeṣ-ika-sūtra
vāja-sa-neyi-saṃ-hitā
vājī-kara-ṇam
vākya-śeṣa
vākya-śeṣaḥ
vaṅga-sena
vaṅga-sena-saṃ-hitā
varā-ha-mihi-ra
vārāhī-kalpa
vā-rāṇa-seya
va-ra-na-si
var-mam
var-man
var-ṇa-saṃ-khyā
var-ṇa-saṅ-khyā
vā-si-ṣṭha
vasiṣṭha-saṃ-hitā
vā-siṣṭha-saṃ-hitā
Va-sistha-Sam-hita-Yoga-Kanda-With-Comm-ent-ary-Kai-valya-Dham
vastra-dhauti
vasu-bandhu
vāta-pit-ta
vāta-pit-ta-kapha
vāta-pit-ta-kapha-śoṇi-ta
vāta-pitta-kapha-śoṇita-san-nipāta-vai-ṣamya-ni-mittāḥ
vāta-pit-ta-śoṇi-ta
vāta-śleṣ-man
vāta-śleṣ-ma-śoṇi-ta
vāta-śoṇi-ta
vātā-tapika
vātsyā-ya-na
vāya-vīya-saṃ-hitā
vedāṅga-rāya
veezhi-nathan
venkat-raman
vid-vad-vara-śrī-gaṇeśa-daiva-jña-vi-racita
vidya-bhu-sana
vi-jaya-siṃ-ha
vi-jñāna-bhikṣu
Vijñāneśvara-vi-racita-mitākṣarā-vyā-khyā-sam-alaṅ-kṛtā
vi-mā-na
vi-mā-na-sthāna
vimāna-sthā-na
vi-racitā
vi-racita-yāmadhu-kośākhya-vyā-khya-yā
vi-recana
vishveshvar-anand
vi-śiṣṭ-āṃśena
vi-suddhi-magga
vi-vi-dha-tṛṇa-kāṣṭha-pāṣāṇa-pāṃ-su-loha-loṣṭāsthi-bāla-nakha-pūyā-srāva-duṣṭa-vraṇāntar-garbha-śalyo-ddharaṇārthaṃ
vṛd-dha-vṛd-dha-tara-vṛd-dha-tamaiḥ
vṛddha-vṛddha-tara-vṛddha-tamaiḥ
vṛnda-mādhava
vyāḍī-ya-pa-ri-bhā-ṣā-vṛtti
vyā-khya-yā
vy-akta-liṅgādi-dharma-yuk-te
vyā-sa-bhā-ṣya-sam-e-tāni
vyati-krāmati
Xiuyao
yādava-bhaṭṭa
yāda-va-śarma-ṇā
yādava-sūri
yājña-valkya-smṛti
yājña-valkya-smṛtiḥ
yantrā-dhyāya
Yantra-rāja-vicāra-viṃśā-dhyāyī
yavanā-cā-rya
yoga-bhā-ṣya-vyā-khyā-rūpaṃ
yoga-cintā-maṇi
yoga-cintā-maṇiḥ
yoga-ratnā-kara
yoga-sāra-mañjarī
yoga-sāra-sam-uc-caya
yoga-sāra-saṅ-graha
yoga-śikh-opa-ni-ṣat
yoga-tārā-valī
yoga-yājña-val-kya
yoga-yājña-valkya-gītāsūpa-ni-ṣatsu
yogi-yājña-valkya-smṛti
yoshi-mizu
yukta-bhava-deva
}
%%%%%%%%%%%%%%%%%%%%
%Sanskrit:
%%%%%%%%%%%%%%%%%%%%
\textsanskrit{\hyphenation{%
    dhanva-ntariṇopa-diṣ-ṭaḥ
suśruta-nāma-dheyena
tac-chiṣyeṇa
    su-śruta-san-dīpana-bhāṣya
    cikitsā-sthāna
tulya-sau-vīrāñjana
indra-gopa
dṛṣṭi-maṇḍala
uc-chiṅga-na
vi-vi-dha-tṛṇa-kāṣṭha-pāṣāṇa-pāṃ-su-loha-loṣṭāsthi-bāla-nakha-pūyā-srāva-duṣṭa-vraṇāntar-garbha-śalyo-ddharaṇārthaṃ
śrī-ḍalhaṇācārya-vi-raci-tayāni-bandha-saṃ-grahākhya-vyā-khyayā
ni-dāna-sthānasyaśrī-gaya-dāsācārya-vi-racitayānyāya-candri-kā-khya-pañjikā-vyā-khyayā
casam-ul-lasi-tāmaharṣiṇāsu-śrutenavi-raci-tāsu-śruta-saṃ-hitā
bhartṛhari-viracitaḥ
śatakatrayādi-subhāṣitasaṃgrahaḥ
mahā-kavi-bhartṛ-hari-praṇīta-tvena
nīti-śṛṅgāra-vai-rāgyādi-nāmnāsamākhyā-tānāṃ
subhāṣitānāṃ
su-pariṣkṛta-saṃgrahaḥ
su-vistṛta-pari-cayātmikyāṅla-prastāvanā-vividha-pāṭhān-tara-pari-śiṣṭādi-sam-anvitaḥ
ācārya-śrī-jina-vijayālekhitāgra-vacanālaṃ-kṛtaś-ca
abhaya-deva-sūri-vi-racita-vṛtti-vi-bhūṣi-tam
abhi-dhar-ma
abhi-dhar-ma-ko-śa
abhi-dhar-ma-ko-śa-bhā-ṣya
abhi-dharma-kośa-bhāṣyam
abhyaṃ-karopāhva-vāsu-deva-śāstri-vi-racita-yā
agni-veśa
āhā-ra-vi-hā-ra-pra-kṛ-tiṃ
ahir-budhnya
ahir-budhnya-saṃ-hitā
akusī-dasya
alter-na-tively
amara-bharati
amara-bhāratī
āmla
amlīkā
ānan-da-rā-ya
anna-mardanādi-bhiś
anu-bhav-ād
anu-bhū-ta-viṣayā-sam-pra-moṣa
anu-bhū-ta-viṣayā-sam-pra-moṣaḥ
anu-māna
anu-miti-mānasa-vāda
ariya-pary-esana-sutta
ārogya-śālā-karaṇā-sam-arthas
ārogya-śālām
ārogyāyopa-kal-pya
arś-āṃ-si
ar-tha
ar-thaḥ
ārya-bhaṭa
ārya-lalita-vistara-nāma-mahā-yāna-sūtra
ārya-mañju-śrī-mūla-kalpa
ārya-mañju-śrī-mūla-kalpaḥ
asaṃ-pra-moṣa
āsana
āsanam
āsanaṃ
asid-dhe
aṣṭāṅga-hṛdaya
aṣṭāṅga-hṛdaya-saṃ-hitā
aṣṭ-āṅga-saṅ-graha
aṣṭ-āṅgā-yur-veda
aśva-gan-dha-kalpa
aśva-ghoṣa
ātaṅka-darpaṇa
ātaṅka-darpaṇa-vyā-khyā-yā
atha-vā
ātu-r-ā-hā-ra-vi-hā-ra-pra-kṛ-tiṃ
aty-al-pam
auṣa-dha-pāvanādi-śālāś
ava-sāda-na
avic-chin-na-sam-pra-dāya-tvād
āyur-veda
āyur-veda-sāra
āyur-vedod-dhāra-ka-vaid-ya-pañc-ānana-vaid-ya-rat-na-rāja-vaid-ya-paṇḍi-ta-rā-ma-pra-sāda-vaid-yo-pādhyā-ya-vi-ra-ci-tā
bahir-deśa-ka
bala-bhadra
bāla-kṛṣṇa
bau-dhā-yana-dhar-ma-sūtra
bhadrā-sana
bhadrā-sanam
bha-ga-vad-gī-tā
bha-ga-vat-pāda
bhaṭṭot-pala-vi-vṛti-sahitā
bhṛtyāva-satha-saṃ-yuktām
bhū-miṃ
bhu--va-na-dī-pa-ka
bīja-pallava
bodhi-sat-tva-bhūmi
brāhmaṇa-pra-mukha-nānā-sat-tva-vyā-dhi-śānty-ar-tham
brāhmaṇa-pra-mukha-nānā-sat-tve-bhyo
brahmāṇḍa-mahā-purā-ṇa
brahmāṇḍa-mahā-purā-ṇam
brāhma-sphu-ṭa-siddhānta
brahma-vi-hāra
brahma-vi-hāras
bṛhad-āraṇya-ka
bṛhad-yā-trā
bṛhad-yogi-yājña-valkya-smṛti
bṛhad-yogī-yājña-valkya-smṛti
bṛhaj-jāta-kam
cak-ra-dat-ta
cak-ra-pā-ṇi-datta
cā-luk-ya
caraka-prati-saṃ-s-kṛta
caraka-prati-saṃ-s-kṛte
cara-ka-saṃ-hitā
ca-tur-thī-vi-bhak-ti
cau-kham-ba
cau-luk-yas
chau-kham-bha
chun-nam
cikit-sā-saṅ-gra-ha
daiva-jñālaṃ-kṛti
daiva-jñālaṅ-kṛti
darśa-nāṅkur-ābhi-dhayāvyā-khya-yā
deva-nagari
deva-nāgarī
dhar-ma-megha
dhar-ma-meghaḥ
dhyā-na-grahopa-deśā-dhyā-yaś
dṛṣṭ-ān-ta
dṛṣṭ-ār-tha
dvāra-tvam
evaṃ-gṛ-hī-tam
evaṃ-vi-dh-a-sya
gala-gaṇḍa
gala-gaṇḍādi-kar-tṛ-tvaṃ
gan-dh-ā-ra
gar-bha-śa-rī-ram
gaurī-kāñcali-kā-tan-tra
gauta-mādi-tra-yo-da-śa-smṛty-ātma-kaḥ
gheraṇḍa-saṃ-hitā
gran-tha-śreṇi
gran-tha-śreṇiḥ
guru-maṇḍala-grantha-mālā
hari-śāstrī
hari-śās-trī
haṭha-yoga
hāyana-rat-na
hema-pra-bha-sūri
hetv-ābhā-sa
hīna-mithy-āti-yoga
hīna-mithy-āti-yogena
hindī-vyā-khyā-vi-marśope-taḥ
hoern-le
idam
ijya-rkṣe
ikka-vālaga
ity-arthaḥ
jābāla-darśanopa-ni-ṣad
jal-pa-kal-pa-tāru
jam-bū-dvī-pa
jam-bū-dvī-pa-pra-jña-pti
jam-bū-dvī-pa-pra-jña-pti-sūtra
jāta-ka-kar-ma-pad-dhati
jinā-agama-grantha-mālā
jī-vā-nan-da-nam
jñā-na-nir-mala
jñā-na-nir-malaṃ
jya-rkṣe
kāka-caṇḍīśvara-kal-pa-tan-tra
kā-la-gar-bhā-śa-ya-pra-kṛ-tim
kā-la-gar-bhā-śa-ya-pra-kṛ-tiṃ
kali-kāla-sarva-jña
kali-kāla-sarva-jña-śrī-hema-candrācārya-vi-raci-ta
kali-kāla-sarva-jña-śrī-hema-candrācārya-vi-raci-taḥ
kali-yuga
kal-pa-sthāna
kar-ma
kar-man
kārt-snyena
katham
kāvya-mālā
keśa-va-śāstrī
kol-ka-ta
kṛṣṇa-pakṣa
kṛtti-kā
kṛtti-kās
kula-pañji-kā
ku-māra-saṃ-bhava
lab-dhāni
mada-na-phalam
mādha-va
Mādhava-karaaita-reya-brāhma-ṇa
Mādhava-ni-dāna
mādhava-ni-dā-nam
madhu-kośa
madhu-kośākhya-vyā-khya-yā
madhya
madhye
ma-hā-bhū-ta-vi-kā-ra-pra-kṛ-tiṃ
mahā-deva
mahā-mati-śrī-mādhava-kara-pra-ṇī-taṃ
mahā-muni-śrī-mad-vyāsa-pra-ṇī-ta
mahā-muni-śrī-mad-vyāsa-pra-ṇī-taṃ
maha-rṣi-pra-ṇīta-dharma-śāstra-saṃ-grahaḥ
mahā-sacca-ka-sutta
mahau-ṣadhi-pari-cchadāṃ
mahā-vra-ta
mahā-yāna-sūtrālaṅ-kāra
mano-ratha-nandin
matsya-purāṇam
me-dhā-ti-thi
medhā-tithi
mithilā-stha
mithilā-stham
mithilā-sthaṃ
mud-rā-yantr-ā-laye
muktā-pīḍa
mūla-pāṭha
nakṣa-tra
nandi-purāṇoktārogya-śālā-dāna-phala-prāpti-kāmo
nara-siṃha
nara-siṃha-bhāṣya
nārā-ya-ṇa-dāsa
nārā-yaṇa-kaṇṭha
nārā-yaṇa-paṇḍi-ta-kṛtā
nava-pañca-mayor
nidā-na-sthā-na-sya
ni-ghaṇ-ṭu
nir-anta-ra-pa-da-vyā-khyā
nir-ṇaya-sā-gara
nir-ṇaya-sā-gara-yantr-ā-laye
nirūha-vasti
niś-cala-kara
ni-yukta-vaidyāṃ
nya-grodha
nya-grodho
nyāya-śās-tra
nyāya-sū-tra-śaṃ-kar
okaḥ-sātmya
okaḥ-sātmyam
okaḥ-sātmyaṃ
oka-sātmya
oka-sātmyam
oka-sātmyaṃ
oṣṭha-saṃ-puṭa
ousha-da-sala
padma-pra-bha-sūri
padma-sva-sti-kārdha-candrādike
paitā-maha-siddhā-nta
pañca-karma
pañca-karma-bhava-rogāḥ
pañca-karmādhi-kāra
pañca-karma-vi-cāra
pāñca-rātrā-gama
pañca-siddh-āntikā
pari-bhāṣā
pari-likh-ya
pātañ-jala-yoga-śās-tra
pātañ-jala-yoga-śās-tra-vi-varaṇa
pat-añ-jali
pāṭī-gaṇita
pāva-suya
pim-pal-gaon
pipal-gaon
pit-ta-kṛt
pit-ta-śleṣma-ghna
pit-ta-śleṣma-medo-meha-hik-kā-śvā-sa-kā-sāti-sā-ra-cchardi-tṛṣṇā-kṛmi-vi-ṣa-pra-śa-ma-naṃ
prā-cya
prā-cya-hindu-gran-tha-śreṇiḥ
prācya-vidyā-saṃ-śodhana-mandira
pra-dhān-āṅ-gaṃ
pra-dhān-in
pra-ka-shan
pra-kṛ-ti
pra-kṛ-tiṃ
pra-mā-ṇa-vārt-tika
pra-saṅ-khyāne
pra-śas-ta-pāda-bhāṣya
pra-śna-pra-dīpa
pra-śnārṇa-va-plava
praśnārṇava-plava
pra-śna-vaiṣṇava
pra-śna-vai-ṣṇava
prati-padyate
pra-ty-akṣa
pra-yat-na-śai-thilyā-nan-ta-sam-ā-pat-ti-bhyām
pra-yat-na-śai-thilyā-nān-tya-sam-ā-pat-ti-bhyāṃ
pra-yatna-śai-thilya-sya
puṇya-pattana
pūrṇi-mā-nta
rāja-kīya
rajjv-ābhyas-ya
rāma-kṛṣṇa
rasa-ratnā-kara
rasa-vai-śeṣika-sūtra
rogi-svasthī-karaṇānu-ṣṭhāna-mātraṃ
rūkṣa-vasti
sād-guṇya
śākalya-saṃ-hitā
sam-ā-mnāya
sāmañña-pha-la-sutta
sama-ran-gana-su-tra-dhara
samā-raṅga-ṇa-sū-tra-dhāra
sama-ra-siṃ-ha
sama-ra-siṃ-haḥ
saṃ-hitā
sāṃ-sid-dhi-ka
saṃ-śo-dhana
sam-ul-lasi-tam
śāndilyopa-ni-ṣad
śaṅ-kara
śaṅ-kara-bha-ga-vat-pāda
Śaṅ-kara-nārā-yaṇa
saṅ-khyā
sāṅ-kṛt-yā-yana
san-s-krit
sap-tame
śāra-dā-tila-ka-tan-tra
śa-raṅ-ga-deva
śār-dūla-karṇā-va-dāna
śā-rī-ra
śā-rī-ra-sthāna
śārṅga-dhara
śārṅga-dhara-saṃ-hitā
sar-va
sarva-darśana-saṃ-grahaḥ
sar-va-dar-śāna-saṅ-gra-ha
sar-va-dar-śāna-saṅ-gra-haḥ
sarv-arthāvi-veka-khyā-ter
sar-va-tan-tra-sid-dhān-ta
sar-va-tan-tra-sid-dhān-taḥ
sarva-yoga-sam-uc-caya
sar-va-yogeśvareśva-ram
śāstrā-rambha-sam-artha-na
śāstrāram-bha-sam-arthana
ṣaṭ-pañcā-śi-kā
sat-tva
saunda-ra-na-nda
sid-dha
sid-dha-man-tra
sid-dha-man-trā-hvayo
sid-dha-man-tra-pra-kāśa
sid-dha-man-tra-pra-kāśaḥ
sid-dha-man-tra-pra-kāśaś
sid-dh-ān-ta
siddhānta-śiro-maṇ
sid-dha-yoga
sid-dhi-sthāna
śi-va-śar-ma-ṇā
ska-nda-pu-rā-ṇa
sneha-basty-upa-deśāt
sodā-haraṇa-saṃ-s-kṛta-vyā-khyayā
śodha-ka-pusta-kaṃ
śo-dha-na-ci-kitsā
so-ma-val-ka
śrī-mad-devī-bhāga-vata-mahā-purāṇa
srag-dharā-tārā-sto-tra
śrī-hari-kṛṣṇa-ni-bandha-bhava-nam
śrī-hema-candrā-cārya-vi-raci-taḥ
śrī-kaṇtha-datta
śrī-kaṇtha-dattā-bhyāṃ
śrī-kṛṣṇa-dāsa
śrī-mad-amara-siṃha-vi-racitam
śrī-mad-aruṇa-dat-ta-vi-ra-ci-tayā
śrī-mad-bhaṭṭot-pala-kṛta-saṃ-s-kṛta-ṭīkā-sahitam
śrī-mad-dvai-pā-yana-muni-pra-ṇītaṃ
śrī-mad-vāg-bha-ṭa-vi-ra-ci-tam
śrī-maṃ-trī-vi-jaya-siṃha-suta-maṃ-trī-teja-siṃhena
śrī-mat-kalyāṇa-varma-vi-racitā
śrīmat-sāyaṇa-mādhavācārya-pra-ṇītaḥ
śrī-vā-cas-pati-vaidya-vi-racita-yā
śrī-vatsa
śrī-vi-jaya-rakṣi-ta
sthānāṅga-sūtra
sthira-sukha
sthira-sukham
strī-niṣevaṇa
śukla-pakṣa
su-śru-ta-saṃ-hitā
sū-tra
sūtrārthānān-upa-patti-sūca-nāt
sūtra-sthāna
su-varṇa-pra-bhāsot-tama-sū-tra
svalpauṣadha-dāna-mā-tram
śvetāśva-taropa-ni-ṣad
tad-upa-karaṇa-tāmra-kaṭāha-kalasādi-pātra-pari-cchada-nānā-vidha-vyādhi-śānty-ucitauṣadha-gaṇa-yathokta-lakṣaṇa-vaidya-nānā-vidha-pari-cāraka-yutāṃ
tājaka-muktā-valeḥ
tājika-kau-stu-bha
tājika-nīla-kaṇṭhī
tājika-yoga-sudhā-ni-dhi
tāmra-paṭṭādi-li-khi-tāṃ
tan-nir-vāhāya
tapo-dhana
tapo-dhanā
tārā-bhakti-su-dhārṇava
tārtīya-yoga-su-sudhā-ni-dhi
tegi-ccha
te-jaḥ-siṃ-ha
trai-lok-ya
trai-lokya-pra-kāśa
tri-piṭa-ka
tri-var-gaḥ
un-mār-ga-gama-na
upa-de-śa
upa-patt-ti
ut-sneha-na
utta-rā-dhyā-ya-na
uttara-sthāna
uttara-tantra
vāchas-pati
vād-ā-valī
vai-śā-kha
vai-ta-raṇa-vasti
vai-ta-raṇok-ta-guṇa-gaṇa-yu-k-taṃ
vājī-kara-ṇam
vāk-patis
vākya-śeṣa
vākya-śeṣaḥ
varā-ha-mihi-ra
va-ra-na-si
vā-rā-ṇa-sī
var-mam
var-man
varṇa-sam-ā-mnāya
va-siṣṭha-saṃ-hitā
vā-siṣṭha-saṃ-hitā
vasu-bandhu
vasu-bandhu
vāta-ghna-pit-talāl-pa-ka-pha
vātsyā-ya-na
vidya-bhu-sana
vidyā-bhū-ṣaṇa
vi-jaya-siṃ-ha
vi-jñāna-bhikṣu
vi-kal-pa
vi-kamp-i-tum
vi-mā-na-sthāna
vi-racita-yā
vishveshvar-anand
vi-śiṣṭ-āṃśena
viṣṇu-dharmot-tara-purāṇa
viśrāma-gṛha-sahitā
vi-suddhi-magga
vopa-de-vīya-sid-dha-man-tra-pra-kāśe
vyādhi-pratī-kārār-tham
vyāḍī-ya-pa-ri-bhā-ṣā-vṛtti
vyati-krāmati
vy-ava-haranti
yādava-bhaṭṭa
yāda-va-śarma-ṇā
yādava-sūri
yājña-valkya-smṛti
yavanā-cā-rya
yoga-ratnā-kara
yoga-sāra-sam-uc-caya
yoga-sāra-sam-uc-cayaḥ
yoga-sūtra-vi-vara-ṇa
yoga-yājña-valkya
yoga-yājña-valkya-gītāsūpa-ni-ṣatsu
yoga-yājña-valkyaḥ
yogi-yājña-valkya-smṛti
yuk-tiḥ
yuk-tis
}}
\normalfontlatin
\endinput
}% should work, but doesn't
% special hyphenations for Sanskrit words tagged in
% Polyglossia.
% *English,\textenglish{},text,and
% *Sanskrit,\textsanskrit{},text.
%
% English (see below for \textsanskrit)
%
\hyphenation{%
    dhanva-ntariṇopa-diṣ-ṭaḥ
    suśruta-nāma-dheyena
    tac-chiṣyeṇa
    kāśyapa-saṃ-hitā
    cikitsā-sthāna
    su-śruta-san-dīpana-bhāṣya
    dṛṣṭi-maṇḍala
    uc-chiṅga-na
    sarva-siddhānta-tattva-cūḍā-maṇi
    tulya-sau-vīrāñja-na
    indra-gopa
    śrī-mad-abhi-nava-guptā-cārya-vi-ra-cita-vi-vṛti-same-tam
    viśva-nātha
śrī-mad-devī-bhāga-vata-mahā-purāṇa
    siddhā-n-ta-sun-dara
    brāhma-sphuṭa-siddh-ānta
    bhū-ta-saṅ-khyā
    bhū-ta-saṃ-khyā
    kathi-ta-pada
    devī-bhā-ga-vata-purāṇa
    devī-bhā-ga-vata-mahā-purāṇa
    Siddhānta-saṃ-hitā-sāra-sam-uc-caya
    sau-ra-pau-rāṇi-ka-mata-sam-artha-na
    Pṛthū-da-ka-svā-min
    Brah-ma-gupta
    Brāh-ma-sphu-ṭa-siddhānta
    siddhānta-sun-dara
    vāsa-nā-bhāṣya
    catur-veda
    bhū-maṇḍala
    jñāna-rāja
    graha-gaṇi-ta-cintā-maṇi
    Śiṣya-dhī-vṛd-dhi-da-tan-tra
    brah-māṇḍa-pu-rā-ṇa
    kūr-ma-pu-rā-ṇa
    jam-bū-dvī-pa
    bhā-ga-vata-pu-rā-ṇa
    kupya-ka
    nandi-suttam
    nandi-sutta
    su-bodhiā-bāī
    asaṅ-khyāta
    saṅ-khyāta
    saṅ-khyā-pra-māṇa
    saṃ-khā-pamāṇa
    nemi-chandra
    anu-yoga-dvāra
    tattvārtha-vārtika
    aka-laṅka
    tri-loka-sāra
    gaṇi-ma-pra-māṇa
    gaṇi-ma-ppa-māṇa
    eka-pra-bhṛti
gaṇaṇā-saṃ-khā
gaṇaṇā-saṅ-khyā
dvi-pra-bhṛti
duppa-bhi-ti-saṃ-khā
vedanābhi-ghāta
Viṣṇu-dharmottara-pu-rāṇa
abhaya-deva-sūri-vi-racita-vṛtti-vi-bhūṣi-tam
abhi-dhar-ma
abhi-dhar-ma-ko-śa
abhi-dhar-ma-ko-śa-bhā-ṣya
abhi-dharma-kośa-bhāṣya
abhi-dharma-kośa-bhāṣyam
abhi-nava
abhyaṃ-karopāhva-vāsu-deva-śāstri-vi-ra-ci-ta-yā
ācārya-śrī-jina-vijayālekhitāgra-vacanālaṃ-kṛtaś-ca
ācāry-opā-hvena
ādhāra
adhi-kāra
adhi-kāras
ādi-nātha
agni-besha
agni-veśa
ahir-budhnya
ahir-budhnya-saṃ-hitā
aita-reya-brāhma-ṇa
akusī-dasya
amara-bharati
Amar-augha-pra-bo-dha
amṛ-ta-siddhi
ānanda-kanda
ānan-da-rā-ya
ānand-āśra-ma-mudraṇā-la-ya
ānand-āśra-ma-saṃ-skṛta-granth-āva-liḥ
anna-pāna-mūlā
anu-ban-dhya-lakṣaṇa-sam-anv-itās
anu-bhav-ād
anu-bhū-ta-viṣayā-sam-pra-moṣa
anu-bhū-ta-viṣayā-sam-pra-moṣaḥ
aparo-kṣā-nu-bhū-ti
app-proxi-mate-ly
ardha-rātrika-karaṇa
ārdha-rātrika-karaṇa
ariya-pary-esana-sutta
arun-dhatī
ārya-bhaṭa
ārya-bhaṭā-cārya-vi-racitam
ārya-bhaṭīya
ārya-bhaṭīyaṃ
ārya-lalita-vistara-nāma-mahā-yāna-sūtra
ārya-mañju-śrī-mūla-kalpa
ārya-mañju-śrī-mūla-kalpaḥ
asaṃ-pra-moṣa
aṣṭāṅga-hṛdaya-saṃ-hitā
aṣṭāṅga-saṃ-graha
asura-bhavana
aśva-ghoṣa
ātaṅka-darpaṇa-vyā-khyā-yā
atha-vā
ava-sāda-na
āyār-aṅga-suttaṃ
ayur-ved
ayur-veda
āyur-veda
āyur-veda-dīpikā
āyur-veda-dīpikā-vyā-khyayā
āyur-ve-da-ra-sā-yana
āyur-veda-sū-tra
ayur-vedic
āyur-vedic
ayur-yog
bādhirya
bahir-deśa-ka
bala-bhadra
bala-kot
bala-krishnan
bāla-kṛṣṇa
bau-dhā-yana-dhar-ma-sūtra
bel-valkar
bhadra-kālī-man-tra-vi-dhi-pra-karaṇa
bhadrā-sana
bhadrā-sanam
bha-ga-vat-pāda
bhaiṣajya-ratnāvalī
bhan-d-ar-kar
bhartṛhari-viracitaḥ
bhaṭṭā-cārya
bhaṭṭot-pala-vi-vṛti-sahitā
Bhiṣag-varāḍha-malla-vi-racita-dīpikā-Kāśī-rāma-vaidya-vi-raci-ta-gūḍhā-rtha-dīpikā-bhyāṃ
bhiṣag-varāḍha-malla-vi-racita-dīpikā-Kāśī-rāma-vaidya-vi-racita-gūḍhārtha-dīpikā-bhyāṃ
bhoja-deva-vi-raci-ta-rāja-mārtaṇḍā-bhi-dha-vṛtti-sam-e-tāni
bhu--va-na-dī-pa-ka
bīja-pallava
bi-kaner
bodhi-sat-tva-bhūmi
brahma-gupta
brahmā-nanda
brahmāṇḍa-mahā-purā-ṇa
brahmāṇḍa-mahā-purā-ṇam
brahma-randhra
brahma-siddh-ānta
brāhma-sphuṭa-siddh-ānta
brāhma-sphu-ṭa-siddhānta
brahma-vi-hāra
brahma-vi-hāras
brahma-yā-mala-tan-tra
Bra-ja-bhāṣā
bṛhad-āraṇya-ka
bṛhad-yā-trā
bṛhad-yogi-yājña-valkya-smṛti
bṛhad-yogī-yājña-valkya-smṛti
bṛhaj-jāta-kam
bṛhat-khe-carī-pra-kāśa
buddhi-tattva-pra-karaṇa
cak-ra-dat-ta
cakra-datta
cakra-pāṇi-datta
cā-luk-ya
caraka-prati-saṃ-s-kṛta
caraka-prati-saṃ-s-kṛte
caraka-saṃ-hitā
casam-ul-lasi-tāmaharṣiṇāsu-śrutenavi-raci-tāsu-śruta-saṃ-hitā
cau-kham-ba
cau-luk-yas
chandi-garh
chara-ka
cha-rīre
chatt-opa-dh-ya-ya
chau-kham-bha
chi-ki-tsi-ta
cid-ghanā-nanda-nātha
ci-ka-ner
com-men-taries
com-men-tary
com-pre-hen-sive-ly
daiva-jñālaṃ-kṛti
daiva-jñālaṅ-kṛti
dāmo-dara-sūnu-Śārṅga-dharācārya-vi-racitā
Dāmodara-sūnu-Śārṅga-dharācārya-vi-racitā
darśanā-ṅkur-ābhi-dhayā
das-gupta
deha-madhya
deha-saṃ-bhava-hetavaḥ
deva-datta
deva-nagari
deva-nāgarī
devā-sura-siddha-gaṇaiḥ
dha-ra-ni-dhar
dharma-megha
dharma-meghaḥ
dhru-vam
dhru-va-sya
dhru-va-yonir
dhyā-na-grahopa-deśā-dhyā-yaś
dṛḍha-śūla-yukta-rakta
dvy-ulbaṇaikolba-ṇ-aiḥ
four-fold
gan-dh-ā-ra
gārgīya-jyoti-ṣa
gārgya-ke-rala-nīla-kaṇṭha-so-ma-sutva-vi-racita-bhāṣyo-pe-tam
garuḍa-mahā-purāṇa
gaurī-kāñcali-kā-tan-tra
gau-tama
gauta-mādi-tra-yo-da-śa-smṛty-ātma-kaḥ
gheraṇḍa-saṃ-hitā
gorakṣa-śata-ka
go-tama
granth-ā-laya
grantha-mālā
gran-tha-śreṇiḥ
grāsa-pramāṇa
guru-maṇḍala-grantha-mālā
gyatso
hari-śāstrī
haṭhābhyāsa-paddhati
haṭha-ratnā-valī
Haṭha-saṅ-keta-candri-kā
haṭha-tattva-kau-mudī
haṭha-yoga
hāyana-rat-na
haya-ta-gran-tha
hema-pra-bha-sūri
hetu-lakṣaṇa-saṃ-sargād
hīna-madhyādhi-kaiś
hindī-vyā-khyā-vi-marśope-taḥ
hoern-le
ijya-rkṣa
ikka-vālaga
indra-dhvaja
indrāṇī-kalpa
indria
Īśāna-śiva-guru-deva-pad-dhati
jābāla-darśanopa-ni-ṣad
jadav-ji
jagan-nā-tha
jala-basti
jal-pa-kal-pa-tāru
jam-bū-dvī-pa-pra-jña-pti
jam-bū-dvī-pa-pra-jña-pti-sūtra
jana-pad-a-sya
jāta-ka-kar-ma-pad-dhati
jaya-siṃha
jinā-agama-grantha-mālā
jin-en-dra-bud-dhi
jīvan-muk-ti-vi-veka
jñā-na-nir-mala
jñā-na-nir-malaṃ
joga-pra-dīpya-kā
jya-rkṣe
Jyo-tiḥ-śās-tra
jyo-ti-ṣa-rāya
jyoti-ṣa-rāya
jyotiṣa-siddhānta-saṃ-graha
jyotiṣa-siddhānta-saṅ-graha
kāka-caṇḍīśvara-kal-pa-tan-tra
kakṣa-puṭa
kali-kāla-sarva-jña
kali-kāla-sarva-jña-śrī-hema-candrācārya-vi-raci-ta
kali-kāla-sarva-jña-śrī-hema-candrācārya-vi-raci-taḥ
kali-yuga
kal-pa
kal-pa-sthāna
kalyāṇa-kāraka
Kāmeśva-ra-siṃha-dara-bhaṅgā-saṃ-skṛta-viśva-vidyā-layaḥ
kapāla-bhāti
karaṇa-tilaka
kar-ma
kar-man
kāṭhaka-saṃ-hitā
kavia-rasu
kavi-raj
keśa-va-śāstrī
ke-vala--rāma
keva-la-rāma
khaṇḍa-khādyaka-tappā
khe-carī-vidyā
knowl-edge
kol-ka-ta
kriyā-krama-karī
kṛṣṇa-pakṣa
kṛtti-kā
kṛtti-kās
kubji-kā-mata-tantra
kula-pañji-kā
kul-karni
ku-māra-saṃ-bhava
kuṭi-pra-veśa
kuṭi-pra-veśika
lakṣ-mī-veṅ-kaṭ-e-ś-va-ra
lit-era-ture
lit-era-tures
locana-roga
mādha-va
mādhava-kara
mādhava-ni-dāna
mādhava-ni-dā-nam
madh-ūni
madhya
mādhyan-dina
madhye
mahā-bhāra-ta
mahā-deva
mahā-kavi-bhartṛ-hari-praṇīta-tvena
maha-mahopa-dhyaya
mahā-maho-pā-dhyā-ya-śrī-vi-jñā-na-bhikṣu-vi-raci-taṃ
mahā-mati-śrī-mādhava-kara-pra-ṇī-taṃ
mahā-mudrā
mahā-muni-śrī-mad-vyāsa-pra-ṇī-ta
mahā-muni-śrī-mad-vyāsa-pra-ṇī-taṃ
maharṣiṇā
maha-rṣi-pra-ṇīta-dharma-śāstra-saṃ-grahaḥ
Maha-rṣi-varya-śrī-yogi-yā-jña-valkya-śiṣya-vi-racitā
mahā-sacca-ka-sutta
mahā-sati-paṭṭhā-na-sutta
mahā-vra-ta
mahā-yāna-sūtrālaṅ-kāra
maitrāya-ṇī-saṃ-hitā
maktab-khānas
māla-jit
māli-nī-vijayot-tara-tan-tra
manaḥ-sam-ā-dhi
mānasol-lāsa
mānava-dharma-śāstra
mandāgni-doṣa
mannar-guḍi
mano-har-lal
mano-ratha-nandin
man-u-script
man-u-scripts
mataṅga-pārame-śvara
mater-ials
matsya-purāṇam
medh-ā-ti-thi
medhā-tithi
mithilā-stha
mithilā-stham
mithilā-sthaṃ
mṛgendra-tantra-vṛtti
mud-rā-yantr-ā-laye
muktā-pīḍa
mūla-pāṭha
muṇḍī-kalpa
mun-sh-ram
Nāda-bindū-pa-ni-ṣat
nāga-bodhi
nāga-buddhi
nakṣa-tra
nara-siṃha
nārā-yaṇa-dāsa
nārā-yaṇa-dāsa
nārā-yaṇa-kaṇṭha
nārā-yaṇa-paṇḍi-ta-kṛtā
nar-ra-tive
nata-rajan
nava-pañca-mayor
nava-re
naya-na-sukho-pā--dhyāya
ni-ban-dha-saṃ-grahā-khya-vyākhya-yā
niban-dha-san-graha
ni-dā-na
nidā-na-sthā-na-sya
ni-dāna-sthānasyaśrī-gaya-dāsācārya-vi-racitayānyāya-candri-kā-khya-pañjikā-vyā-khyayā
nir-anta-ra-pa-da-vyā-khyā
nir-guṇḍī-kalpa
nir-ṇaya-sā-gara
Nir-ṇaya-sāgara
nir-ṇa-ya-sā-gara-mudrā-yantrā-laye
nir-ṇa-ya-sā-ga-ra-yantr-āla-ya
nir-ṇaya-sā-gara-yantr-ā-laye
niśvāsa-kārikā
nīti-śṛṅgāra-vai-rāgyādi-nāmnāsamākhyā-tānāṃ
nityā-nanda
nya-grodha
nya-grodho
nyā-ya-candri-kā-khya-pañji-kā-vyā-khya-yā
nyāya-śās-tra
okaḥ-sātmya
okaḥ-sātmyam
okaḥ-sātmyaṃ
oka-sātmya
oka-sātmyam
oka-sātmyaṃ
oris-sa
oṣṭha-saṃ-puṭa
ousha-da-sala
padma-pra-bha-sūri
Padma-prā-bhṛ-ta-ka
padma-sva-sti-kārdha-candrādike
paitā-maha-siddhā-nta
pañca-karma
pañca-karman
pāñca-rātrā-gama
pañca-siddh-āntikā
paṅkti-śūla
Paraśu-rāma
paraśu-rāma
pari-likh-ya
pāśu-pata-sū-tra-bhāṣya
pātañ-jala-yoga-śās-tra
pātañ-jala-yoga-śās-tra-vi-varaṇa
pat-añ-jali
pat-na
pāva-suya
phiraṅgi-can-dra-cchedyo-pa-yogi-ka
pim-pal-gaon
pipal-gaon
pitta-śleṣ-man
pit-ta-śleṣ-ma-śoṇi-ta
pitta-śoṇi-ta
prā-cīna-rasa-granthaḥ
prā-cya
prā-cya-hindu-gran-tha-śreṇiḥ
prācya-vidyā-saṃ-śodhana-mandira
pra-dhān-in
pra-ka-shan
pra-kaṭa-mūṣā
pra-kṛ-ti-bhū-tāḥ
pra-mā-ṇa-vārt-tika
pra-ṇītā
pra-saṅ-khyāne
pra-śas-ta-pāda-bhāṣya
pra-śna-pra-dīpa
pra-śnārṇa-va-plava
praśnārṇava-plava
pra-śna-vai-ṣṇava
pra-śna-vaiṣṇava
prati-padyate
pra-yatna-śaithilyānan-ta-sam-āpatti-bhyām
prei-sen-danz
punar-vashu
puṇya-pattana
pūrṇi-mā-nta
raghu-nātha
rāja-kīya
rāja-kīya-mudraṇa-yantrā-laya
rāja-śe-khara
rajjv-ābhyas-ya
raj-put
rāj-put
rakta-mokṣa-na
rāma-candra-śāstrī
rāma-kṛṣṇa
rāma-kṛṣṇa-śāstri-ṇā
rama-su-bra-manian
rāmā-yaṇa
rasa-ratnā-kara
rasa-ratnākarāntar-ga-taś
rasa-ratna-sam-uc-caya
rasa-ratna-sam-uc-ca-yaḥ
rasa-vīry-auṣa-dha-pra-bhāvena
rasā-yana
rasendra-maṅgala
rasendra-maṅgalam
rāṣṭra-kūṭa
rāṣṭra-kūṭas
sādhana
śākalya-saṃ-hitā
śāla-grāma-kṛta
śāla-grāma-kṛta
sāmañña-pha-la-sutta
sāmañña-phala-sutta
sama-ran-gana-su-tra-dhara
samā-raṅga-ṇa-sū-tra-dhāra
sama-ra-siṃ-ha
sama-ra-siṃ-haḥ
sāmba-śiva-śāstri
same-taḥ
saṃ-hitā
śāṃ-ka-ra-bhāṣ-ya-sam-etā
sam-rāṭ
saṃ-rāṭ
Sam-rāṭ-siddhānta
Sam-rāṭ-siddhānta-kau-stu-bha
sam-rāṭ-siddhānta-kau-stu-bha
saṃ-sargam
saṃ-sargaṃ
saṃ-s-kṛta
saṃ-s-kṛta-pārasī-ka-pra-da-pra-kāśa
saṃ-śo-dhana
saṃ-śodhitā
saṃ-sthāna
sam-ullasitā
sam-ul-lasi-tam
saṃ-valitā
saṃ-valitā
śāndilyopa-ni-ṣad
śaṅ-kara
śaṅ-kara-bha-ga-vat-pāda
śaṅ-karā-cārya
san-kara-charya
Śaṅ-kara-nārā-yaṇa
sāṅ-kṛt-yā-yana
san-s-krit
śāra-dā-tila-ka-tan-tra
śa-raṅ-ga-deva
śār-dūla-karṇā-va-dāna
śār-dūla-karṇā-va-dāna
śā-rī-ra-sthāna
śārṅga-dhara-saṃ-hitā
Śārṅga-dhara-saṃ-hitā
sar-va-dar-śana-saṅ-gra-ha
sarva-kapha-ja
sarv-arthāvi-veka-khyā-ter
sar-va-śa-rīra-carās
sarva-siddhānta-rāja
Sarva-siddhā-nt-rāja
sarva-vyā-dhi-viṣāpa-ha
sarva-yoga-sam-uc-caya
sar-va-yogeśvareśva-ram
śāstrā-rambha-sam-artha-na
śatakatrayādi-subhāṣitasaṃgrahaḥ
sati-paṭṭhā-na-sutta
ṣaṭ-karma
ṣaṭ-karman
sat-karma-saṅ-graha
sat-karma-saṅ-grahaḥ
ṣaṭ-pañcā-śi-kā
saun-da-ra-nanda
sa-v-āī
schef-tel-o-witz
scholars
sharī-ra
sheth
sid-dha-man-tra
siddha-nanda-na-miśra
siddha-nanda-na-miśraḥ
siddha-nitya-nātha-pra-ṇītaḥ
Siddhānta-saṃ-hitā-sāra-sam-uc-caya
Siddhā-nta-sār-va-bhauma
siddhānta-sindhu
siddhānta-śiro-maṇ
Siddhānta-śiro-maṇi
Siddhā-nta-tat-tva-vi-veka
sid-dha-yoga
siddha-yoga
sid-dhi
sid-dhi-sthā-na
sid-dhi-sthāna
śikhi-sthāna
śiraḥ-karṇā-kṣi-vedana
śiro-bhūṣaṇam
Śivā-nanda-saras-vatī
śiva-saṃ-hitā
śiva-yo-ga-dī-pi-kā
ska-nda-pu-rā-ṇa
śleṣ-man
śleṣ-ma-śoni-ta
sodā-haraṇa-saṃ-s-kṛta-vyā-khyayā
śodha-ka-pusta-kaa
śoṇi-ta
spaṣ-ṭa-krānty-ādhi-kāra
śrī-cakra-pāṇi-datta
śrī-cakra-pāṇi-datta-viracitayā
śrī-ḍalhaṇācārya-vi-raci-tayāni-bandha-saṃ-grahākhya-vyā-khyayā
śrī-dayā-nanda
śrī-hari-kṛṣṇa-ni-bandha-bhava-nam
śrī-hema-candrā-cārya-vi-raci-taḥ
śrī-kaṇtha-dattā-bhyāṃ
śrī-kṛṣṇa-dāsa
śrī-kṛṣṇa-dāsa-śreṣṭhinā
śrīmac-chaṅ-kara-bhaga-vat-pāda-vi-raci-tā
śrī-mad-amara-siṃha-vi-racitam
śrī-mad-bha-ga-vad-gī-tā
śrī-mad-bhaṭṭot-pala-kṛta-saṃ-s-kṛta-ṭīkā-sahitam
śrī-mad-dvai-pā-yana-muni-pra-ṇītaṃ
śrī-mad-vāg-bhaṭa-vi-raci-tam
śrī-maṃ-trī-vi-jaya-siṃha-suta-maṃ-trī-teja-siṃhena
śrī-mat-kalyāṇa-varma-vi-racitā
śrī-mat-sāyaṇa-mādhavācārya-pra-ṇītaḥsarva-darśana-saṃ-grahaḥ
śrī-nitya-nātha-siddha-vi-raci-taḥ
śrī-rāja-śe-khara
śrī-śaṃ-karā-cārya-vi-raci-tam
śrī-vā-cas-pati-vaidya-vi-racita-yā
śrī-vatsa
śrī-veda-vyāsa-pra-ṇīta-mahā-bhā-ratāntar-ga-tā
śrī-veṅkaṭeś-vara
śrī-vi-jaya-rakṣi-ta
sruta-rakta
sruta-raktasya
stambha-karam
sthānāṅga-sūtra
sthira-sukha
sthira-sukham
stra-sthā-na
subhāṣitānāṃ
su-brah-man-ya
su-bra-man-ya
śukla-pakṣa
śukrā-srava
suk-than-kar
su-pariṣkṛta-saṃgrahaḥ
sura-bhi-pra-kash-an
sūrya-dāsa
sūrya-siddhānta
su-shru-ta
su-śru-ta
su-shru-ta-saṃ-hitā
su-śru-ta-saṃ-hitā
su-śru-tena
sutra
sūtra
sūtra-neti
sūtra-ni-dāna-śā-rīra-ci-ki-tsā-kal-pa-sthānot-tara-tan-trātma-kaḥ
sūtra-sthāna
su-varṇa-pra-bhāsot-tama-sū-tra
Su-var-ṇa-pra-bhās-ot-tama-sū-tra
su-varṇa-pra-bhāsotta-ma-sūtra
su-vistṛta-pari-cayātmikyāṅla-prastāvanā-vividha-pāṭhān-tara-pari-śiṣṭādi-sam-anvitaḥ
sva-bhāva-vyādhi-ni-vāraṇa-vi-śiṣṭ-auṣa-dha-cintakās
svā-bhāvika
svā-bhāvikās
sva-cchanda-tantra
śvetāśva-taropa-ni-ṣad
taila-sarpir-ma-dhūni
tait-tirīya-brāhma-ṇa
tājaka-muktā-valeḥ
tājika-kau-stu-bha
tājika-nīla-kaṇṭhī
tājika-yoga-sudhā-ni-dhi
tapo-dhana
tapo-dhanā
tārā-bhakti-su-dhārṇava
tārtīya-yoga-su-sudhā-ni-dhi
tegi-ccha
te-jaḥ-siṃ-ha
ṭhāṇ-āṅga-sutta
ṭīkā-bhyāṃ
ṭīkā-bhyāṃ
tiru-mantiram
tiru-ttoṇṭar-purāṇam
tiru-va-nanta-puram
trai-lok-ya
trai-lokya-pra-kāśa
tri-bhāga
tri-kam-ji
tri-pita-ka
tri-piṭa-ka
tri-vik-ra-mātma-jena
ud-ā-haraṇa
un-mārga-gama-na
upa-ca-ya-bala-varṇa-pra-sādādī-ni
upa-laghana
upa-ni-ṣads
upa-patt-ti
ut-sneha-na
utta-rā-dhya-ya-na
utta-rā-dhya-ya-na-sūtra
uttara-khaṇḍa-khādyaka
uttara-sthāna
uttara-tantra
vācas-pati-miśra-vi-racita-ṭīkā-saṃ-valita
vācas-pati-miśra-vi-racita-ṭīkā-saṃ-valita-vyā-sa-bhā-ṣya-sam-e-tāni
vag-bhata-rasa-ratna-sam-uc-caya
vāg-bhaṭa-rasa-ratna-sam-uc-caya
vaidya-vara-śrī-ḍalhaṇā-cārya-vi-racitayā
vai-śā-kha
vai-śeṣ-ika-sūtra
vāja-sa-neyi-saṃ-hitā
vājī-kara-ṇam
vākya-śeṣa
vākya-śeṣaḥ
vaṅga-sena
vaṅga-sena-saṃ-hitā
varā-ha-mihi-ra
vārāhī-kalpa
vā-rāṇa-seya
va-ra-na-si
var-mam
var-man
var-ṇa-saṃ-khyā
var-ṇa-saṅ-khyā
vā-si-ṣṭha
vasiṣṭha-saṃ-hitā
vā-siṣṭha-saṃ-hitā
Va-sistha-Sam-hita-Yoga-Kanda-With-Comm-ent-ary-Kai-valya-Dham
vastra-dhauti
vasu-bandhu
vāta-pit-ta
vāta-pit-ta-kapha
vāta-pit-ta-kapha-śoṇi-ta
vāta-pitta-kapha-śoṇita-san-nipāta-vai-ṣamya-ni-mittāḥ
vāta-pit-ta-śoṇi-ta
vāta-śleṣ-man
vāta-śleṣ-ma-śoṇi-ta
vāta-śoṇi-ta
vātā-tapika
vātsyā-ya-na
vāya-vīya-saṃ-hitā
vedāṅga-rāya
veezhi-nathan
venkat-raman
vid-vad-vara-śrī-gaṇeśa-daiva-jña-vi-racita
vidya-bhu-sana
vi-jaya-siṃ-ha
vi-jñāna-bhikṣu
Vijñāneśvara-vi-racita-mitākṣarā-vyā-khyā-sam-alaṅ-kṛtā
vi-mā-na
vi-mā-na-sthāna
vimāna-sthā-na
vi-racitā
vi-racita-yāmadhu-kośākhya-vyā-khya-yā
vi-recana
vishveshvar-anand
vi-śiṣṭ-āṃśena
vi-suddhi-magga
vi-vi-dha-tṛṇa-kāṣṭha-pāṣāṇa-pāṃ-su-loha-loṣṭāsthi-bāla-nakha-pūyā-srāva-duṣṭa-vraṇāntar-garbha-śalyo-ddharaṇārthaṃ
vṛd-dha-vṛd-dha-tara-vṛd-dha-tamaiḥ
vṛddha-vṛddha-tara-vṛddha-tamaiḥ
vṛnda-mādhava
vyāḍī-ya-pa-ri-bhā-ṣā-vṛtti
vyā-khya-yā
vy-akta-liṅgādi-dharma-yuk-te
vyā-sa-bhā-ṣya-sam-e-tāni
vyati-krāmati
Xiuyao
yādava-bhaṭṭa
yāda-va-śarma-ṇā
yādava-sūri
yājña-valkya-smṛti
yājña-valkya-smṛtiḥ
yantrā-dhyāya
Yantra-rāja-vicāra-viṃśā-dhyāyī
yavanā-cā-rya
yoga-bhā-ṣya-vyā-khyā-rūpaṃ
yoga-cintā-maṇi
yoga-cintā-maṇiḥ
yoga-ratnā-kara
yoga-sāra-mañjarī
yoga-sāra-sam-uc-caya
yoga-sāra-saṅ-graha
yoga-śikh-opa-ni-ṣat
yoga-tārā-valī
yoga-yājña-val-kya
yoga-yājña-valkya-gītāsūpa-ni-ṣatsu
yogi-yājña-valkya-smṛti
yoshi-mizu
yukta-bhava-deva
}
%%%%%%%%%%%%%%%%%%%%
%Sanskrit:
%%%%%%%%%%%%%%%%%%%%
\textsanskrit{\hyphenation{%
    dhanva-ntariṇopa-diṣ-ṭaḥ
suśruta-nāma-dheyena
tac-chiṣyeṇa
    su-śruta-san-dīpana-bhāṣya
    cikitsā-sthāna
tulya-sau-vīrāñjana
indra-gopa
dṛṣṭi-maṇḍala
uc-chiṅga-na
vi-vi-dha-tṛṇa-kāṣṭha-pāṣāṇa-pāṃ-su-loha-loṣṭāsthi-bāla-nakha-pūyā-srāva-duṣṭa-vraṇāntar-garbha-śalyo-ddharaṇārthaṃ
śrī-ḍalhaṇācārya-vi-raci-tayāni-bandha-saṃ-grahākhya-vyā-khyayā
ni-dāna-sthānasyaśrī-gaya-dāsācārya-vi-racitayānyāya-candri-kā-khya-pañjikā-vyā-khyayā
casam-ul-lasi-tāmaharṣiṇāsu-śrutenavi-raci-tāsu-śruta-saṃ-hitā
bhartṛhari-viracitaḥ
śatakatrayādi-subhāṣitasaṃgrahaḥ
mahā-kavi-bhartṛ-hari-praṇīta-tvena
nīti-śṛṅgāra-vai-rāgyādi-nāmnāsamākhyā-tānāṃ
subhāṣitānāṃ
su-pariṣkṛta-saṃgrahaḥ
su-vistṛta-pari-cayātmikyāṅla-prastāvanā-vividha-pāṭhān-tara-pari-śiṣṭādi-sam-anvitaḥ
ācārya-śrī-jina-vijayālekhitāgra-vacanālaṃ-kṛtaś-ca
abhaya-deva-sūri-vi-racita-vṛtti-vi-bhūṣi-tam
abhi-dhar-ma
abhi-dhar-ma-ko-śa
abhi-dhar-ma-ko-śa-bhā-ṣya
abhi-dharma-kośa-bhāṣyam
abhyaṃ-karopāhva-vāsu-deva-śāstri-vi-racita-yā
agni-veśa
āhā-ra-vi-hā-ra-pra-kṛ-tiṃ
ahir-budhnya
ahir-budhnya-saṃ-hitā
akusī-dasya
alter-na-tively
amara-bharati
amara-bhāratī
āmla
amlīkā
ānan-da-rā-ya
anna-mardanādi-bhiś
anu-bhav-ād
anu-bhū-ta-viṣayā-sam-pra-moṣa
anu-bhū-ta-viṣayā-sam-pra-moṣaḥ
anu-māna
anu-miti-mānasa-vāda
ariya-pary-esana-sutta
ārogya-śālā-karaṇā-sam-arthas
ārogya-śālām
ārogyāyopa-kal-pya
arś-āṃ-si
ar-tha
ar-thaḥ
ārya-bhaṭa
ārya-lalita-vistara-nāma-mahā-yāna-sūtra
ārya-mañju-śrī-mūla-kalpa
ārya-mañju-śrī-mūla-kalpaḥ
asaṃ-pra-moṣa
āsana
āsanam
āsanaṃ
asid-dhe
aṣṭāṅga-hṛdaya
aṣṭāṅga-hṛdaya-saṃ-hitā
aṣṭ-āṅga-saṅ-graha
aṣṭ-āṅgā-yur-veda
aśva-gan-dha-kalpa
aśva-ghoṣa
ātaṅka-darpaṇa
ātaṅka-darpaṇa-vyā-khyā-yā
atha-vā
ātu-r-ā-hā-ra-vi-hā-ra-pra-kṛ-tiṃ
aty-al-pam
auṣa-dha-pāvanādi-śālāś
ava-sāda-na
avic-chin-na-sam-pra-dāya-tvād
āyur-veda
āyur-veda-sāra
āyur-vedod-dhāra-ka-vaid-ya-pañc-ānana-vaid-ya-rat-na-rāja-vaid-ya-paṇḍi-ta-rā-ma-pra-sāda-vaid-yo-pādhyā-ya-vi-ra-ci-tā
bahir-deśa-ka
bala-bhadra
bāla-kṛṣṇa
bau-dhā-yana-dhar-ma-sūtra
bhadrā-sana
bhadrā-sanam
bha-ga-vad-gī-tā
bha-ga-vat-pāda
bhaṭṭot-pala-vi-vṛti-sahitā
bhṛtyāva-satha-saṃ-yuktām
bhū-miṃ
bhu--va-na-dī-pa-ka
bīja-pallava
bodhi-sat-tva-bhūmi
brāhmaṇa-pra-mukha-nānā-sat-tva-vyā-dhi-śānty-ar-tham
brāhmaṇa-pra-mukha-nānā-sat-tve-bhyo
brahmāṇḍa-mahā-purā-ṇa
brahmāṇḍa-mahā-purā-ṇam
brāhma-sphu-ṭa-siddhānta
brahma-vi-hāra
brahma-vi-hāras
bṛhad-āraṇya-ka
bṛhad-yā-trā
bṛhad-yogi-yājña-valkya-smṛti
bṛhad-yogī-yājña-valkya-smṛti
bṛhaj-jāta-kam
cak-ra-dat-ta
cak-ra-pā-ṇi-datta
cā-luk-ya
caraka-prati-saṃ-s-kṛta
caraka-prati-saṃ-s-kṛte
cara-ka-saṃ-hitā
ca-tur-thī-vi-bhak-ti
cau-kham-ba
cau-luk-yas
chau-kham-bha
chun-nam
cikit-sā-saṅ-gra-ha
daiva-jñālaṃ-kṛti
daiva-jñālaṅ-kṛti
darśa-nāṅkur-ābhi-dhayāvyā-khya-yā
deva-nagari
deva-nāgarī
dhar-ma-megha
dhar-ma-meghaḥ
dhyā-na-grahopa-deśā-dhyā-yaś
dṛṣṭ-ān-ta
dṛṣṭ-ār-tha
dvāra-tvam
evaṃ-gṛ-hī-tam
evaṃ-vi-dh-a-sya
gala-gaṇḍa
gala-gaṇḍādi-kar-tṛ-tvaṃ
gan-dh-ā-ra
gar-bha-śa-rī-ram
gaurī-kāñcali-kā-tan-tra
gauta-mādi-tra-yo-da-śa-smṛty-ātma-kaḥ
gheraṇḍa-saṃ-hitā
gran-tha-śreṇi
gran-tha-śreṇiḥ
guru-maṇḍala-grantha-mālā
hari-śāstrī
hari-śās-trī
haṭha-yoga
hāyana-rat-na
hema-pra-bha-sūri
hetv-ābhā-sa
hīna-mithy-āti-yoga
hīna-mithy-āti-yogena
hindī-vyā-khyā-vi-marśope-taḥ
hoern-le
idam
ijya-rkṣe
ikka-vālaga
ity-arthaḥ
jābāla-darśanopa-ni-ṣad
jal-pa-kal-pa-tāru
jam-bū-dvī-pa
jam-bū-dvī-pa-pra-jña-pti
jam-bū-dvī-pa-pra-jña-pti-sūtra
jāta-ka-kar-ma-pad-dhati
jinā-agama-grantha-mālā
jī-vā-nan-da-nam
jñā-na-nir-mala
jñā-na-nir-malaṃ
jya-rkṣe
kāka-caṇḍīśvara-kal-pa-tan-tra
kā-la-gar-bhā-śa-ya-pra-kṛ-tim
kā-la-gar-bhā-śa-ya-pra-kṛ-tiṃ
kali-kāla-sarva-jña
kali-kāla-sarva-jña-śrī-hema-candrācārya-vi-raci-ta
kali-kāla-sarva-jña-śrī-hema-candrācārya-vi-raci-taḥ
kali-yuga
kal-pa-sthāna
kar-ma
kar-man
kārt-snyena
katham
kāvya-mālā
keśa-va-śāstrī
kol-ka-ta
kṛṣṇa-pakṣa
kṛtti-kā
kṛtti-kās
kula-pañji-kā
ku-māra-saṃ-bhava
lab-dhāni
mada-na-phalam
mādha-va
Mādhava-karaaita-reya-brāhma-ṇa
Mādhava-ni-dāna
mādhava-ni-dā-nam
madhu-kośa
madhu-kośākhya-vyā-khya-yā
madhya
madhye
ma-hā-bhū-ta-vi-kā-ra-pra-kṛ-tiṃ
mahā-deva
mahā-mati-śrī-mādhava-kara-pra-ṇī-taṃ
mahā-muni-śrī-mad-vyāsa-pra-ṇī-ta
mahā-muni-śrī-mad-vyāsa-pra-ṇī-taṃ
maha-rṣi-pra-ṇīta-dharma-śāstra-saṃ-grahaḥ
mahā-sacca-ka-sutta
mahau-ṣadhi-pari-cchadāṃ
mahā-vra-ta
mahā-yāna-sūtrālaṅ-kāra
mano-ratha-nandin
matsya-purāṇam
me-dhā-ti-thi
medhā-tithi
mithilā-stha
mithilā-stham
mithilā-sthaṃ
mud-rā-yantr-ā-laye
muktā-pīḍa
mūla-pāṭha
nakṣa-tra
nandi-purāṇoktārogya-śālā-dāna-phala-prāpti-kāmo
nara-siṃha
nara-siṃha-bhāṣya
nārā-ya-ṇa-dāsa
nārā-yaṇa-kaṇṭha
nārā-yaṇa-paṇḍi-ta-kṛtā
nava-pañca-mayor
nidā-na-sthā-na-sya
ni-ghaṇ-ṭu
nir-anta-ra-pa-da-vyā-khyā
nir-ṇaya-sā-gara
nir-ṇaya-sā-gara-yantr-ā-laye
nirūha-vasti
niś-cala-kara
ni-yukta-vaidyāṃ
nya-grodha
nya-grodho
nyāya-śās-tra
nyāya-sū-tra-śaṃ-kar
okaḥ-sātmya
okaḥ-sātmyam
okaḥ-sātmyaṃ
oka-sātmya
oka-sātmyam
oka-sātmyaṃ
oṣṭha-saṃ-puṭa
ousha-da-sala
padma-pra-bha-sūri
padma-sva-sti-kārdha-candrādike
paitā-maha-siddhā-nta
pañca-karma
pañca-karma-bhava-rogāḥ
pañca-karmādhi-kāra
pañca-karma-vi-cāra
pāñca-rātrā-gama
pañca-siddh-āntikā
pari-bhāṣā
pari-likh-ya
pātañ-jala-yoga-śās-tra
pātañ-jala-yoga-śās-tra-vi-varaṇa
pat-añ-jali
pāṭī-gaṇita
pāva-suya
pim-pal-gaon
pipal-gaon
pit-ta-kṛt
pit-ta-śleṣma-ghna
pit-ta-śleṣma-medo-meha-hik-kā-śvā-sa-kā-sāti-sā-ra-cchardi-tṛṣṇā-kṛmi-vi-ṣa-pra-śa-ma-naṃ
prā-cya
prā-cya-hindu-gran-tha-śreṇiḥ
prācya-vidyā-saṃ-śodhana-mandira
pra-dhān-āṅ-gaṃ
pra-dhān-in
pra-ka-shan
pra-kṛ-ti
pra-kṛ-tiṃ
pra-mā-ṇa-vārt-tika
pra-saṅ-khyāne
pra-śas-ta-pāda-bhāṣya
pra-śna-pra-dīpa
pra-śnārṇa-va-plava
praśnārṇava-plava
pra-śna-vaiṣṇava
pra-śna-vai-ṣṇava
prati-padyate
pra-ty-akṣa
pra-yat-na-śai-thilyā-nan-ta-sam-ā-pat-ti-bhyām
pra-yat-na-śai-thilyā-nān-tya-sam-ā-pat-ti-bhyāṃ
pra-yatna-śai-thilya-sya
puṇya-pattana
pūrṇi-mā-nta
rāja-kīya
rajjv-ābhyas-ya
rāma-kṛṣṇa
rasa-ratnā-kara
rasa-vai-śeṣika-sūtra
rogi-svasthī-karaṇānu-ṣṭhāna-mātraṃ
rūkṣa-vasti
sād-guṇya
śākalya-saṃ-hitā
sam-ā-mnāya
sāmañña-pha-la-sutta
sama-ran-gana-su-tra-dhara
samā-raṅga-ṇa-sū-tra-dhāra
sama-ra-siṃ-ha
sama-ra-siṃ-haḥ
saṃ-hitā
sāṃ-sid-dhi-ka
saṃ-śo-dhana
sam-ul-lasi-tam
śāndilyopa-ni-ṣad
śaṅ-kara
śaṅ-kara-bha-ga-vat-pāda
Śaṅ-kara-nārā-yaṇa
saṅ-khyā
sāṅ-kṛt-yā-yana
san-s-krit
sap-tame
śāra-dā-tila-ka-tan-tra
śa-raṅ-ga-deva
śār-dūla-karṇā-va-dāna
śā-rī-ra
śā-rī-ra-sthāna
śārṅga-dhara
śārṅga-dhara-saṃ-hitā
sar-va
sarva-darśana-saṃ-grahaḥ
sar-va-dar-śāna-saṅ-gra-ha
sar-va-dar-śāna-saṅ-gra-haḥ
sarv-arthāvi-veka-khyā-ter
sar-va-tan-tra-sid-dhān-ta
sar-va-tan-tra-sid-dhān-taḥ
sarva-yoga-sam-uc-caya
sar-va-yogeśvareśva-ram
śāstrā-rambha-sam-artha-na
śāstrāram-bha-sam-arthana
ṣaṭ-pañcā-śi-kā
sat-tva
saunda-ra-na-nda
sid-dha
sid-dha-man-tra
sid-dha-man-trā-hvayo
sid-dha-man-tra-pra-kāśa
sid-dha-man-tra-pra-kāśaḥ
sid-dha-man-tra-pra-kāśaś
sid-dh-ān-ta
siddhānta-śiro-maṇ
sid-dha-yoga
sid-dhi-sthāna
śi-va-śar-ma-ṇā
ska-nda-pu-rā-ṇa
sneha-basty-upa-deśāt
sodā-haraṇa-saṃ-s-kṛta-vyā-khyayā
śodha-ka-pusta-kaṃ
śo-dha-na-ci-kitsā
so-ma-val-ka
śrī-mad-devī-bhāga-vata-mahā-purāṇa
srag-dharā-tārā-sto-tra
śrī-hari-kṛṣṇa-ni-bandha-bhava-nam
śrī-hema-candrā-cārya-vi-raci-taḥ
śrī-kaṇtha-datta
śrī-kaṇtha-dattā-bhyāṃ
śrī-kṛṣṇa-dāsa
śrī-mad-amara-siṃha-vi-racitam
śrī-mad-aruṇa-dat-ta-vi-ra-ci-tayā
śrī-mad-bhaṭṭot-pala-kṛta-saṃ-s-kṛta-ṭīkā-sahitam
śrī-mad-dvai-pā-yana-muni-pra-ṇītaṃ
śrī-mad-vāg-bha-ṭa-vi-ra-ci-tam
śrī-maṃ-trī-vi-jaya-siṃha-suta-maṃ-trī-teja-siṃhena
śrī-mat-kalyāṇa-varma-vi-racitā
śrīmat-sāyaṇa-mādhavācārya-pra-ṇītaḥ
śrī-vā-cas-pati-vaidya-vi-racita-yā
śrī-vatsa
śrī-vi-jaya-rakṣi-ta
sthānāṅga-sūtra
sthira-sukha
sthira-sukham
strī-niṣevaṇa
śukla-pakṣa
su-śru-ta-saṃ-hitā
sū-tra
sūtrārthānān-upa-patti-sūca-nāt
sūtra-sthāna
su-varṇa-pra-bhāsot-tama-sū-tra
svalpauṣadha-dāna-mā-tram
śvetāśva-taropa-ni-ṣad
tad-upa-karaṇa-tāmra-kaṭāha-kalasādi-pātra-pari-cchada-nānā-vidha-vyādhi-śānty-ucitauṣadha-gaṇa-yathokta-lakṣaṇa-vaidya-nānā-vidha-pari-cāraka-yutāṃ
tājaka-muktā-valeḥ
tājika-kau-stu-bha
tājika-nīla-kaṇṭhī
tājika-yoga-sudhā-ni-dhi
tāmra-paṭṭādi-li-khi-tāṃ
tan-nir-vāhāya
tapo-dhana
tapo-dhanā
tārā-bhakti-su-dhārṇava
tārtīya-yoga-su-sudhā-ni-dhi
tegi-ccha
te-jaḥ-siṃ-ha
trai-lok-ya
trai-lokya-pra-kāśa
tri-piṭa-ka
tri-var-gaḥ
un-mār-ga-gama-na
upa-de-śa
upa-patt-ti
ut-sneha-na
utta-rā-dhyā-ya-na
uttara-sthāna
uttara-tantra
vāchas-pati
vād-ā-valī
vai-śā-kha
vai-ta-raṇa-vasti
vai-ta-raṇok-ta-guṇa-gaṇa-yu-k-taṃ
vājī-kara-ṇam
vāk-patis
vākya-śeṣa
vākya-śeṣaḥ
varā-ha-mihi-ra
va-ra-na-si
vā-rā-ṇa-sī
var-mam
var-man
varṇa-sam-ā-mnāya
va-siṣṭha-saṃ-hitā
vā-siṣṭha-saṃ-hitā
vasu-bandhu
vasu-bandhu
vāta-ghna-pit-talāl-pa-ka-pha
vātsyā-ya-na
vidya-bhu-sana
vidyā-bhū-ṣaṇa
vi-jaya-siṃ-ha
vi-jñāna-bhikṣu
vi-kal-pa
vi-kamp-i-tum
vi-mā-na-sthāna
vi-racita-yā
vishveshvar-anand
vi-śiṣṭ-āṃśena
viṣṇu-dharmot-tara-purāṇa
viśrāma-gṛha-sahitā
vi-suddhi-magga
vopa-de-vīya-sid-dha-man-tra-pra-kāśe
vyādhi-pratī-kārār-tham
vyāḍī-ya-pa-ri-bhā-ṣā-vṛtti
vyati-krāmati
vy-ava-haranti
yādava-bhaṭṭa
yāda-va-śarma-ṇā
yādava-sūri
yājña-valkya-smṛti
yavanā-cā-rya
yoga-ratnā-kara
yoga-sāra-sam-uc-caya
yoga-sāra-sam-uc-cayaḥ
yoga-sūtra-vi-vara-ṇa
yoga-yājña-valkya
yoga-yājña-valkya-gītāsūpa-ni-ṣatsu
yoga-yājña-valkyaḥ
yogi-yājña-valkya-smṛti
yuk-tiḥ
yuk-tis
}}
\normalfontlatin
\endinput


\lineation{page}
\begingroup
\beginnumbering


\section*{Chapter 1}

\pstart
\edtext{}{
  \Afootnote{\textsc{[pre]} \textsc{(f. 1v)}(From 1v)oṃ namaḥ kamalahastāya|| K; \textsc{[pre]} \textsc{(f. 1v)}(From 1v)namaḥ sar\uwave{vvabuddha}bodhisatvebhyaḥ|| namo nāgārjunapādāya|| N; \textsc{[pre]} \textsc{(f. 1v)}(From 1v)oṃ namo dhanvantaraye || H.}
}
\pend


\pstart
[1] athāto vedotpattim \edtext{adhyāyaṃ}{
  \Afootnote{\textbf{a}dhyāyaṃ K; ādhyāyaṃ N; nāmādhyāyaṃ H.}
} vyākhyāsyāmaḥ | 
\pend


\pstart
\edtext{}{
  \Afootnote{\textsc{[pre]} \textsc{[pre]}yathovāca bhagavān dhanvantariḥ||2|| A.}
}
\pend


\pstart
[2] atha khalu bhagavantam \edtext{amaravaram}{
  \Afootnote{amaravara N.}
} ṛṣigaṇaparivṛttam \edtext{āśramasthaṃ}{
  \Afootnote{āśramastha K.}
} kāśirājaṃ \edtext{divodāsam}{
  \Afootnote{\textsc{[add]} dhanvantarim A H; \textsc{[add]} \textbf{dhanvantari} K.}
} \edtext{aupadhenava}{
  \Afootnote{°navaḥ K.}
} \edtext{vaitaraṇaurabhra}{
  \Afootnote{vaistara° N; vairaṇorabhra H.}
} \edtext{puṣkalāvata}{
  \Afootnote{pauṣka° A N.}
} \edtext{karavīra}{
  \Afootnote{-1karavīrya A.}
} gopurarakṣita \edtext{bhoja}{
  \Afootnote{\textsc{[om]} A.}
} suśruta \edtext{prabhṛtaya}{
  \Afootnote{prabhṛtavya N.}
} ūcuḥ\edlabel{SS.1.1.3-17} \edtext{|}{
  \linenum{|\xlineref{SS.1.1.3-17}}\lemma{ūcuḥ |}\Afootnote{ūcu K.}
}
\pend


\pstart
[3]\emph{\edlabel{SS.1.1.4-0}} \edtext{bhagavañ}{
  \linenum{|\xlineref{SS.1.1.4-0}}\lemma{[3] bhagavañ}\Afootnote{\textsc{[sū.1.4]}bhagavan A.}
} \edtext{śārīramānasāgantubhir\edlabel{SS.1.1.4-2}}{
  \linenum{|\xlineref{SS.1.1.4-0}}\lemma{[3]\ldots śārīramānasāgantubhir}\Afootnote{bhagavac chārī° H.}
\lemma{śārīramānasāgantubhir}  \Afootnote{charī° N; \uwave{chā}rīramānasāgantubhi\textsc{(gap of 1 akṣara, damaged)} K.}
} \edtext{vyādhibhir\edlabel{SS.1.1.4-3}}{
  \linenum{|\xlineref{SS.1.1.4-2}}\lemma{śārīramānasāgantubhir vyādhibhir}\Afootnote{°tuvyādhibhir A.}
\lemma{vyādhibhir}  \Afootnote{vyādhibhi N.}
} \edtext{vividhavedanābhighātopadrutān}{
  \linenum{|\xlineref{SS.1.1.4-3}}\lemma{vyādhibhir vividhavedanābhighātopadrutān}\Afootnote{vyādhibhi\uwave{rvi}\textsc{(l. 2)}dhavedanābhighātopadrutāṃ\uwave{} K.}
\lemma{vividhavedanābhighātopadrutān}  \Afootnote{°drutāṃ H.}
} \edtext{sanāthān\edlabel{SS.1.1.4-5}}{
  \Afootnote{āsa° N; \textsc{[add]} apy A.}
} \edtext{anāthavad}{
  \linenum{|\xlineref{SS.1.1.4-5}}\lemma{sanāthān anāthavad}\Afootnote{sanāthānanā° H.}
} viceṣṭamānān vikrośataś ca \edtext{mānavān}{
  \Afootnote{mānavānam K; mānavānām H.}
} abhisamīkṣya manasi naḥ pīḍābhavat\edlabel{SS.1.1.4-14} \edtext{|}{
  \linenum{|\xlineref{SS.1.1.4-14}}\lemma{pīḍābhavat |}\Afootnote{pīḍā bhavati A.}
}
\pend


\pstart
[4] teṣāṃ sukhaiṣiṇāṃ \edtext{rogopaśamārtham}{
  \Afootnote{rogopagamārtham N.}
} ātmanaś ca \edtext{prāṇayātrārtham}{
  \Afootnote{prāṇayātārtham N; \textsc{[add]} prajāhitahetor A.}
} \edtext{āyurvedaṃ}{
  \Afootnote{āyuśvedam K; \textsc{[add]} śrotum A.}
} icchāma upadiśyamānam\edlabel{SS.1.1.4b-9} \edtext{|}{
  \linenum{|\xlineref{SS.1.1.4b-9}}\lemma{upadiśyamānam |}\Afootnote{ihopa° A.}
} \edtext{atrāyattam}{
  \Afootnote{attrāya untam H.}
} \edtext{aihikam}{
  \Afootnote{auhikam H.}
} āmuṣmikañ ca śreyas tad bhagavantam \edtext{upasannāḥ}{
  \Afootnote{upapannāḥ A; upasannā \uline{K} N H.}
} \edtext{smaḥ}{
  \Afootnote{sma K \uline{N}.}
} śiṣyatveneti\edlabel{SS.1.1.4b-20} \edtext{|}{
  \linenum{|\xlineref{SS.1.1.4b-20}}\lemma{śiṣyatveneti |}\Afootnote{śiṣyatve\textsc{(l. 3)} K.}
}
\pend


\pstart
[5] tān uvāca bhagavān svāgatam vaḥ \edtext{sarva}{
  \Afootnote{sava N.}
} evāmīmāṃsyā \edtext{adhyāpyāś}{
  \Afootnote{a\textbf{dhyā}pyāś K.}
} ca \edtext{bhavanto}{
  \Afootnote{bhagavanto H.}
} vatsāḥ | 
\pend


\pstart
[6] iha khalv \edtext{āyurvedo}{
  \Afootnote{āyurvedaṃ A.}
} nāma\edlabel{SS.1.1.6-4} \edtext{yad}{
  \Afootnote{yam K.}
} \edtext{upāṅgam}{
  \linenum{|\xlineref{SS.1.1.6-4}}\lemma{nāma\ldots upāṅgam}\Afootnote{nāmopāṅgam A.}
\lemma{upāṅgam}  \Afootnote{upāgam N.}
} atharvavedasyoktam\edlabel{SS.1.1.6-7} anutpādyaiva \edtext{ca}{
  \linenum{|\xlineref{SS.1.1.6-7}}\lemma{atharvavedasyoktam\ldots ca}\Afootnote{°dasyānutpādyaiva A.}
} prajāḥ ślokaśatasahasram adhyāyasahasrañ ca kṛtavān svayambhūr \edtext{alpāyuṣkālpamedhastvañ}{
  \Afootnote{tato 'lpāyuṣṭvam alpa° A.}
} cālokya narāṇām bhūyo 'ṣṭadhā praṇītavān \edtext{|}{
  \Afootnote{\textsc{[add]} tad yathā | K.}
}
\pend


\pstart
[7]\emph{\edlabel{SS.1.1.7-0}} tad \edtext{yathā}{
  \linenum{|\xlineref{SS.1.1.7-0}}\lemma{[7]\ldots yathā}\Afootnote{\textsc{[om]} K.}
} śalyaṃ śālākyaṃ kāyacikitsā\edlabel{SS.1.1.7-5} bhūtavidyā \edtext{kaumārabhṛtyam}{
  \linenum{|\xlineref{SS.1.1.7-5}}\lemma{kāyacikitsā\ldots kaumārabhṛtyam}\Afootnote{kāyacikitsābhūtavidyākau° N H; kā\textsc{(gap of 1 akṣara, damaged)}\textsc{(l. 4)}cikitsābhūtavidyākau° K.}
} agadatantraṃ\edlabel{SS.1.1.7-8} \edtext{rasāyanatantraṃ}{
  \linenum{|\xlineref{SS.1.1.7-8}}\lemma{agadatantraṃ rasāyanatantraṃ}\Afootnote{agadatantrara° H.}
} vājīkaraṇatantram iti | 
\pend


\pstart
[8]\emph{\edlabel{SS.1.1.8.1-0}} \edtext{athāsya}{
  \linenum{|\xlineref{SS.1.1.8.1-0}}\lemma{[8] athāsya}\Afootnote{\textsc{(gap of 2 akṣaras, damaged)}sya K.}
} pratyekāṅgalakṣaṇasamāsaḥ\edlabel{SS.1.1.8.1-2} \edtext{|}{
  \linenum{|\xlineref{SS.1.1.8.1-2}}\lemma{pratyekāṅgalakṣaṇasamāsaḥ |}\Afootnote{pratyaṅga° A.}
}
\pend


\pstart
[9] tatra \edtext{śalyan}{
  \Afootnote{śalyaṃ A; śaṃlyan K N.}
} nāma \edtext{vividhatṛṇakāṣṭhapāṣāṇapāṃsulohaloṣṭāsthibālanakhapūyāsrāvaduṣṭavraṇāntargarbhaśalyoddharaṇārthaṃ}{
  \Afootnote{vividhatṛṇakāṣṭhapāṣāṇapāṃśulo° A; vividhatṛṇakāṣṭhapāṣāṇapāṃśuloṣṭāsthibālanakhapūyāśrāva° H; °napūyāsrāva\uwave{du}\textsc{(gap of 2 akṣaras, damaged)}\uwave{ṇā}n\textsc{(gap of 1 akṣara, damaged)}rgarbhaśalyoddharaṇārthaṃ N.}
} \edtext{yantraśastrakṣārāgnipraṇidhānaviniścayārthañ}{
  \Afootnote{°navraṇaviniścayārthaṃ A.}
} ca \edtext{ṣaṣṭyābhidhānair\edlabel{SS.1.1.8.1a-7}}{
  \Afootnote{ṣaṣṭyābhidh\textsc{(gap of 1 akṣara, damaged)}\textsc{(l. 5)}nair K.}
} iti \edtext{|}{
  \linenum{|\xlineref{SS.1.1.8.1a-7}}\lemma{ṣaṣṭyābhidhānair\ldots |}\Afootnote{\textsc{[om]} A.}
}
\pend


\pstart
[10]\emph{\edlabel{SS.1.1.8.2-0}} \edtext{śālākyatantran}{
  \linenum{|\xlineref{SS.1.1.8.2-0}}\lemma{[10] śālākyatantran}\Afootnote{\textsc{[sū.1.8.2]} śālākyaṃ A.}
} \edtext{nāmordhvajatrugatānāṃ}{
  \Afootnote{nāmordhvajātru° H.}
} \edtext{vikārāṇāṃ}{
  \Afootnote{\textsc{[om]} A; vikārāṇā N.}
} \edtext{śravaṇanayanavadanaghrāṇādisaṃśritānāṃ}{
  \Afootnote{°disa\textsc{(gap of 1 akṣara, damaged)}śitānāṃ K; °tānā N.}
} \edtext{vikārāṇām}{
  \Afootnote{vyādhīnām A.}
} upaśamakaraṇārtham\edlabel{SS.1.1.8.2-6} \edtext{|}{
  \linenum{|\xlineref{SS.1.1.8.2-6}}\lemma{upaśamakaraṇārtham |}\Afootnote{°manārtham || A.}
}
\pend


\pstart
[11]\emph{\edlabel{SS.1.1.8.3-0}} \edtext{kāyacikītsā}{
  \linenum{|\xlineref{SS.1.1.8.3-0}}\lemma{[11] kāyacikītsā}\Afootnote{kā° K.}
} \edtext{nāma}{
  \Afootnote{nāma K.}
} \edtext{sarvaśarīrāvasthitānāṃ\edlabel{SS.1.1.8.3-3}}{
  \Afootnote{sarvāṅgasaṃśritānāṃ A; \textsc{[add]} ca N.}
} \edtext{vyādhīnām}{
  \linenum{|\xlineref{SS.1.1.8.3-3}}\lemma{sarvaśarīrāvasthitānāṃ vyādhīnām}\Afootnote{°tā\uuline{}dhīnām K.}
} \edtext{upaśamakaraṇārthaṃ}{
  \Afootnote{\textsc{[om]} A.}
} \edtext{jvaraśophagulmaraktapittonmādāpasmārapramehātisārādīnāñ}{
  \Afootnote{jvararaktapittaśoṣonmādāpasmārakuṣṭhame° A; °hātīsārādīnāñ N \uline{H}.}
} ca\edlabel{SS.1.1.8.3-7} \edtext{|}{
  \linenum{|\xlineref{SS.1.1.8.3-7}}\lemma{ca |}\Afootnote{upaśamanārtham || A.}
}
\pend


\pstart
[12] bhūtavidyā nāma \edtext{devagandharvayakṣarākṣasapitṛpiśācavināyakanāgagrahopasṛṣṭacetasāṃ}{
  \Afootnote{devadānava° H; devadīnava° N; deva\textbf{✗}gandha\textsc{(gap of 2 akṣaras, damaged)}\textsc{(l. 6)}kṣa° K; devāsuragandharvayakṣarakṣaḥpitṛpiśācanāgagrahādyupa° A.}
} śāntikarmabaliharaṇādigrahopaśamanārthaṃ\edlabel{SS.1.1.8.4-4} \edtext{|}{
  \linenum{|\xlineref{SS.1.1.8.4-4}}\lemma{śāntikarmabaliharaṇādigrahopaśamanārthaṃ |}\Afootnote{śāntikarmabaloha° A.}
}
\pend


\pstart
[13]\emph{\edlabel{SS.1.1.8.5-0}} \edtext{kaumārabhṛtyan}{
  \linenum{|\xlineref{SS.1.1.8.5-0}}\lemma{[13] kaumārabhṛtyan}\Afootnote{°bhṛtyaṃ A.}
} nāma \edtext{kumārabharaṇadhātrīkṣīradoṣasaṃśodhanārthaṃ}{
  \Afootnote{kumārābha° H.}
} \edtext{duṣṭastanyagrahasamutthitānāñ}{
  \Afootnote{°mutthānāṃ A.}
} ca vyādhīnām upaśamakaraṇārtham\edlabel{SS.1.1.8.5-7} \edtext{|}{
  \linenum{|\xlineref{SS.1.1.8.5-7}}\lemma{upaśamakaraṇārtham |}\Afootnote{°manārtham || A; °ṇārthaḥ | N.}
}
\pend


\pstart
[14]\emph{\edlabel{SS.1.1.8.6-0}} \edtext{agadatantran}{
  \linenum{|\xlineref{SS.1.1.8.6-0}}\lemma{[14] agadatantran}\Afootnote{°tantraṃ A.}
} nāma \edtext{sarpakīṭadaṣṭaviṣavyañjanārthaṃ}{
  \Afootnote{sarpakīṭalūtāmūśikādida° A; sarppakīṭalūtādaṣṭasarīsṛ\textsc{(l. 5)}pavi° H; °ṭalūtādaṣṭasarīsṛpaviṣavyañjanārtha N.}
} \edtext{vividhaviṣavegopaśamanārthañ}{
  \Afootnote{vividhaviṣasaṃyogo° A; °\uwave{go}\textsc{(gap of 1 akṣara, damaged)}\textsc{(l. 7)}śamanārthañ K.}
} ca | 
\pend


\pstart
[15]\emph{\edlabel{SS.1.1.8.7-0}} \edtext{rasāyanatantran}{
  \linenum{|\xlineref{SS.1.1.8.7-0}}\lemma{[15] rasāyanatantran}\Afootnote{°tantraṃ A.}
} nāma \edtext{vayaḥsthāpanam\edlabel{SS.1.1.8.7-3}}{
  \Afootnote{vayasthā° N.}
} \edtext{āyurmedhākaraṇaṃ}{
  \Afootnote{°dhābalakaraṃ A.}
} \edtext{vyādhyupaśamakaraṇārthañ}{
  \linenum{|\xlineref{SS.1.1.8.7-3}}\lemma{vayaḥsthāpanam\ldots vyādhyupaśamakaraṇārthañ}\Afootnote{va\textsc{(gap of 11 akṣaras, damaged)}pa° K.}
\lemma{vyādhyupaśamakaraṇārthañ}  \Afootnote{rogāpaharaṇasamarthaṃ A.}
} ca| 
\pend


\pstart
[16]\emph{\edlabel{SS.1.1.8.8-0}} \edtext{vājīkaraṇatantran}{
  \linenum{|\xlineref{SS.1.1.8.8-0}}\lemma{[16] vājīkaraṇatantran}\Afootnote{°tantraṃ A.}
} nāma\edlabel{SS.1.1.8.8-2} \edtext{svalpaduṣṭakṣīṇaviśuṣkaretasāṃ}{
  \linenum{|\xlineref{SS.1.1.8.8-2}}\lemma{nāma svalpaduṣṭakṣīṇaviśuṣkaretasāṃ}\Afootnote{nāmālpa° A.}
\lemma{svalpaduṣṭakṣīṇaviśuṣkaretasāṃ}  \Afootnote{svalpaduṣṭakṣī\textsc{(f. 2v)}(From 2v) ṇa° H.}
} \edtext{śukrāpyāyanaprasādopacayajanananimittaṃ}{
  \Afootnote{āpyā° A; śukrāpyāyatana° H.}
} praharṣajananārthañ ca | 
\pend


\pstart
[17] evam ayam āyurvedo 'ṣṭāṅga upadiśyate | atra kasmai kiṃ\edlabel{SS.1.1.9-9} \edtext{varṇyatām}{
  \Afootnote{ucyatām A.}
} iti \edtext{|}{
  \linenum{|\xlineref{SS.1.1.9-9}}\lemma{kiṃ\ldots |}\Afootnote{ki\textsc{(gap of 5 akṣaras, damaged)} K.}
}
\pend


\pstart
[18]\emph{\edlabel{SS.1.1.10-0}} ta ūcur asmākaṃ \edtext{sarvam}{
  \Afootnote{sarveṣām A N.}
} \edtext{eva}{
  \linenum{|\xlineref{SS.1.1.10-0}}\lemma{[18]\ldots eva}\Afootnote{\textsc{[om]} K.}
\lemma{eva}  \Afootnote{\textsc{[add]} na\uuline{nameva}  N.}
} \edtext{śalyajñānam}{
  \Afootnote{°na\uwave{m\textbf{+malaṃkṛ+}} K.}
} \edtext{upadiśatu}{
  \Afootnote{alaṅ kṛtvopa° N \uline{H}; mūlaṃ kṛtvopa° A.}
} bhagavān iti \edtext{|}{
  \Afootnote{\textsc{[add]} \uwave{ sa }\textsc{(gap of 13 akṣaras, damaged)} K.}
}
\pend


\pstart
[19]\emph{\edlabel{SS.1.1.11-0}} sa \edtext{uvācaivam}{
  \Afootnote{uvāca evaṃm N.}
} astv iti \edtext{|}{
  \linenum{|\xlineref{SS.1.1.11-0}}\lemma{[19]\ldots |}\Afootnote{\textsc{[om]} K.}
}
\pend


\pstart
[20]\emph{\edlabel{SS.1.1.12-0}} ta ūcur bhūyo \edtext{'smākaṃ}{
  \Afootnote{'pi bhagavantam A; 'smāka N.}
} \edtext{sarveṣām}{
  \linenum{|\xlineref{SS.1.1.12-0}}\lemma{[20]\ldots sarveṣām}\Afootnote{\textsc{(gap of 6 akṣaras, damaged)}ka(\textbf{ṃ})sa° K.}
\lemma{sarveṣām}  \Afootnote{asmākam A.}
} \edtext{evaikamatīnāṃ}{
  \Afootnote{eka° A; °tīnānām N.}
} matam abhisamīkṣya suśruto bhagavantaṃ prakṣyati\edlabel{SS.1.1.12-11} \edtext{|}{
  \linenum{|\xlineref{SS.1.1.12-11}}\lemma{prakṣyati |}\Afootnote{pra\textsc{(f. 2r)}(From 2r)tyakṣya N; prakṣa\textsc{(l. 3)}\textsc{(gap of 2 akṣaras)} H.}
} \edtext{asyopadiśyamānaṃ}{
  \Afootnote{\textsc{[om]} a° N H; asmai copa° A.}
} vayam apy upadhārayiṣyāmaḥ | 
\pend


\pstart
[21] sa \edtext{uvācaivam}{
  \Afootnote{hovā° A.}
} astv iti | 
\pend


\pstart
\edtext{[22]}{
  \Afootnote{\textsc{[pre]} \textsc{[pre]} vatsa suśruta A.}
} iha khalv \edtext{āyurvedaprayojanaṃ}{
  \Afootnote{āyuveda° N; āyurvvede pra° H.}
} \edtext{vyādhyupasṛṣṭasya}{
  \Afootnote{°sṛṣṭānāṃ A.}
} vyādhiparimokṣaḥ \edtext{svastharakṣaṇañ}{
  \Afootnote{svasthasya ra° A.}
} ca\edlabel{SS.1.1.14-16-7} | \edtext{āyur}{
  \linenum{|\xlineref{SS.1.1.14-16-7}}\lemma{ca\ldots āyur}\Afootnote{cāyur K N.}
} asmin \edtext{vidanty\edlabel{SS.1.1.14-16-11}}{
  \Afootnote{vidyate A.}
} \edtext{anena}{
  \linenum{|\xlineref{SS.1.1.14-16-11}}\lemma{vidanty anena}\Afootnote{vi\textbf{ṃ}da\uwave{ntya}nena K.}
\lemma{anena}  \Afootnote{'nena A.}
} \edtext{vāyur\edlabel{SS.1.1.14-16-13}}{
  \Afootnote{vā+āyur A; vāyu N.}
} \edtext{vidyata\edlabel{SS.1.1.14-16-14}}{
  \Afootnote{vindyata K.}
} \edtext{ity}{
  \linenum{|\xlineref{SS.1.1.14-16-13}}\lemma{vāyur\ldots ity}\Afootnote{\textsc{[om]} H.}
  \linenum{|\xlineref{SS.1.1.14-16-14}}\lemma{vidyata ity}\Afootnote{vindatīty A.}
} āyurvedaḥ\edlabel{SS.1.1.14-16-16} \edtext{|}{
  \linenum{|\xlineref{SS.1.1.14-16-16}}\lemma{āyurvedaḥ |}\Afootnote{cāyu° H.}
} tasyāṅgavaram \edtext{āgamapratyakṣānumānopamānair}{
  \Afootnote{ādyaṃ pratyakṣāgamānu° A; °nu\textbf{nā}mānopamānair K.}
} aviruddham ucyamānam upadhārayadhvam\edlabel{SS.1.1.14-16-22} \edtext{|}{
  \linenum{|\xlineref{SS.1.1.14-16-22}}\lemma{upadhārayadhvam |}\Afootnote{upadharaya || A.}
}
\pend


\pstart
[23] etad dhy aṅgaṃ prathamaṃ pradhānaṃ\edlabel{SS.1.1.17-5} \edtext{prāgabhihitvād}{
  \Afootnote{prāgabhidhānatvād N; °hitatvād K H.}
} \edtext{vraṇasaṃrohaṇakaratvād}{
  \linenum{|\xlineref{SS.1.1.17-5}}\lemma{pradhānaṃ\ldots vraṇasaṃrohaṇakaratvād}\Afootnote{prāgabhighātavraṇasaṃrohāt A.}
} \edtext{yajñaśiraḥpradhānasandhānāc}{
  \Afootnote{yajña\textsc{(l. 2)}śirapra° N; yajñaśiraḥsan° A.}
} ca | śrūyate hi yathā \edtext{purā}{
  \Afootnote{\textsc{[om]} A.}
} \edtext{rudreṇa}{
  \Afootnote{\textsc{[add]} yajñasya A N; \textsc{[add]} \textbf{yajña} H.}
} śiraś \edtext{chinnam}{
  \Afootnote{\textsc{[add]} iti tato devā A.}
} \edtext{aśvibhyāṃ}{
  \Afootnote{aśvināv abhigamyocuḥ bhagavantau naḥ śreṣṭhatamau yuvāṃ bhaviṣyathaḥ bhavadbhyāṃ yajñasya śiraḥ sandhātavyam iti | tāv ūcatur evam astv iti | atha tayor arthe devā indraṃ yajñabhagena A.}
} \edtext{punaḥ}{
  \Afootnote{prāsādayan | tābhyāṃ yajñasya śiraḥ A.}
} \edtext{sandhitam}{
  \Afootnote{saṃhitam A.}
} iti\edlabel{SS.1.1.17-21} \edtext{|}{
  \linenum{|\xlineref{SS.1.1.17-21}}\lemma{iti |}\Afootnote{ity \uline{K} H.}
}
\pend


\pstart
[24]\emph{\edlabel{SS.1.1.18-0}} \edtext{aṣṭānām}{
  \linenum{|\xlineref{SS.1.1.18-0}}\lemma{[24] aṣṭānām}\Afootnote{\textsc{[sū.1.18]} aṣṭāsv A.}
} api \edtext{cāyurvedatantrāṇām}{
  \Afootnote{°tantreṣv A.}
} etad evādhikam \edtext{āśukriyākaraṇād}{
  \Afootnote{abhimataṃ āmāśu° A.}
} yantraśastrakṣārāgnipraṇidhānāt sarvatantrasāmānyāc ca | 
\pend


\pstart
[25] tad \edtext{idaṃ}{
  \Afootnote{ida N.}
} \edtext{śāśvataṃ}{
  \Afootnote{śāśvata N.}
} puṇyaṃ \edtext{svargyaṃ}{
  \Afootnote{svar\uwave{ga}\uuline{ya} K; svargyāṃ yaṃ  H.}
} \edtext{yaśasyam\edlabel{SS.1.1.19-6}}{
  \Afootnote{\textbf{ya}śasyam H.}
} \edtext{āyuṣyaṃ}{
  \linenum{|\xlineref{SS.1.1.19-6}}\lemma{yaśasyam āyuṣyaṃ}\Afootnote{yaśasyamāyuṣyaṃ A.}
} vṛttikarañ ca\edlabel{SS.1.1.19-9} \edtext{|}{
  \linenum{|\xlineref{SS.1.1.19-9}}\lemma{ca |}\Afootnote{ceti || A.}
}
\pend


\pstart
[26]\emph{\edlabel{SS.1.1.20-0}} \edtext{tad}{
  \linenum{|\xlineref{SS.1.1.20-0}}\lemma{[26] tad}\Afootnote{\textsc{[om]} A; ta N.}
} brahmā\edlabel{SS.1.1.20-2} \edtext{provāca}{
  \linenum{|\xlineref{SS.1.1.20-2}}\lemma{brahmā provāca}\Afootnote{brahmovāca H.}
} \edtext{tat}{
  \Afootnote{tataḥ A.}
} prajāpatir \edtext{adhijage}{
  \Afootnote{adhijace | N.}
} tasmād aśvināv aśvibhyām indra indrād \edtext{ahaṃ}{
  \Afootnote{aḥaṃ A K H.}
} mayā tv iha pradeyam arthibhyaḥ prajāhitahetoḥ | 
\pend


\pstart
bha\emph{\edlabel{SS.1.1.20x-0}} \edtext{|
\footnoteC{The Nepalese witnesses use the abbreviation "bha" for "bhavati cātra", introducing the next verse.}
}{
  \linenum{|\xlineref{SS.1.1.20x-0}}\lemma{bha |}\Afootnote{bhavati cātra A.}
}
\pend


\pstart
[27] ahaṃ hi dhanvantarir ādidevo jarārujāmṛtyuharo marāṇāṃ | \edtext{\edlabel{SS.1.1.21-8}}{
  \Afootnote{\textsc{(gap of 1 akṣara, damaged)}lyam K.}
} \edtext{śalyam}{
  \linenum{|\xlineref{SS.1.1.21-8}}\lemma{ śalyam}\Afootnote{śalyāṅgamaṅgair aparair A.}
\lemma{śalyam}  \Afootnote{mahacchastra° N.}
} \edtext{mahacchāstravaraṃ}{
  \Afootnote{upetaṃ A.}
} gṛhītvā \edtext{prāpto}{
  \Afootnote{mi N.}
} \edtext{'smi}{
  \Afootnote{gam H.}
} gāṃ bhūya ihopadeṣṭuṃ | 
\pend


\pstart
[28]\emph{\edlabel{SS.1.1.22-0}} \edtext{tatrāsmin}{
  \linenum{|\xlineref{SS.1.1.22-0}}\lemma{[28] tatrāsmin}\Afootnote{\textsc{[sū.1.22]} asmin A; tatrāsmiñ N H.}
} \edtext{śāstre}{
  \Afootnote{chāstre N.}
} pañcamahābhūtaśarīrisamavāyaḥ puruṣa ity ucyate | \edtext{tasmin}{
  \Afootnote{tasmiṃ N.}
} kriyā so 'dhiṣṭhānaṃ | kasmāt\edlabel{SS.1.1.22-13} \edtext{|}{
  \linenum{|\xlineref{SS.1.1.22-13}}\lemma{kasmāt |}\Afootnote{kasmāl N H.}
} \edtext{lokadvaividhyāl}{
  \Afootnote{laukikavi° N; lokasya dvaividhyāt A.}
} loko hi \edtext{dvividhaḥ\edlabel{SS.1.1.22-18}}{
  \Afootnote{dvividhan K H.}
} sthāvaro \edtext{jaṅgamaś}{
  \linenum{|\xlineref{SS.1.1.22-18}}\lemma{dvividhaḥ\ldots jaṅgamaś}\Afootnote{dvividhasthāvarajaṅ° N.}
} ca | dvividhātmaka evāgneyaḥ saumyaś ca tadbhūyastvāt \edtext{pañcātmako}{
  \Afootnote{pañcātako N.}
} vā | tatra caturvidho bhūtagrāmaḥ saṃsvedajādrijajarāyujāṇḍajasaṃjñaḥ\edlabel{SS.1.1.22-34} \edtext{|}{
  \linenum{|\xlineref{SS.1.1.22-34}}\lemma{saṃsvedajādrijajarāyujāṇḍajasaṃjñaḥ |}\Afootnote{saṃsvedajarāyujāṇḍajodbhijjasaṃjñaḥ A; saṃsvedajo hridajarāyujāṇḍajasaṃjñās H; °dridajajarāyujāṇḍajasaṃ\textsc{(gap of 1 akṣara, damaged)}s N; °saṃjñās K.}
} \edtext{tasmin}{
  \Afootnote{tatra A; tasmis K; tasmiṃ N.}
} puruṣaḥ \edtext{pradhānas}{
  \Afootnote{pradhānaṃ A.}
} tasyopakaraṇam anyat\edlabel{SS.1.1.22-40} \edtext{\textbf{|}}{
  \linenum{|\xlineref{SS.1.1.22-40}}\lemma{anyat \textbf{|}}\Afootnote{anyaṃ N.}
} tasmāt puruṣo 'dhiṣṭhānaṃ\edlabel{SS.1.1.22-44} \edtext{\textbf{|}}{
  \linenum{|\xlineref{SS.1.1.22-44}}\lemma{'dhiṣṭhānaṃ \textbf{|}}\Afootnote{dhiṣṭhāna K.}
}
\pend


\pstart
[29]\emph{\edlabel{SS.1.1.23-0}} \edtext{tadduḥkhasaṃyogā}{
  \linenum{|\xlineref{SS.1.1.23-0}}\lemma{[29] tadduḥkhasaṃyogā}\Afootnote{\textsc{[sū.1.23]} tad duḥkha° A.}
} vyādhaya \edtext{ity}{
  \Afootnote{\textsc{[om]} A.}
} ucyante \edtext{|}{
  \Afootnote{\textsc{[add]}  te caturvidhā āgantavaḥ śārīrā mānasā svābhāvi\textsc{(l. 6)}kāś ceti | teṣv āgantavo bhighātanimittā  śārīrās tv annamūlā vātapittakaphaśoṇitavaiṣamyanimittāḥ  mānasās tu krodhāśokadainyaharṣakāmaviṣāderṣyāsūyāmātsaryalobhādaya icchādveṣanimittāḥ svābhāvikās tu kṣutpipāsājarāmṛtyunidrāprakṛtayaḥ  ta ete manaḥśarīrādhi\textsc{(gap of 2 akṣaras, damaged)}\textsc{(l. 7)} bhavanti | teṣāṃ lekhanabṛmhaṇasaṃśodhanasaṃśamanāhārācārāḥ samyakprayuktā nigrahahetavo bhavanti K; \textsc{[add]} |  te caturvidhā āgantavaḥ śārīrā mānasā svābhāvikāś ceti | teṣv āgantavo bhighātanimittā | śārīrās tv annamūlā vātapittakapha\textsc{(l. 5)}śoṇitavaiṣamyanimittāḥ  mānasās tu krodhaśokadainyaharṣakāmaviṣāderṣyāsūyāmātsaryalobhādaya icchādveṣanimittāḥ|  svābhāvikās tu kṣutpipāsājarāmṛtyunidrāprabhṛtayaḥ  ete manaḥśarīrādhiṣṭhānā bhavanti || teṣāṃ lekhanabṛmhanasaṃśodhanasaṃśamanāhārācārāḥ samyakprayuktā nigrahahetavo bhavanti | N; \textsc{[add]}  te caturvvidhā āgantavaḥ śārīrā mānasā svābhāvikāś ceti | te\textsc{(l. 5)}ṣv āgantavo 'bhighātanimittāḥ  śārīrās tv annamūlā vātapittakaphaśoṇitavaiṣamyanimittāḥ  mānasās tu krodhaśokadainyaharṣakāmaviṣāderṣyāsūyāmātsaryalo\textsc{(l. 6)}bhādaya icchādveṣanimittāḥ  svābhāvikās tu kṣutpipāsājarāmṛtyunidrāprakṛtayaḥ  ta ete manaḥśarīrādhiṣṭhānā bhavanti | teṣāṃ lekhanabṛmhanasaṃśodhanasaṃśamanāhārācārāḥ\textsc{(f. 3v)}(From 3v) samyakprayuktā nigrahahetavo bhavanti H.}
}
\pend


\pstart
[30] te caturvidhā āgantavaḥ śārīrā \edtext{mānasā}{
  \Afootnote{mānasāḥ A.}
} svābhāvikāś ceti |\edlabel{SS.1.1.24-8} teṣv āgantavo 'bhighātanimittāḥ \edtext{|}{
  \linenum{|\xlineref{SS.1.1.24-8}}\lemma{|\ldots |}\Afootnote{\textsc{[om]} A.}
}
\pend


\pstart
\textsc{[sū.1.25.1]} teṣām āgantavo 'bhighātanimittāḥ || 
\pend


\pstart
[32] śārīrās tv \edtext{annamūlā}{
  \Afootnote{annapāna° A.}
} vātapittakaphaśoṇitavaiṣamyanimittāḥ\edlabel{SS.1.1.25.2-4} \edtext{|}{
  \linenum{|\xlineref{SS.1.1.25.2-4}}\lemma{vātapittakaphaśoṇitavaiṣamyanimittāḥ |}\Afootnote{°tasannipātavaiṣamyanimittāḥ || A.}
}
\pend


\pstart
 mānasās tu \edtext{krodhaśokadainyaharṣakāmaviṣāderṣyāsūyāmātsaryalobhādaya}{
  \Afootnote{°kabhayaharṣaviṣāderṣyābhyasūyādainyamātsaryakāmalobhaprabhṛtaya A.}
} icchādveṣanimittāḥ\edlabel{SS.1.1.25.3-4} \edtext{|}{
  \linenum{|\xlineref{SS.1.1.25.3-4}}\lemma{icchādveṣanimittāḥ |}\Afootnote{°ṣabhedair bhavanti || A.}
}
\pend


\pstart
[33] svābhāvikās tu kṣutpipāsājarāmṛtyunidrāprakṛtayaḥ | 
\pend


\pstart
[34] ta ete manaḥśarīrādhiṣṭhānā\edlabel{SS.1.1.26-3} bhavanti \edtext{|}{
  \linenum{|\xlineref{SS.1.1.26-3}}\lemma{manaḥśarīrādhiṣṭhānā\ldots |}\Afootnote{°ṣṭhānāḥ || A.}
}
\pend


\pstart
[35] teṣāṃ \edtext{lekhanabṛmhaṇasaṃśodhanasaṃśamanāhārācārāḥ}{
  \Afootnote{\textsc{[om]} lekha…mhaṇa° A.}
} samyakprayuktā nigrahahetavo\edlabel{SS.1.1.27-4} bhavanti \edtext{|}{
  \linenum{|\xlineref{SS.1.1.27-4}}\lemma{nigrahahetavo\ldots |}\Afootnote{\textsc{[om]} °bhavanti | A.}
}
\pend


\pstart
[30] prāṇināṃ \edtext{punar}{
  \Afootnote{puna N.}
} mūlam āhāro balavarṇaujasāñ ca | sa ṣaṭsu raseṣv āyattaḥ\edlabel{SS.1.1.28-11} | \edtext{rasāḥ}{
  \linenum{|\xlineref{SS.1.1.28-11}}\lemma{āyattaḥ\ldots rasāḥ}\Afootnote{āyattarasāḥ K N H.}
} punar dravyāśrayinaḥ\edlabel{SS.1.1.28-15} \edtext{|}{
  \linenum{|\xlineref{SS.1.1.28-15}}\lemma{dravyāśrayinaḥ |}\Afootnote{dravyāśrayāḥ A; davyāśrayinā N; °yiṇaḥ | H.}
} dravyāṇi punar oṣadhyaḥ\edlabel{SS.1.1.28-19} \edtext{|}{
  \linenum{|\xlineref{SS.1.1.28-19}}\lemma{oṣadhyaḥ |}\Afootnote{oṣadhayaḥ | A; auṣadhyaḥ | H.}
} \edtext{tā}{
  \Afootnote{tās tu A.}
} dvividhā\edlabel{SS.1.1.28-22} \edtext{|}{
  \linenum{|\xlineref{SS.1.1.28-22}}\lemma{dvividhā |}\Afootnote{dvividhāḥ A \uline{H}.}
} sthāvarā jaṅgamāś ca | 
\pend


\pstart
[31] tāsāṃ sthāvarāś caturvidhā\edlabel{SS.1.1.29-3} \edtext{vanaspatayo}{
  \linenum{|\xlineref{SS.1.1.29-3}}\lemma{caturvidhā vanaspatayo}\Afootnote{caturvi\uwave{dhā}\textsc{(f. 2v)}va° K.}
} vṛkṣā \edtext{oṣadhyo}{
  \Afootnote{\textsc{[om]} A; auṣadhyo N.}
} \edtext{vīrudha}{
  \Afootnote{\textsc{[add]} oṣadhaya A.}
} iti | \edtext{tāsv}{
  \Afootnote{tāsu A.}
} \edtext{apuṣpā}{
  \Afootnote{apuṣpāḥ A.}
} \edtext{phalavantyo}{
  \Afootnote{\textsc{[om]} pha° N; phalavanto A.}
} vanaspatayaḥ | puṣpaphalo\edlabel{SS.1.1.29-15} \edtext{ye}{
  \Afootnote{yye N.}
} \edtext{tā}{
  \linenum{|\xlineref{SS.1.1.29-15}}\lemma{puṣpaphalo\ldots tā}\Afootnote{puṣpaphalavanto A.}
} vṛkṣāḥ | phalapākaniṣṭhās\edlabel{SS.1.1.29-20} \edtext{tv}{
  \linenum{|\xlineref{SS.1.1.29-20}}\lemma{phalapākaniṣṭhās tv}\Afootnote{pratānavatyaḥ stambinyaś ca vīrudhaḥ phalapākaniṣṭhā A.}
} oṣadhyaḥ\edlabel{SS.1.1.29-22} \edtext{|}{
  \linenum{|\xlineref{SS.1.1.29-22}}\lemma{oṣadhyaḥ |}\Afootnote{auṣadhyaḥ | N.}
} \edtext{pratānavatyo}{
  \Afootnote{pratānavantyo N; °va\textsc{(gap of 1 akṣara, damaged)} K.}
} \edtext{vīrudha}{
  \linenum{|\xlineref{SS.1.1.29-22}}\lemma{oṣadhyaḥ\ldots vīrudha}\Afootnote{oṣadhaya A.}
\lemma{vīrudha}  \Afootnote{virudha N.}
} iti | 
\pend


\pstart
[32]\emph{\edlabel{SS.1.1.30-0}} \edtext{jaṅgamāḥ}{
  \linenum{|\xlineref{SS.1.1.30-0}}\lemma{[32] jaṅgamāḥ}\Afootnote{jaṅgamā N.}
} khalv api caturvidhāḥ\edlabel{SS.1.1.30-4} \edtext{|}{
  \linenum{|\xlineref{SS.1.1.30-4}}\lemma{caturvidhāḥ |}\Afootnote{catuvidhāḥ A \uline{H}; catuvidhā | K.}
} jarāyujāṇḍajasaṃsvedajodbhidā\edlabel{SS.1.1.30-6} iti \edtext{|}{
  \linenum{|\xlineref{SS.1.1.30-6}}\lemma{jarāyujāṇḍajasaṃsvedajodbhidā\ldots |}\Afootnote{jarāyujā\textsc{(gap of 1 akṣara, damaged)}jasa\textsc{(gap of 1 akṣara, damaged)}svedajodbhidāḥ | iti | K; °jasasvedajodbhidāḥ iti | N.}
} \edtext{teṣām}{
  \linenum{|\xlineref{SS.1.1.30-6}}\lemma{jarāyujāṇḍajasaṃsvedajodbhidā\ldots teṣām}\Afootnote{°jasvedajodbhijjāḥ | tatra A.}
} paśumanuṣyavyālādayo jarāyujāḥ | \edtext{khagasarīsṛpasarpās\edlabel{SS.1.1.30-13}}{
  \Afootnote{\textsc{(gap of 5 akṣaras, damaged)}\textsc{(l. 2)}sṛpa\textbf{sarpā}s K.}
} \edtext{tv}{
  \linenum{|\xlineref{SS.1.1.30-13}}\lemma{khagasarīsṛpasarpās tv}\Afootnote{khagasarpasarīsṛpaprabhṛtayo A.}
} aṇḍajāḥ\edlabel{SS.1.1.30-15} \edtext{|}{
  \linenum{|\xlineref{SS.1.1.30-15}}\lemma{aṇḍajāḥ |}\Afootnote{'ṇḍajāḥ A.}
} \edtext{kṛmikuntapipīlikāprabhṛtayaḥ}{
  \Afootnote{krimikuṣṭhapi° H; kṛmikunthapi° K; kṛmikuṭṭhapi° N; kṛmikīṭapi° A.}
} saṃsvedajāḥ\edlabel{SS.1.1.30-18} \edtext{|}{
  \linenum{|\xlineref{SS.1.1.30-18}}\lemma{saṃsvedajāḥ |}\Afootnote{svedajāḥ A.}
} \edtext{indragopakamaṇḍūkaprabhṛtayaś}{
  \Afootnote{°pamaṇḍūkaprabhṛtaya A.}
} codbhidāḥ\edlabel{SS.1.1.30-21} \edtext{|}{
  \linenum{|\xlineref{SS.1.1.30-21}}\lemma{codbhidāḥ |}\Afootnote{udbhijjāḥ || A.}
}
\pend


\pstart
[33] tatra sthāvarebhyas \edtext{tvakpatrapuṣpaphalamūlakandakṣīraniryāsasārasnehasvarasāḥ}{
  \Afootnote{°daniryāsasvarasādayaḥ A; °rasā N.}
} prayojanavantaḥ | jaṅgamebhyaś carmaromanakharudhirādayaḥ\edlabel{SS.1.1.31-7} \edtext{|}{
  \linenum{|\xlineref{SS.1.1.31-7}}\lemma{carmaromanakharudhirādayaḥ |}\Afootnote{carmanakharomaru° A.}
}
\pend


\pstart
[34]\emph{\edlabel{SS.1.1.32-0}} \edtext{pārthivas}{
  \linenum{|\xlineref{SS.1.1.32-0}}\lemma{[34] pārthivas}\Afootnote{\textsc{[sū.1.32]} pārthivāḥ A.}
} \edtext{tu}{
  \Afootnote{\textsc{[om]} A.}
} suvarṇarajatādayaḥ\edlabel{SS.1.1.32-3} \edtext{|}{
  \linenum{|\xlineref{SS.1.1.32-3}}\lemma{suvarṇarajatādayaḥ |}\Afootnote{°jatamaṇimuktāmanaḥśilāmṛtkapālādayaḥ || A.}
}
\pend


\pstart
[35] kālakṛtās \edtext{tu}{
  \Afootnote{\textsc{[om]} A.}
} pravātanivātātapacchāyātamojyotsnāśītoṣṇavarṣāsu\edlabel{SS.1.1.33-3} samplavāḥ\edlabel{SS.1.1.33-4} \edtext{|}{
  \linenum{|\xlineref{SS.1.1.33-4}}\lemma{samplavāḥ |}\Afootnote{sasam° N.}
} kālaviśeṣās tu nimeṣakāṣṭhākalāmuhūrtāhorātrapakṣamāsartvayanasamvatsarayugaviśeṣāḥ \edtext{|}{
  \linenum{|\xlineref{SS.1.1.33-3}}\lemma{pravātanivātātapacchāyātamojyotsnāśītoṣṇavarṣāsu\ldots |}\Afootnote{°yājyotsnātamaḥśītoṣṇavarṣāhorātrapakṣamāsartvayanādayaḥ saṃvatsaraviśeṣāḥ || A.}
}
\pend


\pstart
\edtext{[36]}{
  \Afootnote{\textsc{[pre]} \textsc{[pre]} ta ete A.}
} svabhāvata eva doṣāṇāṃ \edtext{sañcayaprakopopaśamapratīkārahetavo}{
  \Afootnote{sañcayaprakopa° N; sañcayaprakopapraśa° A.}
} bhavanti\edlabel{SS.1.1.34-5} \edtext{|}{
  \linenum{|\xlineref{SS.1.1.34-5}}\lemma{bhavanti |}\Afootnote{\textsc{[om]} A.}
} prayojanavantaś ca | 
\pend


\pstart
bha\emph{\edlabel{SS.1.1.35-0}} \edtext{|}{
  \linenum{|\xlineref{SS.1.1.35-0}}\lemma{bha |}\Afootnote{\textsc{[sū.1.35]} bhavanti cātra ślokāḥ A.}
}
\pend


\pstart
[37] śārīrāṇāṃ vikārāṇām\edlabel{SS.1.1.35x-2} \edtext{eṣa}{
  \linenum{|\xlineref{SS.1.1.35x-2}}\lemma{vikārāṇām eṣa}\Afootnote{vikārāṇāmeṣa A.}
} \edtext{vargaś}{
  \Afootnote{vargeś N.}
} caturvidhaḥ\edlabel{SS.1.1.35x-5} \edtext{|}{
  \linenum{|\xlineref{SS.1.1.35x-5}}\lemma{caturvidhaḥ |}\Afootnote{catuvidhaḥ |  A K.}
}  prakope praśame caiva \edtext{hetur}{
  \Afootnote{ukta N.}
} uktaś cikitsakaiḥ || 
\pend


\pstart
[38] āgantavas tu ye rogās te dvidhā nipatanti ha\edlabel{SS.1.1.36-8} \edtext{|}{
  \linenum{|\xlineref{SS.1.1.36-8}}\lemma{ha |}\Afootnote{hi |  A.}
}  manasy anye \edtext{śarīre}{
  \Afootnote{'nye A.}
} \edtext{ca}{
  \Afootnote{teṣāṃ A.}
} teṣān tu dvividhā kriyā || 
\pend


\pstart
[39]\emph{\edlabel{SS.1.1.37-0}} \edtext{śarīrapatitānāṃ}{
  \linenum{|\xlineref{SS.1.1.37-0}}\lemma{[39] śarīrapatitānāṃ}\Afootnote{°tānān N H.}
} tu śārīravadupakramaḥ | \edtext{}{
  \Afootnote{mānasānāṃ A H.}
} mānasānān tu śabdādir iṣṭo vargaḥ sukhāvahaḥ || 
\pend


\pstart
[40] evam etat puruṣo vyādhir auṣadhaṃ kriyākāla iti \edtext{samāsena}{
  \Afootnote{\textsc{[om]} A.}
} \edtext{catuṣṭayaṃ}{
  \Afootnote{\textsc{[add]} samāsena A.}
} vyākhyātam bhavati\edlabel{SS.1.1.38-11} \edtext{|}{
  \linenum{|\xlineref{SS.1.1.38-11}}\lemma{bhavati |}\Afootnote{\textsc{[om]} A.}
} tatra puruṣagrahaṇāt tatsaṃbhavadravyasamūho bhūtādir\edlabel{SS.1.1.38-16} uktaḥ \edtext{tadaṅgapratyaṅgavikalpāś}{
  \linenum{|\xlineref{SS.1.1.38-16}}\lemma{bhūtādir\ldots tadaṅgapratyaṅgavikalpāś}\Afootnote{bhūtādiruktastadaṅ° A.}
\lemma{tadaṅgapratyaṅgavikalpāś}  \Afootnote{tadagadgapra° N; °tyaṅgā vikalpāś H.}
} ca tvaṅmāṃsāsirāsnāyvasthisandhiprabhṛtayaḥ\edlabel{SS.1.1.38-20} \edtext{|}{
  \linenum{|\xlineref{SS.1.1.38-20}}\lemma{tvaṅmāṃsāsirāsnāyvasthisandhiprabhṛtayaḥ |}\Afootnote{tvaṅmāṃsāsthisirāsnāyupra° A.}
} \edtext{vyādhigrahaṇād\edlabel{SS.1.1.38-22}}{
  \Afootnote{vyādhigrāha° N.}
} \edtext{vātapittakaphaśoṇitasannipātāgantusvabhāvanimittāḥ}{
  \linenum{|\xlineref{SS.1.1.38-22}}\lemma{vyādhigrahaṇād vātapittakaphaśoṇitasannipātāgantusvabhāvanimittāḥ}\Afootnote{°ṇādvātapittakaphaśoṇitasannipātavaiṣamyanimittāḥ A.}
} sarva eva vyādhayo vyākhyātā bhavanti\edlabel{SS.1.1.38-28} \edtext{|}{
  \linenum{|\xlineref{SS.1.1.38-28}}\lemma{bhavanti |}\Afootnote{\textsc{[om]} A; bhavasvanti | N.}
} oṣadhagrahaṇād dravyarasaguṇavīryavipākānām\edlabel{SS.1.1.38-31} ādeśaḥ \edtext{|}{
  \linenum{|\xlineref{SS.1.1.38-31}}\lemma{dravyarasaguṇavīryavipākānām\ldots |}\Afootnote{°pāko nāmādeśaḥ | H.}
} kriyāgrahaṇāt\edlabel{SS.1.1.38-34} \edtext{snehādīni}{
  \linenum{|\xlineref{SS.1.1.38-34}}\lemma{kriyāgrahaṇāt snehādīni}\Afootnote{°ṇāc chedyādīni A.}
} \edtext{cchedyādīni}{
  \Afootnote{snehādīni A.}
} ca karmāṇy upadiṣṭāni\edlabel{SS.1.1.38-39} bhavanti \edtext{|}{
  \linenum{|\xlineref{SS.1.1.38-39}}\lemma{upadiṣṭāni\ldots |}\Afootnote{vyākhyātāni A.}
} kālagrahaṇāt sarva\edlabel{SS.1.1.38-43} eva kriyākālādeśaḥ\edlabel{SS.1.1.38-45} \edtext{|}{
  \linenum{|\xlineref{SS.1.1.38-43}}\lemma{sarva\ldots |}\Afootnote{sarvakriyākālānām ādeśaḥ || A.}
  \linenum{|\xlineref{SS.1.1.38-45}}\lemma{kriyākālādeśaḥ |}\Afootnote{kriyākālanirdeśaḥ || N; kriyākā\textsc{(l. 6)}lakāladeśaḥ || H.}
}
\pend


\pstart
 bha\edlabel{SS.1.1.39x-1} \edtext{|}{
  \linenum{|\xlineref{SS.1.1.39x-1}}\lemma{bha |}\Afootnote{\textsc{[sū.1.39]} bhavati cātra | A.}
}
\pend


\pstart
[41] bījañ cikitsitasyaitat samāsena prakīrtitam |  saviṃśam adhyāyaśatam asya vyākhyā bhaviṣyati || 
\pend


\pstart
[42] tac ca \edtext{viṃśam}{
  \Afootnote{saviṃśam A.}
} adhyāyaśataṃ pañcasu sthāneṣu ceti\edlabel{SS.1.1.40-7} | tatra \edtext{ślokasthānanidānaśārīracikitsitakalpeṣv\edlabel{SS.1.1.40-10}}{
  \linenum{|\xlineref{SS.1.1.40-7}}\lemma{ceti\ldots ślokasthānanidānaśārīracikitsitakalpeṣv}\Afootnote{sūtrani° A.}
} \edtext{arthavaśād}{
  \Afootnote{arthavasād H.}
} \edtext{vibhajya\edlabel{SS.1.1.40-12}}{
  \linenum{|\xlineref{SS.1.1.40-10}}\lemma{ślokasthānanidānaśārīracikitsitakalpeṣv\ldots vibhajya}\Afootnote{°śārī\textsc{(gap of 11 akṣaras, damaged)}ibhajya K.}
\lemma{vibhajya}  \Afootnote{saṃvi° A.}
} \edtext{uttare}{
  \linenum{|\xlineref{SS.1.1.40-12}}\lemma{vibhajya uttare}\Afootnote{vibhajyottare N.}
} vakṣyāmaḥ\edlabel{SS.1.1.40-14} \edtext{|}{
  \linenum{|\xlineref{SS.1.1.40-14}}\lemma{vakṣyāmaḥ |}\Afootnote{tantre śeṣān arthān vyākhyāsyāmaḥ || A.}
}
\pend


\pstart
 bha\edlabel{SS.1.1.41x-1} \edtext{|}{
  \linenum{|\xlineref{SS.1.1.41x-1}}\lemma{bha |}\Afootnote{\textsc{[sū.1.41]} bhavati cātra | A; \textsc{[om]} K.}
}
\pend


\pstart
[43] svayambhuvā proktam idaṃ sanātanam  \edtext{paṭhet}{
  \Afootnote{dhi A.}
} tu yaḥ kāśipatiprakāśitam |  sa puṇyakarmā bhuvi pūjito nṛpair \edlabel{SS.1.1.41-17} asukṣaye \edtext{śakrasalokatām}{
  \linenum{|\xlineref{SS.1.1.41-17}}\lemma{\ldots śakrasalokatām}\Afootnote{vrajet || A.}
} iyād iti || 
\pend

\newpage
\section*{Chapter 2}
\pstart
\textsc{[1]} athātaḥ śiṣyopanayanīyam adhyāyaṃ vyākhyāsyāmaḥ\edlabel{SS.1.2.1-4} \edtext{||}{
  \linenum{|\xlineref{SS.1.2.1-4}}\lemma{vyākhyāsyāmaḥ ||}\Afootnote{vyā || K.}
}
\pend


\pstart
\edtext{}{
  \Afootnote{\textsc{[pre]} \textsc{[pre]} yathovāca bhagavān dhanvantariḥ || A.}
}
\pend


\pstart
\textsc{[2]} \edtext{brāhmaṇakṣatriyavaiśyānām}{
  \Afootnote{vrāhma° K.}
} anyatamam\edlabel{SS.1.2.3-2} anvayaḥ \edtext{|}{
  \linenum{|\xlineref{SS.1.2.3-2}}\lemma{anyatamam\ldots |}\Afootnote{anyatamamanvaya A.}
} \edtext{vayaḥ\edlabel{SS.1.2.3-5}}{
  \Afootnote{vayaḥśīla A.}
} \edtext{śaurya\edlabel{SS.1.2.3-6}}{
  \linenum{|\xlineref{SS.1.2.3-5}}\lemma{vayaḥśaurya}\Afootnote{vaya\uwave{}saurya H.}
} \edtext{śaucācāra}{
  \linenum{|\xlineref{SS.1.2.3-6}}\lemma{śauryaśaucācāra}\Afootnote{ra K.}
} vinayaśakti \edtext{bala}{
  \Afootnote{vala K.}
} medhādhṛti \edtext{smṛti}{
  \Afootnote{\textsc{[add]} mati A.}
} \edtext{pratipattiyuktan}{
  \Afootnote{°yuktaṃ A K H.}
} \edtext{tanujihvauṣṭhadantāgram\edlabel{SS.1.2.3-15}}{
  \Afootnote{tanujihvauṣṭhadan° N H.}
} \edtext{ṛjuvaktrākṣināsaṃprasannacittavākceṣṭaṃ}{
  \linenum{|\xlineref{SS.1.2.3-15}}\lemma{tanujihvauṣṭhadantāgram ṛjuvaktrākṣināsaṃprasannacittavākceṣṭaṃ}\Afootnote{tanujihvauṣṭhadantākramṛju° A.}
\lemma{ṛjuvaktrākṣināsaṃprasannacittavākceṣṭaṃ}  \Afootnote{ṛjuvaktrākṣināsaṃprasannaci° H.}
} kleśasahañ ca \edtext{śiṣyam}{
  \Afootnote{bhiṣak śiṣyam A.}
} upanayet | sa\edlabel{SS.1.2.3-22} hi guṇavān tasmai \edtext{deyam}{
  \linenum{|\xlineref{SS.1.2.3-22}}\lemma{sa\ldots deyam}\Afootnote{\textsc{[om]} A.}
} ato|| \edtext{viparītaguṇan\edlabel{SS.1.2.3-28}}{
  \Afootnote{°guṇaṃ A K.}
} nopanayet\edlabel{SS.1.2.3-29} \edtext{|}{
  \linenum{|\xlineref{SS.1.2.3-28}}\lemma{viparītaguṇan\ldots |}\Afootnote{°guṇa\uwave{m}nopanayet || H.}
  \linenum{|\xlineref{SS.1.2.3-29}}\lemma{nopanayet |}\Afootnote{noapanayet || A.}
}
\pend


\pstart
\textsc{[3]} śūdram\edlabel{SS.1.2.4-1} api guṇavantam anupanītam adhyāpayed ity eke \edtext{|}{
  \linenum{|\xlineref{SS.1.2.4-1}}\lemma{śūdram\ldots |}\Afootnote{\textsc{[om]} A.}
} \edtext{upanayanīyan}{
  \Afootnote{°nīyaṃ A H.}
} tu brāhmaṇam praśasteṣu \edtext{tithikaraṇamuhūrttanakṣatreṣu}{
  \Afootnote{°ṇamūhūrttanakṣatreṣū N; °treṣū H.}
} \edtext{praśastāyān}{
  \Afootnote{praśastāyāṃ A.}
} diśi \edtext{śucau}{
  \Afootnote{\textsc{[add]} same A.}
} deśe \edtext{gocarmamātraṃ}{
  \Afootnote{caturhastaṃ caturasraṃ A; go ca\uwave{rma} mātraṃ N; go ca \uwave{sma} mātraṃ H.}
} sthaṇḍilam \edtext{upalipya}{
  \Afootnote{\textsc{[add]} gomayena A.}
} \edtext{darbhasaṃstaraṇam\edlabel{SS.1.2.4-21}}{
  \Afootnote{darbhaiḥ saṃstīrya puṣpair A.}
} \edtext{ahitaṃ}{
  \linenum{|\xlineref{SS.1.2.4-21}}\lemma{darbhasaṃstaraṇam ahitaṃ}\Afootnote{°ra\uwave{}hitaṃ H.}
\lemma{ahitaṃ}  \Afootnote{lājabhaktai ratnaiś ca A.}
} \edtext{kṛtvā}{
  \Afootnote{devatāḥ A.}
} \edtext{puṣpair}{
  \Afootnote{pūjayitvā A; puṣpai N.}
} \edtext{dhūpair\edlabel{SS.1.2.4-25}}{
  \Afootnote{viprān bhiṣajaś A; gandhair K.}
} \edtext{janvair\edlabel{SS.1.2.4-26}}{
  \Afootnote{dhūpair K.}
} \edtext{bhankais}{
  \linenum{|\xlineref{SS.1.2.4-25}}\lemma{dhūpair\ldots bhankais}\Afootnote{\uwave{}ś H.}
  \linenum{|\xlineref{SS.1.2.4-26}}\lemma{janvair bhankais}\Afootnote{ca tatrollikhyābhyukṣya ca dakṣiṇato A.}
\lemma{bhankais}  \Afootnote{bhakṣyaiś K.}
} \edtext{ca}{
  \Afootnote{brahmāṇaṃ A.}
} \edtext{pūjayitvā}{
  \Afootnote{sthāpayitvā'gnim A.}
} \edtext{palāśodumvarabilvānāṃ}{
  \Afootnote{upasamādhāya khadirapalāśadevadārubi° A.}
} \edtext{samidbhir}{
  \Afootnote{samidbhi N.}
} \edtext{ghṛtam}{
  \Afootnote{caturṇāṃ vā kṣīravṛkṣāṇāṃ (? nyagrodhodumbarāśvatthamadhūkānāṃ ) A.}
} \edtext{aktābhir}{
  \Afootnote{dadhimadhughṛtāktābhir A; a\textsc{(l. 3)}thaktāni K.}
} \edtext{dārvīhomikenāgnim\edlabel{SS.1.2.4-34}}{
  \Afootnote{dārviho° K.}
} \edtext{upasamādhāyā\edlabel{SS.1.2.4-35}}{
  \linenum{|\xlineref{SS.1.2.4-34}}\lemma{dārvīhomikenāgnim upasamādhāyā}\Afootnote{dārvīhaumikena vidhinā A.}
} \edtext{ṇḍya}{
  \linenum{|\xlineref{SS.1.2.4-35}}\lemma{upasamādhāyā ṇḍya}\Afootnote{°dhāyā\uwave{ṇḍya}ñ K.}
\lemma{ṇḍya}  \Afootnote{sruveṇājyāhutīr A; \textsc{[om]} H.}
} huyāt\edlabel{SS.1.2.4-37} \edtext{|}{
  \linenum{|\xlineref{SS.1.2.4-37}}\lemma{huyāt |}\Afootnote{juhuyāt A K \uline{H}.}
} \edtext{pratidevatam\edlabel{SS.1.2.4-39}}{
  \Afootnote{pratidaivatam H.}
} \edtext{ṛṣibhyaḥ}{
  \linenum{|\xlineref{SS.1.2.4-39}}\lemma{pratidevatam ṛṣibhyaḥ}\Afootnote{sapraṇavābhir mahāvyahṛtibhiḥ tataḥ A.}
\lemma{ṛṣibhyaḥ}  \Afootnote{ṛṣībhyaḥ H.}
} \edtext{śiṣyaṃ}{
  \Afootnote{pratidaivatamṛṣīṃś ca A.}
} \edtext{svāhākāraṃṅ}{
  \Afootnote{\textsc{[add]} kuryāt śiṣyam api A.}
} kārayet | 
\pend


\pstart
\textsc{[4]} \edtext{brāhmaṇas}{
  \Afootnote{brāhmaṇaṃ N.}
} \edtext{trayāṇāṃ}{
  \Afootnote{\textsc{[add]} varṇānām upanayanaṃ kartum arhati A.}
} rājanyo \edtext{dvayasyā}{
  \Afootnote{dvayasya A H.}
} vaiśyo vaisyasyaiva\edlabel{SS.1.2.5-6} \edtext{|}{
  \linenum{|\xlineref{SS.1.2.5-6}}\lemma{vaisyasyaiva |}\Afootnote{vaiśyasyaiveti śūdram api kulaguṇasaṃpannaṃ mantravarjam anupanītam adhyāpayed ity eke || A; vai\uwave{śya}syaiva H.}
}
\pend


\pstart
\textsc{[5]} tato\edlabel{SS.1.2.6-1} \edtext{'gniṃ}{
  \linenum{|\xlineref{SS.1.2.6-1}}\lemma{tato 'gniṃ}\Afootnote{tato'gniṃ H.}
\lemma{'gniṃ}  \Afootnote{\textsc{[add]} triḥ A.}
} pariṇīyāgnisākṣikaṃ śiṣyaṃ brūyāt| kāmakrodhalobhamohamānāhaṃkārerṣyāmātsarya\edlabel{SS.1.2.6-6} \edtext{pārūṣya}{
  \linenum{|\xlineref{SS.1.2.6-6}}\lemma{kāmakrodhalobhamohamānāhaṃkārerṣyāmātsaryapārūṣya}\Afootnote{°\uwave{rya}pāruṣya H.}
\lemma{pārūṣya}  \Afootnote{\uwave{rūṣa} N.}
} \edtext{paiśunyānṛtālasyāsyayaśasyāni}{
  \linenum{|\xlineref{SS.1.2.6-6}}\lemma{kāmakrodhalobhamohamānāhaṃkārerṣyāmātsarya\ldots paiśunyānṛtālasyāsyayaśasyāni}\Afootnote{°rṣyāpāruṣyapaiśunyānṛtālasyāyaśasyāni A.}
\lemma{paiśunyānṛtālasyāsyayaśasyāni}  \Afootnote{°syā\uwave{}\textsc{(l. 2)}\uwave{}yaśasyāni H.}
} hitvā \edtext{kāṣāyavāsasā}{
  \Afootnote{kaṣā° H; \textsc{[om]} A.}
} \edtext{nīcanakharomṇā}{
  \Afootnote{°ro\uwave{} H.}
} \edtext{trirātraṃ}{
  \Afootnote{\textsc{[om]} A; trivāraṃ H.}
} \edtext{śucinā}{
  \Afootnote{\textsc{[add]} kaṣāyavāsasā A.}
} \edtext{satyabrahmacaryābhivādanapareṇa}{
  \Afootnote{satyavratabrahmacaryābhivādanatatpareṇā'vaśyaṃ A.}
} bhavitavyaṃ | \edtext{mamānumatasthānagamanaśayanāsanabhojanādhyayanapareṇa}{
  \Afootnote{madanu° A; mamānumatesthānagamanaśayanāśana° H.}
} bhūtvā matpriyahiteṣu varttitavyam \edtext{ato'nyathā}{
  \Afootnote{ato 'nyathā te A.}
} varttamānasyādharmmo bhavaty aphalā ca vidyā na ca \edtext{prākāśyaṃ}{
  \Afootnote{prākāsyaṃ H.}
} prāpnuyāt\edlabel{SS.1.2.6-30} \edtext{|}{
  \linenum{|\xlineref{SS.1.2.6-30}}\lemma{prāpnuyāt |}\Afootnote{prāpnoti || A.}
}
\pend


\pstart
\textsc{[6]} aham vā tvayi samyag\edlabel{SS.1.2.7-4} \edtext{varttamāne}{
  \linenum{|\xlineref{SS.1.2.7-4}}\lemma{samyag varttamāne}\Afootnote{samyak\textsc{(l. 4)}vartta° H; samyagvarta° A \uline{N}.}
} yady\edlabel{SS.1.2.7-6} \edtext{ananyathādarśā}{
  \linenum{|\xlineref{SS.1.2.7-6}}\lemma{yady ananyathādarśā}\Afootnote{yadyanyathādarśī A.}
} syāt\edlabel{SS.1.2.7-8} \edtext{tadeva}{
  \Afootnote{tadaiva H.}
} \edtext{nāsau}{
  \Afootnote{na sau H.}
} bhāgyavidyāphalabhāk ca bhaveyaṃ \edtext{|}{
  \linenum{|\xlineref{SS.1.2.7-8}}\lemma{syāt\ldots |}\Afootnote{syāmenobhāgbhaveyamaphalavidyaś ca || A.}
}
\pend


\pstart
\textsc{[7]} yasmād \edtext{arogavatā}{
  \Afootnote{roga\uwave{va}tā H.}
} \edtext{dharmmāsau'rthakāmamokṣāḥ}{
  \Afootnote{dharmmārtha° H; dharmmā\uwave{sau} rtha° N.}
} prāpyante\edlabel{SS.1.2.7.a-4} \edtext{|}{
  \linenum{|\xlineref{SS.1.2.7.a-4}}\lemma{prāpyante |}\Afootnote{prāthyante | H.}
}
\pend


\pstart
\textsc{[8]} \edtext{tasmād}{
  \Afootnote{\textsc{[om]} A; tasmā H.}
} \edtext{dvijadaridrasādhvanāthābhyupagatapāśaṇḍasthitānām}{
  \Afootnote{\textsc{[sū.2.8]} dvijagurudaridramitrapravrajitopanatasādhvanāthābhyupagatānāṃ A; °\uwave{pāṣaṇu} sthitānām H.}
} ātmabāndhavānām\edlabel{SS.1.2.8-3} ivātmabheṣajaiḥ\edlabel{SS.1.2.8-4} \edtext{|}{
  \linenum{|\xlineref{SS.1.2.8-3}}\lemma{ātmabāndhavānām\ldots |}\Afootnote{cātmabāndhavānāmiva svabhaiṣajaiḥ A.}
  \linenum{|\xlineref{SS.1.2.8-4}}\lemma{ivātmabheṣajaiḥ |}\Afootnote{iva ātma° H.}
} pratikartavyam\edlabel{SS.1.2.8-6} \edtext{evaṃ}{
  \linenum{|\xlineref{SS.1.2.8-6}}\lemma{pratikartavyam evaṃ}\Afootnote{°rtavyamevaṃ A.}
} \edtext{sādhu}{
  \Afootnote{sādhur H.}
} bhavati | \edtext{vyādhaśākunikapatitapāpakarttṝṇāñ}{
  \Afootnote{vyādhasākunikapatitayāpa° H; °pakāriṇāṃ A.}
} ca na pratikarttavyam evaṃ \edtext{vidyā}{
  \Afootnote{vidyā\uwave{ṃ} H.}
} prakāśate | \edtext{mitradharmakāmayaśāṃ\edlabel{SS.1.2.8-19}}{
  \Afootnote{mitrayaśodharmārthakāmāṃś A.}
} \edtext{micā}{
  \linenum{|\xlineref{SS.1.2.8-19}}\lemma{mitradharmakāmayaśāṃ micā}\Afootnote{°yaśansi\textbf{cā} H.}
\lemma{micā}  \Afootnote{ca A.}
} vāpnoti\edlabel{SS.1.2.8-21} \edtext{||}{
  \linenum{|\xlineref{SS.1.2.8-21}}\lemma{vāpnoti ||}\Afootnote{prāpnoti || A; vā prāpnoti || H.}
}
\pend


\pstart
\textsc{[9]} bhavataś\edlabel{SS.1.2.9X-1} cātra \edtext{||}{
  \linenum{|\xlineref{SS.1.2.9X-1}}\lemma{bhavataś\ldots ||}\Afootnote{bha || H.}
}
\pend


\pstart
\textsc{[10]} \edtext{kṛṣṇāṣṭamī}{
  \Afootnote{\textsc{[sū.2.9]} kṛṣṇe 'ṣṭamī A; kṛ\uwave{ṣtta}ṣūmīn H.}
} \edtext{tannidhane'hanī}{
  \Afootnote{ttanni° N; °ne 'hanī A.}
} dve \edtext{}{
  \Afootnote{śukle \uwave{da}° H; śukle tathā'py A.}
} śuklādaye'py evam ahar\edlabel{SS.1.2.9-7} \edtext{dvisandhyaṃ}{
  \linenum{|\xlineref{SS.1.2.9-7}}\lemma{ahar dvisandhyaṃ}\Afootnote{dvisa\textbf{\uwave{yā}}ndhyaṃ |  H.}
} |  akālavidyutstanayitnughoṣe  svatantrarāṣṭrakṣitipavyathāsu | 
\pend


\pstart
\textsc{[11]} \edtext{śmaśānayānādhvatanāhaveṣu}{
  \Afootnote{śmaśānayānādyata° \uline{A} H.}
} \edlabel{SS.1.2.10-2} \edtext{tathautsavaotpātikadarśaneṣu}{
  \linenum{|\xlineref{SS.1.2.10-2}}\lemma{ tathautsavaotpātikadarśaneṣu}\Afootnote{mahotsavautpā° A; tathotsavotyāti° H.}
} \edtext{|\edlabel{SS.1.2.10-4}}{
  \Afootnote{nādhyoyam H.}
} \edtext{}{
  \linenum{|\xlineref{SS.1.2.10-4}}\lemma{| }\Afootnote{nādhyeyamanyeṣu A.}
} nādhyeyam anyeṣu \edtext{ca}{
  \Afootnote{vaprā  A; viprā  H.}
} \edtext{yeṣu}{
  \Afootnote{nādhī° A \uline{H}.}
} viprāṇ  ṇādhīyate nāśucinā ca nityam || 
\pend


\pstart
\edtext{iti||}{
  \Afootnote{\textsc{[add]} suśrutasaṃhitāyāṃ sūtrasthāne śiṣyopanayanīyo nāma dvitīyo 'dhyāyaḥ || A.}
}
\pend

\newpage

\section*{Chapter 16}

\pstart
athātaḥ \edtext{karṇavyadhavidhim}{
  \Afootnote{°dhabandhavidhim adhyāyaṃ A.}
} vyākhyāsyāmaḥ\edlabel{SS.1.16.1-2} \edtext{||1||}{
  \linenum{|\xlineref{SS.1.16.1-2}}\lemma{vyākhyāsyāmaḥ ||1||}\Afootnote{\textsc{[om]} K.}
}
\pend


\pstart
\edtext{}{
  \Afootnote{\textsc{[pre]} \textsc{[pre]} yathovāca bhagavān dhanavatariḥ || A.}
}
\pend


\pstart
 rakṣābhūṣaṇanimittam bālasya karṇau vyadhayet\edlabel{SS.1.16.3-4} \edtext{|}{
  \linenum{|\xlineref{SS.1.16.3-4}}\lemma{vyadhayet |}\Afootnote{vidhyete | A.}
} tau ṣaṣṭhe \edtext{māse}{
  \Afootnote{māsi A.}
} \edtext{saptame}{
  \Afootnote{\textsc{[om]} N.}
} vā śuklapakṣe praśasteṣu tithikaraṇamuhūrtanakṣatreṣu kṛtamaṅgalaṃ\edlabel{SS.1.16.3-14} \edtext{svastivācanan}{
  \linenum{|\xlineref{SS.1.16.3-14}}\lemma{kṛtamaṅgalaṃ svastivācanan}\Afootnote{°galasvastivācanaṃ A.}
} \edtext{dhātryaṅke}{
  \Afootnote{dhātryaṅkā K; \textsc{[add]} kumāradharāṅke vā A.}
} \edtext{kumāram}{
  \Afootnote{kumārakam N.}
} \edtext{upaveśyābhisāntvayamānaḥ}{
  \Afootnote{upaveśya bālakrīḍanakaiḥ pralobhyābhisāntvayan A.}
} bhiṣag vāmahastenākṛṣya \edtext{karṇan}{
  \Afootnote{karṇaṃ A.}
} daivakṛte \edtext{chidre}{
  \Afootnote{chidra ādityakarāvabhāsite śanaiḥ śanair A.}
} dakṣiṇahastena \edtext{ṛju}{
  \Afootnote{rju A; rma rju K; ṛjum N H.}
} vidhyet | \edtext{pūrvan}{
  \Afootnote{pratanukaṃ sūcyā bahalam ārayā pūrvaṃ A.}
} dakṣiṇaṃ \edtext{kumārasya}{
  \Afootnote{kumārarasya N.}
} vāmaṅ kanyāyāḥ\edlabel{SS.1.16.3-32} \edtext{|}{
  \linenum{|\xlineref{SS.1.16.3-32}}\lemma{kanyāyāḥ |}\Afootnote{kumāryāḥ A.}
} \edtext{pratanuṃ}{
  \Afootnote{tataḥ A; pratanū N H.}
} sūcyā\edlabel{SS.1.16.3-35} \edtext{bahalam}{
  \linenum{|\xlineref{SS.1.16.3-35}}\lemma{sūcyā bahalam}\Afootnote{picuvartiṃ A.}
} ārayā\edlabel{SS.1.16.3-37} \edtext{||2||}{
  \linenum{|\xlineref{SS.1.16.3-37}}\lemma{ārayā ||2||}\Afootnote{praveśayet || A.}
}
\pend


\pstart
\edtext{śoṇitabahutvanivedanāyāṃ}{
  \Afootnote{°hutvena vedanayā A; °hutveti vedanāś N; °nā yāṃ K; °nā yā\uwave{c} H.}
} cānyadeśaviddham iti jānīyāt | \edtext{nirupadravatā}{
  \Afootnote{°vatayā A.}
} taddeśaviddhaliṅgam\edlabel{SS.1.16.4-6} \edtext{||3||}{
  \linenum{|\xlineref{SS.1.16.4-6}}\lemma{taddeśaviddhaliṅgam ||3||}\Afootnote{°viddham iti || A.}
}
\pend


\pstart
\edtext{tatra\emph{\edlabel{SS.1.16.5-0}}}{
  \Afootnote{\textsc{[sū.16.5]} tatrājñena A.}
} \edtext{yadṛcchāviddhāyāṃ}{
  \Afootnote{yadṛcchayā viddhāsu A.}
} \edtext{sirāyām}{
  \Afootnote{sirāsu kālikāmarmarikālohitikāsūpadravā bhavanti | A.}
} \edtext{ajñena}{
  \Afootnote{tatra kālikāyāṃ A.}
} \edtext{jvara}{
  \Afootnote{jvaro A.}
} \edtext{dāhaśvayathu}{
  \Afootnote{dāhaḥ śvayathur A; dāhaśvayathur \uline{N} \uline{H}.}
} \edtext{vedanā}{
  \Afootnote{\textsc{[add]} ca bhavati marmarikāyāṃ vedanā jvaro A.}
} \edtext{granthi}{
  \Afootnote{granthayaś ca lohitikāyāṃ A.}
} manyāstambhāpatānakaśirograhakarṇaśūlāni bhavanti \edtext{||4||}{
  \linenum{|\xlineref{SS.1.16.5-0}}\lemma{tatra\ldots ||4||}\Afootnote{\textsc{[om]} K.}
\lemma{||4||}  \Afootnote{\textsc{[add]} | teṣu yathāsvaṃ pratikurvīt || A.}
}
\pend


\pstart
\edtext{doṣasamudayād\emph{\edlabel{SS.1.16.6-0}}}{
  \Afootnote{°mudāyād kliṣṭajihmāpraśastasūcīvyadhād gāḍhataravartitvād A.}
} apraśastavyadhād \edtext{vā}{
  \Afootnote{\textsc{[add]} yatra saṃrambho vedanā vā bhavati A.}
} tatra vartim \edtext{apahṛtya}{
  \Afootnote{upahṛtyāśu A.}
} \edtext{yavamadhukamañjiṣṭhāgandharvahastamūlair}{
  \Afootnote{madhukairaṇḍamūlamañjiṣṭhāyavatilakalkair A; °gandarvahastamūlai N.}
} madhughṛtapragāḍhair ālepayet \edtext{|}{
  \Afootnote{\textsc{[add]} tāvad yāvat surūḍha iti A.}
} surūḍhañ cainam punar vidhyet \edtext{||5||}{
  \linenum{|\xlineref{SS.1.16.6-0}}\lemma{doṣasamudayād\ldots ||5||}\Afootnote{\textsc{[om]} K.}
\lemma{||5||}  \Afootnote{\textsc{[add]} vidhānaṃ tu pūrvoktam eva || A.}
}
\pend


\pstart
\edtext{}{
  \Afootnote{\textsc{[pre]} \textsc{[pre]} tatra A.}
} samyagviddham\edlabel{SS.1.16.7-1} āmatailapariṣekeṇopacaret\edlabel{SS.1.16.7-2} \edtext{|}{
  \linenum{|\xlineref{SS.1.16.7-2}}\lemma{āmatailapariṣekeṇopacaret |}\Afootnote{āmatailena pariṣecayet A; °ṣekaṇopacaret| H.}
} tryahāt \edtext{tryahād}{
  \Afootnote{\textsc{[add]} ca A.}
} vartiṃ \edtext{sthūlatarāṅ}{
  \Afootnote{sthūlatarīṃ N \uline{H}.}
} \edtext{kurvīta}{
  \Afootnote{dadyāt A.}
} pariṣekañ ca tam eva \edtext{||6||}{
  \linenum{|\xlineref{SS.1.16.7-1}}\lemma{samyagviddham\ldots ||6||}\Afootnote{\textsc{[om]} K.}
}
\pend


\pstart
 atha\edlabel{SS.1.16.8-1} vyapagatadoṣopadrave \edtext{karṇe}{
  \Afootnote{\textsc{[add]} la\textbf{✗} N; \textsc{[add]} lam H.}
} \edtext{pravardhanārthaṃ}{
  \Afootnote{\textsc{[om]} pra° A.}
} \edtext{laghupravardhanakena}{
  \Afootnote{laghuvardhanakaṃ A; \textsc{(l. 2)}\uwave{la}pravardhanakāmo N; °nakāmā H.}
} muñcet\edlabel{SS.1.16.8-6} \edtext{||7||}{
  \linenum{|\xlineref{SS.1.16.8-1}}\lemma{atha\ldots ||7||}\Afootnote{\textsc{[om]} K.}
  \linenum{|\xlineref{SS.1.16.8-6}}\lemma{muñcet ||7||}\Afootnote{kuryāt || A.}
}
\pend


\pstart
 evaṃ\edlabel{SS.1.16.9-1} \edtext{samvarddhitaḥ}{
  \Afootnote{vivardhitaḥ A.}
} karṇaś chidyate tu dvidhā nṛṇāṃ\edlabel{SS.1.16.9-7} \edtext{|}{
  \linenum{|\xlineref{SS.1.16.9-7}}\lemma{nṛṇāṃ |}\Afootnote{nṛṇā|  N.}
} \edtext{}{
  \Afootnote{doṣaṭo N H.}
} doṣato vābhighātād \edtext{vā\edlabel{SS.1.16.9-12}}{
  \Afootnote{sandhānaṃ A.}
} \edtext{sandhānān}{
  \linenum{|\xlineref{SS.1.16.9-12}}\lemma{vā sandhānān}\Afootnote{sandhānāntasya N.}
} tasya me \edtext{śṛṇu}{
  \linenum{|\xlineref{SS.1.16.9-1}}\lemma{evaṃ\ldots śṛṇu}\Afootnote{\textsc{[om]} K.}
} ||8|| 
\pend


\pstart
 tatra\edlabel{SS.1.16.10-1} samāsena \edtext{pañcadaśasandhānākṛtayo}{
  \Afootnote{°śakarṇabandhākṛtayaḥ | A; °dhākṛtayo N.}
} bhavanti\edlabel{SS.1.16.10-4} \edtext{|}{
  \linenum{|\xlineref{SS.1.16.10-4}}\lemma{bhavanti |}\Afootnote{\textsc{[om]} A.}
} tad yathā | nemīsandhānakaḥ\edlabel{SS.1.16.10-9} \edtext{|}{
  \linenum{|\xlineref{SS.1.16.10-9}}\lemma{nemīsandhānakaḥ |}\Afootnote{nemisan° A.}
} utpalabhedyakaḥ | vallūrakaḥ | āsaṅgimaḥ\edlabel{SS.1.16.10-15} \edtext{|}{
  \linenum{|\xlineref{SS.1.16.10-15}}\lemma{āsaṅgimaḥ |}\Afootnote{āsaṅgima ḥ| N.}
} gaṇḍakarṇaḥ| āhāryaḥ | nirvedhimaḥ | vyāyojimaḥ | kapāṭasandhikaḥ | ardhakapāṭasandhikaḥ\edlabel{SS.1.16.10-26} \edtext{|}{
  \linenum{|\xlineref{SS.1.16.10-26}}\lemma{ardhakapāṭasandhikaḥ |}\Afootnote{\textsc{[om]} a° A; \textsc{[om]} N.}
} saṅkṣiptaḥ | hīnakarṇaḥ | vallīkarṇaḥ\edlabel{SS.1.16.10-32} \edtext{|}{
  \linenum{|\xlineref{SS.1.16.10-32}}\lemma{vallīkarṇaḥ |}\Afootnote{vallīka \textsc{(l. 3)}rṇaḥ| N.}
} yaṣṭīkarṇaḥ\edlabel{SS.1.16.10-34} \edtext{|}{
  \linenum{|\xlineref{SS.1.16.10-34}}\lemma{yaṣṭīkarṇaḥ |}\Afootnote{yaṣṭikarṇaḥ A.}
} kākauṣṭhaḥ\edlabel{SS.1.16.10-36} \edtext{|}{
  \linenum{|\xlineref{SS.1.16.10-36}}\lemma{kākauṣṭhaḥ |}\Afootnote{kākauṣṭhaka A; kākauṣṭha\uwave{bhaḥ} H.}
} iti\edlabel{SS.1.16.10-38} \edtext{|}{
  \linenum{|\xlineref{SS.1.16.10-38}}\lemma{iti |}\Afootnote{ti | H.}
} teṣu \edtext{tatra}{
  \Afootnote{\textsc{[om]} A.}
} \edtext{pṛthulāyatasamobhayapālir\edlabel{SS.1.16.10-42}}{
  \Afootnote{pṛthulāyasa° H.}
} nemīsandhānakaḥ\edlabel{SS.1.16.10-43} \edtext{|}{
  \linenum{|\xlineref{SS.1.16.10-42}}\lemma{pṛthulāyatasamobhayapālir\ldots |}\Afootnote{pṛthulātasamone° N.}
  \linenum{|\xlineref{SS.1.16.10-43}}\lemma{nemīsandhānakaḥ |}\Afootnote{nemisan° A.}
} vṛttāyatasamobhayapālir utpalabhedyakaḥ\edlabel{SS.1.16.10-46} \edtext{|}{
  \linenum{|\xlineref{SS.1.16.10-46}}\lemma{utpalabhedyakaḥ |}\Afootnote{°bhedyaḥ| N; °bhedakaḥ| H.}
} hrasvavṛttasamobhayapālir vallūrakarṇakaḥ\edlabel{SS.1.16.10-49} \edtext{|}{
  \linenum{|\xlineref{SS.1.16.10-49}}\lemma{vallūrakarṇakaḥ |}\Afootnote{valūra° N; vallūrakaḥ A.}
} abhyantaradīrghaikapālir āsaṅgimaḥ | \edtext{bāhya}{
  \Afootnote{bāhyaika N H.}
} dīrghaikapālir gaṇḍakarṇakaḥ\edlabel{SS.1.16.10-56} \edtext{|}{
  \linenum{|\xlineref{SS.1.16.10-56}}\lemma{gaṇḍakarṇakaḥ |}\Afootnote{gaṇḍakarṇaḥ | A; gaṇḍaka rṇṇakaḥ | N.}
} apālir \edtext{ubhayato'py}{
  \Afootnote{°to py N.}
} āhāryaḥ | \edtext{pīṭhopamapālir}{
  \Afootnote{\textsc{[add]} ubhayataḥ kṣīṇaputrikāśrito A.}
} nirvedhimaḥ | \edtext{aṇusthūlasamaviṣamapālir}{
  \Afootnote{aśusthū° H; sthūlāṇusa° A.}
} vyāyojimaḥ | abhyantaradīrghaikapālir itarālpapāliḥ kapāṭasandhikaḥ\edlabel{SS.1.16.10-70} \edtext{|}{
  \linenum{|\xlineref{SS.1.16.10-70}}\lemma{kapāṭasandhikaḥ |}\Afootnote{ka\textsc{(l. 1)}vāṭāsan° H.}
} bāhyadīrghaikapālir \edtext{itarālpapāliś}{
  \Afootnote{°pāli ś N.}
} cārdhakapāṭasandhikaḥ\edlabel{SS.1.16.10-74} \edtext{|}{
  \linenum{|\xlineref{SS.1.16.10-74}}\lemma{cārdhakapāṭasandhikaḥ |}\Afootnote{ardha° A; vārddhakavāṭa° H; cārddhakavāpasan° N.}
} \edtext{tatraite}{
  \Afootnote{tatra A.}
} \edtext{daśakarṇasandhivikalpā}{
  \Afootnote{daśaite karṇabandhavikalpāḥ A.}
} bandhyā\edlabel{SS.1.16.10-78} bhavanti \edtext{|}{
  \linenum{|\xlineref{SS.1.16.10-78}}\lemma{bandhyā\ldots |}\Afootnote{sādhyāḥ A.}
} \edtext{teṣān}{
  \Afootnote{teṣāṃ A.}
} \edtext{nāmabhir}{
  \Afootnote{svanā° A.}
} evākṛtayaḥ prāyeṇa vyākhyātāḥ | saṃkṣiptādayaḥ pañcāsādhyāḥ | tatra \edtext{śuṣkaśaṣkulir}{
  \Afootnote{\textsc{[add]} utsannapālir A.}
} itarālpapāliḥ saṃkṣiptaḥ | anadhiṣṭhānapāliḥ paryantayoś\edlabel{SS.1.16.10-96} \edtext{ca}{
  \linenum{|\xlineref{SS.1.16.10-96}}\lemma{paryantayoś ca}\Afootnote{\textsc{[om]} N.}
\lemma{ca}  \Afootnote{\textsc{[om]} A.}
} kṣīṇamāṃso hīnakarṇaḥ | \edtext{tanuviṣamapālir}{
  \Afootnote{°ṣamālpapālir A.}
} vallīkarṇaḥ | \edtext{granthitamāṃsaḥ\edlabel{SS.1.16.10-104}}{
  \Afootnote{granthitamānsaḥ \uline{N} H.}
} \edtext{stabdhasirātatasūkṣmapālir}{
  \linenum{|\xlineref{SS.1.16.10-104}}\lemma{granthitamāṃsaḥ stabdhasirātatasūkṣmapālir}\Afootnote{grathitamāṃsastabdhasirāsaṃta° A.}
\lemma{stabdhasirātatasūkṣmapālir}  \Afootnote{stabdhasirāta ta° H; °ta sūkṣmapāliḥ N.}
} yaṣṭīkarṇaḥ\edlabel{SS.1.16.10-106} \edtext{|}{
  \linenum{|\xlineref{SS.1.16.10-106}}\lemma{yaṣṭīkarṇaḥ |}\Afootnote{yaṣṭikarṇaḥ A; yaṣṭīka rṇṇaḥ| N.}
} \edtext{nirmāṃsasaṃkṣiptāgrālpaśoṇitapāliḥ}{
  \Afootnote{nimāsa° N; nirmmānsasaṃ° H.}
} \edtext{kākauṣṭha}{
  \Afootnote{kākauṣṭhaka A.}
} iti | baddheṣv \edtext{api}{
  \Afootnote{\textsc{[add]} tu śopha A.}
} \edtext{dāha}{
  \Afootnote{\textsc{[add]} rāga A.}
} \edtext{pāka}{
  \Afootnote{\textsc{[add]} piḍakā A.}
} \edtext{srāva}{
  \Afootnote{śrāva H.}
} \edtext{śopha}{
  \Afootnote{\textsc{[om]} A; sopha N.}
} yuktā na siddhim upayānti \edtext{||9||}{
  \linenum{|\xlineref{SS.1.16.10-1}}\lemma{tatra\ldots ||9||}\Afootnote{\textsc{[om]} K.}
}
\pend


\pstart
\edtext{}{
  \Afootnote{\textsc{[pre]} \textsc{[pre]} bhavanti cātra | A.}
}
\pend


\pstart
\edtext{}{
  \Afootnote{\textsc{[pre]} \textsc{[pre]} yasya pālidvayam api karṇasya na bhaved iha | A.}
}
\pend


\pstart
\edtext{}{
  \Afootnote{\textsc{[pre]} \textsc{[pre]} karṇapīṭhaṃ same madhye tasya viddhvā vivardhayet || A.}
}
\pend


\pstart
\edtext{}{
  \Afootnote{\textsc{[pre]} \textsc{[pre]} bāhyāyām iha dīrghāyāṃ sandhir ābhyantaro bhavet | A.}
}
\pend


\pstart
\edtext{}{
  \Afootnote{\textsc{[pre]} \textsc{[pre]} ābhyantarāyāṃ dīrghāyāṃ bāhyasandhir udāhṛtaḥ || A.}
}
\pend


\pstart
\edtext{}{
  \Afootnote{\textsc{[pre]} \textsc{[pre]} ekaiva tu bhavet pāliḥ sthūlā pṛthvī sthirā ca yā | A.}
}
\pend


\pstart
\edtext{}{
  \Afootnote{\textsc{[pre]} \textsc{[pre]} tāṃ dvidhā pāṭayitvā tu chittvā copari sandhayet || A.}
}
\pend


\pstart
\edtext{}{
  \Afootnote{\textsc{[pre]} \textsc{[pre]} gaṇḍād utpāṭya māṃsena sānubandhena jīvatā | A.}
}
\pend


\pstart
\edtext{}{
  \Afootnote{\textsc{[pre]} \textsc{[pre]} karṇapālīm āpāles tu kuryān nirlikhya śāstravit || A.}
}
\pend


\pstart
 \edtext{ato\edlabel{SS.1.16.15-1}}{
  \Afootnote{tato N.}
} \edtext{'nyatamasya}{
  \Afootnote{nyatamaṃ A.}
} bandhañ cikīrṣuḥ \edtext{agropaharaṇīyoktopasambhṛtasambhāraḥ}{
  \Afootnote{agropasaṃha° N; °bhāraṃ A.}
} viśeṣataś \edtext{cātropaharet}{
  \Afootnote{cāgropaharaṇīyāt \uline{N} \uline{H}.}
} \edtext{surāmaṇḍakṣīram}{
  \Afootnote{surāmaṇḍaṃ kṣīram A.}
} udakaṃ \edtext{dhānyāmlakapālacūrṇañ}{
  \Afootnote{dhānyāmlaṃ ka° A.}
} ceti | tato 'ṅganāṃ \edtext{puruṣam}{
  \Afootnote{puruṣañ N.}
} vā grathitakeśāntaṃ \edtext{laghubhuktavantam}{
  \Afootnote{laghu bhu° A N H.}
} āptaiḥ \edtext{suparigṛhītaṃ}{
  \Afootnote{\textsc{[add]} ca A.}
} kṛtvā\edlabel{SS.1.16.15-21} \edtext{ca}{
  \linenum{|\xlineref{SS.1.16.15-21}}\lemma{kṛtvā ca}\Afootnote{\textsc{[om]} N H.}
\lemma{ca}  \Afootnote{\textsc{[om]} A.}
} \edtext{bandhān}{
  \Afootnote{bandham A.}
} \edtext{upadhārya}{
  \Afootnote{upapādya H.}
} \edtext{chedyabhedyalekhyavyadhanair}{
  \Afootnote{\textsc{[add]} upapannair A.}
} upapādya karṇaśoṇitam\edlabel{SS.1.16.15-27} \edtext{avekṣyaitad}{
  \linenum{|\xlineref{SS.1.16.15-27}}\lemma{karṇaśoṇitam avekṣyaitad}\Afootnote{°ṇitata avekṣyetad N.}
\lemma{avekṣyaitad}  \Afootnote{avekṣya A.}
} duṣṭam \edtext{aduṣṭam}{
  \Afootnote{aduṣṭaś N.}
} \edtext{veti}{
  \Afootnote{ceti | N H.}
} \edtext{tato}{
  \Afootnote{tatra A.}
} vātaduṣṭe \edtext{dhānyāmlodakābhyāṃ}{
  \Afootnote{dhānyāvloda° N; dhānyām loda° H; dhānyāmloṣṇoda° A.}
} pittaduṣṭe \edtext{śītodakapayobhyāṃ}{
  \Afootnote{śītodakopa° N.}
} śleṣmaduṣṭe \edtext{surāmaṇḍodakābhyāṃ}{
  \Afootnote{°ḍoṣṇodakābhyāṃ A.}
} prakṣālya \edtext{karṇam}{
  \Afootnote{karṇau A.}
} punar avalikhet\edlabel{SS.1.16.15-42} | \edtext{anunnatam}{
  \linenum{|\xlineref{SS.1.16.15-42}}\lemma{avalikhet\ldots anunnatam}\Afootnote{avalikhyānun° A; °kheta | anunnatam N.}
} ahīnam aviṣamañ ca \edtext{karṇasandhin}{
  \Afootnote{karṇasandhiṃ A \uline{N}.}
} \edtext{niveśya}{
  \Afootnote{sanni° A.}
} sthitaraktaṃ \edtext{sandarśya}{
  \Afootnote{sandadhyāt | tato A.}
} madhughṛtenābhyajya picuplotayor \edtext{anyatareṇāvaguṇṭhya}{
  \Afootnote{°gu\textsc{(l. 4)}ṇṭhyo H.}
} \edtext{nātigāḍhan}{
  \Afootnote{sūtreṇānavagāḍhaman A; nātigāḍhaṃ N.}
} \edtext{nātiśithilaṃ}{
  \Afootnote{ati° A.}
} \edtext{sūtreṇāvabadhya}{
  \Afootnote{ca baddhvā A; °baddha N.}
} kapālacūrṇenāvakīryācārikam upadiśet | dvivraṇīyoktena\edlabel{SS.1.16.15-61} cānnenopacaret\edlabel{SS.1.16.15-62} \edtext{||10||}{
  \linenum{|\xlineref{SS.1.16.15-1}}\lemma{ato\ldots ||10||}\Afootnote{\textsc{[om]} K.}
  \linenum{|\xlineref{SS.1.16.15-61}}\lemma{dvivraṇīyoktena\ldots ||10||}\Afootnote{dvivaṇīyoktena \uwave{upapocaret} N.}
  \linenum{|\xlineref{SS.1.16.15-62}}\lemma{cānnenopacaret ||10||}\Afootnote{ca vidhāne° A.}
}
\pend


\pstart
\edtext{}{
  \Afootnote{\textsc{[pre]} \textsc{[pre]} bhavati cātra | A; \textsc{[pre]} || bha || N.}
}
\pend


\pstart
 \edtext{vighaṭṭanan\edlabel{SS.1.16.16x-1}}{
  \Afootnote{vighaṭṭanaṃ A N.}
} divāsvapnaṃ vyāyāmam atibhojanaṃ |  \edtext{vyavāyam}{
  \Afootnote{āgnisantā\textsc{(f. 14v)}pa N.}
} agnisantāpam vākśramañ ca\edlabel{SS.1.16.16x-10} \edtext{vivarjayet}{
  \linenum{|\xlineref{SS.1.16.16x-1}}\lemma{vighaṭṭanan\ldots vivarjayet}\Afootnote{\textsc{[om]} K.}
  \linenum{|\xlineref{SS.1.16.16x-10}}\lemma{ca vivarjayet}\Afootnote{varjayet || N.}
} ||11|| 
\pend


\pstart
 \edtext{nātiśuddharaktam\edlabel{SS.1.16.17-1}}{
  \Afootnote{\textsc{[sū.16.17]} na cāśu° A; nātisuddha° N.}
} \edtext{atipravṛttaraktaṃ}{
  \Afootnote{°vṛttaṃ raktaṃ N.}
} kṣīṇaraktaṃ vā sandadhyāt | sa hi vātaduṣṭe \edtext{raktabaddho'rūḍho}{
  \Afootnote{rakte rūḍho 'pi A; raktavaddho ruḍho N; raktabaddho rūḍho H.}
} \edtext{paripuṭanavān}{
  \Afootnote{°ṭavām N; °navā\textsc{(f. 33r)}m H.}
} bhavati\edlabel{SS.1.16.17-12} \edtext{|}{
  \linenum{|\xlineref{SS.1.16.17-12}}\lemma{bhavati |}\Afootnote{\textsc{[om]} A.}
} \edtext{pittaduṣṭe}{
  \Afootnote{pittaduṣṭai N.}
} gāḍhapākarāgavān\edlabel{SS.1.16.17-15} \edtext{|}{
  \linenum{|\xlineref{SS.1.16.17-15}}\lemma{gāḍhapākarāgavān |}\Afootnote{dāhapākarāgavedanāvān A.}
} \edtext{śleṣmaduṣṭe}{
  \Afootnote{śleṣaduṣṭe N.}
} \edtext{stabdhakarṇaḥ}{
  \Afootnote{stabdhaḥ A; stabdhavarṇṇaḥ N.}
} kaṇḍūmān \edtext{atipravṛttasrāvaḥ\edlabel{SS.1.16.17-20}}{
  \Afootnote{°ttaśrāvaḥ H.}
} \edtext{śophavān}{
  \linenum{|\xlineref{SS.1.16.17-20}}\lemma{atipravṛttasrāvaḥ śophavān}\Afootnote{°ttarakte śyāvaśophavān A.}
} \edtext{kṣīṇālpamāṃso}{
  \Afootnote{kṣīṇo lpa° \uline{A} N.}
} na vṛddhim upaiti \edtext{||12||}{
  \linenum{|\xlineref{SS.1.16.17-1}}\lemma{nātiśuddharaktam\ldots ||12||}\Afootnote{\textsc{[om]} K.}
}
\pend


\pstart
\edtext{}{
  \Afootnote{\textsc{[pre]} \textsc{[pre]} āmatailena trirātraṃ pariṣecayet trirātrāc ca picuṃ parivartayet | A.}
} sa\edlabel{SS.1.16.18-1} yadā \edtext{rūḍho}{
  \Afootnote{surūḍho A; ruḍho N.}
} nirupadravaḥ \edtext{karṇo}{
  \Afootnote{savarṇo A.}
} bhavati tadainaṃ śanaiḥ śanair abhivardhayet | \edtext{anyathā}{
  \Afootnote{ato 'nyathā A.}
} \edtext{saṃrambhadāhapākavedanāvān}{
  \Afootnote{°karāgavedanāvān A N; °nāvām H.}
} bhavati\edlabel{SS.1.16.18-14} \edtext{|}{
  \linenum{|\xlineref{SS.1.16.18-14}}\lemma{bhavati |}\Afootnote{\textsc{[om]} A.}
} punar \edtext{api}{
  \Afootnote{\textsc{[om]} A.}
} chidyeta\edlabel{SS.1.16.18-18} \edtext{||13||}{
  \linenum{|\xlineref{SS.1.16.18-1}}\lemma{sa\ldots ||13||}\Afootnote{\textsc{[om]} K.}
  \linenum{|\xlineref{SS.1.16.18-18}}\lemma{chidyeta ||13||}\Afootnote{chidyate vā || A.}
}
\pend


\pstart
 \edtext{athāpraduṣṭasyābhivardhanārtham\edlabel{SS.1.16.19-1}}{
  \Afootnote{athāsyāḥ\textsc{(l. 3)} pra° H; \textsc{[sū.16.19]} athāsyāpra° A; °syāvivardhanārtham N.}
} abhyaṅgaḥ \edtext{|}{
  \Afootnote{\textsc{[add]} tad yathā A.}
} \edtext{godhāpratudaviṣkirānūpaudakavasāmajjāpayastailaṃ}{
  \Afootnote{°jjānau payaḥ sarpis tailaṃ A.}
} \edtext{gaurasarṣapajañ}{
  \Afootnote{\textsc{[om]} gaura° N.}
} ca yathālābhaṃ \edtext{saṃbhṛtyārkālarkabalātibalānantāvidārīmadhukajalaśūkaprativāpan}{
  \Afootnote{°lakavalātibalānantāvidārīmadhukajalaśūkaprativāpaṃ N; °tāpāmārgāśvagandhāvidārigandhākṣīraśuklājalaśūkamadhuravargapayasyāprativāpaṃ A.}
} \edtext{tailam}{
  \Afootnote{\textsc{[add]} vā A.}
} pācayitvā \edtext{svanuguptan}{
  \Afootnote{svanuguptaṃ A N.}
} nidadhyāt\edlabel{SS.1.16.19-12} \edtext{||14||}{
  \linenum{|\xlineref{SS.1.16.19-1}}\lemma{athāpraduṣṭasyābhivardhanārtham\ldots ||14||}\Afootnote{\textsc{[om]} K.}
  \linenum{|\xlineref{SS.1.16.19-12}}\lemma{nidadhyāt ||14||}\Afootnote{nidadyāt || N.}
}
\pend


\pstart
 \edtext{svedito\edlabel{SS.1.16.20-1}}{
  \Afootnote{svadito N.}
} \edtext{marditaṅ}{
  \linenum{|\xlineref{SS.1.16.20-1}}\lemma{svedito marditaṅ}\Afootnote{\textsc{[sū.16.20]}sveditonma° A.}
} karṇam \edtext{anena\edlabel{SS.1.16.20-4}}{
  \Afootnote{ane\textbf{✗} N.}
} \edtext{mrakṣayed}{
  \linenum{|\xlineref{SS.1.16.20-4}}\lemma{anena mrakṣayed}\Afootnote{snehenaitena yojayet |  A.}
} budhaḥ\edlabel{SS.1.16.20-6} | \edtext{}{
  \linenum{|\xlineref{SS.1.16.20-6}}\lemma{budhaḥ\ldots }\Afootnote{ato 'nu° A.}
\lemma{}  \Afootnote{tato nu° H; tato nupadravam N.}
} tato'nupadravaḥ samyag balavāṃś ca \edtext{vivardhate}{
  \linenum{|\xlineref{SS.1.16.20-1}}\lemma{svedito\ldots vivardhate}\Afootnote{\textsc{[om]} K.}
} ||15|| 
\pend


\pstart
\edtext{}{
  \Afootnote{\textsc{[pre]} \textsc{[pre]} yavāśvagandhāyaṣṭyāhvais tilaiś codvartanaṃ hitam |  śatāvaryaśvagandhābhyāṃ payasyair aṇḍajīvanaiḥ || A.}
}
\pend


\pstart
\edtext{}{
  \Afootnote{\textsc{[pre]} \textsc{[pre]} tailaṃ vipakvaṃ sakṣīram abhyaṅgāt pālivardhanam |  A.}
} ye\edlabel{SS.1.16.22-1} tu karṇā na vardhante snehasvedopapāditāḥ\edlabel{SS.1.16.22-6} \edtext{|}{
  \linenum{|\xlineref{SS.1.16.22-1}}\lemma{ye\ldots |}\Afootnote{\textsc{[om]} K.}
  \linenum{|\xlineref{SS.1.16.22-6}}\lemma{snehasvedopapāditāḥ |}\Afootnote{svedasnehopa° A.}
}
\pend


\pstart
 teṣām\edlabel{SS.1.16.23-1} \edtext{apāṅge}{
  \Afootnote{apāṅgadeśe A.}
} tv\edlabel{SS.1.16.23-3} \edtext{abahiḥ}{
  \linenum{|\xlineref{SS.1.16.23-3}}\lemma{tv abahiḥ}\Afootnote{tu A.}
\lemma{abahiḥ}  \Afootnote{avarhi N.}
} \edtext{kuryāt}{
  \Afootnote{kuyāt N.}
} \edtext{prachānam}{
  \Afootnote{prachannam H.}
} eva ca\edlabel{SS.1.16.23-8} \edtext{||16||}{
  \linenum{|\xlineref{SS.1.16.23-1}}\lemma{teṣām\ldots ||16||}\Afootnote{\textsc{[om]} K.}
  \linenum{|\xlineref{SS.1.16.23-8}}\lemma{ca ||16||}\Afootnote{tu | A.}
}
\pend


\pstart
\edtext{}{
  \Afootnote{\textsc{[pre]} \textsc{[pre]} bāhyacchedaṃ na kurvīta vyāpadaḥ syus tato dhruvāḥ || A.}
}
\pend


\pstart
\edtext{}{
  \Afootnote{\textsc{[pre]} \textsc{[pre]}baddhamātraṃ tu yaḥ karṇaṃ sahasaivābhivardhayet |  āmakośī samādhmātaḥ kṣipram eva vimucyate || A.}
}
\pend


\pstart
 amitāḥ\edlabel{SS.1.16.26.0-1} \edtext{karṇabandhās}{
  \Afootnote{karṇṇabandho H.}
} \edtext{tu}{
  \Afootnote{stu H.}
} vijñeyāḥ kuśalair iha |  yo \edtext{yathā}{
  \Afootnote{suviśiṣṭaḥ A.}
} suniviṣṭaḥ \edtext{syāt}{
  \Afootnote{taṃ A.}
} tat \edtext{tathā\edlabel{SS.1.16.26.0-14}}{
  \Afootnote{yojaye N.}
} yojayed \edtext{bhiṣak}{
  \linenum{|\xlineref{SS.1.16.26.0-1}}\lemma{amitāḥ\ldots bhiṣak}\Afootnote{\textsc{[om]} K.}
  \linenum{|\xlineref{SS.1.16.26.0-14}}\lemma{tathā\ldots bhiṣak}\Afootnote{viniyojayet || A.}
} ||17|| 
\pend


\pstart
\edtext{}{
  \Afootnote{\textsc{[pre]} \textsc{[pre]}(karṇapālyāmayānnṇnāṃ punar vakṣyāmi suśruta |  karṇapalyāṃ prakupaitā vātapittakaphās trayaḥ || A.}
}
\pend


\pstart
\edtext{}{
  \Afootnote{\textsc{[pre]} \textsc{[pre]}dvidhā vā'pyathā saṃsṛṣṭāḥ kurvanti vividhā rujaḥ |  visphoṭaḥ stabdhatā śophaḥ pālyāṃ doṣe tu vātike ||  dāhavisphiṭajananaṃ śophaḥ pākaś ca paittike |  kaṇḍūḥ saśvayathuḥ stambho gurutvaṃ ca kaphātmake || A.}
}
\pend


\pstart
\edtext{}{
  \Afootnote{\textsc{[pre]} \textsc{[pre]} yathādoṣaṃ ca saṃśodhya kuryātteṣāṃ cikitsitam |  svedābhyaṅgaparīṣekaiḥ pralepāsṛgvimokṣaṇaiḥ || A.}
}
\pend


\pstart
\edtext{}{
  \Afootnote{\textsc{[pre]} \textsc{[pre]}mṛdvīṃ kriyāṃ bṛṃhaṇīyair yathāsvaṃ bhojanais tathā |  ya evaṃ vetti doṣāṇāṃ cikitsāṃ kartum arhati || A.}
}
\pend


\pstart
\edtext{}{
  \Afootnote{\textsc{[pre]} \textsc{[pre]}ata ūrdhvaṃ nāmaliṅgair vakṣye pālyām upadravān |  atpāṭakaś cotpuṭakaḥ śyāvaḥ kaṇḍūyuto bhṛśam || A.}
}
\pend


\pstart
\edtext{}{
  \Afootnote{\textsc{[pre]} \textsc{[pre]}avamanthaḥ sakaṇḍūko granthiko jambulas tathā |  srāvī ca dāhavāṃś caiva śṛṇveṣāṃ kramaśaḥ kriyām || A.}
}
\pend


\pstart
\edtext{}{
  \Afootnote{\textsc{[pre]} \textsc{[pre]} apāmārgaḥ sarjarasaḥ pāṭalālakucatvacau |  utpāṭake pralepaḥ syāttailamebhiś ca pācayet || A.}
}
\pend


\pstart
\edtext{}{
  \Afootnote{\textsc{[pre]} \textsc{[pre]} śampākaśigrupūtīkān godāmedo 'tha tadvasām |  vārāhaṃ gavyamaiṇeyaṃ pittaṃ sarpiś ca saṃsṛjet || A.}
}
\pend


\pstart
\edtext{}{
  \Afootnote{\textsc{[pre]} \textsc{[pre]} lepam utpuṭake dadyāttailamebhiś ca sādhitam |  gaurīṃ sugandhāṃ saśyāmāmanantāṃ taṇḍulīyakam | A.}
}
\pend


\pstart
\edtext{}{
  \Afootnote{\textsc{[pre]} \textsc{[pre]} śyāve pralepanaṃ dadyāttailamebhiś ca sādhitam |  pāṭhāṃ rasāñjanaṃ kṣaudraṃ tathā syāduṣṇakāñjikam || A.}
}
\pend


\pstart
\edtext{}{
  \Afootnote{\textsc{[pre]} \textsc{[pre]} dadyāl lepaṃ sakaṇḍūke tailamebhiś ca sādhitam |  vraṇībhūtasya deyaṃ syādidaṃ tailaṃ vijānatā || A.}
}
\pend


\pstart
\edtext{}{
  \Afootnote{\textsc{[pre]} \textsc{[pre]} madhukakṣīrakākolījīvakādyair vipācitam |  godhāvarāhasarpāṇāṃ vasāḥ syuḥ kṛtabṛṃhaṇe || A.}
}
\pend


\pstart
\edtext{}{
  \Afootnote{\textsc{[pre]} \textsc{[pre]} pralepanam idaṃ dadyād avasicyāvamanthake |  prapauṇḍarīkaṃ madhukaṃ samaṅgāṃ dhavam eva ca || A.}
}
\pend


\pstart
\edtext{}{
  \Afootnote{\textsc{[pre]} \textsc{[pre]} tailam ebhiś ca saṃpakvaṃ śṛṇu kaṇḍūmataḥ kriyām |  sahadevā viśvadevā ajākṣīraṃ sasaindhavam |  etair ālepanaṃ dadyāt tailam ebhiś ca sādhitam || A.}
}
\pend


\pstart
\edtext{}{
  \Afootnote{\textsc{[pre]} \textsc{[pre]} granthike guṭikāṃ pūrvaṃ srāvayed avapāṭya tu |  tataḥ saindhavacūrṇaṃ tu ghṛṣṭvā lepaṃ pradāpayet || A.}
}
\pend


\pstart
\edtext{}{
  \Afootnote{\textsc{[pre]} \textsc{[pre]} likhitvā tatsrutaṃ ghṛṣṭvā cūrṇair lodhrasya jambule |  kṣīreṇa pratisāryainaṃ śuddhaṃ saṃropayet tataḥ || A.}
}
\pend


\pstart
\edtext{}{
  \Afootnote{\textsc{[pre]} \textsc{[pre]} madhuparṇī madhūkaṃ ca ma madhukaṃ madhunā saha |  lepaḥ srāviṇi dātavyas tailam ebhiś ca sādhitam || A.}
}
\pend


\pstart
\edtext{}{
  \Afootnote{\textsc{[pre]} \textsc{[pre]} pañcavalkaiḥ samadhukaiḥ piṣṭais taiś ca ghṛtānvitaiḥ |  jīvakādyaiḥ sasarpiṣkair dahyamānaṃ pralepayet ||) A.}
}
\pend


\pstart
jātaromā\emph{\edlabel{SS.1.16.25-0}} \edtext{suvartmā}{
  \Afootnote{suparmā N; suvarmmā H.}
} ca \edtext{śliṣṭasandhiḥ}{
  \Afootnote{śliṣṭasandhim N.}
} samaḥ sthiraḥ |  surūḍho 'vedano \edtext{yas}{
  \Afootnote{ca A.}
} \edtext{tu}{
  \Afootnote{tat N H.}
} taṃ karṇaṃ vardhayec \edtext{chanaiḥ}{
  \linenum{|\xlineref{SS.1.16.25-0}}\lemma{jātaromā\ldots chanaiḥ}\Afootnote{\textsc{[om]} K.}
} ||18|| 
\pend


\pstart
\edtext{viśleṣitāyām\emph{\edlabel{SS.1.16.27-0}}}{
  \Afootnote{°tāyās tv A.}
} atha \edtext{nāsikāyāṃ}{
  \Afootnote{nāsikāyā  A N.}
}  vakṣyāmi sandhānavidhiṃ yathāvat \edtext{|}{
  \Afootnote{°māṇa N.}
} \edtext{}{
  \Afootnote{°hāṇam N.}
} \edtext{nāsāpramāṇaṃ}{
  \Afootnote{patra N.}
} pṛthivīruhāṇāṃ  patraṃ gṛhītvā \edtext{tv}{
  \linenum{|\xlineref{SS.1.16.27-0}}\lemma{viśleṣitāyām\ldots tv}\Afootnote{\textsc{[om]} K.}
} avalambi tasya ||19|| 
\pend


\pstart
 tena\edlabel{SS.1.16.28-1} pramāṇena hi gaṇḍapārśvād  \edtext{utkṛtya}{
  \Afootnote{baddhaṃ A; vandhra H.}
} vadhraṃ tv atha nāsikāgraṃ |  vilikhya \edtext{cāśu}{
  \Afootnote{tat A.}
} \edtext{pratisandadhīta}{
  \Afootnote{sādhubandhair A; sādhuvaddha N.}
}  taṃ \edtext{sādhubaddham}{
  \linenum{|\xlineref{SS.1.16.28-1}}\lemma{tena\ldots sādhubaddham}\Afootnote{\textsc{[om]} K.}
} bhiṣag apramattaḥ ||20|| 
\pend


\pstart
\edtext{susīvitaṃ\emph{\edlabel{SS.1.16.29-0}}}{
  \Afootnote{\textsc{[sū.16.29]} susaṃhitaṃ A; susīvita N; suśīvitaṃ H.}
} samyag ato yathāvan  nāḍīdvayenābhisamīkṣya\edlabel{SS.1.16.29-5} \edtext{nahyet}{
  \linenum{|\xlineref{SS.1.16.29-5}}\lemma{nāḍīdvayenābhisamīkṣya nahyet}\Afootnote{baddhvā |  A.}
} |\edlabel{SS.1.16.29-7} \edtext{}{
  \linenum{|\xlineref{SS.1.16.29-7}}\lemma{| }\Afootnote{prānnamya cainām A.}
} \edtext{unnāmayitvā}{
  \Afootnote{°rṇayet tu  A.}
} \edtext{tv}{
  \Afootnote{pattrāṅga° H; pataṅga° A \uline{N}.}
} avacūrṇayīta \edtext{}{
  \linenum{|\xlineref{SS.1.16.29-0}}\lemma{susīvitaṃ\ldots }\Afootnote{\textsc{[om]} K.}
} pattāṅgayaṣṭīmadhukāñjanaiś ca ||21|| 
\pend


\pstart
saṃchādya\emph{\edlabel{SS.1.16.30-0}} samyak picunā \edtext{vraṇan\edlabel{SS.1.16.30-3}}{
  \Afootnote{vraṇa N.}
} \edtext{tu}{
  \linenum{|\xlineref{SS.1.16.30-3}}\lemma{vraṇan tu}\Afootnote{sitena  A.}
\lemma{tu}  \Afootnote{tun  N.}
}  tailena siñced asakṛt tilānāṃ |  ghṛtañ ca pāyyaḥ sa naraḥ \edtext{sujīrṇe}{
  \Afootnote{virecya N H.}
} \edlabel{SS.1.16.30-18} \edtext{snigdho}{
  \linenum{|\xlineref{SS.1.16.30-0}}\lemma{saṃchādya\ldots snigdho}\Afootnote{\textsc{[om]} K.}
  \linenum{|\xlineref{SS.1.16.30-18}}\lemma{ snigdho}\Afootnote{sa ya° A; °deśaḥ || N.}
} virecyaḥ svayathopadeśaṃ ||22|| 
\pend


\pstart
rūḍhañ\emph{\edlabel{SS.1.16.31-0}} ca \edtext{sandhānam}{
  \Afootnote{sandhām N.}
} \edtext{upāgataṃ}{
  \Afootnote{upāgataś H.}
} \edtext{vai}{
  \Afootnote{syāt  A; cai  H.}
} \edtext{}{
  \Afootnote{tad ardhaśeṣaṃ A; tadvadhraseṣan N.}
} tadvadhraśeṣaṃ tu punar nikṛntet \edtext{|}{
  \Afootnote{hīnāṃ A.}
}  hīnam \edtext{punar}{
  \Afootnote{yatetaḥ  N.}
} \edtext{vardhayituṃ}{
  \Afootnote{samāṃ A.}
} yateta  \edtext{samañ\edlabel{SS.1.16.31-17}}{
  \Afootnote{°ddhamānsam N.}
} ca kuryād\edlabel{SS.1.16.31-19} ativṛddhamāṃsam \edtext{iti}{
  \linenum{|\xlineref{SS.1.16.31-0}}\lemma{rūḍhañ\ldots iti}\Afootnote{\textsc{[om]} K.}
  \linenum{|\xlineref{SS.1.16.31-17}}\lemma{samañ\ldots iti}\Afootnote{°māṃsām || A.}
  \linenum{|\xlineref{SS.1.16.31-19}}\lemma{kuryād\ldots iti}\Afootnote{\textsc{[om]} N.}
} ||23|| om || 
\pend


\pstart
\edtext{}{
  \Afootnote{\textsc{[pre]} \textsc{[pre]} nāḍīyogaṃ vinauṣṭhasya nāsāsandhānavad vidhim |  ya evam eva jānīyāt sa rājñaḥ kartum arhati || A.}
}
\pend



\endnumbering
\endgroup
\end{document}